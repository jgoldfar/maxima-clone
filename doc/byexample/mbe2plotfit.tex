% e. woollett
% revised june 2013 to take out white space at end
% revised jan 2012 to add path to file names
% using function mkp and global variable bpath
%  revised may 2011
%   feb 08, april may 08 july 09
% file work2/plotfit.tex
% maxima by example, ch.2 Plots, Files, Read, Write, and Fit
% april 30, 09, convert to new style
% C:\work2\mbe1-start.tex
% sec 1.4.5 used makelist to create xylist
% sec 1.5.10 and 1.5.12 used apply('matrix, datalist) to get matrix
% april 9, 08, sec -> subsec renumbering
% mar 23, 08: sec variables in maxima : quote from macrakis about don't need to declare x
%             using the maxima interface:  closing bott. window w keybd cmmds
%needs step.eps
% edit with Notepad++, then load into LED for latexing
% add write data to file example
\documentclass[11pt]{article}
\usepackage[dvips,top=1.5cm,left=1.5cm,right=1.5cm,foot=1cm,bottom=1.5cm]{geometry}
\usepackage{times,amsmath,amsbsy,graphicx,fancyvrb,url}
\usepackage[usenames]{color}
%\definecolor{MyDarkBlue}{rgb}{0,0.08,0.45}
\definecolor{mdb}{rgb}{0.1,0,0.55}
\newcommand{\tcdb}{\textcolor{mdb}}
\newcommand{\tcbr}{\textcolor{BrickRed}}
\newcommand{\tcb}{\textcolor{blue}}
\newcommand{\tcr}{\textcolor{red}}
\urldef\tedhome\url{ http://www.csulb.edu/~woollett/  }
\urldef\tedmail\url{ woollett@charter.net}
%1.  this is for maxima code: red framed bold, footnotesize 
\DefineVerbatimEnvironment%
   {myVerbatim}%
   {Verbatim}%
   {fontfamily=courier,fontseries=b,fontsize=\footnotesize ,frame=single,rulecolor=\color{BrickRed}}
\DefineVerbatimEnvironment%
   {myVerbatim1}%
   {Verbatim}%
   {fontfamily=courier,fontseries=b,fontsize=\scriptsize ,frame=single,rulecolor=\color{BrickRed}}
%2.  this is for blue framed bold 
\DefineVerbatimEnvironment%
   {myVerbatim2}%
   {Verbatim}%
   {fontfamily=courier,fontseries=b,frame=single,rulecolor=\color{blue}}
\DefineVerbatimEnvironment%
   {myVerbatim2s}%
   {Verbatim}%
   {fontfamily=courier,fontseries=b,fontsize=\small,frame=single,rulecolor=\color{blue}}
\DefineVerbatimEnvironment%
   {myVerbatim2f}%
   {Verbatim}%
   {fontfamily=courier,fontseries=b,fontsize=\footnotesize,frame=single,rulecolor=\color{blue}}
% 3.  this is for black framed  bold
\DefineVerbatimEnvironment%
   {myVerbatim3}%
   {Verbatim}%
   {fontfamily= courier, fontseries=b, frame=single}
% 4.  this is for no frame bold
\DefineVerbatimEnvironment%
   {myVerbatim4}%
   {Verbatim}%
   {fontfamily=courier, fontseries=b,fontsize=\small}
% 6.  for defaults use usual verbatim
\newcommand{\mv}{\Verb[fontfamily=courier,fontseries=b]}
\newcommand{\mvs}{\Verb[fontfamily=courier,fontseries=b,fontsize=\small]}
\newcommand{\mvf}{\Verb[fontfamily=courier,fontseries=b,fontsize=\footnotesize]}

\renewcommand{\thefootnote}{\ensuremath{\fnsymbol{footnote}}}
%%%%%%%%%%%%%%%%%%%%%%%%%%%%%%%%%%%%%%%%%%%%%%%%%%%%%%%%%%%%%%%%%%%%%%%%
%   title page
%%%%%%%%%%%%%%%%%%%%%%%%%%%%%%%%%%%%%%%%%%%%%%%%%%%%%%%%%%%%%%%%%%%%%%
%% \title{\color{mdb}\  Maxima by Example:\\ Ch. 2,
\title{ Maxima by Example:\\ Ch. 2, Plots, Files, Read, Write, and Fit
             \thanks{This version uses \textbf{Maxima 5.31.2}. \;
			 Check \; \textbf{ \tedhome } \; for the latest version of these notes. Send comments and
			 suggestions to \textbf{\tedmail} } }

%\author{Edwin L. Woollett \thanks{Dept. of Physics and Astron, Calif. State University at Long Beach,\\
%                    author's email: \tedmail} }
%%  \author{\color{BrickRed}\  Edwin
\author{ Edwin L. Woollett}
\date{\today}
%\date{}
%%%%%%%%%%%%%%%%%%%%%%%%%%%%%%%%%%%%%%%%%
%          document
%%%%%%%%%%%%%%%%%%%%%%%%%%%%%%%%%%%%%%%%%%
\begin{document}
%\SMALL
%\small
\maketitle
\small
\tableofcontents
\normalsize
\numberwithin{equation}{section}
\newpage
%\normalsize
\begin{myVerbatim2} 
COPYING AND DISTRIBUTION POLICY    
This document is part of a series of notes titled
"Maxima by Example" and is made available
via the author's webpage http://www.csulb.edu/~woollett/
to aid new users of the Maxima computer algebra system.	
	
NON-PROFIT PRINTING AND DISTRIBUTION IS PERMITTED.
	
You may make copies of this document and distribute them
to others as long as you charge no more than the costs of printing.	

Keeping a set of notes about using Maxima up to date is easier
than keeping a published book up to date, especially in view of
the regular changes introduced in the Maxima software updates.
\end{myVerbatim2}	
\smallskip
\noindent \tcbr{Feedback from readers is the best way for this series of notes
  to become more helpful to new users of Maxima}.
\tcdb{\emph{All} comments and suggestions for improvements will be appreciated and
  carefully considered}.
\smallskip
\begin{myVerbatim2s}
LOADING FILES
The defaults allow you to use the brief version load(fft) to load in the
Maxima file fft.lisp.
To load in your own homemade file, such as qxxx.mac 
using the brief version load(qxxx), you either need to place 
qxxx.mac in one of the folders Maxima searches by default, or
else put a line like:
  
file_search_maxima : append(["c:/work2/###.{mac,mc}"],file_search_maxima )$
  
in your personal startup file maxima-init.mac (see Ch. 1, Introduction to
  Maxima for more information about this).

Otherwise you need to provide a complete path in double quotes,
as in load("c:/work2/qxxx.mac"),
 
We always use the brief load version in our examples, which are generated 
using the XMaxima graphics interface on a Windows XP computer, and copied
into a fancy verbatim environment in a latex file which uses the fancyvrb
and color packages.
\end{myVerbatim2s} 
\smallskip
\begin{myVerbatim}
Maxima.sourceforge.net. Maxima, a Computer Algebra System. Version 5.31.2
 (2013). http://maxima.sourceforge.net/
\end{myVerbatim}   
\newpage
%\normalsize
\setcounter{section}{2}
\pagestyle{headings}
\subsection{Introduction to \textbf{plot2d}}
\small
You should be able to use any of our examples with either \textbf{wxMaxima}
  or \textbf{Xmaxima}.
If you substitute the word \mv|wxplot2d| for the word \mv|plot2d|, you should get
  the same plot (using \textbf{wxMaxima}), but the plot will be drawn ``inline'' 
  in your notebook rather than in a separate gnuplot window, and the vertical axis
  labeling will be rotated by 90 degrees as compared to the gnuplot window
  graphic produced by \mv|plot2d|.\\
  
\noindent To save a plot as an image file, using \textbf{wxMaxima}, right click on the inline
  plot, choose a name and a destination folder, and click ok.\\
  
\noindent The \textbf{XMaxima} interface default plot mode is set to ``Separate'', which the author
  recommends you use normally.
You can check this setting in the \textbf{XMaxima} window by selecting \textbf{Options},
  \textbf{Plot Windows}, and see that \textbf{Separate} is checked.
(You can then back out by pressing Esc repeatedly.)\\

\noindent To save a \textbf{XMaxima} plot drawn in a separate \textbf{gnuplot} window,
 left click the left-most icon in the icon bar of the gnuplot window, which is labeled
  ``Copy the plot to clipboard'', and
  then open any utility which can open a picture file and select Edit, Paste, and then
  File, Save As.
A standard utility which comes with \textbf{Windows XP} is the accessory \textbf{Paint}, which will work fine
  in this role to save the clipboard image file.
The freely available \textbf{Infanview} is a combination picture viewer and editor which can
  also be used for this purpose.
Saving the image via the gnuplot window route results in a larger image.  \\

\noindent We discuss later how to use an optional plot2d list to force a particular
  type of image output. 
A simple example is
\begin{myVerbatim2s}
    plot2d (sin(u),['u,0,%pi], [gnuplot_term,'pdf])$
\end{myVerbatim2s}
 which will create the image file \mv|maxplot.pdf| in your
  current working directory.
\subsubsection{First Steps with \textbf{plot2d}}
The syntax of \mv|plot2d| is
\begin{myVerbatim2}
 plot2d( object-list, draw-parameter-list, other-option-lists).
\end{myVerbatim2}
The required object list (the first item) \tcbr{may} be simply one object (not a list).
The object types may be expressions (or functions), all depending on the same draw
  parameter, discrete data objects, and parametric objects.
If at least one of the plot objects involves a draw parameter, say \mv|p|, then a draw parameter range 
  list of the form \mv|[p, pmin, pmax]| should follow the object list.\\
 
\noindent We start with the simplest version  which only controls how much of the expression to 
  plot, and does not try to control the canvas width or height.
\begin{myVerbatim}
(%i1) plot2d ( sin(u), ['u, 0, %pi/2] )$
\end{myVerbatim}
%\newpage
which produces (on the author's Windows XP system) approximately:

\smallskip
\begin{figure} [h]  
   \centerline{\includegraphics[scale=.4]{ch2p01.eps} }
	\caption{ plot2d ( sin(u), ['u, 0, \%pi/2] ) }
\end{figure}

%plot2d(sin(u),['u,0,%pi/2],[psfile,"ch2p01.eps"])$
\newpage
\normalsize
\noindent While viewing the resulting plot, use of the two-key command \mv|Alt-Spacebar|,
  which normally (in Windows)
  brings up a resizing menu, instead switches from the  gnuplot figure to
  a raw gnuplot window.
You can get back to the figure using \mv|Alt-Tab|, but the raw gnuplot
  window remains in the background. 
You should resize the figure window by clicking on the Maximize (or Restore)
  icon in the upper right-hand corner.
Both the figure and the raw gnuplot window disappear when you close (using, say
  \mv|Alt-F4|, twice: once to close the figure window, and once to close the
  raw gnuplot window).\\
  
\noindent Returning to the drawn plot, we see that \textbf{plot2d} has made the 
  canvas width only as wide as the drawing
  width, and has made the canvas height only as high as the drawing height.
Now let's add a horizontal range (canvas width) control list in the
  form \mv|['x,-0.2,1.8]|.
Notice the special role the symbol \mv|x| plays here in \mv|plot2d|.
\mv|u| is a plot parameter, and \mv|x| is a horizontal range control parameter.
\begin{myVerbatim}
(%i2) plot2d ( sin(u),['u,0,%pi/2],['x, -0.2, 1.8] )$
\end{myVerbatim}
%\newpage
which produces approximately:

\smallskip
\begin{figure} [h]  
   \centerline{\includegraphics[scale=.6]{ch2p02.eps} }
	\caption{ plot2d ( sin(u), ['u, 0, \%pi/2], ['x,-0.2,1.8] ) }
\end{figure}

%plot2d(sin(u),['u,0,%pi/2],['x,-0.2,1.8],
%         [psfile,"ch2p02.eps"])$
\noindent We see that we now have separate draw width and canvas width controls 
  included.
If we try to put the canvas width control list before the draw width control list, we get an
  error message:
\begin{myVerbatim}
(%i3) plot2d(sin(u),['x,-0.2,1.8], ['u,0,%pi/2] )$
set_plot_option: unknown plot option: u
 -- an error. To debug this try: debugmode(true);
\end{myVerbatim}
However, if the expression variable \tcbr{happens} to be \mv|x|, the following
  command includes both draw width and canvas width using separate \mv|x|
  symbol control lists, and results in the correct plot:
\begin{myVerbatim}
(%i4) plot2d ( sin(x), ['x,0,%pi/2], ['x,-0.2,1.8] )$
\end{myVerbatim}
\noindent in which the first (required) \mv|x| drawing parameter list determines the drawing range, and the
   second (optional) \mv|x| control list determines the canvas width.\\
   
\noindent Despite the special role the symbol \mv|y| also plays in \textbf{plot2d},
  the following command produces the same plot as above.
\begin{myVerbatim}
(%i5) plot2d ( sin(y), ['y,0,%pi/2], ['x,-0.2,1.8] )$
\end{myVerbatim}
\newpage
\noindent The \textbf{optional} vertical canvas height control list uses the special symbol \mv|y|,
  as shown in
\begin{myVerbatim}
(%i6) plot2d ( sin(u), ['u,0,%pi/2], ['x,-0.2,1.8], ['y,-0.2, 1.2] )$
\end{myVerbatim}
  which produces
\smallskip
\begin{figure} [h]  
   \centerline{\includegraphics[scale=.6]{ch2p03.eps} }
	\caption{plot2d ( sin(u), ['u,0,\%pi/2], ['x,-0.2,1.8], ['y,-0.2, 1.2] )}
\end{figure}

%plot2d(sin(u),['u,0,%pi/2],['x,-0.2,1.8],
%           ['y,-0.2, 1.2],[psfile,"ch2p03.eps"])$
  and the following alternatives produce exactly the same plot.
\begin{myVerbatim}
(%i7) plot2d ( sin(u), ['u,0,%pi/2], ['y,-0.2, 1.2], ['x,-0.2,1.8] )$
(%i8) plot2d ( sin(x), ['x,0,%pi/2], ['x,-0.2,1.8], ['y,-0.2, 1.2] )$
(%i9) plot2d ( sin(y), ['y,0,%pi/2], ['x,-0.2,1.8], ['y,-0.2, 1.2] )$
\end{myVerbatim}
%\newpage
\subsubsection{\textbf{Parametric} Plots}
For orientation, we will draw a sine curve using the parametric plot object
  syntax and using a parametric parameter \mv|t|.
\begin{myVerbatim}
(%i1) plot2d ( [parametric, t, sin(t), ['t, 0, %pi] ] )$
\end{myVerbatim}
%\newpage
which produces

\smallskip
\begin{figure} [h]  
   \centerline{\includegraphics[scale=.4]{ch2p04.eps} }
	\caption{plot2d ( [parametric, t, sin(t), ['t, 0, \%pi] ] )}
\end{figure}

\newpage
\noindent As the \mv|plot2d| section of the manual asserts, the general syntax
  for a \mv|plot2d| parametric plot is
\begin{myVerbatim2}
  plot2d (... [parametric,x_expr,y_expr,t_range],...)  
\end{myVerbatim2}
  in which \mv|t_range| has the form of a list: \mv|['t,tmin,tmax]| if the
  two expressions are functions of \mv|t|, say.
There is no restriction on the name used for the parametric parameter.\\

\noindent We see that
\begin{myVerbatim2}
  plot2d ( [ parametric,  fx(t),  fy(t), [ 't, tmin, tmax ] ] )$
\end{myVerbatim2}
  plots pairs of points \mv| ( fx (ta), fy(ta) ) | for \mv|ta| in the interval \mv|[tmin, tmax]|.
We have used \textbf{no} canvas width control list \mv|[ 'x, xmin, xmax ]| in this minimal version.\\
\subsubsection{Can We Draw A Circle?}
This is a messy subject.
We will only consider the separate gnuplot window mode (not the
  embedded plot mode) and assume  a maximized gnuplot window 
  (as large as the monitor allows).\\
  
\noindent We use a parametric plot to create a ``circle'', letting
  \mv|fx(t) = cos(t)| and \mv|fy(t) = sin(t)|, and again adding no canvas width or height
  control lists.  
  
\begin{myVerbatim}
(%i2) plot2d ([parametric, cos(t), sin(t), ['t,-%pi,%pi]])$
\end{myVerbatim}  
If this plot is viewed in a maximized gnuplot window, the height
  to width ratio is about 0.6 on the author's equipment.
The corresponding eps file for the figure included here has a height to width ratio
  of about 0.7 when viewed with GSView, and about the same ratio in
  this pdf file:
  
\smallskip
\begin{figure} [h]  
   \centerline{\includegraphics[scale=.5]{ch2p05.eps} }
	\caption{ plot2d ( [ parametric, cos(t), sin(t), ['t, -\%pi, \%pi] ] ) }
\end{figure}

%plot2d ([parametric, cos(t), sin(t), ['t,-%pi,%pi]],
%                  [psfile,"ch2p05.eps"])$
\newpage 
\noindent There are two approaches to better ``roundness''. 
The first approach is to use the \mv|plot2d| option\\
  \mv|[gnuplot_preamble,"set size ratio 1;"]|, as in
\begin{myVerbatim2}
(%i3) plot2d ([parametric, cos(t), sin(t), ['t,-%pi,%pi]],
          [gnuplot_preamble,"set size ratio 1;"])$
\end{myVerbatim2}
With this command, the author gets a height to width ratio of about 0.9 using the fullscreen
  gnuplot window choice.  
The eps file save of this figure, produced with the code
\begin{myVerbatim2}
plot2d ([parametric, cos(t), sin(t), ['t,-%pi,%pi]],
           [gnuplot_preamble,"set size ratio 1;"],
           [psfile,"ch2p06.eps"] )$
\end{myVerbatim2}
 had a height to width ratio close to 1 when viewed with GSView 
  and close to that value in this pdf file:

\smallskip
\begin{figure} [h]  
   \centerline{\includegraphics[scale=.7]{ch2p06.eps} }
	\caption{ plot2d ( adding set size ratio 1 option ) }
\end{figure}

% plot2d ([parametric, cos(t), sin(t), ['t,-%pi,%pi]],
%           [gnuplot_preamble,"set size ratio 1;"],
%           [psfile,"ch2p06.eps"] )$
  
We conclude that the gnuplot preamble method of getting an approximate
  circle works quite well for an eps graphics file, and we will use that
  method to produce figures for this pdf file.\\
  
\noindent The alternative approach is to handset the x-range and y-range
  experimentally until the resulting ``circle'' measures the same
  width as height, for example forcing the horizontal canvas width
  to be, say, 1.6 as large as the vertical canvas height.
\begin{myVerbatim}
(%i4) plot2d ([parametric, cos(t), sin(t), ['t, -%pi, %pi] ], ['x,-1.6,1.6])$
\end{myVerbatim}
 has a height to width ratio of about 1.0 (fullscreen Gnuplot window) on the
  author's equipment.
Notice above that the vertical range is determined by the curve properties
  and extends over \mv|(y= -1, y = 1)|.
The y-range here is 2, the x-range is 3.2, so the x-range is 1.6 times
  the y-range.
  

\newpage
\noindent We now make a plot consisting of two plot objects, the first being the explicit
  expression \mv|u^3|, and the second being the parametric plot object used above.
We now need the syntax
\begin{myVerbatim2}
plot2d ([plot-object-1, plot-object-2], possibly-required-draw-range-control,
                   other-option-lists )
\end{myVerbatim2}
Here is an example :
\begin{myVerbatim}
(%i5) plot2d ([u^3,[parametric, cos(t), sin(t), ['t,-%pi,%pi]]],
                ['u,-1.1,1.1],['x,-1.5,1.5],['y,-1.5,1.5],
                 [gnuplot_preamble,"set size ratio 1;"])$
\end{myVerbatim}
  in which \mv|['u,-1.1,1.1]| is required to determine the drawing range of \mv|u^3|,
  and we have added separate horizontal and vertical canvas control lists
  as well as the gnuplot\_preamble option to approximate a circle, since it is quicker than fiddling
  with the x and y ranges by hand.\\  

\noindent To get the corresponding eps file figure for incorporation in this pdf file,
  we used the code 
\begin{myVerbatim2}
plot2d ([u^3,[parametric, cos(t), sin(t), ['t,-%pi,%pi]]],
                ['u,-1.1,1.1],['x,-1.5,1.5],['y,-1.5,1.5],
                 [gnuplot_preamble,"set size ratio 1;"],
                 [psfile,"ch2p07.eps"])$
\end{myVerbatim2}
 which produces
 
\smallskip
\begin{figure} [h]  
   \centerline{\includegraphics[scale=.7]{ch2p07.eps} }
	\caption{ Combining an Explicit Expression with a Parametric Object }
\end{figure}

%
% plot2d ([u^3,[parametric, cos(t), sin(t), ['t,-%pi,%pi]]],
%                ['u,-1.1,1.1],['x,-1.95,1.95],['y,-1.5,1.5],
%                 [psfile,"ch2p07.eps"])$


 in which the horizontal axis label (by default) is \mv|u|.  
\newpage
\noindent We now add a few more options to make this combined plot
  look cleaner and brighter (more about some of these later).
\begin{myVerbatim}
(%i6) plot2d (
              [ [parametric, cos(t), sin(t), ['t,-%pi,%pi],[nticks,200]],u^3],
                ['u,-1,1], ['x,-1.2,1.2] , ['y,-1.2,1.2], 
                [style, [lines,8]],  [xlabel," "], [ylabel," "],
                [box,false], [axes, false],              
                [legend,false],[gnuplot_preamble,"set size ratio 1;"])$
\end{myVerbatim}
  and the corresponding eps file (with \mv|[lines,20]| for increased line width) produces:
  
\smallskip
\begin{figure} [h]  
   \centerline{\includegraphics[scale=1]{ch2p08.eps} }
	\caption{ Drawing of $\mathbf{u^3}$ Over a Circle }
\end{figure}

%plot2d (
%              [ [parametric, cos(t), sin(t), ['t,-%pi,%pi],[nticks,200]],u^3],
%                ['u,-1,1], ['x,-1.2,1.2] , ['y,-1.2,1.2], 
%                [style, [lines,20]],  [xlabel," "], [ylabel," "],
%                [box,false], [axes, false],              
%                [legend,false],[gnuplot_preamble,"set size ratio 1;"],
%                [psfile,"ch2p08.eps"])$

\noindent The default value of \textbf{nticks} inside \textbf{plot2d} is \mv|29|, and
  using \mv|[nticks, 200]| yields a much smoother parametric curve.
\subsubsection{Line Width and Color Controls}
Each element to be included in the plot can have a separate
  \textbf{[ lines, nlw, nlc ]} entry in the \textbf{style} option
  list, with \textbf{nlw} determining the line width and \textbf{nlc} determining the line color.
The default value of \textbf{nlw} is \mv|1|, a very thin weak line.
The use of \textbf{nlw = 5} creates a strong wider line.\\

\noindent The default color choices (if you don't provide a specific vlue of \mv|nlc|)
   consist of a rotating color scheme which starts with
  \textbf{nlc = 1} (blue)  and then progresses through \textbf{nlc = 6} (cyan = greenish-blue)
  and then repeats.\\


\noindent You will see the colors with the associated values of \textbf{nlc},
   using  the following code which draws a set of vertical lines in various colors
   (a version of a histogram).
This code also shows an example of using \textbf{discrete} list objects (to
  draw straight lines), and the  use of various options available.
You should run this code with your own hardware-software setup, to see what
  the default \mv|plot2d| colors are with your system.
\begin{myVerbatim}
(%i1) plot2d(
         [ [discrete,[[0,0],[0,5]]], [discrete,[[2,0],[2,5]]],
           [discrete,[[4,0],[4,5]]],[discrete,[[6,0],[6,5]]],
           [discrete,[[8,0],[8,5]]],[discrete,[[10,0],[10,5]]],
           [discrete,[[12,0],[12,5]]],[discrete,[[14,0],[14,5]]] ],
		   
           [style,  [lines,6,0],[lines,6,1],[lines,6,2],
           [lines,6,3],[lines,6,4],[lines,6,5],[lines,6,6],
           [lines,6,7]],
		   
           [x,-2,20],   [y,-2,8],
           [legend,"0","1","2","3","4","5","6","7"],
           [xlabel," "],   [ylabel," "],
           [box,false],[axes,false])$
\end{myVerbatim}
Note that none of the objects being drawn are expressions or functions, so
  a draw parameter range list is not only not necessary but would make no 
   sense,  and that the optional horizontal canvas width control list above
   is \mv|['x,-2,20]|.\\
  
\noindent The interactive gnuplot \textbf{plot2d} colors 
  available on a Windows XP system (using XMaxima with Maxima ver. 5.31) thus 
  are: \textbf{0 = cyan, 1 = blue, 2 = red, 3 = green},
   \textbf{4 = majenta, 5 = black, 6 = cyan, 7 = blue, 8 = red, ...}\\
   
\noindent Adding the element \mv|[psfile,"c:/work2/ztest2.eps"]| to the above
  \mv|plot2d| code produces the figure displayed here:

\smallskip
\begin{figure} [h]  
   \centerline{\includegraphics[scale=.8]{ztest2.eps} }
	\caption{Cyclic \textbf{plot2d} Colors with a Windows XP System }
\end{figure}

\newpage
\noindent For a simple example which uses color and line width controls, we plot
  the expressions \mv|u^2| and \mv|u^3| on the same canvas,
   using lines in black and red colors, and add a height control list,
  which has the syntax \mvs| ['y, ymin, ymax]|.
\begin{myVerbatim}
(%i2) plot2d( [u^2,u^3],['u,0,2], ['x, -.2, 2.5],
          [style, [lines,5,5],[lines,5,2]],
           ['y,-1,4] )$
plot2d: some values were clipped.
\end{myVerbatim}
The plot2d warning should not be of concern here.\\

\noindent The width and height control list parameters have been chosen to make it easy
  to see where the two curves cross for positive \textbf{u}.
If you move the cursor over the crossing point, you can read off the coordinates
  from the cursor position printout in the lower left corner of the plot window.
This produces the plot:

\smallskip
\begin{figure} [h]  
   \centerline{\includegraphics[scale=.7]{ch2p11.eps} }
	\caption{ Black and Red Curves }
\end{figure}

\subsubsection{\textbf{Discrete} Data Plots: Point Size, Color, and Type Control}
We have seen some simple parametric plot examples above.
Here we make a more elaborate plot which includes discrete data points which locate on the
  curve special places, with information on the key legend about those
  special points.
The actual parametric curve color is chosen to be black (5) with some
  thickness (4), using \mv|[lines,4,5]| in the \mv|style| list.
We force large size points with special color choices, using the maximum amount
  of control in the \mv|[points, nsize, ncolor, ntype]| style assignments.
\begin{myVerbatim}
(%i1) obj_list : [ [parametric, 2*cos(t),t^2,['t,0,2*%pi]],
                     [discrete,[[2,0]]],[discrete,[[0,(%pi/2)^2]]],
              [discrete,[[-2,%pi^2]]],[discrete,[[0,(3*%pi/2)^2]]] ]$
(%i2) style_list : [style, [lines,4,5],[points,5,1,1],[points,5,2,1],
                 [points,5,3,1],[points,5,4,1]]$
(%i3) legend_list : [legend, " ","t = 0","t = pi/2","t = pi",
                " t = 3*pi/2"]$
(%i4) plot2d( obj_list, ['x,-3,4], ['y,-1,40],style_list,
                  [xlabel,"X = 2 cos( t ), Y = t ^2 "],
                   [ylabel, " "] ,legend_list )$
\end{myVerbatim}
\newpage
\noindent This produces the plot:
\smallskip
\begin{figure} [h]  
   \centerline{\includegraphics[scale=.7]{ch2p12.eps} }
	\caption{ Parametric Plot with Discrete Points }
\end{figure}

\noindent The \textbf{points} style option has any of the following
  forms: \mvs|[points]|, or \mvs|[points, point_size]|, 
  or\\
  \mvs|[points, point_size, point_color]|, or
  \mvs|[points, point_size, point_color, point_type]|, in order of increasing
  control.\\
  
\noindent The default \tcbr{point colors} are the same cyclic colors used by the \textbf{lines}
  style.
The default \tcbr{point type} is a cyclic order starting with 
  \textbf{1 = filled circle} ,
  then  continuing \textbf{2 = open circle}, \textbf{3 = plus sign},
  \textbf{4 = diagonal crossed lines as capital X},
  \textbf{5 = star}, \textbf{6 = filled square}, {7 = open square},
  {8 = filled triangle point up}, \textbf{9 = open triangle point up},
  \textbf{10 = filled triangle point down}, \textbf{11 = open triangle point down}
  \textbf{12 = filled diamond}, \textbf{13 = open diamond = 0},
  \textbf{14 = filled circle}, which is the same as \textbf{1}, and repeating the same
  cycle.  
Thus if you use \mvs|[points,5]| you will get good size points, and both the color and shape
  will cycle through the default order.
If you use \mvs|[points,5,2]| you will force a red color but the shape will depend on
  the contents and order of the rest of the objects list.\\
  
\noindent You can see the default points cycle through both colors and shapes
  with the code:
\begin{myVerbatim2}
plot2d([ [discrete,[[0,.5]]],[discrete,[[0.1,.5]]],
                  [discrete,[[0.2,.5]]],[discrete,[[0.3,.5]]],
                 [discrete,[[0.4,.5]]],[discrete,[[0.5,.5]]],
                  [discrete,[[0.6,.5]]],[discrete,[[0.7,.5]]],
                  [discrete,[[0,0]]],[discrete,[[0.1,0]]],
                  [discrete,[[0.2,0]]],[discrete,[[0.3,0]]],
                  [discrete,[[0.4,0]]],[discrete,[[0.5,0]]],
                  [discrete,[[0.6,0]]],[discrete,[[0.7,0]]]   ],
\end{myVerbatim2}
\newpage
\begin{myVerbatim2}
                [style,[points,5]],[xlabel,""],[ylabel,""],
                ['x,-0.2,1],['y,-0.2,0.7],
                [box,false],[axes,false],[legend,false])$
\end{myVerbatim2}

which will produce something like

\smallskip
\begin{figure} [h]  
   \centerline{\includegraphics[scale=.6]{shapes.eps} }
	\caption{default point colors and styles, 1-8, then 9-16 }
\end{figure}

\noindent You can also experiment with one shape at a time by defining a function \mv|dopt(n)|
  which selects the shape with the integer \mv|n| and leaves the color blue:
\begin{myVerbatim2}
dopt(n) := plot2d ([discrete,[[0,0]]],[style,[points,15,1,n]],
                   [box,false],[axes,false],[legend,false])$
\end{myVerbatim2}

\noindent Next we combine a list of twelve (x,y) pairs of points with the key word \textbf{discrete} to
  form a discrete object type for \textbf{plot2d}, and then look at the data points
  without adding the optional canvas width control.
Note that  using only one discrete list for all the points  results in
  all data points being displayed with the same size, color and shape.
\begin{myVerbatim}
(%i5) data_list : [discrete,
         [ [1.1,-0.9],[1.45,-1],[1.56,0.3],[1.88,2],
              [1.98,3.67],[2.32,2.6],[2.58,1.14],
           [2.74,-1.5],[3,-0.8],[3.3,1.1],
           [3.65,0.8],[3.72,-2.9] ] ]$
(%i6) plot2d( data_list, [style,[points]])$
\end{myVerbatim}
\newpage

\noindent This produces the plot
\smallskip
\begin{figure} [h]  
   \centerline{\includegraphics[scale=.6]{ch2p13.eps} }
	\caption{ Twelve Data Points with Same Size, Color, and Shape}
\end{figure}

%plot2d( data_list, [style,[points]],
%         [psfile,"ch2p13.eps"])$

\noindent We now combine the data points with a curve which
  is a possible fit to these data points over the draw parameter range  \mvs|[u, 1, 4]|.
\begin{myVerbatim}
(%i7) plot2d( [sin(u)*cos(3*u)*u^2, data_list],
               ['u,1,4], ['x,0,5],['y,-10,8],
          [style,[lines,4,1],[points,4,2,1]])$
plot2d: some values were clipped.
\end{myVerbatim}
%\newpage
which produces (approximately) the plot
\smallskip
\begin{figure} [h]  
   \centerline{\includegraphics[scale=.6]{ch2p14.eps} }
	\caption{ Curve Plus Data }
\end{figure}
 
 
%plot2d( [sin(u)*cos(3*u)*u^2, data_list],
%               [u,1,4], ['x,0,5],['y,-10,8],
%          [style,[lines,4,1],[points,4,2,1] ],
%           [psfile,"ch2p14.eps"])$
\newpage
\subsubsection{More \textbf{gnuplot\_preamble} Options}
Here is an example of using the \textbf{gnuplot\_preamble} options
  to add a grid, a title, and position the plot key at the bottom
  center of the canvas.
Note the use of a semi-colon between successive gnuplot instructions.
\begin{myVerbatim}
(%i1) plot2d([ u*sin(u),cos(u)],['u,-4,4] ,['x,-8,8],
             [style,[lines,5]],
         [gnuplot_preamble,"set grid; set key bottom center;
              set title 'Two Functions';"])$
\end{myVerbatim}
%\newpage
which produces (approximately) the plot
\smallskip
\begin{figure} [h]  
   \centerline{\includegraphics[scale=1.2]{ch1p47.eps} }
	\caption{ Using the gnuplot\_preamble Option }
\end{figure}

\noindent Another option you can use is \mv|set zeroaxis lw 5;| to get more
  prominent \mv|x| and \mv|y| axes.
Another example of a key location would be \mv|set key top left;|.
We have also previously used \mv|set size ratio 1;| to get a ``rounder'' circle.
\newpage
\subsubsection{Creating Various Kinds of Graphics Files Using \textbf{plot2d}}
The first method of exporting \mv|plot2d| drawings as special graphics files is
  to have the plot drawn in the Gnuplot window (either the XMaxima route, or using
   \mv|wxplot2d| rather than \mv|plot2d| if you are  using \mv|wxmaxima|).
Then left-click the left-most Gnuplot icon near the top of the Gnuplot
  window (``copy the plot to clipboard").
Then open an application which accomodates the graphics format you desire,
  and paste the clipboard image into the application, and then use Save As,
  selecting the graphics type of save desired.\\
  
\noindent The second method of exporting \mv|plot2d| drawings as special graphics files is
  to use the \mv|gnuplot_term|  option as part of your \mv|plot2d| command.
If you do not add an additional option of the form (for example)
\begin{myVerbatim2}
  [gnuplot_out_file, "c:/k1/myname.ext"]
\end{myVerbatim2}
  where \mv|ext| is replaced by an
  appropriate graphics type extension, then \mv|plot2d| creates
  a file with the name \mv|maxplot.ext| in your current working directory.\\
  
\noindent For example,
\begin{myVerbatim}
(%i1) plot2d (sin(u),['u,0,%pi], [gnuplot_term,'jpeg])$
\end{myVerbatim}
  will create the graphics file \mv|maxplot.jpeg|, and a further 
  commmand
\begin{myVerbatim}
(%i2) plot2d (cos(u),['u,0,%pi], [gnuplot_term,'jpeg])$
\end{myVerbatim}
 will overwrite the previous file \mv|maxplot.jpeg| to create the
   new graphics file for the plot of \mv|cos|.
To provide a different name for different plots, you would
  write, for example,
\begin{myVerbatim}
(%i3) plot2d (cos(u),['u,0,%pi],
        [gnuplot_out_file,"c:/work2/mycos1.jpeg"],
       [gnuplot_term,'jpeg])$
\end{myVerbatim}
If you do not supply the complete path, the file is written in the
 \mv|/bin| folder of the Maxima program installation.
Turning to other graphics file formats, and ignoring the naming option
  part, 
\begin{myVerbatim}
(%i4) plot2d (sin(u),['u,0,%pi],
       [gnuplot_term,'png])$
\end{myVerbatim}
  will create \mv|maxplot.png|.
\begin{myVerbatim}
(%i5) plot2d (sin(u),['u,0,%pi],
       [gnuplot_term,'eps])$
\end{myVerbatim}
will create \mv|maxplot.eps|.
\begin{myVerbatim}
(%i6) plot2d (sin(u),['u,0,%pi],
       [gnuplot_term,'svg])$
\end{myVerbatim}
will create \mv|maxplot.svg| (a ``scalable vector graphics'' file openable by \mv|inkscape|).
\begin{myVerbatim}
(%i7) plot2d (sin(u),['u,0,%pi],
       [gnuplot_term,'pdf])$
\end{myVerbatim}
will create \mv|maxplot.pdf|, which will be the cleanest plot of the above cases.
\subsubsection{Using \textbf{qplot} for Quick Plots of One or More Functions}
The file \textbf{qplot.mac} is posted with Ch. 2 and
  contains a function called \textbf{qplot} which
  can be used for quick plotting of functions in place of \textbf{plot2d}.\\
  
\noindent The function \textbf{qplot} (\mv|q| for ``quick'') accepts the default cyclic
  colors but always uses thicker lines than the \textbf{plot2d} default,
  adds more prominent x and y axes to the plot, and adds a grid (which can be
  switched off using the third from the left gnuplot icon).
Here are some examples of use.
(We include use with \textbf{discrete} lists only for completeness, since
  there is no way to get the \textbf{points} style with \textbf{qplot}.)

\begin{myVerbatim}
(%i1) load(qplot);
(%o1)                         c:/work2/qplot.mac
(%i2) qplot(sin(u),['u,-%pi,%pi])$
(%i3) qplot(sin(u),['u,-%pi,%pi],['x,-4,4])$
(%i4) qplot(sin(u),['u,-%pi,%pi],['x,-4,4],['y,-1.2,1.2])$
(%i5) qplot([sin(u),cos(u)],['u,-%pi,%pi])$
(%i6) qplot([sin(u),cos(u)],['u,-%pi,%pi],['x,-4,4])$
(%i7) qplot([sin(u),cos(u)],['u,-%pi,%pi],['x,-4,4],['y,-1.2,1.2])$
(%i8) qplot ([parametric, cos(t), sin(t), ['t,-%pi,%pi]],
              ['x,-2.1,2.1],['y,-1.5,1.5])$
\end{myVerbatim}
The last use involved only a parametric object, and the list
  \mv|['x,-2.1,2.1]| is interpreted as a horizontal canvas width control 
  list based on the symbol \mv|x|.\\

\small
\noindent While viewing the resulting plot, use of the two-key command \mv|Alt-Spacebar|,
  which normally
  brings up a resizing menu, instead switches from the  gnuplot figure to
  a raw gnuplot window.
You can get back to the figure using \mv|Alt-Tab|, but the raw gnuplot
  window remains in the background. 
You should resize the figure window by clicking on the Maximize (or Restore)
  icon in the upper right-hand corner.
Both the figure and the raw gnuplot window disappear when you close (using, say
  \mv|Alt-F4|, twice: once to close the figure window, and once to close the
  raw gnuplot window).\\
  
\noindent The next example includes both an expression depending on the parameter \mv|u|
  and a parametric object depending on a parameter \mv|t|, so we must have a 
  expression draw list of the form: \mv|['u,umin,umax]|.
\begin{myVerbatim}
(%i9) qplot ([ u^3,
               [parametric, cos(t), sin(t), ['t,-%pi,%pi]]],
               ['u,-1,1],['x,-2.25,2.25],['y,-1.5,1.5])$
\end{myVerbatim}
\noindent Here is approximatedly the result with the author's system:

\smallskip
\begin{figure} [h]  
   \centerline{\includegraphics[scale=.5]{ch2p17.eps} }
	\caption{ qplot example }
\end{figure}
 
%plot2d([ u^3, [parametric, cos(t), sin(t), ['t,-%pi,%pi]]],
%               ['u,-1,1],['x,-2.1,2.1],['y,-1.5,1.5],
%              [style,[lines,15]], [nticks,100],
%            [gnuplot_preamble, "set grid; set zeroaxis lw 15;"],
%             [legend,false],[ylabel, " "],
%            [psfile,"ch2p17.eps"])$

\newpage
\normalsize
\noindent To get the same plot using \mv|plot2d| from scratch requires the
  code:
\begin{myVerbatim2}
plot2d([ u^3, [parametric, cos(t), sin(t), ['t,-%pi,%pi]]],
               ['u,-1,1],['x,-2.25,2.25],['y,-1.5,1.5],
              [style,[lines,5]], [nticks,100],
            [gnuplot_preamble, "set grid; set zeroaxis lw 5;"],
             [legend,false],[ylabel, " "])$
\end{myVerbatim2}
\noindent Here are two \textbf{discrete} examples which draw vertical lines.
\begin{myVerbatim}
(%i10) qplot([discrete,[[0,-2],[0,2]]],['x,-2,2],['y,-4,4])$
(%i11) qplot( [ [discrete,[[-1,-2],[-1,2]]],
                  [discrete,[[1,-2],[1,2]]]],['x,-2,2],['y,-4,4])$
\end{myVerbatim}
Here is the code (in qplot.mac) which defines the Maxima function
  \textbf{qplot}.
\begin{myVerbatim2}
qplot ( exprlist, prange, [hvrange]) := 
    block([optlist, plist],
      optlist : [ [nticks,100], [legend, false], 
           [ylabel, " "], [gnuplot_preamble, "set grid; set zeroaxis lw 5;"] ], 
      optlist : cons ( [style,[lines,5]], optlist ),
      if length (hvrange) = 0 then plist : [] 
            else   plist : hvrange, 
      plist : cons (prange,plist), 
      plist : cons (exprlist,plist),
      plist : append ( plist, optlist ),
      apply (plot2d, plist ) )$
\end{myVerbatim2}
In this code, the third argument is an optional argument.
The local \textbf{plist} accumulates the arguments to be passed
  to \textbf{plot2d} by use of \textbf{cons} and \textbf{append}, and is then
   passed to \textbf{plot2d} by the use of  \textbf{apply}. 
The order of using \textbf{cons} makes sure that \textbf{exprlist} will be
  the first element, (and \textbf{prange} will be the second) seen
  by \textbf{plot2d}.
In this example you can see several tools used for programming with lists.\\

\noindent Several choices have been made in the \textbf{qplot} code to get
  quick and uncluttered plots of one or more functions.
One choice was to add a grid and stronger \mv|x| and \mv|y| axis lines.
Another choice was to eliminate the key legend by using the
  option \mv|[legend, false]|.
If you want a key legend to appear when plotting multiple functions,
  you should remove that option from the code and reload \textbf{qplot.mac}.
\newpage
\subsubsection{Plot of a Discontinuous Function}
Here is an example of a definition and plot of a discontinous function.
\begin{myVerbatim}
(%i12) fs(x) := if x >= -1 and x <= 1 then 3/2 else 0$
(%i13) plot2d (fs(u),['u,-2,2],['x,-3,3],['y,-.5,2],
        [style, [lines,5]],[ylabel,""],
         [xlabel,""])$
\end{myVerbatim}

 which produces (approximately) the plot:
 
\smallskip
\begin{figure} [h]  
   \centerline{\includegraphics[scale=.6]{ch2p22.eps} }
	\caption{Plot of a Discontinuous Function}
\end{figure}
  
%plot2d (fs(u),['u,-2,2],['x,-3,3],['y,-.5,2],
%        [style, [lines,10]],[ylabel,""],
%         [xlabel,""],[psfile,"ch2p22.eps"])$
\noindent or we can use \mv|qplot|:
\begin{myVerbatim}
(%i14) load(qplot);
(%o14)                          c:/work2/qplot.mac
(%i15) qplot (fs(u),['u,-2,2],['x,-3,3],['y,-.5,2],[xlabel,""])$
\end{myVerbatim}
  to get the same plot.
\subsection{Working with Files Using the Package \textbf{mfiles.mac}}
The chapter 2 package file \mv|mfiles.mac| (which loads \mv|mfiles1.lisp|)
  has some Maxima tools for working with both text and data files.
Both files are available on the author's webpage.
Unless you use your initialization file to automatically load \mv|mfiles.mac|
  when you start or restart Maxima, you will need to do an explicit load to
  use these functions.
\subsubsection{Check File Existence with \textbf{file\_search} or \textbf{probe\_file}}
\small
To check for the existence of a file, you can use the
  standard Maxima function \mv|file_search| or the \mv|mfiles| package
  function \mv|probe_file| (the latter is experimental and
  seems to work in a M.S. Windows version of Maxima).
In the following, the file \mv|ztemp.txt| exists in the current
  working directory (\mv|c:/work2/|), and the file \mv|ytemp.txt| does not
   exist in this work folder.
Both functions return \mv|false| when the file is not found.
You need to supply the full file name including any extension, as a string.\\

\noindent The XMaxima output shown here uses \mv|display2d:true| (the default).
If you use the non-default setting\\
 (\mv|display2d:false|), strings will appear
   surrounded by double-quotes, but unpleasant backslash escape characters \mv|\|
   will appear in the output.\\
   
\newpage
\normalsize
\noindent In the following, we have not yet loaded \mv|mfiles.mac|.
\begin{myVerbatim}
(%i1) file_search ("ztemp.txt");
(%o1)                         c:/work2/ztemp.txt
(%i2) probe_file ("ztemp.txt");
(%o2)                        probe_file(ztemp.txt)
(%i3) load(mfiles);
(%o3)                         c:/work2/mfiles.mac
(%i4) probe_file ("ztemp.txt");
(%o4)                                false
(%i5) probe_file ("c:/work2/ztemp.txt");
(%o5)                         c:/work2/ztemp.txt
(%i6) myf : mkp("ztemp.txt");
(%o6)                         c:/work2/ztemp.txt
(%i7) probe_file (myf);
(%o7)                         c:/work2/ztemp.txt
(%i8) file_search ("ztemp");
(%o8)                                false
(%i9) file_search ("ytemp.txt");
(%o9)                                false
\end{myVerbatim}
Although the core Maxima function \mv|file_search| does not need the complete
  path (due to our \mv|maxima.init| file contents),
  our homemade functions, such as \mv|probe_file| do need a complete
    path, due to recent changes in Maxima.
To ease the pain of typing the full path, you can use a function
  which we call \mv|mkp| (``make path'') which we define in our
  \mv|maxima.init.mac| file, whose contents are:
\begin{myVerbatim2}
/* this is c:\Documents and Settings\Edwin Woollett\maxima\maxima-init.mac  */
maxima_userdir: "c:/work2" $     
maxima_tempdir : "c:/work2"$
file_search_maxima : append(["c:/work2/###.{mac,mc}"],file_search_maxima )$
file_search_lisp : append(["c:/work2/###.lisp"],file_search_lisp )$

bpath : "c:/work2/"$
mkp (_fname) := sconcat (bpath,_fname)$
\end{myVerbatim2}
We have used this function, \mv|mkp| above, to get \mv|probe_file| to work.
The string processing function \mv|sconcat| creates a new string from
  two given strings (``string concatenation''):
\begin{myVerbatim}
(%i10) display2d:false$
(%i11) sconcat("a bc","xy z");
(%o11) "a bcxy z"
\end{myVerbatim}
Note that we defined (in our maxima-init.mac file) a global 
  variable \mv|bpath| (``base of path'' or 
	``beginning of path'') instead of using the global variable
	\mv|maxima_userdir|.
This makes it more convenient for the user to redefine \mv|bpath| ``on the fly''
   (inside a maxima session)
   instead of opening and editing \mv|maxima-init.mac|, and restarting Maxima
   to have the changes take effect.\\
   
\noindent We see that \mv|file_search| is easier to use than \mv|probe_file|.
\subsubsection{Check for File Existence using \textbf{ls} or \textbf{dir}}
The \mv|mfiles.mac| package functions \mv|ls| and \mv|dir| accept
  the use of a wildcard pathname.
Both functions are experimental and are not guaranteed to
  work with Maxima engines compiled with a Lisp version 
  different from GCL (Gnu Common Lisp) (which is the Lisp
  version used for the Windows binary used by the author).\\
\newpage 
\noindent Again, you must use the full path, and can make use
  of the  \mv|mkp| function as in the above examples.
The last examples below refer to a folder different than
  the working folder \mv|c:/work2/|.
\begin{myVerbatim}
(%i12) ls(mkp("*temp.txt"));
(%o12) [c:/work2/REPLACETEMP.TXT, c:/work2/temp.txt, c:/work2/ztemp.txt]
(%i13) ls("c:/work2/*temp.txt");
(%o13) [c:/work2/REPLACETEMP.TXT, c:/work2/temp.txt, c:/work2/ztemp.txt]
(%i14) dir(mkp("*temp.txt"));
(%o14)              [REPLACETEMP.TXT, temp.txt, ztemp.txt]
(%i15) dir("c:/work2/*temp.txt");
(%o15)              [REPLACETEMP.TXT, temp.txt, ztemp.txt]
(%i16) ls ("c:/work3/dirac.*");
(%o16)             [c:/work3/dirac.mac, c:/work3/dirac.tex]
(%i17) dir ("c:/work3/dirac.*");
(%o17)                      [dirac.mac, dirac.tex]
\end{myVerbatim}
\subsubsection{Type of File, Number of Lines, Number of Characters}
The text file lisp1w.txt is a file with Windows line ending control
  characters which has three lines of text and no blank lines.
\begin{myVerbatim}
(%i18) file_search("lisp1w.txt");
(%o18)                        c:/work2/lisp1w.txt
(%i19) file_lines("lisp1w.txt");
openr: file does not exist: lisp1w.txt
#0: file_lines(fnm=lisp1w.txt)(mfiles.mac line 637)
 -- an error. To debug this try: debugmode(true);
(%i20) file_lines(mkp("lisp1w.txt"));
(%o20)                              [3, 3]
(%i21) file_lines("c:/work2/lisp1w.txt");
(%o21)                              [3, 3]
\end{myVerbatim}
The output of the \mv|file_lines| function returns the list \\
\mv|[number of non-blank lines, total number of lines]|.
\begin{myVerbatim}
(%i22) myf:mkp("lisp1w.txt");
(%o22)                        c:/work2/lisp1w.txt
(%i23) ftype(myf);
(%o23)                              windows
(%i24) file_length(myf);
(%o24)                                131
(%i25) file_info(myf);
(%o25)                       [3, 3, windows, 131]
\end{myVerbatim}
The function \mv|ftype| (file type) returns either \mv|windows|, \mv|unix|, or \mv|mac|
  depending on the nature of the ``end of line chars''.
The function \mv|file_length| returns the number of characters (``chars'') in the
  file, including the end of line chars.
The function \mv|file_info| combines the line number info,  the file type, and
  the number of characters into one list.
\subsubsection{Print All or Some Lines of a File to the Console}
For a small file the packge function \mv|print_file(file)| is useful.
For a larger file\\ \mv|print_lines(file,start,end)| is useful.
\begin{myVerbatim}
(%i26) print_file (myf)$
Lisp (or LISP) is a family of computer programming
languages with a long history and a distinctive, fully
parenthesized syntax.
(%i27) myf : mkp("lisp2.txt");
(%o27)                        c:/work2/lisp2.txt
\end{myVerbatim}
\newpage
\begin{myVerbatim}
(%i28) file_info(myf);
(%o28)                       [8, 8, windows, 504]
(%i29) print_lines(myf,3,5)$
parenthesized syntax. Originally specified in 1958, Lisp is 
the second-oldest high-level programming language in 
widespread use today; only Fortran is older (by one year).
\end{myVerbatim}
\subsubsection{Rename a File using \textbf{rename\_file}}
The \mv|mfiles.mac| package function \mv|rename_file (oldname,newname)|
  is an experimental function which works as follows in a Windows
  version of Maxima (again we must use the complete path):
\begin{myVerbatim}
(%i30) file_search("foo1.txt");
(%o30)                               false
(%i31) file_search("bar1.txt");
(%o31)                         c:/work2/bar1.txt
(%i32) rename_file(mkp("bar1.txt"),mkp("foo1.txt"));
(%o32)                         c:/work2/foo1.txt
(%i33) file_search("foo1.txt");
(%o33)                         c:/work2/foo1.txt
(%i34) file_search("bar1.txt");
(%o34)                               false
\end{myVerbatim}
\subsubsection{Delete a File with \textbf{delete\_file}}
The \mv|mfiles.mac| package function \mv|delete_file (filename)|
  does what its name implies:
\begin{myVerbatim}
(%i35) file_search("bar2.txt");
(%o35)                         c:/work2/bar2.txt
(%i36) delete_file(mkp("bar2.txt"));
(%o36)                               done
(%i37) file_search("bar2.txt");
(%o37)                               false
\end{myVerbatim}
\subsubsection{Copy a File using \textbf{copy\_file}}
The \mv|mfiles.mac| package function \mv|copy_file(fsource,fdest)|
  will preserve the file type, but will not warn you if you
  are over-writing a previous file name.
\begin{myVerbatim}
(%i38) file_search("foo2.txt");
(%o38)                               false
(%i39) file_info(mkp("foo1.txt"));
(%o39)                        [3, 3, windows, 59]
(%i40) copy_file(mkp("foo1.txt"),mkp("foo2.txt"));
(%o40)                         c:/work2/foo2.txt
(%i41) file_info(mkp("foo2.txt"));
(%o41)                        [3, 3, windows, 59]
\end{myVerbatim}
\subsubsection{Change the File Type using \textbf{file\_convert}}
Here, ``file type'' refers to the text file types \mv|unix|, \mv|windows|,
  and \mv|mac|, each distinguised by the use of different conventions
  in indicating the end of a text file line.\\
  
\noindent Syntax: \mv|file_convert(file, newtype)| or\\
\mv|file_convert (oldfile, newfile,newtype)|\\
The acceptable values of \mv|newtype| are \mv|windows|, \mv|unix|,
  and \mv|mac|.\\
  
\noindent For example, we can change a unix file to a windows file using
  \mv|file_convert(f,windows)| which replaces the previous
  file, or we can change a unix file to a windows file
  with a new name using \mv|file_convert(fold,fnew,windows)|.\\
  
\noindent It is easy to check the end of line characters using the Notepad2
   View menu.
Notepad2 also easily lets you change the end of line characters, so writing
  Maxima code for this task is somewhat unneeded, though instructive as
  a coding challenge.
\begin{myVerbatim}
(%i42) file_search("bar1.txt");
(%o42)                          c:/work2/bar1.txt
(%i43) file_search("bar2.txt");
(%o43)                          c:/work2/bar2.txt
(%i44) print_file(mkp("bar1.txt"))$
This is line one.
This is line two.
This is line three.
(%i45) file_info(mkp("bar1.txt"));
(%o45)                        [3, 3, windows, 59]
(%i46) file_convert(mkp("bar1.txt"),mkp("bar11u.txt"),unix);
(%o46)                        c:/work2/bar11u.txt
(%i47) file_info(mkp("bar11u.txt"));
(%o47)                         [3, 3, unix, 56]
(%i48) print_file(mkp("bar2.txt"));
This is line one.
This is line two.
This is line three.
(%o48)                         c:/work2/bar2.txt
(%i49) file_info(mkp("bar2.txt"));
(%o49)                        [3, 3, windows, 59]
(%i50) file_convert(mkp("bar2.txt"),mac);
(%o50)                         c:/work2/bar2.txt
(%i51) file_info(mkp("bar2.txt"));
(%o51)                          [3, 3, mac, 56]
(%i52) print_file(mkp("bar2.txt"));
This is line one.
This is line two.
This is line three.
(%o52)                         c:/work2/bar2.txt
\end{myVerbatim}
\subsubsection{Breaking File Lines with \textbf{pbreak\_lines} or \textbf{pbreak()}}
Four paragraphs of the Lisp entry from part of Paul Graham's web site were copied
  into a text file \mv|ztemp.txt|.
In Notepad2, each paragraph was one long line.\\

\noindent Inside Maxima, we then used the \mv|mfiles| package 
  function \mv|pbreak_lines (file,nmax)| to break the lines
  (at a space) and print the results to the console screen of Xmaxima.
\begin{myVerbatim}
(%i53) file_info(mkp("ztemp.txt"));
(%o53)                      [7, 13, windows, 1082]
(%i54) print_lines(mkp("ztemp.txt"),1,1);
6. Programs composed of expressions. Lisp programs are trees ...[continues]
(%o54)                        c:/work2/ztemp.txt
(%i55) pbreak_lines(mkp("ztemp.txt"),60)$
6. Programs composed of expressions. Lisp programs are
trees of expressions, each of which returns a value. (In
some Lisps expressions can return multiple values.) This is
in contrast to Fortran and most succeeding languages, which
distinguish between expressions and statements.

It was natural to have this distinction in Fortran because
(not surprisingly in a language where the input format was
punched cards) the language was line-oriented. You could
not nest statements. And so while you needed expressions
\end{myVerbatim}
\newpage
\begin{myVerbatim}
for math to work, there was no point in making anything
else return a value, because there could not be anything
waiting for it.

This limitation went away with the arrival of
block-structured languages, but by then it was too late.
The distinction between expressions and statements was
entrenched. It spread from Fortran into Algol and thence to
both their descendants.

When a language is made entirely of expressions, you can
compose expressions however you want. You can say either
(using Arc syntax)

(if foo (= x 1) (= x 2))

or

(= x (if foo 1 2))
\end{myVerbatim}
Once the line breaking text appears on the console screen,
  one can copy and paste into a text file for further use.\\

\noindent It is simpler to use the function \mv|pbreak()| which
  has \mv|nmax = 72| hardwired in the code, as well as the
  name\\
  \mv|"ztemp.txt"|; this could be used
  in the above example as \mv|pbreak()|, and is easy to use
  since you don't have to type either the text file name or
  the value of \mv|nmax|.\\

\noindent The package function \mv|pbreak()| uses the current definition
  of \mv|bpath| and \mv|mkp|, as well as the file name \mv|"ztemp.txt"|.
\begin{myVerbatim}
(%i56) pbreak();
6. Programs composed of expressions. Lisp programs are trees of
expressions, each of which returns a value. (In some Lisps expressions
can return multiple values.) This is in contrast to Fortran and most
succeeding languages, which distinguish between expressions and
statements.

It was natural to have this distinction in Fortran because (not
surprisingly in a language where the input format was punched cards)
the language was line-oriented. You could not nest statements. And so
while you needed expressions for math to work, there was no point in
making anything else return a value, because there could not be
anything waiting for it.

This limitation went away with the arrival of block-structured
languages, but by then it was too late. The distinction between
expressions and statements was entrenched. It spread from Fortran into
Algol and thence to both their descendants.

When a language is made entirely of expressions, you can compose
expressions however you want. You can say either (using Arc syntax)

(if foo (= x 1) (= x 2))

or

(= x (if foo 1 2))
(%o56)                               done
\end{myVerbatim}
\newpage 
\noindent Alternatively, one can employ the \mv|mfiles| package
  function \mv|break_file_lines (fold,fnew,nmax)| to dump the
  folded lines into created file \mv|fnew|.
\begin{myVerbatim}
(%i57) break_file_lines (mkp("ztemp.txt"),mkp("ztemp1.txt"),72);
(%o57)                        c:/work2/ztemp1.txt
(%i58) print_lines(mkp("ztemp1.txt"),1,2);
6. Programs composed of expressions. Lisp programs are trees of 
expressions, each of which returns a value. (In some Lisps expressions 
(%o58)                        c:/work2/ztemp1.txt
(%i59) file_info(mkp("ztemp.txt"));
(%o59)                      [7, 13, windows, 1082]
(%i60) file_info(mkp("ztemp1.txt"));
(%o60)                      [20, 26, windows, 1095]
\end{myVerbatim}
\subsubsection{Search Text Lines for Strings with \textbf{search\_file}}
the default two arg behavior:\\
\mv|     search_file(filename, substring)|\\
   is to return line number and line text only for
   lines in which \mv|substring| is a distinct word, as
   defined by the package function \mv|sword|, used by
    the package function \mv|wsearch|.\\
       
\noindent Using the form\\
\mv|      search_file (filename,substring,word)|\\
  produces exactly the same as the two arg default mode above.\\
        
\noindent Using the form\\
\mv|       search_file (filename,substring,all)|\\
   will return line numbers and line text for all lines in which 
   the package function \mv|ssearch| returns an integer,
   ie., all lines in which the substring appears, 
   regardless of being a distinct word. \\
 
\noindent The simplest syntax \mv|search_file (file, search-string)| is first demonstrated
  with two searches of the file \mv|ndata1.dat|, which happens to be a purely
  text file.
\begin{myVerbatim}
(%i61) file_info (mkp("ndata1.dat"));
(%o61)                       [5, 9, windows, 336]
(%i62) print_file (mkp("ndata1.dat"))$
The calculation of the effective cross section is much simplified if only
 
those collisions are considered for which the impact parameter is large, so
 
that the field U is weak and the angles of deflection are small. The
 
calculation can be carried out in the laboratory system, and the center
 
of mass frame need not be used.
(%i63) search_file (mkp("ndata1.dat"),"is")$
c:/work2/ndata1.dat 
 1  The calculation of the effective cross section is much simplified if only 
 3  those collisions are considered for which the impact parameter is large, so 
 5  that the field U is weak and the angles of deflection are small. The 
 
(%i64) search_file (mkp("ndata1.dat"),"is much")$
c:/work2/ndata1.dat 
 1  The calculation of the effective cross section is much simplified if only 
 
\end{myVerbatim}
\newpage
\noindent We next demonstrate all three possible syntax forms with a purely
  text file \mv|text1.txt|.
\begin{myVerbatim}
(%i65) file_info (mkp("text1.txt"));
(%o65)                       [5, 5, windows, 152]
(%i66) print_file (mkp("text1.txt"))$
is this line one? Yes, this is line one.
This might be line two.
Here is line three.
I insist that this be line four.
This is line five, isn't it?
(%i67) search_file (mkp("text1.txt"),"is")$
c:/work2/text1.txt 
 1  is this line one? Yes, this is line one. 
 3  Here is line three. 
 5  This is line five, isn't it? 
 
(%i68) search_file (mkp("text1.txt"),"is",word)$
c:/work2/text1.txt 
 1  is this line one? Yes, this is line one. 
 3  Here is line three. 
 5  This is line five, isn't it?  

(%i69) search_file (mkp("text1.txt"),"is",all)$
c:/work2/text1.txt 
 1  is this line one? Yes, this is line one. 
 2  This might be line two. 
 3  Here is line three. 
 4  I insist that this be line four. 
 5  This is line five, isn't it? 
 
\end{myVerbatim}
%\newpage
\subsubsection{Search for a Text String in Multiple Files with \textbf{search\_mfiles}}
The most general syntax is \mv|search_mfiles (file or path,string,options...)|
  in which the options recognised are \mv|word, all, cs, ic|, used in the
  same way as described above for \mv|search_file|.
The simplest syntax is \mv|search_mfiles (file or path,string)| which 
  defaults to case sensitive (\mv|cs|) and isolated  word (\mv|word|) as options.
An example of over-riding the default behavior (\mv|cs| and \mv|word|) would be\\
  \mv|search_mfiles (file or path, string,ic, all)| and the
  options args can be in either order.\\

\noindent First an example of searching one file in the working directory.
\begin{myVerbatim}
(%i1) load(mfiles);
(%o1)                         c:/work2/mfiles.mac
(%i2) print_file(mkp("text1.txt"))$
is this line one? Yes, this is line one.
This might be line two.
Here is line three.
I insist that this be line four.
This is line five, isn't it?
(%i3) search_mfiles(mkp("text1.txt"),"is")$
c:/work2/text1.txt 
 1  is this line one? Yes, this is line one. 
 3  Here is line three. 
 5  This is line five, isn't it? 
 
(%i4) search_mfiles(mkp("text1.txt"),"Is")$
(%i5) 
\end{myVerbatim}
\newpage
\noindent Next we use a wildcard type file name for a search in the
  working directory.
\begin{myVerbatim}
(%i5) search_mfiles(mkp("ndata*.dat"),"is")$
c:/work2/ndata1.dat 
 1  The calculation of the effective cross section is much simplified if only 
 3  those collisions are considered for which the impact parameter is large, so 
 5  that the field U is weak and the angles of deflection are small. The 
 
\end{myVerbatim}
Next we return to a search of the single file \mv|text1.txt|, but look
  for lines containing the string \mv|"is"| whether or not it is an instance
  of an isolated word.
\begin{myVerbatim}
(%i6) search_mfiles(mkp("text1.txt"),"is",all)$
c:/work2/text1.txt 
 1  is this line one? Yes, this is line one. 
 2  This might be line two. 
 3  Here is line three. 
 4  I insist that this be line four. 
 5  This is line five, isn't it? 
 
\end{myVerbatim}
\noindent We now use \mv|search_mfiles| to look for a text string in a file
  \mv|atext1.txt| which is \textbf{not} in the current working directory.
\begin{myVerbatim}
(%i1) load(mfiles);
(%o1)                         c:/work2/mfiles.mac
(%i2) search_mfiles ("c:/work2/temp1/atext1.txt","is");
c:/work2/temp1/atext1.txt 
 2  Is this line two? Yes, this is line two. 
 6  This is line six, Isn't it? 
 
(%o2)                               done
\end{myVerbatim}
If you want to search \textbf{all} files in the folder \mv|c:/work2/temp1|, you 
  use the syntax:
\begin{myVerbatim}
(%i3) search_mfiles ("c:/work2/temp1/","is")$
c:/work2/temp1/atext1.txt 
 2  Is this line two? Yes, this is line two. 
 6  This is line six, Isn't it? 
 
c:/work2/temp1/atext2.txt 
 2  Is this line two? Yes, this is line two. 
 6  This is line six, Isn't it? 
 
c:/work2/temp1/calc1news.txt 
 9  The organization of chapter six is then: 
 96  The Maxima output is the list of the vector curl components in the 
 98   a reminder to the user of what the current coordinate system is 
 102  Thus the syntax is based on lists and is similar to (although better 
 105  There is a separate function to change the current coordinate system. 
 112    plotderiv(..) which is useful for "automating" the plotting 
 
c:/work2/temp1/ndata1.dat 
 1  The calculation of the effective cross section is much simplified if only 
 3  those collisions are considered for which the impact parameter is large, so 
 5  that the field U is weak and the angles of deflection are small. The 
 
c:/work2/temp1/stavros-tricks.txt 
 34  Not a bug, but Maxima doesn't know that the beta function is symmetric: 
\end{myVerbatim}
\newpage
\begin{myVerbatim}
 
c:/work2/temp1/text1.txt 
 1  is this line one? Yes, this is line one. 
 3  Here is line three. 
 5  This is line five, isn't it? 
 
c:/work2/temp1/trigsimplification.txt 
 13  (1) Is there a Maxima command that indicates whether expr is a product of  
 76  > (1) Is there a Maxima command that indicates whether expr is a product of 
 91  Well, that is inherent in their definition.  Trigreduce replaces all 
 94  is sin(x)*cos(x), since the individual terms are not products of trigs. 
 95  There is no built-in function which tries to find the smallest expression, 
 151  is better. If the user wants to expand the contents of sin to discover 
 153  is right that Maxima avoids potentially very expensive operations in 
 
c:/work2/temp1/wu-d.txt 
 1  As a dedicated windows xp user who is delighted to have windows binaries 
 3  all who are considering windows use that there is no problem with keeping previous 
 
\end{myVerbatim}
%\newpage
\subsubsection{Replace Text in File with \textbf{ftext\_replace}}
The simplest syntax \mv|ftext_replace(file,sold,snew)| replaces 
  distinct substrings \mv|sold| (separate words) by \mv|snew|.\\  
  
\noindent The four arg syntax \mv|ftext_replace(file,sold,snew,word)|
   does exactly the same thing.\\
   
\noindent The four arg syntax \mv|ftext_replace(file,sold,snew,all)|
  instead replaces \textbf{all} substrings \mv|sold| by \mv|snew|,
   whether or not they are distinct words.\\
   
\noindent In all cases, the text file type (unix, windows, or mac) is
   preserved.\\
   
The package function \mv|ftext_replace| calls the package function\\
  \mv|replace_file_text (fsource, fdest, sold,snew, optional-mode)|
  which allows the replacement to occur in a newly created file with 
  a name \mv|fdest|.        
\begin{myVerbatim}
(%i4) file_info(mkp("text1w.txt"));
(%o4)                       [5, 5, windows, 152]
(%i5) print_file(mkp("text1w.txt"));
is this line one? Yes, this is line one.
This might be line two.
Here is line three.
I insist that this be line four.
This is line five, isn't it?
(%o5)                        c:/work2/text1w.txt
(%i6) ftext_replace(mkp("text1w.txt"),"is","was");
(%o6)                        c:/work2/text1w.txt
(%i7) print_file(mkp("text1w.txt"));
was this line one? Yes, this was line one.
This might be line two.
Here was line three.
I insist that this be line four.
This was line five, isn't it?
(%o7)                        c:/work2/text1w.txt
(%i8) file_info(mkp("text1w.txt"));
(%o8)                       [5, 5, windows, 156]
(%i9) ftext_replace(mkp("text1w.txt"),"was","is");
(%o9)                        c:/work2/text1w.txt
\end{myVerbatim}
\newpage
\begin{myVerbatim}
(%i10) print_file(mkp("text1w.txt"));
is this line one? Yes, this is line one.
This might be line two.
Here is line three.
I insist that this be line four.
This is line five, isn't it?
(%o10)                        c:/work2/text1w.txt
\end{myVerbatim}
%\newpage
\subsubsection{Email Reply Format Using \textbf{reply\_to}}
The package function \mv|reply_to (name-string)| reads an email
  message (or part of an email message) which has been dumped into
  the current working directory file called \mv|ztemp.txt| (the
  name chosen so the file is easy to find) and writes a version
  of that file to the console screen with a supplied name
  prefixing each line, suitable for a copy/paste into a 
  reply email message.\\  

\noindent It will be obvious if the message lines need breaking.
  If so, then use \mv|pbreak()|, which will place the
  broken line message on the Xmaxima console screen, 
  which can be copied and pasted over the original contents
  of \mv|ztemp.txt|.\\
  
\noindent Once you are satisfied with the appearance of the message
  in \mv|ztemp.txt|, use \mv|reply_to("Ray")| for example, which will
  print out on the console screen the email message with each line
  prefixed by \mv|"Ray"|.
This output can then be copied from the Xmaxima screen and
  pasted into the email message being designed as a reply.
   
\begin{myVerbatim}
(%i11) reply_to("");
>Could you file a bug report on this?  I know about some of these issues
>and am working on them (slowly).  The basic parts work, but the corner
>cases need more work (especically since my approach fails in some cases
>where the original gave correct answers).
>
(%o11)                               done
\end{myVerbatim}
or
\begin{myVerbatim}
(%i12) reply_to(" ray")$
 ray>Could you file a bug report on this?  I know about some of these issues
 ray>and am working on them (slowly).  The basic parts work, but the corner
 ray>cases need more work (especically since my approach fails in some cases
 ray>where the original gave correct answers).
 ray>
 \end{myVerbatim}
\subsubsection{Reading a Data File with \textbf{read\_data}}
An important advantage of \mv|read_data| is the ability to
  work correctly with all three types of text files (unix,
  windows, and mac).\\
  
\noindent Our first example data file has the data items separated by
  commas on the first line and by spaces on the second line.
The data items are a mixture of integers, rational numbers,
  strings, and floating point numbers.
None of the strings contain spaces.\\

\noindent The function \mv|read_data| places the items of each line in
  the data file into a separate list.
\begin{myVerbatim}
(%i13) print_file(mkp("ndata2w.dat"))$
2 , 4.8, -3/4, "xyz", -2.8e-9
3 22.2  7/8 "abc" 4.4e10
\end{myVerbatim}
\newpage
\begin{myVerbatim}
(%i14) read_data(mkp("ndata2w.dat"));
                   3
(%o14) [[2, 4.8, - -, xyz, - 2.7999999999999998E-9], 
                   4
                                                              7
                                                    [3, 22.2, -, abc, 4.4E+10]]
                                                              8
(%i15) display2d:false$
(%i16) read_data(mkp("ndata2w.dat"));
(%o16) [[2,4.8,-3/4,"xyz",-2.7999999999999998E-9],[3,22.2,7/8,"abc",4.4E+10]]
(%i17) fpprintprec:8$
(%i18) read_data(mkp("ndata2w.dat"));
(%o18) [[2,4.8,-3/4,"xyz",-2.8E-9],[3,22.2,7/8,"abc",4.4E+10]]
\end{myVerbatim}
This simplest syntax mode of \mv|read_data| does not care where
  the commas and spaces are, they can be randomly used as data
  item separators, and the data is read into lists correctly:
\begin{myVerbatim}
(%i19) print_file(mkp("ndata2wa.dat"))$
2 , 4.8  -3/4, "xyz" -2.8e-9
3 22.2,  7/8 "abc",  4.4e10
(%i20) read_data(mkp("ndata2wa.dat"));
(%o20) [[2,4.8,-3/4,"xyz",-2.8E-9],[3,22.2,7/8,"abc",4.4E+10]]
\end{myVerbatim}
Next is a case in which the data item separator is consistently
   a semicolon \mv|;|.
In such a case we must include as a second argument to \mv|read_data|
  the string \mv|";"|.   
\begin{myVerbatim}
(%i21) print_file(mkp("ndata3w.dat"))$
2.0; -3/7; (x:1,y:2,x+y); block([fpprec:24],bfloat(%pi)); foo
(%i22) read_data(mkp("ndata3w.dat"),";");
(%o22) [[2.0,-3/7,(x:1,y:2,y+x),block([fpprec:24],bfloat(%pi)),foo]]
\end{myVerbatim}
(If some of the data items include semicolons, then you would not
  want to use semicolons as data item separators; rather you might
  choose a dollar sign \mv|$| as the separator, and so indicate
  as the second argument.)\\
  
\noindent Our next example is a data file which includes some strings
  which include spaces inside the strings. 
The data file should use commas as data item separators, and
  the correct syntax to read the data is
  \mv|read_data(file,",")|.
\begin{myVerbatim}
(%i23) print_file(mkp("ndata6.dat"));
1, 2/3, 3.4, 2.3e9, "file ndata6.dat"
"line two" , -3/4 , 6 , -4.8e-7 , 5.5
7/13, "hi there", 8, 3.3e4, -7.3
4,-3/9,"Jkl", 44.6, 9.9e-6
(%o23) "c:/work2/ndata6.dat"
(%i24) read_data(mkp("ndata6.dat"),",");
(%o24) [[1,2/3,3.4,2.3E+9,"file ndata6.dat"],["line two",-3/4,6,-4.8E-7,5.5],
        [7/13,"hi there",8,33000.0,-7.3],[4,-1/3,"Jkl",44.6,9.9E-6]]
\end{myVerbatim}
The package function \mv|read_data| ignores blank lines in
  the data file (as it should):
\begin{myVerbatim}
(%i25) print_file(mkp("ndata10w.dat"))$
2  4.8  -3/4  "xyz"  -2.8e-9
 
2   4.8  -3/4  "xyz"  -2.8e-9
 
2   4.8  -3/4  "xyz"  -2.8e-9
 
(%i26) read_data(mkp("ndata10w.dat"));
(%o26) [[2,4.8,-3/4,"xyz",-2.8E-9],[2,4.8,-3/4,"xyz",-2.8E-9],
        [2,4.8,-3/4,"xyz",-2.8E-9]]
(%i27) file_info (mkp("ndata10w.dat"));
(%o27) [3,6,windows,98]
\end{myVerbatim}
\newpage
\subsubsection{File Lines to List of Strings using \textbf{read\_text}}
The package function \mv|read_text(path)| preserves blank lines in the source
     file, and returns a list of strings, one for each physical
     line in the source file.
\begin{myVerbatim}
(%i28) print_file(mkp("ndata1.dat"))$
The calculation of the effective cross section is much simplified if only
 
those collisions are considered for which the impact parameter is large, so
 
that the field U is weak and the angles of deflection are small. The
 
calculation can be carried out in the laboratory system, and the center
 
of mass frame need not be used.

(%i29) read_text(mkp("ndata1.dat"));
(%o29) ["The calculation of the effective cross section is much simplified if only",
        "",
        "those collisions are considered for which the impact parameter is large, so",
        "",
        "that the field U is weak and the angles of deflection are small. The",
        "",
        "calculation can be carried out in the laboratory system, and the center",
        "","of mass frame need not be used."]
\end{myVerbatim}
\subsubsection{Writing Data to a Data File One Line at a Time Using \textbf{with\_stdout}}
The core Maxima function \mv|with_stdout| can be used to write loop results to
  a file instead of to the screen.
This can be used to create a separate data file as a byproduct of your
  Maxima work.
The function has the syntax\\
  \mv|with_stdout (file,expr1,expr2,...)| and
  writes any output generated with \mv|print|, \mv|display|, or \mv|grind| (for example)
  to the indicated file, overwriting any pre-existing file, and creating a
  unix type file.
\begin{myVerbatim}
(%i30) with_stdout (mkp("tmp.out"),
           for i thru 10 do
             print (i,",",i^2,",",i^3))$
(%i31) print_file (mkp("tmp.out"))$
1 , 1 , 1 
2 , 4 , 8 
3 , 9 , 27 
4 , 16 , 64 
5 , 25 , 125 
6 , 36 , 216 
7 , 49 , 343 
8 , 64 , 512 
9 , 81 , 729 
10 , 100 , 1000 
(%i32) read_data (mkp("tmp.out"));
(%o32) [[1,1,1],[2,4,8],[3,9,27],[4,16,64],[5,25,125],[6,36,216],[7,49,343],
        [8,64,512],[9,81,729],[10,100,1000]]
(%i33) file_info (mkp("tmp.out"));
(%o33) [10,10,unix,134]
\end{myVerbatim}
Notice that if you don't provide the full path to your
  work directory with the file name (with the Maxima
  function \mv|with_stdout|), the file will be created
  in the \mv|../bin/| folder of Maxima.
\newpage
\subsubsection{Creating a Data File from a Nested List Using \textbf{write\_data}}
The core Maxima function \mv|write_data| can be used to write
  the contents of a nested list to a named file, writing one line for each 
  sublist, overwriting the contents of any pre-existing file of that name,
  creating a unix type file with space separator as the default.
The simplest syntax, \mv|write_data (list, filename)| produces the default
  space separation of the sublist items.
You can get comma separation or semicolon separation by using respectively
  \mv|write_data (list,filename,comma)| or \mv|write_data (list,filename,semicolon)|.\\
  
\noindent Note again that you need to supply the full path as well
  as the file name, or else the file will be created by \mv|write_data|
  in \mv|.../bin/|.\\
  
\noindent (This same function can also be used to write a Maxima matrix object
  to a data file.)
\begin{myVerbatim}
(%i34) dataL : [[0,2],[1,3],[2,4]]$
(%i35) write_data(dataL,mkp("tmp.out"))$
(%i36) file_search("tmp.out");
(%o36) "c:/work2/tmp.out"
(%i37) print_file(mkp("tmp.out"))$
0 2
1 3
2 4
(%i38) file_info(mkp("tmp.out"));
(%o38) [3,3,unix,12]
(%i39) write_data(dataL,mkp("tmp.out"),comma)$
(%i40) print_file(mkp("tmp.out"))$
0,2
1,3
2,4
(%i41) file_info(mkp("tmp.out"));
(%o41) [3,3,unix,12]
(%i42) write_data(dataL,mkp("tmp.out"),semicolon)$
(%i43) print_file(mkp("tmp.out"))$
0;2
1;3
2;4
\end{myVerbatim}
Here is a simple example taken from a question from the Maxima mailing list.
Suppose you compute the expression \mv|(1 - z)/(1 + z)| for a number of values
  of \mv|z|, and you want to place the numbers \mv|z , f(z)| in a data file for
  later use.\\
  
\noindent The easiest way to do this in Maxima is to create a list of
  sublists, with each sublist being a row in your new data file.
You can then use the Maxima function \mv|write_data|, which has the
  syntax: \mv|write_data ( datalist, filename )|.\\

\noindent Let's keep the example really simple and just use five integral values of
  \mv|z|, starting with an empty list we will call \mv|dataL|.
(The final \mv|L| is useful (but not necessary) to remind us that it
  stands for a list.)
\begin{myVerbatim}
(%i44) dataL : []$
(%i45) for x thru 5 do (
          px : subst (x,z, (1-z)/(1+z)),
          dataL : cons ( [x, px], dataL ))$
(%i46) dataL;
(%o46) [[5,-2/3],[4,-3/5],[3,-1/2],[2,-1/3],[1,0]]
(%i47) dataL : reverse (dataL);
(%o47) [[1,0],[2,-1/3],[3,-1/2],[4,-3/5],[5,-2/3]]
\end{myVerbatim}
\newpage
\begin{myVerbatim}
(%i48) write_data (dataL, mkp("mydata1.dat"))$
(%i49) print_file (mkp("mydata1.dat"))$
1 0
2 -1/3
3 -1/2
4 -3/5
5 -2/3
(%i50) read_data (mkp("mydata1.dat"));
(%o50) [[1,0],[2,-1/3],[3,-1/2],[4,-3/5],[5,-2/3]]
(%i51) file_info (mkp("mydata1.dat"));
(%o51) [5,5,unix,32]
\end{myVerbatim}
If you open the unix text file \mv|mydata1.dat| using the older version of the  Windows
  text file application Notepad, you may only see one line, which looks like:
\begin{myVerbatim2}
1 02 -1/33 -1/24 -3/55 -2/3
\end{myVerbatim2}
  which occurs because Maxima creates text files having Unix style 
  line endings which older native Windows applications don't recognise.\\
	
\noindent In order to see the two columns of numbers (using a text editor), you should use the
  freely available Windows text editor Notepad2. (Just Google it.)
Notepad2 recognises unix, mac and windows line ending control characters, and in fact has a
  signal (LF for unix) at the bottom of the screen which tells you what the line ending
  control characters are.\\
  
\noindent The alternative choice provided by the standard Maxima system to
  created a nested list from a data file is the function \mv|read_nested_list|.
This is not as good a choice as our package function \mv|read_data|, as is shown
  here:
\begin{myVerbatim}
(%i52) read_nested_list (mkp("mydata1.dat"));
(%o52) [[1,0],[2,-1,\/,3],[3,-1,\/,2],[4,-3,\/,5],[5,-2,\/,3]]
(%i53) display2d:true$
(%i54) read_nested_list (mkp("mydata1.dat"));
(%o54) [[1, 0], [2, - 1, /, 3], [3, - 1, /, 2], [4, - 3, /, 5], 
                                                                [5, - 2, /, 3]]
\end{myVerbatim}
\subsection{Least Squares Fit to Experimental Data}
\subsubsection{Maxima and Least Squares Fits: \textbf{lsquares\_estimates}}
Suppose we are given a list of \mv|[x,y]| pairs which are thought to
  be roughly described by the relation \mv|y = a*x^b + c|, where the three
  parameters are all of order \mv|1|.
We can use the data of \mv|[x,y]| pairs to find the ``best'' values of
  the unknown parameters \mv|[a, b, c]|, such that the data is described
  by the equation \mv|y = a*x^b + c| (a three parameter fit to the data).\\
  
\noindent We are using one of the Manual examples for \textbf{lsquares\_estimates}.
\begin{myVerbatim}
(%i1) dataL : [[1, 1], [2, 7/4], [3, 11/4], [4, 13/4]]$
(%i2) display2d:false$
(%i3) dataM : apply ('matrix, dataL);
(%o3) matrix([1,1],[2,7/4],[3,11/4],[4,13/4])
(%i4) load (lsquares);
(%o4) "C:/PROGRA~1/MAXIMA~3.2/share/maxima/5.31.2/share/lsquares/lsquares.mac"
(%i5) fpprintprec:6$
(%i6) lsquares_estimates (dataM, [x,y], y=a*x^b+c,
               [a,b,c], initial=[3,3,3], iprint=[-1,0] );
(%o6) [[a = 1.37575,b = 0.7149,c = -0.4021]]
(%i7) myfit : a*x^b + c , % ;
(%o7) 1.37575*x^0.7149-0.4021
\end{myVerbatim}
Note that we must use \mv|load (lsquares);| to use this method.
We can now make a plot of both the discrete data points and the
  least squares fit to those four data points.
\begin{myVerbatim}
(%i8) plot2d ([myfit,[discrete,dataL]],[x,0,5],
        [style,[lines,5],[points,4,2,1]],
         [legend,"myfit", "data"],
         [gnuplot_preamble,"set key bottom;"])$
\end{myVerbatim}

\noindent which produces the plot

\smallskip
\begin{figure} [h]  
   \centerline{\includegraphics[scale=.4]{ch2p19.eps} }
	\caption{Three Parameter Fit to Four Data Points }
\end{figure}

%plot2d([fit,[discrete,dataL]],[x,0,5],
%        [style,[lines,8],[points,6,2,1]],
%         [legend,"fit", "data"],
%         [gnuplot_preamble,"set key bottom;"],
%         [psfile,"ch2p19.eps"] )$
\subsubsection{Syntax of \textbf{lsquares\_estimates}}
The \tcbr{minimal} syntax is
\begin{myVerbatim2s}
  lsquares_estimates (data-matrix, data-variable-list, fit-eqn, param-list );
\end{myVerbatim2s}
 in which the \mv|data-variable-list| assigns a variable name to the
 corresponding column of the \mv|data-matrix|, and the \mv|fit-eqn| is an equation
 which is a relation among the data variable symbols and the equation parameters
 which appear in \mv|param-list|.
 The function returns the "best fit" values of the equation parameters
   in the form \mv|[ [p1 = p1val, p2  = p2val, ...] ]|.\\
   
\noindent In the example above, the data variable list was \mv|[x, y]| and
  the parameter list was \mv|[a, b, c]|.\\
  
\noindent If an exact solution cannot be found, a numerical approximation
  is attempted using \textbf{lbfgs}, in which case, all the elements of the
  data matrix should be "numbers" \mv|numberp(x) ->  true|.
This means that \mv|%pi| and \mv|%e|, for example, should be converted to explicit
  numbers before use of this method.
\begin{myVerbatim}
(%i1) expr : 2*%pi + 3*exp(-4);
                                            - 4
(%o1)                           2 %pi + 3 %e
(%i2) listconstvars:true$
(%i3) listofvars(expr);
(%o3)                              [%e, %pi]
(%i4) map('numberp,%);
(%o4)                           [false, false]
(%i5) fullmap('numberp,expr);
                               true
(%o5)                     false     true + false true
(%i6) float(expr);
(%o6)                          6.338132223845789
(%i7) numberp(%);
(%o7)                                true
\end{myVerbatim}

\noindent Optional arguments to \textbf{lsquares\_estimates} are (in any order)
\begin{myVerbatim2}
 initial = [p10, p20,...], iprint = [n, m], tol = search-tolerance
\end{myVerbatim2}
The list \mv|[p10, p20, ...]| is the optional list of initial values of
  the equation parameters, and without including your own guess for starting values
  this list defaults (in the code) to \mv|[1, 1, ...]|.\\
  
\noindent The first integer \mv|n| in the \mv|iprint| list controls how
  often progress messages are printed to the screen.
The default is \mv|n = 1| which causes a new progress message printout
  each iteration of the search method.
Using \mv|n = -1| surpresses all progress messages.
Using \mv|n = 5| allows one progress message every five iterations.\\

\noindent The second integer \mv|m| in the \mv|iprint| list controls
  the verbosity, with \mv|m = 0| giving minimum information 
  and \mv|m = 3| giving maximum information.\\
  
\noindent The option \mv|iprint = [-1,0]| will hide the details of the
  search process.\\
  
\noindent The default value of the \mv|search-tolerance| is \mv|1e-3|,
  so by using the option \mv|tol = 1e-8| you might find a more accurate
  solution.\\
  
\noindent Many examples of the use of the \textbf{lsquares} package are found in the file
  \mv|lsquares.mac|, which is found in the \mv|...share/contrib| folder.
You can also see great examples of efficient programming in the Maxima language in that file.
\subsubsection{Coffee Cooling Model}
"Newton's law of cooling" (only approximate and not a law) assumes the rate 
 of decrease of temperature (celsius degrees per minute) is proportional to 
 the instantaneous difference between the temperature $\mathbf{T(t)}$ of the
 coffee in the cup and the surrounding 
  ambient temperature $\mathbf{T_{s}}$, the latter being treated as a constant.
If we use the symbol $\mathbf{r}$ for the "rate constant" of proportionality,
  we then assume the cooling of the coffee obeys the first order
  differential equation
\begin{equation}
\mathbf{ \frac{d\,T}{d\,t} = - r \, ( T(t) - T_{s} ) }
\end{equation}
Since $\mathbf{T}$ has dimension degrees Celsius, and $\mathbf{t}$ has dimension minute,
   the dimension of the rate constant $\mathbf{r}$ must be \mv|1/min|.\\

\noindent (This attempt to employ a rough mathematical model which
  can be used for the cooling of a cup of coffee avoids a bottom-up approach
  to the problem, which would require 
  mathematical descriptions of the four distinct physical mechanisms which contribute
  to the decrease of thermal energy in the system hot coffee plus cup to 
  the surroundings: thermal radiation 
  (net electromagnetic radiation flux, approximately black body )  energy transport across the
  surface of the liquid and cup, collisional heat conduction due to the presence of the surrounding 
  air molecules, convective energy transport due to local air temperature rise,  
  and finally evaporation which is the escape of the fastest coffee molecules
  which are able to escape the binding surface forces at the liquid surface.
If the temperature difference between the coffee and the ambient surroundings is not
  too large, experiment shows that the simple relation above is roughly true.)\\

\noindent This differential equation is easy to solve "by hand" , since we can write
\begin{equation}
\mathbf{\frac{d\,T}{d\,t} = \frac{d\,(T - T_{s})}{d\,t} = \frac{d\,y}{d\,t} }
\end{equation}
 and then divide both sides by $\mathbf{y = (T - T_{s} ) }$, multiply both
 sides by $\mathbf{d\,t}$, and use $\mathbf{ d\,y/y = d\,\boldsymbol{\ln}(y)}$
 and finally integrate both sides over corresponding intervals to get
 $\mathbf{ \boldsymbol{\ln(y) - \ln(y_{0}) = \ln}(y/y_{0}) = - r \, t}$, where 
 $\mathbf{y_{0} = T(0) - T_{s}}$ involves the initial temperature at $\mathbf{t = 0}$.
Since
\begin{equation}
\mathbf{ e^{\boldsymbol{\ln}(A)} = A, }
\end{equation}
  by equating the exponential of the left side to that of the right side, we get 
\begin{equation}  \label{Eq:newton}
\mathbf{T(t) = T_{s} + ( T(0) - T_{s} )\,e^{ - r \, t}.}
\end{equation}
Using \textbf{ode2}, \textbf{ic1}, \textbf{expand}, and \textbf{collectterms},
  we can also use Maxima just for fun:
\begin{myVerbatim}
(%i1) de : 'diff(T,t) + r*(T - Ts);
                                dT
(%o1)                           -- + r (T - Ts)
                                dt
(%i2) gsoln : ode2(de,T,t);
                                - r t    r t
(%o2)                     T = %e      (%e    Ts + %c)
(%i3) de, gsoln, diff, ratsimp;
(%o3)                                  0
(%i4) ic1 (gsoln, t = 0, T = T0);
                             - r t          r t
(%o4)                  T = %e      (T0 + (%e    - 1) Ts)
(%i5) expand (%);
                             - r t        - r t
(%o5)                  T = %e      T0 - %e      Ts + Ts
(%i6) Tcup : collectterms ( rhs(%), exp(-r*t) );
                              - r t
(%o6)                       %e      (T0 - Ts) + Ts
(%i7) Tcup, t = 0;
(%o7)                                 T0
\end{myVerbatim}
We arrive at the same solution as found ``by hand''.
We have checked the particular solution for the initial condition and checked
  that our original differential equation is satisfied by the general solution.
\subsubsection{Experiment Data for Coffee Cooling }
Let's take some ``real world'' data for this problem (p. 21, An Introduction to Computer
  Simulation Methods, 2nd ed., Harvey Gould and Jan Tobochnik, Addison-Wesley, 1996) which is
  in a data file \mvs|c:\work2\coffee.dat| on the author's Window's XP computer (data file
  available with this chapter on the author's webpage).\\
  
\noindent This file contains three columns of tab separated numbers, column one
  being the elapsed time in minutes, column two is the Celsius temperature of the
  system glass plus coffee for black coffee, and column three is the Celsius temperature
  for the system glass plus creamed coffee.
The glass-coffee temperature was recorded with an estimated accuracy of $\mathbf{0.1^{\circ}C }$.
The ambient temperature of the surroundings was $\mathbf{17^{\;\circ}C}$.
The function \mv|read_data| automatically replaces tabs (ascii(9)) in the data
  by spaces (ascii(32)) as each line is read in.\\
  
\noindent We need to remind the reader that we are using a function
  \mv|mkp| to create a complete path to a file name.
This function was discussed at the beginning of the section on
  file manipulation methods.
For convenience, we repeat some of that discussion here:\\

\noindent To ease the pain of typing the full path, you can use a function
  which we call \mv|mkp| (``make path'') which we define in our
  \mv|maxima.init.mac| file, whose contents are:
\begin{myVerbatim2}
/* this is c:\Documents and Settings\Edwin Woollett\maxima\maxima-init.mac  */
maxima_userdir: "c:/work2" $     
maxima_tempdir : "c:/work2"$
file_search_maxima : append(["c:/work2/###.{mac,mc}"],file_search_maxima )$
file_search_lisp : append(["c:/work2/###.lisp"],file_search_lisp )$

bpath : "c:/work2/"$
mkp (_fname) := sconcat (bpath,_fname)$
\end{myVerbatim2}
We will use this function, \mv|mkp|, below for example
  with \mv|print_file| and \mv|read_data|.
The string processing function \mv|sconcat| creates a new string from
  two given strings (``string concatenation''):
\begin{myVerbatim}
(%i8) display2d:false$
(%i9) sconcat("a bc","xy z");
(%o9) "a bcxy z"
\end{myVerbatim}
Note that we used a global variable \mv|bpath| (``base of path'' or 
	``beginning of path'') instead of the global variable
	\mv|maxima_userdir|.
This makes it more convenient for the user to redefine \mv|bpath| ``on the fly''
   instead of opening and editing \mv|maxima-init.mac|, and restarting Maxima
   to have the changes take effect.\\
\begin{myVerbatim}
(%i10) file_search("coffee.dat");
(%o10)                         c:/work2/coffee.dat
(%i11) (display2d:false,load(mfiles));
(%o11) "c:/work2/mfiles.mac"
(%i12) print_file(mkp("coffee.dat"))$
0	82.3	68.8
2	78.5	64.8
4	74.3	62.1
6	70.7	59.9
8	67.6	57.7
10	65.0	55.9
12	62.5	53.9
14	60.1	52.3
16	58.1	50.8
18	56.1	49.5
20	54.3	48.1
22	52.8	46.8
24	51.2	45.9
26	49.9	44.8
28	48.6	43.7
30	47.2	42.6
32	46.1	41.7
34	45.0	40.8
36	43.9	39.9
38	43.0	39.3
40	41.9	38.6
42	41.0	37.7
44	40.1	37.0
\end{myVerbatim}
We now use \mv|read_data| which will create a list of sublists, one sublist
  per row.
\begin{myVerbatim}
(%i13) fpprintprec:6$
(%i14) cdata : read_data(mkp("coffee.dat"));
(%o14) [[0,82.3,68.8],[2,78.5,64.8],[4,74.3,62.1],[6,70.7,59.9],[8,67.6,57.7],
       [10,65.0,55.9],[12,62.5,53.9],[14,60.1,52.3],[16,58.1,50.8],
       [18,56.1,49.5],[20,54.3,48.1],[22,52.8,46.8],[24,51.2,45.9],
       [26,49.9,44.8],[28,48.6,43.7],[30,47.2,42.6],[32,46.1,41.7],
       [34,45.0,40.8],[36,43.9,39.9],[38,43.0,39.3],[40,41.9,38.6],
       [42,41.0,37.7],[44,40.1,37.0]]
\end{myVerbatim}
We now use \textbf{makelist} to create a (time, temperature) list based on the
  \tcbr{black} coffee data and then based on the \tcbr{white} (creamed coffee) data.
\begin{myVerbatim}
(%i15) black_data : makelist( [first(cdata[i]),second(cdata[i])],
                       i,1,length(cdata));
(%o15) [[0,82.3],[2,78.5],[4,74.3],[6,70.7],[8,67.6],[10,65.0],[12,62.5],
       [14,60.1],[16,58.1],[18,56.1],[20,54.3],[22,52.8],[24,51.2],[26,49.9],
       [28,48.6],[30,47.2],[32,46.1],[34,45.0],[36,43.9],[38,43.0],[40,41.9],
       [42,41.0],[44,40.1]]
\end{myVerbatim}
\newpage
\begin{myVerbatim}
(%i16) white_data : makelist( [first(cdata[i]),third(cdata[i])],
                       i,1,length(cdata));
(%o16) [[0,68.8],[2,64.8],[4,62.1],[6,59.9],[8,57.7],[10,55.9],[12,53.9],
       [14,52.3],[16,50.8],[18,49.5],[20,48.1],[22,46.8],[24,45.9],[26,44.8],
       [28,43.7],[30,42.6],[32,41.7],[34,40.8],[36,39.9],[38,39.3],[40,38.6],
       [42,37.7],[44,37.0]]
\end{myVerbatim}
\subsubsection{Least Squares Fit of Coffee Cooling Data}
We now use \textbf{lsquares\_estimates} to use a least squares fit with
  each of our data sets to our phenomenological model, that is finding
  the "best" value of the cooling rate constant \textbf{r} that appears
  in Eq. (\ref{Eq:newton}).
The function \textbf{lsquares\_estimates(data\_matrix, eqnvarlist,eqn,paramlist)}
  is available after using \textbf{load(lsquares)}.\\

\noindent To save space in this chapter we use \textbf{display2d:false} to surpress the
  default two dimensional display of a Maxima \textbf{matrix} object.
\begin{myVerbatim}
(%i17) black_matrix : apply ( 'matrix, black_data);
(%o17) matrix([0,82.3],[2,78.5],[4,74.3],[6,70.7],[8,67.6],[10,65.0],[12,62.5],
             [14,60.1],[16,58.1],[18,56.1],[20,54.3],[22,52.8],[24,51.2],
             [26,49.9],[28,48.6],[30,47.2],[32,46.1],[34,45.0],[36,43.9],
             [38,43.0],[40,41.9],[42,41.0],[44,40.1])
(%i18) white_matrix : apply ( 'matrix, white_data);
(%o18) matrix([0,68.8],[2,64.8],[4,62.1],[6,59.9],[8,57.7],[10,55.9],
              [12,53.9],[14,52.3],[16,50.8],[18,49.5],[20,48.1],[22,46.8],
              [24,45.9],[26,44.8],[28,43.7],[30,42.6],[32,41.7],[34,40.8],
              [36,39.9],[38,39.3],[40,38.6],[42,37.7],[44,37.0])
\end{myVerbatim}
We now load \textbf{lsquares.mac} and calculate the "best fit" values of the
  cooling rate constant \textbf{r} for both cases.
For the black coffee case, \mv|T0 = 82.3 deg C| and \mv|Ts = 17 deg C| and we surpress the
  units.
\begin{myVerbatim}
(%i19) load(lsquares);
(%o19) "C:/PROGRA~1/MAXIMA~1.1-G/share/maxima/5.25.1/share/contrib/lsquares.mac"
(%i20) black_eqn : T = 17 + 65.3*exp(-r*t);
(%o20) T = 65.3*%e^-(r*t)+17
(%i21) lsquares_estimates ( black_matrix, [t,T], black_eqn, [r],
                               iprint = [-1,0] );
(%o21) [[r = 0.02612]]
(%i22) black_fit : rhs ( black_eqn ), %;
(%o22) 65.3*%e^-(0.02612*t)+17
\end{myVerbatim}
Thus \textbf{rblack} is roughly \mv|0.026 min^(-1)|.\\

\noindent For the white coffee case, \mv|T0 = 68.8 deg C| and \mv|Ts = 17 deg C|.
\begin{myVerbatim}
(%i23) white_eqn : T = 17 + 51.8*exp(-r*t);
(%o23) T = 51.8*%e^-(r*t)+17
(%i24) lsquares_estimates ( white_matrix, [t,T], white_eqn, [r],
                              iprint = [-1,0] );
(%o24) [[r = 0.02388]]
(%i25) white_fit : rhs ( white_eqn ), %;
(%o25) 51.8*%e^-(0.02388*t)+17
\end{myVerbatim}
Thus \textbf{rwhite} is roughly \mv|0.024 min^(-1)|, a slightly smaller value than for
  the black coffee ( which is reasonable since a black body is a better radiator of thermal
  energy than a white surface).
\newpage  
\noindent A prudent check on mathematical reasonableness can be made by using, say, the
  two data points for $\mathbf{t = 0}$ and $\mathbf{t = 24 \, min}$ to solve for
  a rough value of \textbf{r}.
For this rough check, the author concludes that \textbf{rblack} is roughly
   \mv|0.027 min^(-1)| and \textbf{rwhite} is roughly \mv|0.024 min^(-1)|.\\
  
\noindent We can now plot the temperature data against the best fit model curve,
  first for the black coffee case.
\begin{myVerbatim}
(%i26) plot2d( [ black_fit ,[discrete,black_data] ],
           [t,0,50], [style, [lines,5], [points,2,2,6]],
           [ylabel," "] ,
           [xlabel," Black Coffee T(deg C) vs. t(min) with r = 0.026/min"],
           [legend,"black fit","black data"] )$
\end{myVerbatim}  

which produces the plot


\smallskip
\begin{figure} [h]  
   \centerline{\includegraphics[scale=.35]{ch2p20.eps} }
	\caption{Black Coffee Data and Fit }
\end{figure}


%plot2d( [ black_fit ,[discrete, black_data] ],
%           [t,0,50], [style,[lines,10],[points,4,2,6]],
%           [ylabel," "] ,
%           [xlabel," Black Coffee T(deg C) vs. t(min) with r = 0.026/min"],
%           [legend,"black fit","black data"],
%           [psfile,"ch2p20.eps"] )$


\noindent and next plot the white coffee data and fit:
\begin{myVerbatim}
(%i27) plot2d( [ white_fit ,[discrete, white_data] ],
           [t,0,50], [style, [lines,5], [points,2,2,6]],
           [ylabel," "] ,
           [xlabel," White Coffee T(deg C) vs. t(min) with r = 0.024/min"],
           [legend,"white fit","white data"] )$
\end{myVerbatim}
%\newpage
which yields the plot

\smallskip
\begin{figure} [h]  
   \centerline{\includegraphics[scale=.35]{ch2p21.eps} }
	\caption{White Coffee Data and Fit }
\end{figure}

%plot2d( [ white_fit ,[discrete, white_data] ],
%           [t,0,50], [style,[lines,10],[points,4,2,6]],
%           [ylabel," "] ,
%           [xlabel," White Coffee T(deg C) vs. t(min) with r = 0.024/min"],
%           [legend,"white fit","white data"],
%           [psfile,"ch2p21.eps"] )$
\newpage
\subsubsection*{Cream at Start or Later?}
Let's use the above approximate values for the cooling rate constants to 
  find the fastest method to use to get the temperature of hot coffee down
  to a drinkable temperature.
Let's assume we start with a glass of very hot coffee, $\mathbf{T_{0} = 90^{\circ}C}$,
  and want to compare two methods of getting the temperature down to $\mathbf{75^{\circ}C}$,
  which we assume is low enough to start sipping.
We will assume that adding cream lowers the temperature of the coffee by $\mathbf{5^{\circ}C}$
  for both options we explore.
Option 1 (white option) is to immediately add cream and let the creamed coffee cool down
  from $\mathbf{85^{\circ}C}$ to $\mathbf{75^{\circ}C}$.
We first write down a general expression as a function of
  \mv|T0| and \mv|r|, and then substitute values appropriate to the
  white coffee cooldown.
\begin{myVerbatim}
(%i28) T : 17 + (T0 -17)*exp(-r*t);
(%o28) %e^-(r*t)*(T0-17)+17
(%i29) T1 : T, [T0 = 85, r = 0.02388];
(%o29) 68*%e^-(0.02388*t)+17
(%i30) t1 : find_root(T1 - 75,t,2,10);
(%o30) 6.661
\end{myVerbatim}  
The "white option" requires about $\mathbf{6.7 \, min}$ for the coffee to be sippable.\\

\noindent Option 2 (the black option) lets the black coffee cool from $\mathbf{90^{\circ}C}$ 
   to $\mathbf{80^{\circ}C}$, and then adds cream, immediately getting the temperature
   down from $\mathbf{80^{\circ}C}$  to $\mathbf{75^{\circ}C}$
\begin{myVerbatim}
(%i31) T2 : T, [T0 = 90, r = 0.02612];
(%o31) 73*%e^-(0.02612*t)+17
(%i32) t2 : find_root(T2 - 80,t,2,10);
(%o32) 5.6403
\end{myVerbatim}
The black option (option 2) is the fastest method to cool the coffee, taking about $\mathbf{5.64 \,min}$
  which is about $\mathbf{61 \,sec}$ less than the white option 1.

\end{document}

ver 5.31.2 xmaxima jan 1, 2014
if you look at global variable plot_options,
you see the element
(cyan is greenish-blue)

 [color, blue, red, green, magenta, black, cyan]
 
 0: cyan, 1: dark blue, 2: red, 3: green,
4: majenta, 5: black, 6: cyan, 7: dark blue

color bar: need to shrink line width to 6 for the
legend numbers to be seen.
as in:

use default colors for bars: plot_options; returns the 
element [color,blue,red,green,majenta,black,cyan]
using [lines,6] for width allows legend numbers to
be seen.

plot2d(
[ [discrete,[[0,0],[0,5]]], [discrete,[[2,0],[2,5]]],
[discrete,[[4,0],[4,5]]],[discrete,[[6,0],[6,5]]],
[discrete,[[8,0],[8,5]]],[discrete,[[10,0],[10,5]]],
[discrete,[[12,0],[12,5]]],[discrete,[[14,0],[14,5]]] ],
[style, [lines,6]],
[x,-2,20], [y,-2,8],
[legend,"1","2","3","4","5","6","7","8"],
[xlabel," "], [ylabel," "],
[box,false],[axes,false])$

