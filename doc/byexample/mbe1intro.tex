% file intro.tex
%
% Maxima by Example, Ch. 1, Introduction to Maxima
%   Copyright 2009 by Edwin L. Woollett
%   I release this file under the terms of the GNU General Public License.
%
%   Edwin L Woollett, July, 2009
%   woollett@charter.net
%   http://www.csulb.edu/~woollett
%   last edit  08-11-09 with Maxima 5.19.0
%   history: feb 08, april may 08, August 09
%   things to add
% add comparison of facsum with ratsimp see 
% compare expand(e,0,0) with expand (e,1,0) with expand (e,0,1)
%
% edit with Notepad++,  http://notepad-plus.sourceforge.net/uk/site.htm ,
% then load into LED, http://www.latexeditor.org/ ,  for latexing
% latex by miktex 2.7 , http://miktex.org/ , -->  intro.dvi  
\documentclass[11pt]{article}
\usepackage[dvips,top=1.5cm,left=1.5cm,right=1.5cm,foot=1cm,bottom=1.5cm]{geometry}
\usepackage{times,amsmath,amsbsy,graphicx,fancyvrb,url}
\usepackage[usenames]{color}
%\definecolor{MyDarkBlue}{rgb}{0,0.08,0.45}
\definecolor{mdb}{rgb}{0.1,0,0.55}
\newcommand{\tcdb}{\textcolor{mdb}}
\newcommand{\tcbr}{\textcolor{BrickRed}}
\newcommand{\tcb}{\textcolor{blue}}
\newcommand{\tcr}{\textcolor{red}}
\urldef\tedhome\url{ http://www.csulb.edu/~woollett/  }
\urldef\tedmail\url{ woollett@charter.net}
%1.  this is for maxima code: red framed bold, footnotesize 
\DefineVerbatimEnvironment%
   {myVerbatim}%
   {Verbatim}%
   {fontfamily=courier,fontseries=b,fontsize=\footnotesize ,frame=single,rulecolor=\color{BrickRed}}
\DefineVerbatimEnvironment%
   {myVerbatim1}%
   {Verbatim}%
   {fontfamily=courier,fontseries=b,fontsize=\scriptsize ,frame=single,rulecolor=\color{BrickRed}}
%2.  this is for blue framed bold 
\DefineVerbatimEnvironment%
   {myVerbatim2}%
   {Verbatim}%
   {fontfamily=courier,fontseries=b,frame=single,rulecolor=\color{blue}}
\DefineVerbatimEnvironment%
   {myVerbatim2s}%
   {Verbatim}%
   {fontfamily=courier,fontseries=b,fontsize=\small,frame=single,rulecolor=\color{blue}}
\DefineVerbatimEnvironment%
   {myVerbatim2f}%
   {Verbatim}%
   {fontfamily=courier,fontseries=b,fontsize=\footnotesize,frame=single,rulecolor=\color{blue}}
% 3.  this is for black framed  bold
\DefineVerbatimEnvironment%
   {myVerbatim3}%
   {Verbatim}%
   {fontfamily= courier, fontseries=b, frame=single}
% 4.  this is for no frame bold
\DefineVerbatimEnvironment%
   {myVerbatim4}%
   {Verbatim}%
   {fontfamily=courier, fontseries=b,fontsize=\small}
% 6.  for defaults use usual verbatim
\newcommand{\mv}{\Verb[fontfamily=courier,fontseries=b]}
\newcommand{\mvs}{\Verb[fontfamily=courier,fontseries=b,fontsize=\small]}
\newcommand{\mvf}{\Verb[fontfamily=courier,fontseries=b,fontsize=\footnotesize]}

\renewcommand{\thefootnote}{\ensuremath{\fnsymbol{footnote}}}
%%%%%%%%%%%%%%%%%%%%%%%%%%%%%%%%%%%%%%%%%%%%%%%%%%%%%%%%%%%%%%%%%%%%%%%%
%   title page
%%%%%%%%%%%%%%%%%%%%%%%%%%%%%%%%%%%%%%%%%%%%%%%%%%%%%%%%%%%%%%%%%%%%%%

\title{Maxima by Example:\\ Ch. 1, Introduction to Maxima
             \thanks{This is a live document. This version uses \textbf{Maxima 5.19.0}.
			 Check \; \textbf{ \tedhome } \; for the latest version of these notes. Send comments and
			 suggestions for improvements to \textbf{\tedmail} } }

%\author{Edwin L. Woollett \thanks{Dept. of Physics and Astron, Calif. State University at Long Beach,\\
%                    author's email: \tedmail} }
\author{Edwin L. Woollett}
\date{\today}
%\date{}
%%%%%%%%%%%%%%%%%%%%%%%%%%%%%%%%%%%%%%%%%
%          document
%%%%%%%%%%%%%%%%%%%%%%%%%%%%%%%%%%%%%%%%%%
\begin{document}
%\SMALL
\small
\maketitle
\tableofcontents
\numberwithin{equation}{section}
\newpage
\normalsize
\begin{myVerbatim2} 
COPYING AND DISTRIBUTION POLICY    
This document is part of a series of notes titled
"Maxima by Example" and is made available
via the author's webpage http://www.csulb.edu/~woollett/
to aid new users of the Maxima computer algebra system.	
	
NON-PROFIT PRINTING AND DISTRIBUTION IS PERMITTED.
	
You may make copies of this document and distribute them
to others as long as you charge no more than the costs of printing.	

These notes (with some modifications) will be published in book form
eventually via Lulu.com in an arrangement which will continue
to allow unlimited free download of the pdf files as well as the option
of ordering a low cost paperbound version of these notes.
\end{myVerbatim2}	
\smallskip
\noindent \tcbr{Feedback from readers is the best way for this series of notes
  to become more helpful to new users of Maxima}.
\tcdb{\emph{All} comments and suggestions for improvements will be appreciated and
  carefully considered}.
\smallskip
\begin{myVerbatim2s}
LOADING FILES
The defaults allow you to use the brief version load(fft) to load in the
Maxima file fft.lisp.
To load in your own file, such as qxxx.mac 
using the brief version load(qxxx), you either need to place 
qxxx.mac in one of the folders Maxima searches by default, or
else put a line like:
  
file_search_maxima : append(["c:/work2/###.{mac,mc}"],file_search_maxima )$
  
in your personal startup file maxima-init.mac (see later in this chapter
for more information about this).

Otherwise you need to provide a complete path in double quotes,
as in load("c:/work2/qxxx.mac"),
 
We always use the brief load version in our examples, which are generated 
using the Xmaxima graphics interface on a Windows XP computer, and copied
into a fancy verbatim environment in a latex file which uses the fancyvrb
and color packages.
\end{myVerbatim2s} 
\smallskip
\begin{myVerbatim2s}
  Maxima, a Computer Algebra System.
  Some numerical results depend on the Lisp version used.
  This chapter uses Version 5.19.0 (2009) using Lisp GNU 
  Common Lisp (GCL) GCL 2.6.8 (aka GCL).
  http://maxima.sourceforge.net/
\end{myVerbatim2s} 
\newpage
%\normalsize
\section*{Acknowledgements}
Some of the examples used in these notes are from the Maxima html help 
 manual or the Maxima mailing list:\\
  \mvs|http://maxima.sourceforge.net/maximalist.html|.\\
  
\noindent Our discussion begins with some of the 
  ``nuts and bolts'' of using the software in a Windows XP environment, and continues with 
  information useful to a  new user.\\
  
\noindent The author would like to thank the Maxima developers for their friendly help via 
  the Maxima mailing list.
\setcounter{section}{1}
\pagestyle{headings}
\subsection{What is \textbf{Maxima}?}
Maxima is a powerful computer algebra system (CAS) which combines symbolic,
  numerical, and graphical abilities.
See the Maxima sourceforge webpage  \mvs|http://maxima.sourceforge.net/|.\\

\noindent A cousin of the commercial Macsyma CAS (now available but without support ),
  Maxima is a freely available and open source program which is being
    continuously improved by a team of volunteers.
When compared with Mathematica or Maple, Maxima has a more basic interface,
 but has the advantage in price (!).
Students, teachers, and researchers can ``own'' multiple copies for home, laptop, and desktop without the
 expense of buying licenses for each copy.\\
 
\noindent There are known  ``bugs'' in the present version (a new version
  is available about three times each year), and the volunteer developers and
  programming experts are dealing with these known bugs as time permits.\\

\noindent Maxima is not only ``free'' and will always stay that way, but also comes with
  a copy of the underlying source code (in a dialect of the Lisp language), which
  a user can modify to suit her own research needs and then share with the Maxima
  community of users.\\  

\noindent A self-installing binary for Windows users is available, making it easy for Windows
  users to get a fast start.\\  
  
\noindent Here is a more technical description of Maxima, taken
  from the Unix/Linus man document:
\begin{quote}
Maxima is a version of the MIT-developed MACSYMA system, modified to run under
CLISP.  It is an interactive expert system and programming environment
for symbolic and numerical mathematical manipulation.  Written in
Lisp, it allows differentiation, integration, solution of linear or
polynomial equations, factoring of polynomials, expansion of functions
in Laurent or Taylor series, computation of Poisson series, matrix and
tensor manipulations, and two- and three-dimensional graphics.\\

\noindent Procedures may be written using an ALGOL-like syntax, and both
Lisp-like functions and pattern matching facilities are provided.
Files containing Maxima objects may be read from and written to disk files. Pre-written
Maxima commands may be read from a file and executed, allowing batch-mode
use.\\

\noindent Maxima is a complex system. It includes both known and unknown bugs.
Use at your own risk. The Maxima bug database is available at\\
\mvs|http://sourceforge.net/tracker/?atid=104933&group_id=4933&func=browse|.\\

\noindent New bug reports are always appreciated. Please include the output of
the Maxima function \textbf{build\_info()} with the report.
\end{quote}

\noindent Information about the history of Maxima and its relation to 
  Macsyma can be found in the last section of this chapter.\\
  
\noindent You should first familiarize yourself with the Maxima work environment
  by downloading and installing Maxima on your computer, and starting up either
  the wxMaxima or the XMaxima interface.
\subsection{Which Maxima Interface Should You Use?}
New users generally like to start with \textbf{wxMaxima} since there are
  convenient icons which help locate Maxima functions for common tasks.
\textbf{wxMaxima} allows the user to construct a combination text, calculation,
  and plot document which can be saved and used as a homework submission,
  used as a tutorial for others and/or simply used as a ``permanent'' record
  of work on some topic.  
For new users, the menus and buttons allow a gradual learning of
  Maxima syntax, by reading what the menus and buttons do with an expression,
  and the user can pick up a knowledge of most of the ``power tools'' in Maxima
  in this way.
The reader should be warned, however, that no menu and button system can include
  every Maxima function which might be either useful or needed for a particular
  task.\\
  
\noindent Experienced users tend to split between \textbf{wxMaxima} and \textbf{Xmaxima}, switching
  to \textbf{Xmaxima} because they already know the names of common 
  Maxima functions which help in getting the job done, and appreciate a 
  simple stable interface without distractions.
Experienced users tend to become good ``touch typists'', able to
  type most things without looking at the keyboard and using both hands.
Such experienced users usually find that it is faster to just type the
  name than to reach for the mouse and manipulate the cursor to the right
    button.
The \textbf{Xmaxima} interface is quite stable between new versions of Maxima,
  whereas the \textbf{wxMaxima} interface is being actively developed and changes
  to the appearance and behavior occur frequently during this period of
  rapid development.\\
  
\noindent \textbf{Xmaxima} is also a faster environment for testing and playing with code
  ideas, and the session record can be easily copied and pasted into a Latex
  verbatim environment with zero hassle.
The current version of \textbf{wxMaxima} does not provide this hassle free transfer to
  a latex document (although one can save output as an image, but then one
  must go through the hassle of converting to eps file image format if one is
  using the conventional latex to dvi to pdf route).
\subsection{Using the \textbf{wxMaxima} Interface }
From the \mvs|wxMaxima| webpage \mvf|http://wxmaxima.sourceforge.net/wiki/index.php/Main_Page|
  which could be accessed via 
\begin{myVerbatim2s}
start, My Programs, Maxima-5.19.0, wxMaxima on the Web
\end{myVerbatim2s}  
  one finds the information:
\begin{myVerbatim2s}
wxMaxima features include:
  * 2D formatted math display: wxMaxima implements its 
	  own math display engine to nicely display maxima output.
  * Menu system: most Maxima commands are available through menus.
      Most used functions are also available through a button 
	  panel below the document.
  * Dialogs: commands which require more that one argument can 
	  be entered through dialogs so that there is no need to 
	  remember the exact syntax.
  * Create documents: text can be mixed with math
      calculations to create documents. Documents can be 
	  saved and edited again later.
  * Animations: version 0.7.4 adds support for simple animations. 
\end{myVerbatim2s}
On that page is a link to a set of \mvs|wxMaxima| tutorials on the
  page\\
  \mvf|http://wxmaxima.sourceforge.net/wiki/index.php/Tutorials|.\\
  
\noindent There you can download tutorials in the form of zip files, which, when
  unzipped, become wxMaxima document format (wxm) files which can then be loaded
  into your wxMaxima work session.\\
  
\noindent \mv|10minute.zip| contains a ``ten minute'' wxMaxima tutorial.
\mv|usingwxmaxima.zip| provides general information about the cell
  structure of wxMaxima.  
\subsubsection*{Starting wxMaxima}
One can use the Windows Start menu:
\begin{myVerbatim2}
start, All Programs, Maxima-5.19.0, wxMaxima
\end{myVerbatim2}
During the Windows binary setup process you can select the options which
  place icons for both \mvs|wxMaxima| and \mvs|XMaxima| on your desktop
  for a convenient start, and you can later copy any shortcut and paste it into
  your work folder for an alternative start method.
\subsubsection*{Quitting wxMaxima}
The quick way to quit is the two-key command \textbf{Ctrl + q}.
\subsubsection*{The Maxima Manual}  
\noindent To access the Maxima manual from within \mvs|wxMaxima|, you can use function key \mv|F1| or
  the menu item\\
  \mvs|Help, Maxima Help|.
\subsubsection*{The Online wxMaxima Forum}
\noindent You can access the \mvs|wxMaxima| online forum at the web page
\begin{myVerbatim2s}
http://sourceforge.net/forum/forum.php?forum_id=435775
\end{myVerbatim2s}
which can be accessed using
\begin{myVerbatim2s}
Start, My Programs, Maxima-5.19.0, wxMaxima Online Forum
\end{myVerbatim2s}
and search for topics of interest.
\subsubsection*{The Cell Structure of wxMaxima}
\noindent In the second tutorial, Using wxMaxima, the cell structure of a wxMaxima
  document is explained (we will use parts of that discussion here and we give only a 
  rough idea of the actual appearance here):\\

\noindent \tcdb{The top of the cell bracket is actually a triangle. The following
  is a ``text cell'' which has no Maxima code.}
\begin{myVerbatim4}
-----
|
| Unlike "command-line Maxima"' (such as XMaxima or Console mode)
| which works in a simple input-output manner, wxMaxima introduces
| the concept of a live mathematical document, in which you mix 
| text, calculations and plots. 
|
| Each wxMaxima document consists of a number of so called "cells".
| The cell is the basic building block of a wxMaxima document. Each cell has a
| bracket on the left border of the document, indicating where the cell
| begins and ends. Cells are of different types. You can have a "title
| cell", a "section cell", and a "text cell" like this one. The most 
| important cell type is the "input cell".
---
\end{myVerbatim4}
\newpage
\noindent The next cell is an example of an \tcr{input cell}, which has the
  input prompt \mv|>>| at the top. Note that we can include a  text
  comment within an input cell (which will be ignored by the Maxima
  computational engine) by putting
  the comment between the delimiters \mv|/* | and  \mv| */| :
\begin{myVerbatim4}
----
| >> /* this is an input cell - it holds Maxima code and can be
| evaluated by first left-clicking once anywhere in the cell and
| then using the two-key command SHIFT-ENTER. The code entered in this cell 
| will be sent to Maxima when you press SHIFT-ENTER. Before 
| wxMaxima sends code to Maxima, it checks if the contents 
| of each code command in this input cell ends with a ';' or a '$'.
| If it doesn't, wxMaxima adds a ';' at the end. Maxima requires that 
| each completed Maxima statement end with either ';' or '$'.
| Note that this does not mean you have to have each statement on
| one line. An input cell could have the contents
|     sin
|     (
|      x
|       );
|
| and this would be accepted as a complete Maxima input, equivalent to
|    sin(x);
|
| Any *output* wxMaxima gets from Maxima will be attached to the end of
| the input cell. Try clicking in this cell and pressing SHIFT-ENTER. */
|
|   /*  example Maxmima code: */
|
|   print("Hello, world!")$
|   integrate(x^2, x);
---
\end{myVerbatim4}
If you click once somewhere inside this cell the vertical left line
  changes to (if you have not changed the default) \tcr{bright red}, 
  indicating that the cell has been selected for some action.
If you then press the two-key combination \textbf{Shift + Enter},
  the prompt \mv|>>| will be replaced by \mv|(%i1)|, and the results of 
  the two separate Maxima commands will be shown at the bottom of the
  cell as:
\begin{myVerbatim4}
|  Hello, world!
|             3
|            x
|    (%o2)  ---
|            3		 
----
 \end{myVerbatim4}
except that the results are shown using pixel graphics with much better looking
  results than we show here.\\

\noindent There is no \mv|(%o1)| since the \textbf{print} command
  was followed by the \mv|$| sign, which surpresses the normal default
  output.\\
  
\noindent If you have either a blank screen or a horizontal line present
  and just start typing, an input type cell is automatically created.
(You can instead, prior to typing, use the function key \mv|F7| to create a new input cell
  and can use the function key \mv|F6| to create a new text cell).\\
  
\noindent In the current version of wxMaxima, if you change
  windows to some other task unrelated to wxMaxima, a junk input
  cell may be created in wxMaxima containing something which looks like
  a percent sign \mv|>> %| which you will see when you come
  back to your wxMaxima window.
If you want to have an input cell at that location, just backspace to get rid
  of the weird symbol and continue.\\
  
\noindent If you then want to delete this junk cell, 
 (or any cell you might be working on) just left-click on the bottom of the cell
  bracket (which will highlight the bracket) and press the \textbf{Delete} key
  (or else select \textbf{Edit, Cell, Cut Cell}).\\
  
\noindent An alternative method is to use the horizontal line as a vertical position cursor
  which acts on cells.
If your cursor is still inside a cell you want to delete, use the DOWN key to get the
  cursor out of the cell, and the horizontal black line appears.
At that point you can press the backspace (or delete) key twice to delete the cell
  above the black horizontal line.
 Or you can press SHIFT-UP once to select that cell and then press DELETE.
 Using SHIFT-UP more than once will select a group of cells for action.\\
  
\noindent In the author's copy of \textbf{wxMaxima} (ie., using Windows XP),
  trying to use the usual Windows menu key combination\\
  \textbf{Alt + E}, for example, does not actuate the \textbf{wxMaxima} Edit menu icon; 
   one must left-click on that Edit icon at the top of the screen to 
   choose among the options.\\  
  
\noindent The next cell is a ``text cell'' which does not need
  Maxima comment delimiters \mv|/*   | and \mv|   */|.
\begin{myVerbatim4}
----
|
| Again, there is a triangle at the top of both text and
|  input type cells which we don't show. An open triangle is the
|  default, but if you click on the triangle, it will turn solid 
|  black and a) for a text cell, the text content is hidden, and
|  b) for an input type cell, the Maxima ouput of the cell is hidden.
|  Clicking that solid black triangle again will restore the hidden
|  portions to view.
|
|  Editing cells is easy. To modify the contents of a cell, 
|  click into it (ie., left-click anywhere in the cell once).
|  A cursor should appear and the left cell bracket should turn red,
|  (in default style mode) indicating that the cell is ready to be edited.
|
|  The usual Windows methods can be used to select parts of a cell.
|  One method is to hold the left mouse button down while you drag
|  over your selection. A second method combines the SHIFT key with
|  other keys to make the selection. For example, left-click at 
|  a starting location. Then use the two-key command SHIFT-END to 
|  select to the end of the line (say) or SHIFT-RIGHTARROW to 
|  select part of the line. You can then use CTRL-C  to copy your
|  selection to the clipboard. You can then left-click somewhere 
|  else and use CTRL-V to paste the clipboard contents at that location.
|  (This is the usual behvior: if you experience lack of correct pasting,
|  you can temporarily revert to the longer process of using the 
|  Edit, Copy, and Edit, Paste menu items to get wxMaxima in the
|   right spirit).
|  The DOWN arrow key will step down one line at a time through the
|  cell lines and then make a horizontal line under the cell. You can
|  also simply left-click in the space between cells to create the
|  active horizontal line, which you can think of as a sort of
|  cursor for vertical position. With the horizontal line present,
|  simply start typing to automatically create a new input type cell.
|  Or press the function key F6 to create a new text type cell
|  at that vertical location.
--
\end{myVerbatim4}  
When you're satisfied with the document you've created, save it using
  the two-key \textbf{Ctrl + s} command or the \textbf{File, Save} menu command.
Note that the \tcr{output} parts of input cells will not be saved.
When you load your document later, you can evaluate all cells by
 using the \textbf{Edit, Cell, Evaluate all cells} menu command or 
 the  shortcut two-key command \textbf{Ctrl + r}.\\
If the shortcut two-key command doesn't work the first time,
   just repeat once and it should work.
The present version of \textbf{wxMaxima} is a little cranky still.
This will evaluate all the cells in the order they were originally
   created.\\
 
\noindent Some common Maxima commands are available through the seven
  menu icons: \textbf{Maxima, Equations, Algebra, Calculus, Simplify, Plot and 
  Numeric}.
All of the menu choices which end with \mv|.....| will open a dialog, 
  to help you formulate your desired command.
The resulting command will be inserted at the current horizontal cursor's position or
below the currently active cell.
\tcr{The chosen command will also be evaluated}.
\subsubsection*{Configuring the Buttons and Fonts}
The bottom button panel can be configured through 
 \textbf{Edit, Configure, Options Tab, Button Panel, Full or Basic}, and 
  the font type and size can be configured through 
  \textbf{Edit, Configure, Style Tab, Default Font, Choose Font, Times New Roman, Regular, 12 },
 for 12 pt generic roman for example.
This is smaller than the startup default font, and may be too small for some
  users, but will allow more information to be seen per screen.\\
  
\noindent The default text cell background color is a six variable custom green
  that is nice.
In the following, the italic box is left unchecked unless mentioned otherwise.
To change a color, click the color bar once.
The author combines the 12 pt roman choice with input labels in bold red, 
  Maxima input in blue bold, text cell in bold black, strings in black bold
  italic, Maxima questions in blue bold, and all of the following in
  bold black: output labels, function names, variables, numbers, special constants,
  greek constants.\\
  
\noindent When you want to save your choices, select the Save button, which will write
  a file \textbf{style.ini} to the folder in which wxMaxima was started.\\
  
\noindent (The author uses a shortcut to wxMaxima placed in his \mvs|c:\work2| windows xp
  folder, since that folder is where the author expects saved wxm and xml files
  to be saved. By starting with the contents of that folder in view, the author
  can then simply click on the wxMaxima shortcut link to begin work with wxMaxima
  in that folder.)  \\

\noindent The author chose the \textbf{Full} bottom button option (the default is \textbf{Basic}),
  which draws twenty buttons in two rows at the bottom of the screen.
The top row contains the \textbf{Simplify, Simplify(r), Factor, Expand, Simplify(tr),
  Expand(tr), Reduce(tr), Rectform, Sum..., Product...} buttons.\\
\noindent The bottom row contains the \textbf{Solve..., Solve ODE...,
   Diff..., Integrate..., Limit..., Series..., Subst..., Map..., Plot2D..., Plot2D...}
   buttons.

\noindent Note that if you have made a selection before you left-click a button panel
command or a menu command, that selection will be used as the main (or only) argument to the command.
Selection of a previous output or part of an input will work in a similar manner. 
The selected button function will act on the input cell contents or selection and
  immediately proceed with evaluation unless there is an argument dialog which
  must be completed.
\subsubsection*{Multiple Maxima Requests in One Input Cell}
Here is a calculation of the cross sectional area \mv|a| and volume \mv|v| of a right
  circular cylinder in terms of the cross section radius \mv|r| and the
  height \mv|h| (ignoring units).
We first create an input cell as described above, by just
  starting to type the following lines, ending each line with the
  ENTER key to allow entry of the following line.
\begin{myVerbatim4}
|---
| >>  (r : 10, h : 100)$
|      a : %pi * r^2;
|      v : a * h;
---
\end{myVerbatim4}
Note that we put two Maxima assignment statements on one line,
  using the syntax \textbf{ ( job\_1, job\_2 )\$}.
Naturally, you can put more than two statements between the
  beginning and ending parentheses.
Thus far, the Maxima computational engine has not been invoked.
The left cell bracket will be \tcr{bright red} (in the default style), and you will have a
  blinking cursor in the cell somewhere.\\
  
\noindent If you now use the two-key command SHIFT-ENTER, all
 Maxima commands will be carried out and the ouputs will
  show up (in the same order as the inputs) at the bottom
 of that input cell, looking like
\begin{myVerbatim4}
----
| (%i1)  (r : 10, h : 100)$
|         a : %pi * r^2;
|         v : a * h;
| (%o2)   100*%pi
| (%o3)   10000*%pi
---
\end{myVerbatim4}
with the horizontal (vertical location) line present underneath.
The input prompt \mv|>>| was replaced with the input number \mvs|(%i1)|.
Output \mv|(%o1)| was not printed to the screen because the first input
  ended with \mv|$|.\\
  
\noindent Now just start typing  to start the next input cell:
\begin{myVerbatim4}
----
| >>   [ a, v ], numer ;
---
\end{myVerbatim4}
With the cell bracket highlighted in red and the blinking
  cursor inside the cell, press SHIFT-ENTER to carry out the operation,
  which returns the numerical value of the cross-sectional area \mv|a| and
  the cylinder volume \mv|v| as a list.
\begin{myVerbatim4}
----
| (%i4)    [ a, v ], numer ;
| (%o4)   [314.1592653589793,31415.92653589793]
\end{myVerbatim4}
\subsubsection{Rational Simplification with \textbf{ratsimp} and \textbf{fullratsimp}} \label{rat}
A rational expression is a ratio of polynomials or a sum of such. A particular case
  of a rational expression is a polynomial (whether expanded out or factored).
\textbf{ratsimp} and \textbf{fullratsimp} can be useful tools for expressions,
  part of whose structure is of this form.
Here is a simple example.
If you are using wxMaxima, just start typing the expression you see entered
  below:
\begin{myVerbatim4}
----
| >>  (x+2)*(x-2)
---
\end{myVerbatim4}
When you type a leading parenthesis \mv|'('|, the trailing parenthesis \mv|')'| also
  appears automatically.
However, if you type an ending parenthesis anyway, wxMaxima will not use it, but
  rather will properly end with one parenthesis.
However, you should pay attention to this feature at first so you understand
  how it works.
Un-needed parentheses will result in a Maxima error message.\\

\noindent Now press HOME and then SHIFT-END to select the whole expression,
  \textbf{(x+2)*(x-2)}, then click on the
  \textbf{Simplify} button at the bottom of the screen.
Maxima will apply a default type of simplification using the 
  function \textbf{ratsimp}, and the cell contents evaluation will show
\begin{myVerbatim4}
|(%i5) ratsimp((x+2)*(x-2));
|        2
|(%o5)  x  - 4
\end{myVerbatim4}
Thus the \textbf{Simplify} button uses \textbf{ratsimp}
  to simplify an expression.
Instead of using the \textbf{Simplify} button, you could use the menu
 selection: \textbf{Simplify, Simplify Expression}.
 The word ``simplify'' is being used here in a loose fashion, since
   Maxima has no way of knowing what a particular user will consider
   a useful simplification.\\
   
\noindent A nice feature of wxMaxima is that you can left-click on a Maxima
  function name (in an input cell) like \textbf{ratsimp} and then press the
  function key \textbf{F1}, and the Maxima manual will open up to the entry
  for that function.
\newpage
\noindent Note that if you had started with the input cell contents
\begin{myVerbatim4}
----
| >> log( (x+2)*(x-2) ) + log(x) 
---
\end{myVerbatim4}
and then highlighted just the \mvs|(x+2)*(x-2)| portion, followed by clicking
  on the \textbf{Simplify} button, you would get the same input and
  output we see above.
Thus Maxima will ignore the rest of the expression.
On the other hand, if you highlighted the \mvs|whole expression|
  and then clicked on \textbf{Simplify}, you would get
\begin{myVerbatim4}
----
| (%i5)    ratsimp(log((x+2)*(x-2))+log(x));
|                                   2
| (%o5)                        log(x  - 4) + log(x)
---
\end{myVerbatim4}
\subsubsection*{\textbf{fullratsimp}}
According to the Maxima manual
\begin{quote}
\textbf{fullratsimp} repeatedly applies \textbf{ratsimp} followed
  by non-rational simplification to an expression until no further 
  change occurs, and returns the result. 
When non-rational expressions are involved, one call to \textbf{ratsimp}
  followed as is usual by non-rational ("general") simplification may not 
  be sufficient to return a simplified result.
Sometimes, more than one such call may be necessary.
\textbf{fullratsimp} makes this process convenient. 
\end{quote}
Here is the Maxima manual example for \textbf{fullratsimp}, making use of
  the \textbf{Xmaxima} interface.
\begin{myVerbatim}
(%i1) expr: (x^(a/2) + 1)^2*(x^(a/2) - 1)^2/(x^a - 1);

                              a/2     2   a/2     2
                            (x    - 1)  (x    + 1)
(%o1)                       -----------------------
                                     a
                                    x  - 1
(%i2) expr, ratsimp;
                                 2 a      a
                                x    - 2 x  + 1
(%o2)                           ---------------
                                     a
                                    x  - 1
(%i3) expr, fullratsimp;
                                     a
(%o3)                               x  - 1
(%i4) rat (expr);

                              a/2 4       a/2 2
                            (x   )  - 2 (x   )  + 1
(%o4)/R/                    -----------------------
                                     a
                                    x  - 1
\end{myVerbatim}
We see that \textbf{fullratsimp} returned a completely simplified result.
\textbf{rat(expr)} converts \textbf{expr} to a \textbf{canonical rational expression} (CRE)
   form by expanding and combining all terms over a common denominator
   and cancelling out the greatest common divisor of the numerator and
   denominator, as well as converting floating point numbers to rational 
   numbers within a tolerance of \textbf{ratepsilon} (see the Manual for more
   information about \textbf{rat}).
\newpage
\subsection{Using the \textbf{Xmaxima} Interface}
To start up \textbf{Xmaxima}, you can use the Windows Start Menu route
\begin{myVerbatim2s}
 Start, All Programs, Maxima 5.19.0, xmaxima
\end{myVerbatim2s}
  or click on the desktop icon for Xmaxima.
You can copy the desktop icon to the Clipboard and paste it into
  any work folder for quick access.\\
  
\noindent If you are new to Maxima, go through the quick start
  tutorial in the bottom window of Xmaxima.
You can either click on the expression to have it evaluated, or you can
  enter the expression, followed by a semi-colon (\mv|;|) in the upper window 
  and press enter to have Maxima carry out the operation.
In the top Xmaxima window, note that the two key command \textbf{Alt+p}
 will type in the previous entry, and you can keep entering \textbf{Alt+p} 
 until you get the entry you want (to edit or simply rerun).\\

\noindent Once you know what you are doing, you can close the 
  bottom window of Xmaxima by using the Menu bar.
On that menu bar, choose \textbf{Options, Toggle Browser Visibility}.
To use only keyboard commands to close the bottom window, use 
  the succession: \textbf{Alt + e}, \textbf{RightArrow}, \textbf{Enter}.\\
  
\noindent To quit Xmaxima, choose \textbf{File, Exit} on the menu bar 
   (or \textbf{Alt+f}, \textbf{x}).\\

\noindent The second short introduction can be found on the
  \textbf{Start,All Programs, Maxima 5.19.0, Introduction} link.
Written by Cornell University (Dept. of Theoretical and Applied Mechanics) Professor Richard
  Rand\\
  (\mvs|http://tam.cornell.edu/|), this is an excellent short introduction to
  many basic features of Maxima.
The advice in Rand's introduction to exit Maxima by typing \mvs|quit();| is relevant
  if you are using the ``command line maxima'' version, aka ``maxima console''.\\
  
\noindent One important thing to remember when using Xmaxima is to never press \textbf{Enter}
  (to execute code) with a space between the semicolon (or the dollar sign) and the postion
  of the cursor.
At least on the author's Windows XP machine, Xmaxima will ``hang'' and refuse to
  issue the next input prompt, and you will have to click on \textbf{File, Restart} on
  the Xmaxima menu bar to start a new session.
This type of error can creep in easily if you are copying code you have previously
  saved to a text file, and have an extra space in the file.
If you then select, copy, and paste that code fragment into Xmaxima, with the
  space at the end intact, you should carefully backspace to either the semicolon
  or the dollar sign before pressing \textbf{Enter}.
The safest path is to make sure your original text selection for copy does
  not include a space beyond the dollar sign.
\subsubsection*{The Maxima Help Manual}
The most important continuous source of information about Maxima syntax and
  reserved words is the Maxima Manual, which you should leave open in a separate window.
To open a separate Maxima Manual window from inside the XMaxima interface,
  click on the XMaxima menu item: \textbf{Help, Maxima Manual} 
   ( you can use the shortcut \textbf{Alt+h} to open the Help menu).\\

\noindent Move around this reference manual via either \textbf{Contents} or
   \textbf{Index}.
For example, left-click \textbf{Index} and start typing \mvs|integrate|.
By the time you have typed in \mvs|inte|, you are close to the \mvs|integrate| entry,
  and you can either continue to type the whole word, or use the down arrow
  key to select that entry.
Then press the \textbf{Enter} key.
On the right side will be the Maxima Manual entry for \mvs|integrate|.\\

\noindent To scroll the Maxima Manual page, double-click on the page, and then 
  use the \textbf{PageDown} and \textbf{PageUp} keys and the \textbf{UpArrow} 
  and \textbf{DownArrow} keys.
To return to the index entry panel, left click that panel, and type \mvs|diff|,
  which will take you to the section of the Maxima Manual which explains how to
  evaluate a derivative. 
\subsubsection*{The Xmaxima Manual}
If you look at the Xmaxima manual via \textbf{Help, Xmaxima Manual (Web Browser) }, your
  default browser will come up with a somewhat out of date manual with the sections:
  1. Command-line options, 2. Xmaxima Window, 3. Entering commands, 4. Session control,
  5. Openmath plots, 6. The browser, 7. Getting Help, and Concept Index.\\
  
\noindent The first section ``1. Command-line options'' is not relevant
  for Windows XP Xmaxima.
\subsubsection*{Xmaxima Font Choices}
In Sec 2, Xmaxima Window, you will find the statement:
\begin{quote}
You can also choose different types and sizes for the fonts, in the 
  section `Preferences' of the Options menu; those settings will be 
  saved for future sessions. 
\end{quote}
The defaults are Times New Roman with size adjustment 3 for proportional fonts
  and Courier New with size adjustment 2 for fixed fonts.
Using the menu with \textbf{Options, Preferences}, you can click on the 
  typeface buttons to select a different font type, and can
  click on the size number button to select another font size. 
You then should click the \textbf{Apply and Quit} button.  
%\newpage
\subsubsection*{Entering Your Expression}
In Xmaxima, every input prompt, like \mv|(%i1)|, is waiting for 
  an input which conforms to Maxima's syntax expectations.
There is no such thing as a ``text cell'', although you can include
  text comments anywhere as long as they are delimited by the standard
  comment delimiters \mv|/*  | and \mv|  */|, which only need to occur
  at the very beginning and end of the comment, even if the comment extends 
  over many lines.\\

\noindent Here is the beginning of Sec.3, Entering Commands, from
  the Xmaxima manual.
We have replaced some irrelevant or obsolete material with updated
  instructions.
\begin{quote}
Most commonly, you will enter Maxima commands in the last input
  line that appears on the text Window.
That text will be rendered in  weak green.
If you press \textbf{Enter}, without having written a command-termination character
  (either \mv|;| or \mv|$|) at the end, the text will remain green and
  you can continue to write a multi-line command.
When you type a command-end character and press the \textbf{Enter} key,
  the text will become light blue and a response from Maxima 
  should appear in light black.
You can also use the \textbf{UpArrow} or \textbf{DownArrow} keys to move to a new
line without sending the input for Maxima evaluation yet.
If you want to \textbf{clear} part  of the current input from the beginning to some
  point, position your cursor at that point (even if the region thereby selected spans
  several lines) and then use \textbf{Edit, Clear input} or the two-key command \textbf{Ctrl+u}.\\

\noindent If you move the cursor over the \mv|(%i1)| input label, or any other label 
  or output text (in black), you will not be able to type any text there;
  that feature will prevent you from trying to enter a command 
  in the wrong place, by mistake.
If you really want to insert some additional text 
  to modify Maxima's output, and which will not be interpreted by 
  Maxima, you can do that using cut and paste (we will cover that later).\\

\noindent You can also write a new input command for Maxima on
   top of a previous input line (in blue), for instance, if 
   you do not want to write down again everything but
   just want to make a slight change.
Once you press the \textbf{Enter} key, the text you modified will 
  appear at the last input line, as if you had written 
   it down there; the input line you modified 
   will continue the same in Xmaxima's and Maxima's memory,
   in spite of having changed in the screen.
\newpage
\noindent For example, suppose you entered \mvs|a: 45;| in input line \mvs|(%i1)|,
  and something else in \mvs|(%i2)|.
\begin{myVerbatim}
(%i1) a:45;
(%o1)                                 45
(%i2) b:30;
(%o2)                                 30
\end{myVerbatim}
You then move up over the \mvs|(%i1) a: 45;| and change the \mv|5| to \mv|8|.
You can then press \textbf{End} to get the cursor at the end of the command, and 
  then press \textbf{Enter} to submit the new (edited) command.
At that point the screen looks like:
\begin{myVerbatim}
(%i1) a:48;
(%o1)                                 45
(%i2) b:30;
(%o2)                                 30
(%i3) a:48;
(%o3)                                 48
\end{myVerbatim}  
But if you now enter \mvs|(%i1);| as input \mv|(%i4)| and press \textbf{Enter},
 the output \mv|(%o4)| will be \mv|a: 45|.
The screen will now look like:
\begin{myVerbatim}
(%i1) a:48;
(%o1)                                 45
(%i2) b:30;
(%o2)                                 30
(%i3) a:48;
(%o3)                                 48
(%i4) (%i1);
(%o4)                               a : 45
\end{myVerbatim}
Maxima knows the current binding of both \mv|(%i1)| (which is
  the output \mv|(%o4)|) and \mv|a|
\begin{myVerbatim}
(%i5) a;
(%o5)                                 48
\end{myVerbatim}
If you navigate through the input lines history (see next section),
 you will also see that the first input keeps its original value.
\end{quote}
\subsubsection*{Speeding up Your Work with XMaxima}
When you want to edit your previous command, use \textbf{Alt+p} to enter 
  the previous input (or use enough repetitions of \textbf{Alt+p} to retrieve
  the command you want to start with).
If the code extends over several screen lines, and/or your editing  will
  include deleting lines, etc., delete the command-completion character
  at the end (\mv|;| or \mv|$|) first, and then edit, and then restore
  the command completion character to run the edited code.\\

\noindent The use of the keyboard keys \textbf{Home}, \textbf{End}, \textbf{PageUp},
  \textbf{PageDown}, \textbf{Ctrl+Home}, and
  \textbf{Ctrl+End} (as well as \textbf{UpArrow} and \textbf{DownArrow}
   greatly speeds up working with Xmaxima\\
  
\noindent For example to rerun as is or to copy a command which is located
  up near the top of your current Xmaxima session, first use \textbf{Home} to put 
  the cursor at the \textbf{beginning} of the current line, then \textbf{PageUp}
   or \textbf{Ctrl+Home} to get up to the top region fast.\\
  
\noindent If you simply want to rerun that command as is, use \textbf{End} to get the
  cursor at the end of the entry (past the command-completion character),
  and simply press \textbf{Enter} to retry that command.
The command will be evaluated  with the state of information Maxima has after
  the last input run.
That entry will be automatically entered into your session (with new output)
  at the bottom of your session screen.
You can get back to the bottom using the  \textbf{Ctrl+End} two-key command.\\
  
\noindent Alternatively, if you don't want the retry the exact same command, but something
  similar, then select the part of the code you want to use as a starting point
  for editing and  press \textbf{Ctrl+c} to copy your selection to the Window's Clipboard.
To select, you can either drag over the code with the mouse while you hold
  down the left mouse button, or else hold down the \textbf{Shift} key with your left hand
   and and combine with the \textbf{End}, the \textbf{LeftArrow},
   and the \textbf{DownArrow} keys  to help you select a region to copy.\\
  
\noindent Once you have selected and copied, press \textbf{Ctrl+End} to have the cursor move to 
  the bottom of your workspace where XMaxima is waiting for your next input and
  press \textbf{Ctrl+v} to paste your selection.
If the selection extends over multiple lines, use the down cursor key to find
  the end of the selection.
If your selection included the command-completion character, remove it 
 (backspace over that final symbol) before starting to edit your copied selection.\\
  
\noindent You are then in the driver's seat and can move around the code and make
  any changes without danger of Xmaxima pre-emptively sending your work to the
  Maxima engine until you finally have completed your editing, and move to the
  very end and provide the proper ending (either \mv|;| or \mv|$|) and then
  press \textbf{Enter} to evaluate the entry.
\subsubsection*{Using the Input Lines History}
If your cursor is positioned next to the active input prompt at the bottom of the screen
  (ie., where \textbf{Ctrl+End} places the cursor),
  you can use the key combinations \textbf{Alt+p} and \textbf{Alt+n} to recover the previous
 or next command that you entered.
For example, if you press \textbf{Alt+n}, you will enter the first input \mv|(%i1)|,
 and if you continue to press \textbf{Alt+n}, you will get in sucession \mv|(%i2)|, \mv|(%i3)|,...\\

\noindent Alternatively, if the active input prompt is \mv|(%i10)| and you 
  press \textbf{Alt+p} repeatedly, you will get inputs \mv|(%i9)|, \mv|(%i8)|,
 \mv|(%i7)|,  ...
\subsubsection*{Searching for a String in Your Previous Inputs}
Those same two-key combinations can also be used to search for a previous input
  lines containing a particular string.
Suppose you have one or more previous lines that included (among other functions) \textbf{sin(something)}.
At the last input prompt, type \textbf{sin} and then use either of the two-key
  commands \textbf{Alt+p} or \textbf{Alt+n} repeatedly until the particular input line
  containing the instance of \textbf{sin} you are looking for appears.
You can then either immediately rerun that input line (press \textbf{End} to get the cursor at
  the end of the input and then press \textbf{Enter}) or you can edit the input line using 
  \textbf{RightArrow} and \textbf{LeftArrow}, \textbf{Home}, and \textbf{End} 
  to move around, and finally complete your editing and press \textbf{Enter}.
In summary, you first write down the string to search, and then \textbf{Alt+p},
   to search backwards, or \textbf{Alt+n} to search forward.
Pressing those key combinations repeatedly will allow you to cycle
  through all the lines that contain the string.
If you want to try a different string in the middle of the search,
  you can delete the current input, type the new string, and 
  start the search again.   
\subsubsection*{Cutting and Pasting}
You can cut or copy a piece of text that you select,
  from anywhere on the text window (ie., the main top window of Xmaxima);
  not only from the input lines but also from the output text in black.
To select the text, you can drag the cursor with the mouse 
  while you keep its left button depressed, or you can
   hold the \textbf{Shift} key with one finger, while you move the cursor
   with the mouse or with the arrow keys.\\

\noindent Once you have selected the text, you can either
   cut it, with \textbf{Edit, cut} or the shortcut \textbf{Ctrl+x},
   \textbf{or} copy it to an internal buffer using \textbf{Edit, copy} or  \textbf{Ctrl+c}.
The text that has been cut or copied most recently can be 
  pasted anywhere, even in the output fields, 
  using \textbf{Edit, paste} or \textbf{Ctrl+v}.\\
  
\noindent There is a command similar to `cut', called `kill',
  accessed via either \textbf{Edit, kill} or \textbf{Ctrl+k}, 
  with two major differences: it only works in input fields (blue or green)
  and instead of cutting a text that was selected, 
  it will cut all the text from the cursor until
  the \textbf{end} of the input line.\\
  
\noindent The command \textbf{Edit, Clear input} or \textbf{Ctrl+u} is similar to 
	\textbf{Edit, kill}, but it will only work on the last input line (ie the
	current input line) and will clear all from the beginning of that input line 
	to the cursor position.\\

\noindent To paste the \textbf{last} text that you have cut with
  either `kill' or `clear input', you should use the `yank'
   command  \textbf{Edit, yank} or  \textbf{Ctrl+y}.
If you use the Clear Input command, \textbf{Ctrl+u}, you can immediately restore the line with the yank
  command \textbf{Ctrl+y} in a sort of ``UnDo''.
\subsubsection*{Other Keyboard Shortcuts}
Other useful key combinations are:\\
 \textbf{Ctrl+f}, the same as \textbf{RightArrow},\\
 \textbf{Ctrl+b}, the same as \textbf{LeftArrow},\\
 \textbf{Ctrl+p}, the same as \textbf{UpArrow},\\ 
 \textbf{Ctrl+n}, the same as \textbf{DownArrow},\\
Either \textbf{Ctrl+a} or \textbf{Home} moves to the left end of the current 
           line (to the left of \mv|(%xn)|),\\
Either \textbf{Ctrl+e} or \textbf{End} moves to the right end of the current line,\\
 \textbf{Ctrl+Home} moves to the first character at the top of the text window,\\
 \textbf{Ctrl+End}, moves to the last character at the bottom of the text window. 
\subsubsection*{Save Xmaxima Session Record as a Text File}
The menu command \textbf{Edit, Save Console to File} will bring up a dialog
  which allows you to select the folder and file name with the default
  extension \textbf{.out}, and Xmaxima will save the current session
  screen, as it appears to you, to a text file with that name.
This can be a convenient way to keep a record of your work, particularly if
  you use the day's date as part of the name.
You can then open that text file with any text editor, such as
  \textbf{Notepad2}, and edit it as usual, and you can also change
  the name of the file.
This session record is not in the form of inputs which you could use immediately
  with Xmaxima, although you could copy and paste command inputs one at a time
  into a later Xmaxima session.
\subsubsection*{Save All Xmaxima Inputs as Lisp Code for Later Use in Maxima}
The menu command \textbf{File, Save Expressions to File} will open a dialog
  which will save every input as Lisp code with a file name like \textbf{sat.bin} or
  \textbf{sat.sav}.
Although you can read such a file with a normal text editor, its main use to
  to rerun all the inputs in a later session by using \textbf{load("sat.bin")}, for
  example.
All the variable assignments and function definitions will be absorbed by Maxima,
  but you don't see any special output on the screen.
Instead of that menu route, you could just type the input
  \mv|save("C:/work2/sat1.bin",all);| to create the Lisp code record of session inputs.
\subsubsection*{Save All Xmaxima Inputs in a Batch File Using stringout}
If you type the input \mv|stringout("c:/work2/sat1.mac",input);|, the inputs will
  be saved in the form of Maxima syntax which can later be batched into a later
  Maxima session.
In a later session, you type the input\\
  \mv|batch("c:/work2/sat1.mac");| 
  and on the screen will appear each input line as well as what each
  output is in detail. 
(Or you could use the menu route \textbf{File, Batch File}, which will open a dialog
  which lets you select the file).
\subsubsection*{Quiet Batch Input}
If you use the menu route \textbf{File, Batch File Silently}, you will not see the
  record of inputs and outputs in detail, but all the bindings and definitions
  will be absorbed by Maxima.
There will be no return value and no indication which file has been loaded,
  so you might prefer typing \mvs|load("c:/work2/sat1.mac");|,
  or \mvs|batchload("c:/work2/sat1.mac");| just to have that record in your work.
\newpage
\subsection{Creating and Using a Startup File: \textbf{maxima-init.mac}}
You can create a startup file which will be read by Maxima at the start (or restart)
   of a new Maxima session.
If you either have not already created your startup file or have not
  interactively changed the binding of \mvs|maxima_userdir|, you can find
  where Maxima expects to find a startup file as follows.
We purposely start Xmaxima from a link to \textbf{..bin/xmaxima.exe} on the
  desktop, so Xmaxima has no clue where our work folder is.
\begin{myVerbatim}
(%i1) maxima_userdir;
(%o1)           C:/Documents and Settings/Edwin Woollett/maxima
\end{myVerbatim}
Before we show the advantages of using your startup file, let's show
  what you need to do to load one of your \textbf{*.mac} files in your
  work folder into Maxima without that startup file helping out.
In my work folder \textbf{c:/work2} is a file \textbf{qfft.mac},
  (available with Ch. 11 material on the author's web page) and here
  is an effort to load the file in Xmaxima.
\begin{myVerbatim}
(%i2) load(qfft);
Could not find `qfft' using paths in file_search_maxima,file_search_lisp.
 -- an error.  To debug this try debugmode(true);
(%i3) load("qfft.mac");
Could not find `qfft.mac' using paths in file_search_maxima,file_search_lisp.
 -- an error.  To debug this try debugmode(true);
(%i4) load("c:/work2/qfft.mac");
 type qfft_syntax(); to see qfft and qift syntax 
(%o4)                          c:/work2/qfft.mac
\end{myVerbatim}
On the other hand, if we have a link to \textbf{..bin/xmaxima.exe} sitting
  in our work folder \textbf{c:/work2} and we start Xmaxima using that folder
  link we get
\begin{myVerbatim}
(%i2) load(qfft);
Could not find `qfft' using paths in file_search_maxima,file_search_lisp.
 -- an error.  To debug this try debugmode(true);
(%i3) load("qfft.mac");
 type qfft_syntax(); to see qfft and qift syntax 
(%o3)                              qfft.mac
\end{myVerbatim}
which shows one of the virtues of starting up Xmaxima using a link in your
  work folder.\\
  
\noindent Now that you know where Maxima is going to look for your startup file, 
  make sure such a folder exists and create a file named \textbf{maxima-init.mac} in that folder.
You can have Maxima display a simple message when it reads your startup file
  as an added check on what exactly is going on.
For example, you could have the line \textbf{disp ("Hi, Cindy")\$}.\\

\noindent After saving the current version of that file in the correct folder, that message should 
  appear after the Maxima version and credits message.\\
  
\noindent Below is the author's startup file, in which he has chosen to tell Maxima to look in
  the \textbf{c:/work2} folder for \mvs| *.mac | or \mvs| *.mc | files, as well
  as the usual search paths.
The chapter 1 utility file \textbf{mbe1util.mac} is loaded into the session.
\begin{myVerbatim2s}
/* this is c:\Documents and Settings\Edwin Woollett\maxima\maxima-init.mac  */
/* last edit: 7-28-09  */
maxima_userdir: "c:/work2" $     
maxima_tempdir : "c:/work2"$
file_search_maxima : append(["c:/work2/###.{mac,mc}"],file_search_maxima )$
file_search_lisp : append(["c:/work2/###.lisp"],file_search_lisp )$
load(mbe1util)$
print("   mbe1util.mac functions ", functions)$
disp("Maxima is the Future!")$
\end{myVerbatim2s}
Naturally, you need to replace \mvs|c:/work2| by the path to your own work folder.
\newpage
\noindent With this startup file in place, here is the opening screen of Maxima,
  using the link to Xmaxima in my work folder to start the session, and
  setting input \mv|(%i1)| to be a  request for the binding of \mvs|maxima_userdir|:
\begin{myVerbatim} 
Maxima 5.19.0 http://maxima.sourceforge.net
Using Lisp GNU Common Lisp (GCL) GCL 2.6.8 (aka GCL)
Distributed under the GNU Public License. See the file COPYING.
Dedicated to the memory of William Schelter.
The function bug_report() provides bug reporting information.
   mbe1util.mac functions  [qplot(exprlist, prange, [hvrange]), rtt(e), 
ts(e, v), to_atan(e, y, x), to_atan2(e, y, x), totan(e, v), totan2(e, v), 
mstate(), mattrib(), mclean(), fll(x) ] 
                             Maxima is the Future!

(%i1) maxima_userdir;
(%o1)                              c:/work2
\end{myVerbatim}
Now it is easy to load in the package \textbf{qfft.mac}, and see the large increase
  in the number of user defined functons.
\begin{myVerbatim}
(%i2) load(qfft)$ 
(%i3) functions;
(%o3) [qplot(exprlist, prange, [hvrange]), rtt(e), ts(e, v), to_atan(e, y, x), 
to_atan2(e, y, x), totan(e, v), totan2(e, v), mstate(), mattrib(), mclean(), 
fll(x), nyquist(ns, fs), sample(expr, var, ns, dvar), vf(flist, dvar), 
current_small(), setsmall(val), _chop%(ex), fchop(expr), fchop1(expr, small), 
_fabs%(e), kg(glist), spectrum1(glist, nlw, ymax), 
spectrum(glist, nlw, ymax, [klim]), spectrum_eps(glist, nlw, ymax, fname, 
[klim])]
\end{myVerbatim}
We can use the utility function \textbf{mclean()} to remove those \textbf{qfft} functions.
\begin{myVerbatim}
(%i4) mclean();
----- clean start 
   
(%o0)                                   
(%i1) functions;
(%o1) [qplot(exprlist, prange, [hvrange]), rtt(e), ts(e, v), to_atan(e, y, x), 
to_atan2(e, y, x), totan(e, v), totan2(e, v), mstate(), mattrib(), mclean(), 
fll(x)]
\end{myVerbatim}
If we had instead used \textbf{kill ( all ) }, the intial set of utilities
  loaded in from \textbf{mbe1util.mac} would have also vanished.
\begin{myVerbatim}
(%i2) kill(all);
(%o0)                                done
(%i1) functions;
(%o1)                                 []
\end{myVerbatim}
Of course you could then use a separate \textbf{load(mbe1util)} command to get them back.
\newpage
\subsection{Maxima Expressions, Numbers, Operators, Constants and Reserved Words}
The basic unit of information in Maxima is the \tcbr{expression}.
 An \tcbr{expression} is made up of a combination of operators, numbers, variables, and constants.
 \tcbr{Variables} should have names which are not reserved words (see below),
   and can represent any type of data structue; there is no requirement to
   ``declare'' a variable to be  of a certain type.
\subsubsection*{Numbers in Maxima}
Maxima uses 
\begin{itemize}
\item Integers, such as  \textbf{123456},
\item Rational numbers, such as $\mathbf{3/2}$, ratios of integers,
\item Floats and bigfloats such as \textbf{1.234}, \textbf{1.234e-6}, and \textbf{1.234b5},
\item Complex numbers, such as \mvs|4 + 2*%i| and \mvs|a + b*%i|.
Maxima assumes the symbols \mv|a| and \mv|b| represent real numbers by default.
\end{itemize}
\subsubsection*{Operators in Maxima}
The table below lists some Maxima operators in order of priority, from lowest to highest.\\
The input \mvs|x^2+3| means $\mathbf{x^{2}+3}$, and not $\mathbf{x^{2+3}}$.
The exponentiation has higher precedence than addition.
The input \mvs|2^3^4| stands for \mvs|2^(3^4)|.
Parentheses can be used to force order of operations, or simply for clarity.
\begin{myVerbatim}
(%i1) x^2+3;
                                     2
(%o1)                               x  + 3
(%i2) 2^3^4;
(%o2)                      2417851639229258349412352
(%i3) 2^(3^4);
(%o3)                      2417851639229258349412352
(%i4) (2^3)^4;
(%o4)                                4096
\end{myVerbatim}
Since the operator $\mathbf{*}$ has precedence over $\mathbf{+}$, \; $\mathbf{a + b * c}$ \;
  means \; $\mathbf{a + (b * c)}$, \; rather than \; $\mathbf{(a+b)*c}$.

%\newpage
%\small
\begin{table}[h]
\begin{center}
  \begin{tabular}{|ll|}
    \hline
    Operator & Description \\ \hline
	$+$ & addition \\
	$-$  & subtraction \\
	$\ast$  & multiplication    \\
    $/$&  division   \\
	$-$ & negation \\
	\verb|^| & exponentiation  \\
	$.$  &  non-commutative multiplication \\
	\verb|^^|  &  non-commutative exponentiation  \\ 
	$!$ &  factorial  \\
	$!!$ & double factorial \\ \hline
	\end{tabular}  
\caption{Operators in Order of Increasing Precedence}
\end{center}
\end{table}
%\normalsize  

\newpage
\subsubsection*{Constants in Maxima}
The following table summarizes predefined constants.
%\small
\begin{table}[h]
\begin{center}
  \begin{tabular}{|ll|}
    \hline
    Constant & Description \\ \hline
	\textbf{\%e} & Base of the natural logarithms $\mathbf{(e)}$ \\
	\textbf{\%i}  & The square root of $(-1)$ \; $\mathbf{(i)}$ \\
	\textbf{\%pi}  & The transcendental constant pi $\boldsymbol{(\pi)}$    \\
    \textbf{\%phi}&  The golden mean $\boldsymbol{(1+\sqrt{5})/2}$   \\
	\textbf{\%gamma} & The Euler-Mascheroni constant \\
	\textbf{inf} & Real positive infinity $\mathbf{(\infty)}$ \\
	\textbf{minf} & Real negative infinity $\mathbf{(-\infty)}$  \\ \hline
	\end{tabular}  
\caption{Maxima Predefined Constants}
\end{center}
\end{table}
%\normalsize

\noindent Here are the numerical values to $16$ digit precision.
\begin{myVerbatim}
(%i5) float( [%e,%pi,%phi,%gamma] );
(%o5) [2.718281828459045, 3.141592653589793, 1.618033988749895, 
                                                              0.57721566490153]
\end{myVerbatim}
%\newpage
\subsubsection*{Reserved Words}
There are a number of reserved words which, if used as variable or function names,
  might be confusing to both the user and Maxima.
Their use might cause a possibly cryptic syntax error. 
Here are some of the ``well known'' and ``less well known but short'' reserved words
\small
\begin{table}[h]
\begin{center}
  \begin{tabular}{|p{1in} p{1in} p{1in} p{1in}|}
    \hline
	af & else & ic2 & plog\\
    and & elseif & if &	psi\\
	av & erf & ift & product\\
	args & ev & ilt & put\\
	array & exp & in & rat\\
	at & f90 & ind & rem\\
	bc2 & fft & inf & rk\\
	carg & fib & inrt & some\\
	cf & fix & integrate & step\\
	cint & for & is & sum\\
	col & from & li & then	\\
    cov & gcd & limit & thru\\
	cv & gd & min & und\\
	del & get & next & unless\\
	diag & go & not & vers\\
	diff & hav & op & while\\
	do & ic1 & or &  zeta\\ \hline	
	\end{tabular}  
\caption{Some Simple Reserved Words}
\end{center}
\end{table}
\normalsize  
Of course there are many other Maxima function names, global option variables,
  and system option variables which you might also try to avoid when naming your own
  variables and function.
One quick way to check on ``name conflicts'' is to keep the html Maxima help manual
  up in a separate window, and have the Index panel available to type in a
  name you want to use.
The painful way to check on name conflicts is to wait for Maxima to give you a strange
  error message as to why what you are trying to do won't work.
If you get strange results, try changing the names of your variables and or functions.\\

\noindent An important fact is that Maxima is case sensitive, so you can avoid name
  conflicts by capitalizing the names of your user defined Maxima functions.
Your \textbf{Solve} will not conflict with Maxima's \textbf{solve}.
This is a dumb example, but illustrates the principle:
\begin{myVerbatim}
(%i6) Solve(x):= x^2;
                                             2
(%o6)                           Solve(x) := x
(%i7) Solve(3);
(%o7)                                  9
\end{myVerbatim}
Of course, it takes more typing effort to use capitalized function names, which is why
  they are not popular among power users.
\subsection{Input and Output Examples}
In the following we are using  the \textbf{Xmaxima} interface.\\

\noindent As discussed in Sec. \ref{rat}, a \tcbr{rational expression} 
  is a ratio of polynomials or a sum of such.
A special case is a polynomial.
A rational expression is a special case of what is called an \tcbr{expression} in
  Maxima.\\

\noindent Here we write a rational expression without binding it to any particular symbol.
\begin{myVerbatim}
(%i1) x/(x^3+1);
                                      x
(%o1)                               ------
                                     3
                                    x  + 1
\end{myVerbatim}
\small
The input line is typed in a ``one dimensional'' version and, if the input completion character
  is a semi-colon \mv|;| then Xmaxima displays the expression in a text based two-dimensional
  notation.
You can force one-dimensional output if you set \textbf{display2d} to \textbf{false}:
\begin{myVerbatim}
(%i2) display2d:false$
(%i3) %o1;
(%o3) x/(x^3+1)
(%i4) display2d:true$
(%i5) %o1;
                                      x
(%o5)                               ------
                                     3
                                    x  + 1
\end{myVerbatim}
When you want to show a piece of code to the Maxima mailing list,
  it is recommended that you show the output in the one-dimensional form
  since otherwise, in the received message, exponents can appear in a shifted position which
  may be hard to interpret.\\
  
\noindent The output \mv|(%o6)| is a symbolic expression which can be manipulated as a
  data structure.
For example, you can add it to itself.
The Maxima symbol \mv|%| refers to the last output line.
\begin{myVerbatim}
(%i6) % + %;
                                     2 x
(%o6)                               ------
                                     3
                                    x  + 1
\end{myVerbatim}
Notice the automatic simplification carried out by default.
Maxima could have left the result as a sum of two terms, but instead recognised that
  the summands were identical and added them together producing a one term result.
Maxima will automatically (in default behavior) perform many such simplifications.\\

\noindent Here is an example of automatic simplification of a trig function:
\begin{myVerbatim}
(%i7) sin(x - %pi/2);
(%o7)                              - cos(x)
\end{myVerbatim}  
\normalsize
\newpage
\subsection{Maxima Power Tools at Work}
\subsubsection{The Functions \textbf{apropos} and \textbf{describe}}
\subsubsection*{Function \textbf{apropos}}
\mv|apropos("foo")| returns a list of core Maxima names 
  which have \mv|foo| appearing anywhere within them.
For example, \mv|apropos ("exp")| returns a list of all the core flags and functions 
  which have \textbf{exp} as part of their names, such as \textbf{expand}, 
  \textbf{exp}, and \textbf{ratexpand}.
Thus if you can only remember part of the name of something, 
  you can use this command to find the correct complete name. 
Here is an example:
\begin{myVerbatim}
(%i1) apropos ("exp");
(%o1) [askexp, auto_mexpr, besselexpand, beta_expand, cfexpand, comexp, 
domxexpt, dotexptsimp, errexp, errexp1, errexp2, errexp3, exp, exp-form, 
expand, expandwrt, expandwrt_denom, expandwrt_factored, expandwrt_nonrat, 
expansion, expint, expintegral_chi, expintegral_ci, expintegral_e, 
expintegral_e1, expintegral_ei, expintegral_hyp, expintegral_li, 
expintegral_shi, expintegral_si, expintegral_trig, expintexpand, expintrep, 
explicit, explose, expon, exponentialize, expop, expr, exprlist, expt, 
exptdispflag, exptisolate, exptsubst, Expt, facexpand, factorial_expand, 
gamma_expand, logexpand, macroexpand, macroexpand1, macroexpansion, matrixexp, 
poisexpt, psexpand, radexpand, ratexpand, ratsimpexpons, sexplode, 
solveexplicit, sumexpand, taylor_logexpand, texput, trigexpand, 
trigexpandplus, trigexpandtimes, tr_exponent, tr_warn_fexpr]
\end{myVerbatim}
\subsubsection*{Function \textbf{describe}}
\textbf{describe(e), describe(e, exact), describe(e, inexact) }\\
\textbf{describe(e)} is equivalent to \textbf{describe(e, exact)} and
      prints to the screen the manual documentation of \textbf{e}.\\
	  
	  
\noindent \textbf{describe(e, inexact)} prints to the screen a numbered list of all
  items documented in the manual which contain ``e'' as part of their name.
If there is more than one list element, Maxima asks the user to 
  select an element or elements to display.\\

\noindent \tcbr{SHORTCUTS}:\\
At the interactive prompt, \mv|? foo| (with a space between \mv|?| and \mv|foo|) and
  NO ENDING SEMICOLON (just press Enter) is equivalent to either
  \textbf{describe(foo)} or \textbf{describe(foo, exact)},
  and \mv|?? foo| (with a space between \mv|??| and \mv|foo|) and
  NO ENDING SEMICOLON (just press Enter) is equivalent to \textbf{describe(foo, inexact)}. \\

\noindent In the latter case, the user will be asked to type either a set of space separated numbers
  to select some of the list elements, such as \mv|2 3 5| followed by Enter, or
  the word \mv|all| followed by Enter, or the word \mv|none| followed by Enter.\\
  
\noindent Here is an example of interactive use of the single question mark
  shortcut.
\begin{myVerbatim}
(%i2) ? exp
 -- Function: exp (<x>)
     Represents the exponential function.  Instances of `exp (<x>)' in
     input are simplified to `%e^<x>'; `exp' does not appear in
     simplified expressions.

     `demoivre' if `true' causes `%e^(a + b %i)' to simplify to `%e^(a
     (cos(b) + %i sin(b)))' if `b' is free of `%i'. See `demoivre'.

     `%emode', when `true', causes `%e^(%pi %i x)' to be simplified.
     See `%emode'.

     `%enumer', when `true' causes `%e' to be replaced by 2.718...
     whenever `numer' is `true'. See `%enumer'.

\end{myVerbatim}
\newpage
\begin{myVerbatim}
  There are also some inexact matches for `exp'.
  Try `?? exp' to see them.

(%o2)                                true
\end{myVerbatim}
\subsubsection{The Function \textbf{ev} and  the Properties \textbf{evflag} and \textbf{evfun}}
 \subsubsection*{Function \textbf{ev}}
\textbf{ev} is a ``jack-of-all-trades swiss army knife'' which is frequently useful,
  occasionally dangerous, and complex to describe.\\

\noindent In brief, the syntax is
\begin{myVerbatim2}
     ev ( expr, options );
        or more explicitly,
     ev (expr, arg_1, ..., arg_n) 
\end{myVerbatim2}
  with the interactive mode shortcut
\begin{myVerbatim2}
     expr, options ; 
       or again more explicitly
     expr, arg_1, ..., arg_n ; 
\end{myVerbatim2}
The interactive mode shortcut form cannot be used inside user defined Maxima
  functions or blocks.\\
  
\noindent The Manual description of \textbf{ev} begins
\begin{myVerbatim2s}
Evaluates the expression expr in the environment specified by the 
  arguments arg_1, ..., arg_n. The arguments are switches (Boolean flags),
  assignments, equations, and [Maxima] functions. ev returns the result
  (another expression) of the evaluation. 
\end{myVerbatim2s}
  
\noindent One option has the form \mv|V : e|, or \mv|V = e|,
  which causes \mv|V| to be bound to the value of \mv|e| during the evaluation of \mv|expr|.
If more than one argument to \textbf{ev} is of this type, then the binding is done in \tcbr{parallel},
  as shown in the following example. 
\begin{myVerbatim}
(%i1) x+y, x  =  a+y;
(%o1)                               2 y + a
(%i2) %, y = 2;
(%o2)                                a + 4
(%i3) x+y, x = a+y, y = 2;
(%o3)                              y + a + 2
(%i4) x+y, [x = a+y, y = 2];
(%o4)                              y + a + 2
\end{myVerbatim}
If \mv|V| is a non-atomic expression, then a substitution rather than a binding is
  performed.\\
  
\noindent This example illustrates the subtlety of the way \textbf{ev} in designed to work,
  and shows that some experimentation should be carried out to gain confidence in the
  result returned by \textbf{ev}.\\
  
\noindent The next example of \textbf{ev} shows its use to check the correctness of solutions
  returned by \textbf{solve}.
\begin{myVerbatim}
(%i1) eqns : [-2*x -3*y = 3, -3*x +2*y = -4]$
(%i2) solns : solve (eqns);
                                     17      6
(%o2)                        [[y = - --, x = --]]
                                     13      13
(%i3) eqns, solns;
(%o3)                         [3 = 3, - 4 = - 4]
\end{myVerbatim}
\newpage
\noindent Our final example shows the use of \textbf{rectform} and \textbf{ratsimp} to make explicit
  the fourth roots of $\mathbf{-1}$ returned by \textbf{solve}.
\begin{myVerbatim}
(%i1) solve ( a^4 + 1 );
                1/4                1/4             1/4              1/4
(%o1) [a = (- 1)    %i, a = - (- 1)   , a = - (- 1)    %i, a = (- 1)   ]
(%i2) % , rectform, ratsimp;
           sqrt(2) %i - sqrt(2)        sqrt(2) %i + sqrt(2)
(%o2) [a = --------------------, a = - --------------------, 
                    2                           2
                                sqrt(2) %i - sqrt(2)      sqrt(2) %i + sqrt(2)
                          a = - --------------------, a = --------------------]
                                         2                         2
(%i3) %^4, ratsimp;
                     4         4         4         4
(%o3)              [a  = - 1, a  = - 1, a  = - 1, a  = - 1]
\end{myVerbatim}
Both \textbf{rectform} and \textbf{ratsimp} are Maxima functions which have the property \textbf{evfun}
   (see next entry), which means that \mv|ev(expr, rectform, ratsimp)| is equivalent to
   \mv|ratsimp ( rectform ( ev(expr) ) ) |.
\subsubsection*{Property \textbf{evflag}}
When a symbol \mv|x| has the \textbf{evflag} property, the expressions
   \mv|ev(expr, x)| and \mv|expr, x| (at the interactive prompt) are equivalent
   to \mv|ev(expr, x = true)|.
That is, \mv|x| is bound to \textbf{true} while \mv|expr| is evaluated.\\
The expression \mv|declare(x, evflag)| gives the \textbf{evflag} property to the variable \mv|x|.\\
The flags which have the \textbf{evflag} property by default are the following:
\begin{myVerbatim2s}
 algebraic, cauchysum, demoivre, dotscrules, %emode, %enumer, exponentialize,
 exptisolate, factorflag, float, halfangles, infeval, isolate_wrt_times,
 keepfloat, letrat, listarith, logabs, logarc, logexpand, lognegint, lognumer,
 m1pbranch, numer_pbranch, programmode, radexpand, ratalgdenom, ratfac,
 ratmx, ratsimpexpons, simp, simpsum, sumexpand, and trigexpand. 
\end{myVerbatim2s}
Even though the Boolean switch \textbf{numer} does not have the property \textbf{evflag}, it
  can be used as if it does.
\begin{myVerbatim}
(%i4) [exponentialize,float,numer,simp];
(%o4)                    [false, false, false, true]
(%i5) properties(exponentialize);
(%o5) [system value, transfun, transfun, transfun, transfun, transfun, 
                                                              transfun, evflag]
(%i6) properties(numer);
(%o6)                  [system value, assign property]
(%i7) properties(float);
(%o7) [system value, transfun, transfun, transfun, transfun, transfun, 
                                                    evflag, transfun, transfun]
(%i8) properties(simp);
(%o8)                      [system value, evflag]
\end{myVerbatim}
Here are some examples of use.
\begin{myVerbatim}
(%i9) [ ev (exp(3/29), numer ), ev (exp(3/29), float) ];
(%o9)              [1.108988430411017, 1.108988430411017]
(%i10) [ ev (exp (%pi*3/29), numer), ev (exp (%pi*3/29), float) ];
                                        0.10344827586207 %pi
(%o10)            [1.384020049155809, %e                    ]
(%i11) 2*cos(w*t) + 3*sin(w*t), exponentialize, expand;
                    %i t w                     - %i t w
             3 %i %e           %i t w   3 %i %e             - %i t w
(%o11)     - ------------- + %e       + --------------- + %e
                   2                           2
\end{myVerbatim}
(For the use of the ``key word'' \textbf{expand} in this context, see below.)
\subsubsection*{Property \textbf{evfun}}
When a Maxima \tcbr{function} \mv|F| has the \textbf{evfun} property, the expressions 
  \mv|ev(expr, F)| and \mv|expr, F| (at the interactive prompt) are 
  equivalent to \mv|F ( ev (expr))|.\\
If two or more \textbf{evfun} functions \mv|F|, \mv|G|, etc., are specified,
  then \textbf{ev ( expr, F, G )} is equivalent to \textbf{G ( F ( ev(expr) ) )}.\\
  
\noindent The command \mv|declare(F, evfun)| gives the \textbf{evfun} property to the function \mv|F|.\\
The functions which have the \textbf{evfun} property by \tcbr{default} are
  the following very useful and single argument funtions: \textbf{bfloat,}
   \textbf{factor, fullratsimp, logcontract},
  \textbf{polarform, radcan, ratexpand, ratsimp, rectform},
  \textbf{rootscontract, trigexpand, and trigreduce}.\\
  
\noindent Note that \textbf{rat}, and \textbf{trigsimp} do not, by default, 
   have the property \textbf{evfun}.\\
   
\noindent Some of the other ``key words'' which can be used with \textbf{ev} are
 \textbf{expand, nouns, diff, integrate} (even though they are not obviously
   boolean switches and do not have the property \textbf{evflag}).
\begin{myVerbatim}
(%i12) [diff,expand,integrate,nouns];
(%o12)                 [diff, expand, integrate, nouns]
(%i13) properties(expand);
(%o13)        [transfun, transfun, transfun, transfun, transfun]
(%i14) (a+b)*(c+d);
(%o14)                          (b + a) (d + c)
(%i15) (a+b)*(c+d),expand;
(%o15)                       b d + a d + b c + a c
(%i16) ev((a+b)*(c+d),expand);
(%o16)                       b d + a d + b c + a c
\end{myVerbatim}
The undocumented property \textbf{transfun} apparently has something to do with
  translation to Lisp.

\subsubsection{The List \textbf{functions} and the Function \textbf{fundef}}
\subsubsection*{\textbf{functions}}
\textbf{functions} is the name of a list maintained by Maxima and this list
  is printed to the screen (as would any list) with the names of the current 
  available non-core Maxima functions (either user defined interactively 
  or defined by any packages loaded into the current session).
Here is the example we saw previously based on the author's startup file:
\begin{myVerbatim}
(%i1) functions;
(%o1) [qplot(exprlist, prange, [hvrange]), rtt(e), ts(e, v), to_atan(e, y, x), 
to_atan2(e, y, x), totan(e, v), totan2(e, v), mstate(), mattrib(), mclean(), 
fll(x)]
\end{myVerbatim}
\subsubsection*{\textbf{fundef}}
\textbf{fundef(name)} prints out the definition of a non-core Maxima function
  (if currently known by Maxima).
Here is an example related to the previous example which said that Maxima knew
  about a function called \textbf{rtt}.
\begin{myVerbatim}
(%i2) fundef(rtt);
(%o2)               rtt(e) := radcan(trigrat(trigsimp(e)))
(%i3) fundef(cos);
cos is not the name of a user function.
 -- an error.  To debug this try debugmode(true);
\end{myVerbatim}
\newpage
\subsubsection{The Function \textbf{kill} and the List \textbf{values}}
\subsubsection*{ \textbf{kill}}
 \textbf{kill(a,b)} will eliminate the objects \textbf{a} and \textbf{b}.
Special cases are \textbf{kill(all)} and \textbf{kill(allbut(x,y))}.
Here is an example which removes our user defined Maxima function \textbf{rtt}.
\begin{myVerbatim}
(%i1) kill(rtt);
(%o1)                                done
(%i2) functions;
(%o2) [qplot(exprlist, prange, [hvrange]), ts(e, v), to_atan(e, y, x), 
to_atan2(e, y, x), totan(e, v), totan2(e, v), mstate(), mattrib(), mclean(), 
fll(x)]
\end{myVerbatim}
\subsubsection*{\textbf{values}}
\textbf{values} is another list maintained by Maxima which contains     
  the names of currently assigned scalar values which have been set by the user interactively
   or by packages which have been loaded.
Here is a simple example.
\begin{myVerbatim}
(%i3) [ a:2, b:5, e: x^2/3 ];
                                          2
                                         x
(%o3)                             [2, 5, --]
                                         3
(%i4) values;
(%o4)                              [a, b, e]   
\end{myVerbatim}
\subsubsection{Examples of \textbf{map}, \textbf{fullmap}, \textbf{apply}, \textbf{grind}, and \textbf{args}}
Let \mv|f| be an unbound symbol.
When \mv|f| is then mapped onto a list or expression, we can see what will
  happen if we \textbf{map} a Maxima function on to the same kind of object.
\begin{myVerbatim}
(%i1) map('f, [x, y, z] );
(%o1)                         [f(x), f(y), f(z)]
(%i2) map('f, x + y + z);
(%o2)                         f(z) + f(y) + f(x)
(%i3) map('f, [a*x,b*exp(y),c*log(z)]);
                                       y
(%o3)                   [f(a x), f(b %e ), f(c log(z))]
(%i4) map('f, a*x + b*exp(y) + c*log(z) );
                                            y
(%o4)                   f(c log(z)) + f(b %e ) + f(a x)
\end{myVerbatim}
The use of \textbf{map} with the Maxima function \textbf{ratsimp} allows
  a separate simplification to be carried out on each term of the following
  expression.
\begin{myVerbatim}
(%i5) e : x/(x^2+x)+(y^2+y)/y ;
                                 2
                                y  + y     x
(%o5)                           ------ + ------
                                  y       2
                                         x  + x
(%i6) e, ratsimp;
                               (x + 1) y + x + 2
(%o6)                          -----------------
                                     x + 1
(%i7) map('ratsimp,e);
                                       1
(%o7)                            y + ----- + 1
                                     x + 1
\end{myVerbatim}
We create a list of equations from two lists using \textbf{map}.
\begin{myVerbatim}
(%i8) map( "=", [x,y,z],[a,b,c] );
(%o8)                       [x = a, y = b, z = c]
\end{myVerbatim}
We compare \textbf{map} and \textbf{fullmap} when appied to an expression.
\begin{myVerbatim}
(%i9) expr : 2*%pi + 3*exp(-4);
                                            - 4
(%o9)                           2 %pi + 3 %e
(%i10) map('f, expr);
                                              - 4
(%o10)                        f(2 %pi) + f(3 %e   )
(%i11) fullmap('f, expr);
                                                f(- 4)
(%o11)                   f(2) f(%pi) + f(3) f(%e)
\end{myVerbatim}
\subsubsection*{apply}
The Maxima function \textbf{apply} can be used with a list only (not an expression).
\begin{myVerbatim}
(%i12) apply('f, [x,y,z]);
(%o12)                            f(x, y, z)
(%i13) apply("+",[x,y,z]);
(%o13)                             z + y + x
(%i14) dataL : [ [1,2], [2,4] ]$
(%i15) dataM : apply('matrix, dataL );
                                   [ 1  2 ]
(%o15)                              [      ]
                                   [ 2  4 ]
(%i16) grind(%)$
matrix([1,2],[2,4])$
\end{myVerbatim}
The function \textbf{grind} allows one dimensional display suitable for
  copying into another input line.
The last example above created a Maxima \textbf{matrix} object from a nested list.
Maxima has many tools to work with \textbf{matrix} objects.
To achieve one dimensional display of such \textbf{matrix} objects (saving space),
  you can use \mv|display2d:false|.
To go from a Maxima \textbf{matrix} object to a nested list, use \textbf{args}.
\begin{myVerbatim}
(%i17) args(dataM);
(%o17)                          [[1, 2], [2, 4]]
\end{myVerbatim}
\subsubsection{Examples of \textbf{subst}, \textbf{ratsubst}, \textbf{part}, and \textbf{substpart}}
\subsubsection*{subst}
To replace \mv|x| by  \mv|a| in an expression \mv|expr|,
  you can use either of two forms.
\begin{myVerbatim2}
   subst ( a, x, expr );
     or
   subst ( x = a, expr );	 
\end{myVerbatim2}
To replace \mv|x| by \mv|a| and also \mv|y| by \mv|b| one
  can use
\begin{myVerbatim2}
  subst ( [ x = a, y = b ], expr );
\end{myVerbatim2}
 \mv|x| and \mv|y| must be  \tcbr{atoms} (ie., a number, a string,
    or a symbol) or a complete \tcbr{subexpression} of \mv|expr|.
 When \mv|x| does not have these characteristics, use \mv|ratsubst( a, x, expr )|
     instead.\\
\newpage
\noindent Some examples:
\begin{myVerbatim}
(%i1) e : f*x^3 + g*cos(x);
                                              3
(%o1)                           g cos(x) + f x
(%i2) subst ( x = a, e );
                                            3
(%o2)                           cos(a) g + a  f
(%i3) e1 : subst ( x = a + b, e );
                                                 3
(%o3)                      cos(b + a) g + (b + a)  f
(%i4) e2 : subst ( a + b = y, e1 );
                                              3
(%o4)                           g cos(y) + f y
(%i5) e3 : f*x^3 + g*cos(y);
                                              3
(%o5)                           g cos(y) + f x
(%i6) e4 : subst ( [x = a+b, y = c+d],e3 );
                                                 3
(%o6)                      cos(d + c) g + (b + a)  f
(%i7) subst ([a+b = r, c+d = p],e4 );
                                   3
(%o7)                           f r  + g cos(p)
\end{myVerbatim}
\subsubsection*{ratsubst}
\begin{myVerbatim}
(%i8) e : a*f(y) + b*g(x);
(%o8)                           a f(y) + b g(x)
(%i9) e1 : ratsubst( cos(y),f(y),e );
(%o9)                          a cos(y) + b g(x)
(%i10) e2 : ratsubst( x^3*sin(x),g(x),e1 );
                                          3
(%o10)                       a cos(y) + b x  sin(x)
\end{myVerbatim}	 
\subsubsection*{part and substpart}
The Maxima functions \textbf{part} and \textbf{substpart} are best
  defined by a simple example.
\begin{myVerbatim}
(%i11) e : a*log(f(y))/(b*exp(f(y)));
                                 - f(y)
                             a %e       log(f(y))
(%o11)                        --------------------
                                      b
(%i12) length(e);
(%o12)                                  2
(%i13) [part(e,0),part(e,1),part(e,2)];
                                 - f(y)
(%o13)                    [/, a %e       log(f(y)), b]
(%i14) substpart("+",e,0);
                               - f(y)
(%o14)                      a %e       log(f(y)) + b
(%i15) length(part(e,1));
(%o15)                                  3
(%i16) [part(e,1,0),part(e,1,1),part(e,1,2),part(e,1,3)];
                                   - f(y)
(%o16)                     [*, a, %e      , log(f(y))]
(%i17) length(part(e,1,3));
(%o17)                                  1
(%i18) [part(e,1,3,0),part(e,1,3,1)];
(%o18)                             [log, f(y)]
\end{myVerbatim}
\newpage
\begin{myVerbatim}
(%i19) substpart(sin,e,1,3,0);
                                 - f(y)
                             a %e       sin(f(y))
(%o19)                       --------------------
                                      b
(%i20) length( part(e,1,3,1) );
(%o20)                                 1
(%i21) [part(e,1,3,1,0), part(e,1,3,1,1)];
(%o21)                              [f, y]
(%i22) substpart(x,e,1,3,1,1);
                                           - f(y)
                             a log(f(x)) %e
(%o22)                       --------------------
                                      b
\end{myVerbatim}
By judicious use of \textbf{part} and \textbf{substpart}, we see that we can
  modify a given expression in all possible ways.
\subsubsection{Examples of \textbf{coeff}, \textbf{ratcoef}, and \textbf{collectterms}}
\subsubsection*{\textbf{collectterms}}
Use the syntax
\begin{myVerbatim2}
   collectterms ( expr, arg1, arg2, ...)
\end{myVerbatim2}
This Maxima function is best explained with an example.
Consider the expression:
\begin{myVerbatim}
(%i1) ex1 :  a1*(b + c/2)^2 + a2*(d + e/3)^3 , expand;
              3         2                         2
          a2 e    a2 d e        2         3   a1 c                 2
(%o1)     ----- + ------- + a2 d  e + a2 d  + ----- + a1 b c + a1 b
           27        3                          4
\end{myVerbatim}
How can we return this expanded expression to the original form?
We first use \textbf{collectterms}, and then \textbf{factor} with \textbf{map}.
\begin{myVerbatim}
(%i2) collectterms(ex1,a1,a2);
                     3      2                     2
                    e    d e     2      3        c           2
(%o2)           a2 (-- + ---- + d  e + d ) + a1 (-- + b c + b )
                    27    3                      4
(%i3) map('factor, %);
                                     3               2
                         a2 (e + 3 d)    a1 (c + 2 b)
(%o3)                    ------------- + -------------
                              27               4
\end{myVerbatim}
Maxima's core simplification rules prevent us from getting the $3^{3}=27$ into
  the numerator of the first term, and also from getting the $2^{2}=4$ into the
   numerator of the second term, unless we are willing to do a lot of \textbf{substpart}
   piecework (usually not worth the trouble).
\subsubsection*{\textbf{coeff}}
The syntax
\begin{myVerbatim2}
     coeff ( expr, x, 3 )
\end{myVerbatim2}
  will return the coefficient of \mv|x^3| in \mv|expr|.\\
\newpage
\noindent The syntax
\begin{myVerbatim2}
     coeff ( expr, x )
\end{myVerbatim2}
  will return the coefficient of \mv|x| in \mv|expr|.\\
\mv|x| may be an atom or a complete subexpression of \mv|expr|,
  such as \mv|sin(y)|, \mv|a[j]|, or \mv|(a+b)|.   
Sometimes it may be necessary to expand or factor \mv|expr| in
  order to make \mv|x^n| explicit.
This  preparation is not done automatically by \textbf{coeff}. 
\begin{myVerbatim}
(%i4) coeff(%, a2);
                                           3
                                  (e + 3 d)
(%o4)                             ----------
                                      27
(%i5) coeff(%, e + 3*d, 3);
                                      1
(%o5)                                 --
                                      27
\end{myVerbatim}
\subsubsection*{\textbf{ratcoef} (also \textbf{ratcoeff})}
The function \textbf{ratcoef} (or \textbf{ratcoeff}) has the same syntax as \textbf{coeff}
  (except that \mv|n| should not be negative), but expands and 
 rationally simplifies the expression before finding the coefficient, and thus 
  can produce answers different from \textbf{coeff}, which is purely syntactic.
\begin{myVerbatim}
(%i6) ex2 : (a*x + b)^2;
                                           2
(%o6)                             (a x + b)
(%i7) coeff(ex2,x);
(%o7)                                  0
(%i8) ratcoeff(ex2,x);
(%o8)                                2 a b
(%i9) ratcoef(ex2,x);
(%o9)                                2 a b
(%i10) ratcoeff(ex2, x, 0);
                                       2
(%o10)                                 b
(%i11) ratcoef(ex2, x, 0);
                                       2
(%o11)                                 b
\end{myVerbatim}
Both \textbf{coeff} and \textbf{ratcoef} can be used with equations (as well as
  expressions).
\begin{myVerbatim}
(%i12) eqn : (a*sin(x) + b*cos(x))^3 = c*sin(x)*cos(x);
                                        3
(%o12)             (a sin(x) + b cos(x))  = c cos(x) sin(x)
(%i13) ratcoef(eqn,sin(x),0);
                                 3    3
(%o13)                          b  cos (x) = 0
(%i14) ratcoef(eqn,sin(x));
                                2    2
(%o14)                     3 a b  cos (x) = c cos(x)
\end{myVerbatim}
\newpage
\subsubsection{Examples of \textbf{rat, diff, ratdiff,  ratexpand, expand, factor, gfactor} and \textbf{partfrac}} 
Maxima provides tools which allow the user to write an expression in multiple forms.
Consider the tenth order polynomial in x given by
\begin{myVerbatim}
(%i1) e:(x + 3)^10;
                                          10
(%o1)                              (x + 3)
\end{myVerbatim}
We can use \textbf{diff ( expr, x )} to find the first derivative of \textbf{expr},
  and \textbf{diff ( expr, x, 2 )} to find the second derivative of \textbf{expr}, and so on.
\begin{myVerbatim}  
(%i2) de1 : diff ( e, x );
                                            9
(%o2)                             10 (x + 3)
\end{myVerbatim}
Because this expression is a polynomial in \mv|x|, we can also use \textbf{ratdiff ( expr, x )}:
\begin{myVerbatim}
(%i3) de1r : ratdiff ( e, x );
          9        8         7          6           5           4           3
(%o3) 10 x  + 270 x  + 3240 x  + 22680 x  + 102060 x  + 306180 x  + 612360 x
                                                          2
                                                + 787320 x  + 590490 x + 196830
(%i4) factor ( de1r );
                                            9
(%o4)                             10 (x + 3)
\end{myVerbatim}
We show three tools for expansion of this polynomial.
\begin{myVerbatim}
(%i5) rat(e);
          10       9        8         7          6          5           4
(%o5)/R/ x   + 30 x  + 405 x  + 3240 x  + 17010 x  + 61236 x  + 153090 x
                                               3           2
                                     + 262440 x  + 295245 x  + 196830 x + 59049
(%i6) ratexpand (e);
       10       9        8         7          6          5           4
(%o6) x   + 30 x  + 405 x  + 3240 x  + 17010 x  + 61236 x  + 153090 x
                                               3           2
                                     + 262440 x  + 295245 x  + 196830 x + 59049
(%i7) expand (e);
       10       9        8         7          6          5           4
(%o7) x   + 30 x  + 405 x  + 3240 x  + 17010 x  + 61236 x  + 153090 x
                                               3           2
                                     + 262440 x  + 295245 x  + 196830 x + 59049
(%i8) factor (%);
                                          10
(%o8)                              (x + 3)
\end{myVerbatim}
If you want the lowest powers displayed first, you have to change the
  setting of \textbf{powerdisp} to \textbf{true}.
\begin{myVerbatim}
(%i9) powerdisp;
(%o9)                                false
(%i10) powerdisp : true$
(%i11) %o7;
                                  2           3           4          5
(%o11) 59049 + 196830 x + 295245 x  + 262440 x  + 153090 x  + 61236 x
                                             6         7        8       9    10
                                    + 17010 x  + 3240 x  + 405 x  + 30 x  + x
(%i12) powerdisp : false$
\end{myVerbatim}
\newpage
\noindent Next we consider a sum of rational expressions:
\begin{myVerbatim}
(%i13) expr: (x - 1)/(x + 1)^2 + 1/(x - 1);
                                x - 1       1
(%o13)                          -------- + -----
                                      2   x - 1
                               (x + 1)
(%i14) expand ( expr );
                           x              1           1
(%o14)                 ------------ - ------------ + -----
                       2              2             x - 1
                      x  + 2 x + 1   x  + 2 x + 1
(%i15) ratexpand ( expr );
                               2
                            2 x                 2
(%o15)                  --------------- + ---------------
                        3    2            3    2
                       x  + x  - x - 1   x  + x  - x - 1
\end{myVerbatim}
We see that \textbf{ratexpand} wrote all fractions with a common denominator, whereas
  \textbf{expand} did not.
In general, \textbf{rat} and \textbf{ratexpand} is more efficient at expanding rational expressions.
Here is a very large expression with many \mv|%pi|'s which can be simplified easily by
  the ``rat tribe'', but \textbf{expand} takes a very long time and returns a mess.
\begin{myVerbatim}
(%i16) e :
   ((2/%pi-1)*(((%pi/2-1)/(%pi-1)-1)*((6*(2/%pi-2*(4-%pi)/%pi)/%pi-(6*(2*(4-%pi)
     /%pi-(%pi-2)/(%pi/2-1))/(%pi-1)-6*((%pi-2)/(%pi/2-1)-2)*(%pi/2-1)
     /(%pi*(%pi-1)))/(2*((2/%pi-1)*(%pi/2-1)/(2*(%pi-1)) +2)))     
      /(((%pi/2-1)/(%pi-1)-1)/(2*((2/%pi-1)*(%pi/2-1)/(2*(%pi-1))+2))+2)
     -(6*(2*(2*%pi-5)/%pi-2/%pi)/%pi-(6*(2/%pi-2*(4-%pi)/%pi)/%pi
     -(6*(2*(4-%pi)/%pi-(%pi-2)/(%pi/2-1))/(%pi-1)-6*((%pi-2)/(%pi/2-1)-2)
     *(%pi/2-1)/(%pi*(%pi-1)))/(2*((2/%pi-1)*(%pi/2-1)/(2*(%pi-1))+2)))
     /(2*(((%pi/2-1)/(%pi-1)-1)/(2*((2/%pi-1)*(%pi/2-1)/(2*(%pi-1))+2))+2)))
     /(2*(2-1/(4*(((%pi/2-1)/(%pi-1)-1)/(2*((2/%pi-1)*(%pi/2-1)/(2*(%pi-1))+2))
     +2)))*(((%pi/2-1)/(%pi-1)-1)/(2*((2/%pi-1)*(%pi/2-1)/(2*(%pi-1))
     +2))+2)))/((2/%pi-1)*(%pi/2-1)/(2*(%pi-1))+2)+(6*(2*(4-%pi)/%pi-(%pi-2)
     /(%pi/2-1))/(%pi-1)-6*((%pi-2)/(%pi/2-1)-2)*(%pi/2-1)/(%pi*(%pi-1)))
     /((2/%pi-1)*(%pi/2-1)/(2*(%pi-1))+2))/2+6*((%pi-2)/(%pi/2-1)-2)/%pi)
     *(x^3-x)/6  + 2*x$
(%i17) ratsimp (e);
              2                   3           3          2
      (128 %pi  - 520 %pi + 528) x  + (194 %pi  - 248 %pi  + 400 %pi - 528) x
(%o17) -----------------------------------------------------------------------
                                  3         2
                            97 %pi  - 60 %pi  - 60 %pi
(%i18) rat (e);
(%o18)/R/ 
                2                   3           3          2
        (128 %pi  - 520 %pi + 528) x  + (194 %pi  - 248 %pi  + 400 %pi - 528) x
        -----------------------------------------------------------------------
                                    3         2
                              97 %pi  - 60 %pi  - 60 %pi
\end{myVerbatim}
\newpage
\begin{myVerbatim}
(%i19) ratexpand (e);
                     3                           3
                528 x                   128 %pi x
(%o19) -------------------------- + ---------------------
            3         2                  2
      97 %pi  - 60 %pi  - 60 %pi   97 %pi  - 60 %pi - 60
               3                                                    2
          520 x                      528 x                   194 %pi  x
 - --------------------- - -------------------------- + ---------------------
         2                       3         2                  2
   97 %pi  - 60 %pi - 60   97 %pi  - 60 %pi  - 60 %pi   97 %pi  - 60 %pi - 60
         248 %pi x                 400 x
 - --------------------- + ---------------------
         2                       2
   97 %pi  - 60 %pi - 60   97 %pi  - 60 %pi - 60
\end{myVerbatim}
There is a more general form \textbf{expand (expr, p, n)} available to control which parts 
  of fractions are expanded and how (see the Manual index entry for \textbf{expand}).
There is also a more general form for \mv|rat (expr, x_1, ..., x_n)|.
\subsubsection*{gfactor}
The ``g'' in \textbf{gfactor} comes from factorization over the Gaussian
  integers.
\begin{myVerbatim}
(%i20) gfactor ( x^2 + 1 );
(%o20)                          (x - %i) (x + %i)
\end{myVerbatim}
\subsubsection*{partfrac}
We have earlier emphasized the usefulness of \textbf{ratsimp}.
A Maxima function which in some cases can be considered the ``opposite'' of
  \textbf{ratsimp} in its results is \textbf{partfrac}, which has the
  syntax \mv|partfrac ( expr, var )|, and which expands \mv|expr| in partial
  fractions with respect to \mv|var|.
Here is the example presented in the Manual.  
\begin{myVerbatim}
(%i21) e : 1/(1+x)^2 - 2/(1+x) + 2/(2+x);
                             2       2        1
(%o21)                      ----- - ----- + --------
                           x + 2   x + 1          2
                                           (x + 1)
(%i22) e, ratsimp;
                                        x
(%o22)                        - -------------------
                                3      2
                               x  + 4 x  + 5 x + 2
(%i23) partfrac (%, x);
                             2       2        1
(%o23)                      ----- - ----- + --------
                           x + 2   x + 1          2
                                           (x + 1)
\end{myVerbatim}
\newpage
\subsubsection{Examples of \textbf{integrate}, \textbf{assume}, \textbf{facts}, and \textbf{forget}}
An important symbolic tool is indefinite integration, much of which is performed
  algorithmically.
The original integration package, written by Joel Moses, incorporates the non-algebraic
  case of the Risch integration algorithm, and the package is called \textbf{sin.lisp}
  ( ``sin'' is a mnemonic from ``Symbolic INtegrator'' ).\\
  
\noindent Consider the indefinite integral of a rational expression:
\begin{myVerbatim}
(%i1) e : x/(x^3 + 1);
                                      x
(%o1)                               ------
                                     3
                                    x  + 1
(%i2) ie : integrate (e, x);
                                        2 x - 1
                      2            atan(-------)
                 log(x  - x + 1)        sqrt(3)    log(x + 1)
(%o2)            --------------- + ------------- - ----------
                        6             sqrt(3)          3
\end{myVerbatim}  
Indefinite integration can usually be checked by differentiating the
  result:
\begin{myVerbatim}
(%i3) diff (ie, x);
                        2               2 x - 1           1
(%o3)           ------------------ + -------------- - ---------
                            2            2            3 (x + 1)
                   (2 x - 1)         6 (x  - x + 1)
                3 (---------- + 1)
                       3
\end{myVerbatim}
This answer, as it stands, is not identical to the starting integrand.
The reason is that Maxima merely differentiates term by term.
Maxima automatically applies only those simplification rules which are
  considered ``obvious'' and ``always desirable''.
A particular simplification rule might make one expression smaller
  but might make another expression larger.
In such cases, Maxima leaves the decision to the user.\\

\noindent In this example, the starting integrand can be recovered by
  "\tcb{rational simplification}" using the \textbf{ratsimp} function.
\begin{myVerbatim}
(%i4) %, ratsimp;
                                      x
(%o4)                               ------
                                     3
                                    x  + 1
\end{myVerbatim}
We met \textbf{ratsimp} and \textbf{fullratsimp} (Sec. \ref{rat}) in our brief 
  discussion of the buttons and menu system of \textbf{wxMaxima}, where the
  \textbf{Simplify} button made use of \textbf{ratsimp}, a reflection of 
  the frequent use you will make of this function.
You can think of the word \textbf{ratsimp} being a mnenomic using the
  capitalised letters in RATional SIMPlification.\\
  
\noindent If you try to use \textbf{ratdiff} on the above indefinite integration
  result, you get an obscure error message:
\begin{myVerbatim}
(%i5) ratdiff (ie, x);
`ratdiff' variable is embedded in kernel
 -- an error.  To debug this try debugmode(true);
\end{myVerbatim}
The error message is returned because \textbf{ratdiff (expr, x)} can only be used for expressions
  which are either a polynomial in \mv|x| or a ratio of polynomials in \mv|x|.  
\newpage
\noindent \textbf{integrate} consults the \textbf{assume} database when making algebraic
  decisions.
The content of that database is returned by \mv|facts();|.
You can sometimes avoid questions from \textbf{integrate} if you first
  provide some assumptions with \textbf{assume}.
Here is an example of a definite integral in which we start with no 
  assumptions, and need to answer a  question from \textbf{integrate}.
\begin{myVerbatim}
(%i1) facts();
(%o1)                                 []
(%i2) integrate(x*exp(-a*x)*cos(w*x),x,0,inf);
Is  a  positive, negative, or zero?

p;
                                      2    2
                                     w  - a
(%o2)                         - -----------------
                                 4      2  2    4
                                w  + 2 a  w  + a
\end{myVerbatim}
We can now tell Maxima that the parameter \mv|a| should be treated as a positive number.
We thereby avoid the question.
\begin{myVerbatim}
(%i3) ( assume(a > 0), facts() );
(%o3)                               [a > 0]
(%i4) integrate(x*exp(-a*x)*cos(w*x),x,0,inf);
                                      2    2
                                     w  - a
(%o4)                         - -----------------
                                 4      2  2    4
                                w  + 2 a  w  + a
\end{myVerbatim}
You can now tell Maxima to \textbf{forget} that assumption.
\begin{myVerbatim}
(%i5) (forget(a > 0), facts() );
(%o5)                                 []
\end{myVerbatim}
\subsubsection{Numerical Integration and Evaluation: \textbf{float, bfloat,} and  \textbf{quad\_qags}}  
Let's try a harder integral, this time a definite integral which is expressed by Maxima in terms of
  a ``special function''.
\begin{myVerbatim}
(%i1) is : integrate( exp(x^3),x,1,2 );
                                         1                          1
      (sqrt(3) %i - 1) (gamma_incomplete(-, - 8) - gamma_incomplete(-, - 1))
                                         3                          3
(%o1) ----------------------------------------------------------------------
                                        6
(%i2) float(is);
(%o2) 0.16666666666667 (- 719.2849028287006 %i
 - 1.0 (0.66608190774162 - 3.486369946257051 %i) - 412.6003937374765)
 (1.732050807568877 %i - 1.0)
(%i3) expand(%);
(%o3)                          275.5109837634787
(%i4) ival:%;
(%o4)                          275.5109837634787
\end{myVerbatim}
You can look up the \mvs|gamma_incomplete(a,z)| definition in the Maxima manual, where 
  you will find a reference to the incomplete upper gamma function \textbf{A\&S 6.5.2},
  which refers to Eq. 6.5.2 in \textbf{Handbook of Mathematical Functions}, Edited by
  Milton Abramowitz and Irene A. Stegun, Dover Publications, N.Y., 1965.\\
  
\noindent We know that since we are integrating along the real x axis from 1 to 2,
  our integrand in \mv|(%i1)| is real, so that the numerical value of this integral
  must be a real number.
In using \textbf{float}, we see that the real answer will be obtained by cancellation
  of imaginary pieces, so we may have some roundoff errors.\\

\noindent Let's check the accuracy of using \textbf{float} here by comparing that
  answer to the answer returned by the use of \textbf{bfloat} together with
  \textbf{fpprec} set to 20.
\begin{myVerbatim}
(%i5) bfloat(is),fpprec:20;
(%o5) 1.6666666666666666667b-1 (1.7320508075688772935b0 %i - 1.0b0)
 (- 1.0b0 (6.6608190774161802181b-1 - 3.4863699462570511861b0 %i)
 - 7.1928490282826659929b2 %i - 4.1260039373722578387b2)
(%i6) expand(%),fpprec:20;
(%o6)       5.7824115865893569814b-19 %i + 2.7551098376331160126b2
(%i7) tval20: realpart(%);
(%o7)                       2.7551098376331160126b2
(%i8) abs(ival - tval20);
(%o8)                        1.670912297413452b-10
\end{myVerbatim}
We see that the numerical answer found using \textbf{float} only has 
  twelve digit accuracy.
Let's look at the numerical values of the \mvs|gamma_incomplete| functions
  involved here as returned by \textbf{float}:
\begin{myVerbatim}
(%i9) float(gamma_incomplete(1/3,-8));
(%o9)             - 719.2849028287006 %i - 412.6003937374765
(%i10) float(gamma_incomplete(1/3,-1));
(%o10)              0.66608190774162 - 3.486369946257051 %i
\end{myVerbatim}
Finally, let's use a pure numerical integration routine ,
     \mvs|quad_qags|, which returns
  a list consisting of\\
  \mvs|[answer, est-error, num-integrand-eval, error-code]|.
\begin{myVerbatim}
(%i11) quad_qags(exp(x^3),x,1,2);
(%o11)         [275.5109837633116, 3.2305615936931465E-7, 21, 0]
(%i12) abs(first(%) - tval20);
(%o12)                       2.842170943040401b-14
\end{myVerbatim}
and we see that the use of \mvs|quad_qags| for a numerical value was more accurate
  than the comination of \textbf{integrate} and \textbf{float}.
\subsubsection{Taylor and Laurent Series Expansions with \textbf{taylor}}  
When computing exact symbolic answers is intractable, one can often resort
  to series approximations to get approximate symbolic results.
Maxima has an excellent Taylor series program.
As an example, we can get truncated series representations for \textbf{sin(x)}
  and \textbf{cos(x)} as follows.
According to the Maxima manual
\begin{quote}
\textbf{taylor (expr, x, a, n)} expands the expression \textbf{expr} in a 
   truncated Taylor or Laurent series in the variable \textbf{x} around the
   point $\mathbf{x = a}$, containing terms through $\mathbf{(x - a)^{n}}$. 
\end{quote}
Here we expand around the point $\mathbf{x = 0}$.
\begin{myVerbatim}
(%i1) taylor( sin(x),x,0,5 );
                                  3    5
                                 x    x
(%o1)/T/                     x - -- + --- + . . .
                                 6    120
(%i2) taylor( cos(x),x,0,5);
                                   2    4
                                  x    x
(%o2)/T/                      1 - -- + -- + . . .
                                  2    24
\end{myVerbatim}
A truncated series in denoted by the "\mv|/T/|" symbol next to the line label and
  also by the trailing dots. 
Maxima retains certain information about such expansions, such as the ``quality'' of the
  approximation, so when several series expansions are combined, no terms are computed
  beyond the degree of the approximation.
In our example, our starting expansions are only good through terms of order $\mathbf{x^{5}}$,
  so if we multiply the expansions, any terms smaller (in order of magnitude) than
  $\mathbf{x^{5}}$ are neglected.\\
  
\noindent Here we use \mv|%th(n)|, which refers to the n'th to last line.
\begin{myVerbatim}
(%i3) % * %th(2);
                                   3      5
                                2 x    2 x
(%o3)/T/                    x - ---- + ---- + . . .
                                 3      15
\end{myVerbatim}
The name ``taylor'' is only an artifact of history, since the Maxima function \textbf{taylor}
  can handle expansions of functions with poles and branch points and automatically
  generates Laurent series when appropriate as shown here:
\begin{myVerbatim}
(%i4) taylor( 1/ (cos(x) - sec(x))^3, x,0,5 );
                                                2          4
             1     1       11      347    6767 x    15377 x
(%o4)/T/   - -- + ---- + ------ - ----- - ------- - -------- + . . .
              6      4        2   15120   604800    7983360
             x    2 x    120 x
\end{myVerbatim}
\newpage
\subsubsection{Solving Equations: \textbf{solve, allroots, realroots,} and \textbf{find\_root}}
The \textbf{solve} function is described in the Maxima manual
\begin{quote}
Function: \textbf{solve (expr, x)}\\ 
Function: \textbf{solve (expr)}\\ 
Function: \textbf{solve} \mvs|([eqn_1, ..., eqn_n], [x_1, ..., x_n])|\\ 
Solves the algebraic equation \textbf{expr} for the variable \textbf{x} 
   and returns a list of solution equations for \textbf{x}. If \textbf{expr}
   is not an equation, the equation $\mathbf{expr = 0}$ is assumed in its place.
   \textbf{x} may be a function (e.g. f(x)), or other non-atomic 
   expression except a sum or product.
   \textbf{x} may be omitted if \textbf{expr} contains only one variable.
   \textbf{expr} may be a rational expression, and may contain trigonometric
   functions, exponentials, etc. 
\end{quote}
\subsubsection*{Example 1}
\noindent Here we look for solutions of the equation $\mathbf{x^{6} = 1}$ or $\mathbf{x^{6} - 1 = 0}$.
Once we have found candidate solutions, we check them one at a time.
\begin{myVerbatim}
(%i1) eqn : x^6 - 1 = 0$
(%i2) solns : solve(eqn);
           sqrt(3) %i + 1      sqrt(3) %i - 1                 sqrt(3) %i + 1
(%o2) [x = --------------, x = --------------, x = - 1, x = - --------------, 
                 2                   2                              2
                                                         sqrt(3) %i - 1
                                                   x = - --------------, x = 1]
                                                               2
(%i3) for i thru length(solns) do
        disp ( ev( eqn, solns[i], ratsimp ) )$
                                     0 = 0
                                     0 = 0
                                     0 = 0
                                     0 = 0
                                     0 = 0
                                     0 = 0
\end{myVerbatim}
One often uses \textbf{solve} in the hope of finding useful symbolic expressions
  for roots of an equation.
It may still be illuminating to look at the numerical values of the roots
  returned by the polynomial root finder \textbf{allroots}, which we do for the
  above example:
\begin{myVerbatim}
(%i4) fpprintprec:8$
(%i5) nsolns : allroots(x^6 - 1);
(%o5) [x = 0.866025 %i + 0.5, x = 0.5 - 0.866025 %i, x = 0.866025 %i - 0.5, 
                                   x = - 0.866025 %i - 0.5, x = 1.0, x = - 1.0]
(%i6) for i thru length(nsolns) do
        disp ( ev( x^6 - 1, nsolns[i],expand ))$
                      - 4.4408921E-16 %i - 5.55111512E-16
                       4.4408921E-16 %i - 5.55111512E-16
                       4.4408921E-16 %i - 5.55111512E-16
                      - 4.4408921E-16 %i - 5.55111512E-16
                                      0.0
                                      0.0
\end{myVerbatim}
The numerical roots found by \textbf{allroots} satisfy the starting equation
  to within floating point errors.
\newpage
\subsubsection*{Example 2}  
You can use \textbf{solve} to look for solutions to a set of equations.
Here we find four sets of solutions satisfying two simultaneous equations,
  and we check the solutions.
\begin{myVerbatim}
(%i1) fpprintprec:8$
(%i2) eqns : [4*x^2 - y^2 - 12 = 0, x*y - x - 2 = 0]$
(%i3) solns : solve(eqns,[x,y]);
(%o3) [[x = 2, y = 2], [x = 0.520259 %i - 0.133124, 
y = 0.0767838 - 3.6080032 %i], [x = - 0.520259 %i - 0.133124, 
y = 3.6080032 %i + 0.0767838], [x = - 1.7337518, y = - 0.153568]]
(%i4) eqns,solns[1],ratsimp;
(%o4)                           [0 = 0, 0 = 0]
(%i5) for i:2 thru 4 do
        disp ( ev (eqns, solns[i],expand) )$
        [1.77635684E-15 - 6.66133815E-16 %i = 0, - 2.22044605E-16 = 0]
        [6.66133815E-16 %i + 1.77635684E-15 = 0, - 2.22044605E-16 = 0]
                  [- 1.13954405E-6 = 0, - 9.38499825E-8 = 0]
(%i6) solns[4];
(%o6)                  [x = - 1.7337518, y = - 0.153568]
\end{myVerbatim}
We only use \textbf{ratsimp} on the first (integral) solution, which is an exact solution.
For the numerical solutions, we use instead \textbf{expand}.
The second and third (numerical) solutions have an accuracy of about 15 digits.
The fourth (numerical) solution has about five digit accuracy.
\subsubsection*{Example 3: \textbf{realroots}}
A Maxima user asked (on the Mailing List) how to find the value of \mv|r| implied
  by the equation
\begin{myVerbatim2}
  275.0 * exp(-r) + 275.0 * exp(-2*r) + 275.0 * exp(-3*r) 
    + 275.0 * exp(-4*r) + 5275.0 * exp(-5*r) = 4750.0
\end{myVerbatim2}
 Straightforward use of \textbf{solve} does not return an explicit solution.
Here is an approach suggested by Maxima developer Stavros Macrakis.\\

\noindent First use \textbf{rat} to convert approximate numbers (275.0) to exact (275),
  replace \mv|exp(r)| by \mv|z|, and then try \textbf{solve}.
\begin{myVerbatim}
(%i1) fpprintprec:8$
(%i2) eqn : 275.0 * exp(-r) + 275.0 * exp(-2*r) + 275.0 * exp(-3*r) 
    + 275.0 * exp(-4*r) + 5275.0 * exp(-5*r) - 4750.0 = 0$
(%i3) rat(eqn),ratprint:false;
(%o3)/R/ 
                r 5          r 4          r 3          r 2         r
        4750 (%e )  - 275 (%e )  - 275 (%e )  - 275 (%e )  - 275 %e  - 5275
      - ------------------------------------------------------------------- = 0
                                         r 5
                                      (%e )
(%i4) zeqn : ratsubst(z,exp(r),%);
                    5        4        3        2
              4750 z  - 275 z  - 275 z  - 275 z  - 275 z - 5275
(%o4)       - ------------------------------------------------- = 0
                                      5
                                     z
(%i5) soln : solve(zeqn);
                         5       4       3       2
(%o5)          [0 = 190 z  - 11 z  - 11 z  - 11 z  - 11 z - 211]
\end{myVerbatim}
We see that \textbf{solve} returned a list containing one element: a 
  simplified form of our polynomial (in \mv|z|)
  equation, but was unable to return an algebraic solution.
We must then consider an approximate solution, and suppose we are only
  interested in real roots \mv|z|, in which case we can use \textbf{realroots}.
\begin{myVerbatim}
(%i6) rr : realroots(soln[1]);
                                     35805935
(%o6)                           [z = --------]
                                     33554432
(%i7) zeqn, rr, ratsimp;
                     1011365300358912057317883590855138
(%o7)             - ------------------------------------- = 0
                    2354155174380926220896751357249338375
(%i8) %, numer;
(%o8)                         - 4.29608596E-4 = 0
\end{myVerbatim}
We see that the returned real root is not very accurate, so we override the default
  value of \textbf{rootsepsilon} (which is one part in $\mathbf{10^{7}}$)
\begin{myVerbatim}
(%i9) rr : realroots( soln[1],1.0e-16);
                                 38446329182573765
(%o9)                       [z = -----------------]
                                 36028797018963968
(%i10) zeqn, rr, numer;
(%o10)                        - 6.5731657E-13 = 0
\end{myVerbatim}
If $\mathbf{z = e^{r}}$, then $\mathbf{\boldsymbol{\ln}(z) = \boldsymbol{\ln}(e^{r}) = r}$.
We then get an accurate numerical answer for \mv|r| via
\begin{myVerbatim}
(%i11) rval : log(rhs(rr[1])),numer;
(%o11)                             0.0649447
(%i12) eqn, r = rval;
(%o12)                       - 1.70530257E-12 = 0
\end{myVerbatim}
\subsubsection*{Example 4: Using \textbf{find\_root}}
Let's go thru the first manual example for \textbf{find\_root}, which has the
syntax
\begin{myVerbatim2}
    find_root ( expr, var, a, b )
\end{myVerbatim2}
  which assumes \textbf{expr} is a function of \textbf{var}, and Maxima is to look for
    a numerical root of the equation \mv|expr = 0| in the range \mv| a <= var <= b|.\\
	
\noindent Here we find the value of \mv|x| such that \mv|sin(x) - x/2 = 0|.
  The function \textbf{solve} cannot cope with this type of nonlinear problem.
\begin{myVerbatim}
(%i1) e : sin(x) - x/2;
                                           x
(%o1)                             sin(x) - -
                                           2
(%i2) solve(e);
(%o2)                           [x = 2 sin(x)]
\end{myVerbatim}
\newpage
\noindent Let's use \textbf{plot2d} for a quick look at this expression for positive \mv|x|.
\begin{myVerbatim}
(%i3) plot2d(e,[x,0,2*%pi],
          [style,[lines,5]],[ylabel," "],
           [xlabel," sin(x) - x/2 " ],
           [gnuplot_preamble,"set zeroaxis lw 2"])$
\end{myVerbatim}
%\newpage
 which produces the plot:
\smallskip
\begin{figure} [h]  
   \centerline{\includegraphics[scale=.6]{ch1p18.eps} }
	\caption{ $\sin(x) - x/2$ }
\end{figure}
%(%i8) plot2d([e,[discrete,[[0,0],[2*%pi,0]]]],[x,0,2*%pi],
%          [style,[lines,5,1],[lines,3,4]],[ylabel," "],
%           [xlabel," sin(x) - x/2 " ],
%           [psfile,"ch1p18.eps"],
%           [gnuplot_preamble,"set nokey"])$

\noindent We see from the plot that there is a root in the range \mv| 0 < x < %pi |.
We can now ask \textbf{find\_root} to locate this root numerically.
\begin{myVerbatim}
(%i4) xr : find_root(e,x,0.1,%pi);
(%o4)                          1.895494267033981
(%i5) e, x = xr;
(%o5)                                 0.0
\end{myVerbatim}
\subsubsection*{Example 5: Helping \textbf{solve} Find the Exact Roots}
Let's find the values of \textbf{x} which satisfy the equation 
\begin{equation}
\mathbf{a\,( 1 - \boldsymbol{\sin}(x) ) = 2 \, b\, \boldsymbol{\cos}(x) }
\end{equation}
As this equation stands, \textbf{solve} is not able to make progress.
\begin{myVerbatim}
(%i1) eqn : a*(1-sin(x)) - 2*b*cos(x) = 0$
(%i2) solve ( eqn, x );
                                      2 b cos(x) - a
(%o2)                     [sin(x) = - --------------]
                                            a
\end{myVerbatim}
\newpage
\noindent There are several ways one can help out \textbf{solve} to get the solutions we want.\\

\noindent \textbf{Method 1} is to convert the trig functions to their complex exponential forms
  using \textbf{exponentialize}.
\begin{myVerbatim}
(%i3) eqn, exponentialize;
                   %i x     - %i x
             %i (%e     - %e      )             %i x     - %i x
(%o3)     a (---------------------- + 1) - b (%e     + %e      ) = 0
                       2
(%i4) solns : solve ( %, x );
                      2 %i b         a
(%o4) [x = - %i log(---------- + ----------), 
                    2 b - %i a   2 b - %i a
                                                          a          2 %i b
                                         x = - %i log(---------- - ----------)]
                                                      2 b - %i a   2 b - %i a
(%i5) solns : solns, rectform, ratsimp;
                                          2    2
                      %pi              4 b  - a     4 a b
(%o5)            [x = ---, x = - atan2(---------, ---------)]
                       2                  2    2     2    2
                                       4 b  + a   4 b  + a
(%i6) eqn, solns[1], ratsimp;
(%o6)                                0 = 0
(%i7) eqn, solns[2], ratsimp;
(%o7)                                0 = 0
(%i8) x2 : atan2(a^2 - 4*b^2, 4*a*b)$
(%i9) eqn, x = x2, ratsimp;
                            4      2  2    4         2    3
                 a sqrt(16 b  + 8 a  b  + a ) - 4 a b  - a
(%o9)          - ------------------------------------------ = 0
                                    2    2
                                 4 b  + a
(%i10) scanmap( 'factor, % );
(%o10)                               0 = 0
\end{myVerbatim}
\textbf{Method 2} is to enlarge the system of equations to include the trig identity
\begin{equation}
\mathbf{ \boldsymbol{\cos}^{2} x + \boldsymbol{\sin}^{2} x = 1 }
\end{equation}  
  which one would hope would be used creatively by \textbf{solve} to find the
  solutions of our equation.\\
  
Let's replace \mv|cos(x)| by the symbol \mv|c| and
  likewise replace \mv|sin(x)| by the symbol \mv|s|, and start with a
  pair of equations.
\begin{myVerbatim}
(%i1) eqns : [a*(1-s) - 2*b*c = 0, c^2 + s^2 - 1 = 0]$
(%i2) solns : solve ( eqns, [s,c] );
                                        2    2
                                     4 b  - a         4 a b
(%o2)        [[s = 1, c = 0], [s = - ---------, c = ---------]]
                                        2    2         2    2
                                     4 b  + a       4 b  + a
(%i3) eqns, solns[1], ratsimp;
(%o3)                          [0 = 0, 0 = 0]
(%i4) x1soln : solve ( sin(x) = 1, x );
solve: using arc-trig functions to get a solution.
Some solutions will be lost.
                                       %pi
(%o4)                             [x = ---]
                                        2
(%i5) eqn, x1soln, ratsimp;
(%o5)                               0 = 0
\end{myVerbatim}
\newpage
\begin{myVerbatim}
(%i6) eqns, solns[2], ratsimp;
(%o6)                          [0 = 0, 0 = 0]
(%i7) tan(x) = s/c, solns[2], ratsimp;
(%o7) tan(x) = -(4*b^2-a^2)/(4*a*b)
(%i8) x2soln : solve (%, x), ratsimp;
solve: using arc-trig functions to get a solution.
Some solutions will be lost.
                                          2    2
                                       4 b  - a
(%o8)                      [x = - atan(---------)]
                                         4 a b
\end{myVerbatim}
Bearing in mind that $\mathbf{\boldsymbol{\arctan}(-x)  = - \boldsymbol{\arctan}(x) }$,
  we get the same set of solutions using method 2.
\subsubsection{Non-Rational Simplification: \textbf{radcan, logcontract, rootscontract,} and \textbf{radexpand}}
The wxMaxima interface has the  \textbf{Simplify(r)} button allows uses
 the \textbf{radcan} function on a selection.
The ``r'' stands for \textbf{radcan} although you could just as well
  think "radical", since it is equivalent to
  the wxMaxima menu choice \textbf{Simplify, Simplify Radicals}.
You can also think of the work \textbf{radcan} as a mnenomic constructed
  from the capitalised letters in ``RADical CANcellation'', but this
  does not hint at the full power of this function.\\
  
\noindent The manual asserts that \textbf{radcan} is useful in simplifying
  expressions containing logs, exponentials, and radicals.
Here is an expression containing exponentials which \textbf{radcan} simplifies.
\begin{myVerbatim}
(%i1) expr : (exp(x)-1)/(exp(x/2)+1);
                                      x
                                    %e  - 1
(%o1)                              ---------
                                     x/2
                                   %e    + 1
(%i2) expr, ratsimp;
                                      x
                                    %e  - 1
(%o2)                              ---------
                                     x/2
                                   %e    + 1
(%i3) expr, radcan;
                                     x/2
(%o3)                              %e    - 1
\end{myVerbatim}
We see that \textbf{radcan} is able to simplify this expression
  containing exponentials, whereas \textbf{ratsimp} cannot.\\

\noindent Here is some Xmaxima work showing the differences between
  the use of \textbf{ratsimp}, \textbf{radcan}, and \textbf{logcontract}
  on an expression which is a sum of logs.
\begin{myVerbatim}
(%i4) logexpr : log ( (x+2)*(x-2) ) + log(x);
(%o4)                    log((x - 2) (x + 2)) + log(x)
(%i5) logexpr, ratsimp;
                                  2
(%o5)                        log(x  - 4) + log(x)
(%i6) logexpr, fullratsimp;
                                  2
(%o6)                        log(x  - 4) + log(x)
(%i7) logexpr, radcan;
(%o7)                  log(x + 2) + log(x) + log(x - 2)
(%i8) %, logcontract;
                                      3
(%o8)                            log(x  - 4 x)
\end{myVerbatim}
\newpage
\begin{myVerbatim}
(%i9) logexpr, logcontract;
                                      3
(%o9)                            log(x  - 4 x)
\end{myVerbatim}
We see that use of \textbf{logcontract} provided the same quality
  of simplification as the two step route \textbf{radcan} followed by
  \textbf{logcontract}.\\
  
\noindent Here we expolore the use of \textbf{radcan} and
  \textbf{rootscontract} to simplify square roots:
\begin{myVerbatim}
(%i10) sqrt(2)*sqrt(3), radcan;
(%o10)                          sqrt(2) sqrt(3)
(%i11) sqrt(2)*sqrt(3), rootscontract;
(%o11)                              sqrt(6)
(%i12) sqrt(6)*sqrt(3), radcan;
(%o12)                             3 sqrt(2)
(%i13) sqrt(6)*sqrt(3), rootscontract;
(%o13)                             3 sqrt(2)
(%i14) sqrt(6)/sqrt(3), radcan;
(%o14)                              sqrt(2)
(%i15) sqrt(6)/sqrt(3), rootscontract;
(%o15)                              sqrt(2)
\end{myVerbatim}  
\subsubsection*{radexpand}
The Maxima manual describes the option variable \textbf{radexpand}
  which can have the values \textbf{true} (the default), \textbf{all}, or \textbf{false}.
The setting of \textbf{radexpand} controls the automatic simplifications of radicals.\\

\noindent In the default case (\textbf{radexpand} set to \textbf{true}),
 \mvs|sqrt(x^2)| simplifies to \mvs|abs(x)|.
This is because, by default, all symbols, in the absence of special 
  declarations, are considered to represent real numbers.
You can declare that all symbols should (unless declared otherwise) be considered
  to represent complex numbers using the \textbf{domain} flag, which
  has the default value \textbf{real}.
\begin{myVerbatim}
(%i16) sqrt(x^2);
(%o16)                              abs(x)
(%i17) domain:complex$
(%i18) sqrt(x^2);
                                         2
(%o18)                             sqrt(x )
(%i19) domain:real$
(%i20) sqrt(x^2);
(%o20)                              abs(x)
\end{myVerbatim} 
Next we show the simplifications which occur to the
  expression \mvs|sqrt(16*x^2)| for the three possible values
  of \textbf{radexpand}.
\begin{myVerbatim}
(%i21) radexpand;
(%o21)                               true
(%i22) sqrt(16*x^2);
(%o22)                             4 abs(x)
(%i23) radexpand:all$
(%i24) sqrt(16*x^2);
(%o24)                                4 x
(%i25) radexpand:false$
(%i26) sqrt(16*x^2);
                                           2
(%o26)                            sqrt(16 x )
\end{myVerbatim}
\subsubsection{Trigonometric Simplification: \textbf{trigsimp, trigexpand, trigreduce,} and \textbf{trigrat}}
Here are simple examples of the important trig
  simplification functions \textbf{trigsimp}, \textbf{trigexpand},
   \textbf{trigreduce}, and \textbf{trigrat}.\\
   
\noindent \textbf{trigsimp} converts trig functions to sines and cosines and also
  attempts to use the identiy $\mathbf{\boldsymbol{\cos}(x)^{2} + \boldsymbol{\sin}(x)^{2} = 1}$.
\begin{myVerbatim}
(%i1) trigsimp(tan(x));
                                    sin(x)
(%o1)                               ------
                                    cos(x)
(%i2) sin(x+y), trigexpand;
(%o2)                    cos(x) sin(y) + sin(x) cos(y)
(%i3) x+3*cos(x)^2 - sin(x)^2, trigreduce;
                      cos(2 x)      cos(2 x)   1        1
(%o3)                 -------- + 3 (-------- + -) + x - -
                         2             2       2        2
(%i4) trigrat ( sin(3*a)/sin(a+%pi/3) );
(%o4)                   sqrt(3) sin(2 a) + cos(2 a) - 1
\end{myVerbatim}
The manual entry for \textbf{trigsimp} advises that \textbf{trigreduce}, \textbf{ratsimp},
  and \textbf{radcan} may be able to further simplify the result.
The author has the following function defined in \textbf{mbe1util.mac}, which
   is loaded in during startup.
\begin{myVerbatim2}
    rtt(e) := radcan ( trigrat ( trigsimp (e)))$
\end{myVerbatim2}
This function can be equivalent to \textbf{trigreduce} if only sines and cosines
  are present, but otherwise may return different forms, as we show here,
  comparing the opposide effects of \textbf{trigexpand} and \textbf{trigreduce}.
Note that \textbf{trigexpand} expands trig functions of sums of angles and/or
  trig functions of multiple angles (like \mvs|2*x| at the ``top level'', and
  the user should read the manual description to see options available.
\begin{myVerbatim}
(%i5) e: sin(x+y), trigexpand;
(%o5)                    cos(x) sin(y) + sin(x) cos(y)
(%i6) e, trigreduce;
(%o6)                             sin(y + x)
(%i7) rtt (e);
(%o7)                             sin(y + x)
(%i8) e :  tan(x+y), trigexpand;
                                tan(y) + tan(x)
(%o8)                          -----------------
                               1 - tan(x) tan(y)
(%i9) e, trigreduce;
                           tan(y)              tan(x)
(%o9)               - ----------------- - -----------------
                      tan(x) tan(y) - 1   tan(x) tan(y) - 1
(%i10) %, ratsimp;
                                 tan(y) + tan(x)
(%o10)                        - -----------------
                                tan(x) tan(y) - 1
(%i11) rtt (e);
                                  sin(y + x)
(%o11)                            ----------
                                  cos(y + x)
(%i12) trigsimp( tan(x+y) );
                                  sin(y + x)
(%o12)                            ----------
                                  cos(y + x)
(%i13) e :  cosh(x + y), trigexpand;
(%o13)                 sinh(x) sinh(y) + cosh(x) cosh(y)
(%i14) e, trigreduce;
(%o14)                            cosh(y + x)
\end{myVerbatim}
\newpage
\begin{myVerbatim}
(%i15) rtt (e);
                            - y - x    2 y + 2 x
                          %e        (%e          + 1)
(%o15)                    ---------------------------
                                       2
(%i16) expand(%);
                                y + x     - y - x
                              %e        %e
(%o16)                        ------- + ---------
                                 2          2
\end{myVerbatim}
In the following, note the difference when using \mv|trigexpand(expr)|
  compared with \mv|ev ( expr, trigexpand )|.
This difference is due to the fact that \textbf{trigexpand} has both the \textbf{evfun} and
  the \textbf{evflag} properties.
\begin{myVerbatim}
(%i1) e1 : trigexpand( tan(2*x + y) );
                               tan(y) + tan(2 x)
(%o1)                         -------------------
                              1 - tan(2 x) tan(y)
(%i2) e2 : tan (2*x + y), trigexpand;
                                       2 tan(x)
                             tan(y) + -----------
                                             2
                                      1 - tan (x)
(%o2)                        --------------------
                                 2 tan(x) tan(y)
                             1 - ---------------
                                          2
                                   1 - tan (x)
(%i3) rtt (e1);
                                 sin(y + 2 x)
(%o3)                            ------------
                                 cos(y + 2 x)
(%i4) rtt (e2);
                                 sin(y + 2 x)
(%o4)                            ------------
                                 cos(y + 2 x)
(%i5) trigsimp (e1);
                        cos(2 x) sin(y) + sin(2 x) cos(y)
(%o5)                 - ---------------------------------
                        sin(2 x) sin(y) - cos(2 x) cos(y)
(%i6) trigsimp (e2);
                       2
                 (2 cos (x) - 1) sin(y) + 2 cos(x) sin(x) cos(y)
(%o6)          - -----------------------------------------------
                                                    2
                 2 cos(x) sin(x) sin(y) + (1 - 2 cos (x)) cos(y)
\end{myVerbatim}
\newpage
\subsubsection{Complex Expressions: \textbf{rectform, demoivre, realpart, imagpart,} and \textbf{exponentialize}}
Sec. 6.2 of the Maxima manual has the introduction
\begin{quote}
A complex expression is specified in Maxima by adding the real 
  part of the expression to \mvs|%i| times the imaginary part.
Thus the roots of the equation \mvs|x^2 - 4*x + 13 = 0| 
  are \mvs|2 + 3*%i| and \mvs|2 - 3*%i|.
Note that simplification of products of complex expressions 
   can be effected by expanding the product.
Simplification of quotients, roots, and other functions of 
   complex expressions can usually be accomplished by using 
   the \textbf{realpart}, \textbf{imagpart}, \textbf{rectform}, 
   \textbf{polarform}, \textbf{abs}, \textbf{carg} functions. 
\end{quote}
Here is a brief look at some of these functions, starting with
  \textbf{exponentialize}.
\begin{myVerbatim}
(%i1) cform : [cos(x),sin(x),cosh(x)], exponentialize;
             %i x     - %i x          %i x     - %i x     x     - x
           %e     + %e          %i (%e     - %e      )  %e  + %e
(%o1)     [-----------------, - ----------------------, -----------]
                   2                      2                  2
(%i2) cform, demoivre;
                                            x     - x
                                          %e  + %e
(%o2)                    [cos(x), sin(x), -----------]
                                               2
(%i3) cform, rectform;
                                            x     - x
                                          %e  + %e
(%o3)                    [cos(x), sin(x), -----------]
                                               2
(%i4) realpart ( cform );
                                            x     - x
                                          %e  + %e
(%o4)                    [cos(x), sin(x), -----------]
                                               2
(%i5) imagpart ( cform );
(%o5)                              [0, 0, 0]
\end{myVerbatim}
\subsubsection{Are Two Expressions Numerically Equivalent?   \textbf{zeroequiv}}
The function \textbf{zeroequiv( expr, var )} uses a series of randomly chosen values of
  the variable \textbf{var} to test if \textbf{expr} is equivalent to zero.
Although there are cases where \textbf{dontknow} is returned (see the manual
  cautions), this function can be useful if you arrange that \textbf{expr}
  is the difference of two other expressions which you suspect are numerically
  equivalent (or not).\\
  
\noindent As a pedagogical example consider showing that  
\begin{equation}
\mathbf{\boldsymbol{\cos}^{2}(x-1) =  \frac{1}{2}\,(\boldsymbol{\sin(2) \,\sin(2\,x) + \cos(2)\,\cos(2x)} + 1)}
\end{equation}
  is true.
Although it is not difficult to show the equivalence using the standard Maxima
  tools (see Sec. 10.3.6 in Ch.10), we use here \textbf{zeroequiv}.
\begin{myVerbatim}
(%i1) e1:cos(x-1)^2;
                                     2
(%o1)                             cos (x - 1)
(%i2) e2 : (sin(2)*sin(2*x)+cos(2)*cos(2*x)+1)/2;
                     sin(2) sin(2 x) + cos(2) cos(2 x) + 1
(%o2)                -------------------------------------
                                       2
(%i3) zeroequiv ( e1-e2, x );
(%o3)                                true
\end{myVerbatim}  
An alternative (and complementary) approach is to plot both expressions \mv|e1| and \mv|e2| on
  the same plot.
\newpage
\subsection{User Defined Maxima Functions: \textbf{define}, \textbf{fundef}, \textbf{block}, and \textbf{local}}
A Maxima function can be defined using the \mv|:=| operator.
The left side of the function definition should be the name of the function
  followed by comma separated formal parameters enclosed in parentheses.
The right side of the Maxima function definition is the ``function body''.
When a Maxima function is called, the formal parameters are bound to the
  call arguments, any ``free'' variables in the function body take on the values that they have
  at the time of the function call, and the function body is evaluated.
You can define Maxima functions which are recursive to an ``arbitrary'' depth  
  (of course there are always practical limits due to memory, speed,...).
After a Maxima function is defined, its name is added to the Maxima list \textbf{functions}.\\

\noindent Problems may sometimes arise when passing to a Maxima function (as one
  of the calling arguments) an expression which contains a variable with the same
  name as a formal parameter used in the function body (name conflicts).


\subsubsection{A Function Which Takes a Derivative}
In this section we discuss a question sent in to the Maxima
  Mailing List (which we paraphrase).
\begin{myVerbatim2}
I want to make the derivative of a function a function itself.
Naively, I tried the following.
First define a Maxima function f(x) as a simple polynomial.
Next define a second Maxima function fp(x) ("f-prime")
  which will take the first derivative of f(x).
Then try to use fp(x) to get the derivative of f at x =1.

This did not work, as shown here:
------------------------------------------------
(%i1) f(x):= -2*x^3 + 3*x^2 + 12*x - 13$
(%i2) fp(x):= diff(f(x),x)$
(%i3) fp(1);
diff: second argument must be a variable; found 1
#0: fp(x=1)
 -- an error.  To debug this try debugmode(true);
---------------------------------------------
What can I do?
Thanks
\end{myVerbatim2}  
The error message returned by Maxima in response to input \mv|%i3| means that
  Maxima determined that the job requested was to return the result of the
  operation \mv|diff( f(1), 1 )|.
In other words, ``variable substitution'' occurred before differentiation, whereas, what
  is wanted is first differentiation, and then variable substitution.\\
  
\noindent Let's pick a simpler function, and compare three ways to try to get
  this job done.
\begin{myVerbatim}
(%i1) f(y) := y^3;
                                           3
(%o1)                             f(y) := y
(%i2) f(z);
                                       3
(%o2)                                 z
(%i3) f1(x) := diff ( f(x), x);
(%o3)                       f1(x) := diff(f(x), x)
(%i4) f2(x) := ''(diff ( f(x), x));
                                             2
(%o4)                            f2(x) := 3 x
(%i5) define ( f3(x), diff ( f(x), x) );
                                             2
(%o5)                            f3(x) := 3 x
\end{myVerbatim}
\newpage
\begin{myVerbatim}
(%i6) [f1(z), f2(z), f3(z) ];
                                  2     2     2
(%o6)                         [3 z , 3 z , 3 z ]
(%i7) [f1(1), f2(1), f3(1) ];
diff: second argument must be a variable; found 1
#0: f1(x=1)
 -- an error.  To debug this try debugmode(true);
(%i8) [f2(1), f3(1)];
(%o8)                               [3, 3]
\end{myVerbatim}
As long as we request the derivative of f(z), where z is treated as an undefined
  variable (ie., \textbf{z} has not been bound to a number, in which
  case \mv|numberp(z)| will return \mv|false|), all three definitions 
  give the correct result.
But when we request the derivative at a specific numerical point, only the
  last two methods give the correct result without errors, and the first
  ("delayed assignment operator method") does not work.\\
  
\noindent Note that the ``quote-quote method'' requires \tcbr{two single quotes}, and
  we need to surround the whole expression with parentheses: \mv|''(...)|.
This ``quote-quote'' method is available only in interactive use, and will not work
  inside another function, as you can verify.
The conventional wisdom denigrates the quote-quote method and advises 
  use of the \textbf{define} method.\\

\noindent The \textbf{define} function method can not only be used interactively,
  but also when one needs to define a function like this inside another
  Maxima function, as in
\begin{myVerbatim}
(%i9) dplay(g,a,b) := block ( [val1,val2 ],
        local (dg),
          define ( dg(y), diff (g(y), y)),
          val1 : g(a) + dg(a),
          val2 : g(b) + dg(b),
          display (val1,val2),
          val1 + val2)$
(%i10) g(x) := x;
(%o10)                             g(x) := x
(%i11) dplay(g,1,1);
                                   val1 = 2

                                   val2 = 2

(%o11)                                 4
(%i12) g(x) := x^2;
                                           2
(%o12)                            g(x) := x
(%i13) dplay(g,1,2);
                                   val1 = 3

                                   val2 = 8

(%o13)                                11
(%i14) dg(3);
(%o14)                               dg(3)
\end{myVerbatim}
You can find the manual discussion of the quote-quote operator near the top of
  the index list where misc. symbols are recorded.
There you find the comment ``Applied to the operator of an expression, 
  quote-quote changes the operator from a noun to a verb (if it is not already a verb).''\\
  
\noindent The use of \textbf{local} inside our Maxima function \textbf{dplay}
  prevents our definition of the Maxima function \textbf{dg} from leaking out 
  into  the top level context and keeps the knowledge of \textbf{dg} inside \textbf{dplay} 
   and any user defined Maxima functions which are called by \textbf{dplay}, and inside
   any user defined Maxima functions which are called by those functions, etc...   \\
   
\noindent Maxima developer Stavros Macrakis has posted a good example which 
  emphasizes the difference between the delayed assignment operator \mv|:=| and
  the \textbf{define} function.
This example uses Maxima's \textbf{print} function.
\begin{myVerbatim}
(%i1) f1(x):=print (x)$
(%i2) define ( f2(x), print (x) )$
x 
\end{myVerbatim}
Notice that the delayed assignment definition for \mv|f1| did not result in the actual
  operation of \textbf{print(x)}, whereas the \textbf{define} definition of \mv|f2| resulted
  in the operation of \textbf{print(x)}, as you can see on your screen after input \mv|%i2|.
Quoting from Macrakis:
\begin{quote}
In the case of \mv|':='|, both arguments are unevaluated.  This is the normal way
function definitions work in most programming languages.\\

\noindent In the case of \mv|'define'|, the second argument is evaluated 
  normally [in the process of absorbing the definition], not
  unevaluated.  This allows you to \mv|*calculate*| a definition.
\end{quote}
We can see the difference between \textbf{f1} and \textbf{f2}  by using \textbf{fundef},
  which returns what Maxima has accepted as a definition in the two cases.
\begin{myVerbatim}
(%i3) fundef (f1);
(%o3)                          f1(x) := print(x)
(%i4) fundef (f2);
(%o4)                             f2(x) := x
\end{myVerbatim}
Let's now compare these definitions at runtime.
 \mv|f1| will print its argument to the screen, but \mv|f2| will not, since in the latter
   case, all reference to Maxima's \textbf{print} function has been lost.
\begin{myVerbatim}
(%i5) f1(5);
5 
(%o5)                                  5
(%i6) f2(5);
(%o6)                                  5
\end{myVerbatim}
Macrakis continues:\\
\begin{quote}
\mv|'define'| is useful for calculating function definitions which will later be
applied to arguments.  For example, compare:
\end{quote}
\begin{myVerbatim}
(%i7) define (f(x), optimize (horner (taylor (tanh(x),x,0,8),x)));
                                2  x (%1 (%1 (42 - 17 %1) - 105) + 315)
(%o7) f(x) := block([%1], %1 : x , ------------------------------------)
                                                   315
(%i8) f(0.1);
(%o8)                          0.099667994603175
(%i9) g(x) := optimize (horner (taylor (tanh(x),x,0,8),x));
(%o9)        g(x) := optimize(horner(taylor(tanh(x), x, 0, 8), x))
(%i10) g(0.1);
Variable of expansion cannot be a number: 0.1
#0: g(x=0.1)
 -- an error.  To debug this try debugmode(true);
\end{myVerbatim}
\begin{quote}
\qquad \mv|g(.1) => ERROR| (because it is trying to do \mv|taylor(tanh(.1),.1,0,8)| ).
\end{quote}  
\newpage
\subsubsection{Lambda Expressions}
Lambda notation is used to define unnamed ``anonymous'' Maxima functions. 
Lambda expressions allow you to define a special purpose function which 
  will not be used in other contexts (since the function has no name,
  it cannot be used later).\\
  
\noindent The general syntax is\\
\textbf{lambda ( arglist, expr1, expr2, ...,exprn )} \\
where arglist has the form \mv|[x1, x2, ...,xm]| and provide formal parameters in terms
  of which the \mv|expr1, expr2,...| can be written, (use can also be made
  of \mv|%%| to refer to the result of the previous expression \mv|exprj|).
When this function is evaluated, unbound local variables \mv|x1,x2,...| are
  created, and then \mv|expr1| through \mv|exprn| are evaluated in turn.
The return value of this function is \mv|exprn|.
\begin{myVerbatim}
(%i1) functions;
(%o1) [qplot(exprlist, prange, [hvrange]), rtt(e), ts(e, v), to_atan(e, y, x), 
to_atan2(e, y, x), totan(e, v), totan2(e, v), mstate(), mattrib(), mclean(), 
fll(x)]
(%i2) f : lambda([x,y],x^2 + y^3);
                                            2    3
(%o2)                       lambda([x, y], x  + y )
(%i3) f (1,1);
(%o3)                                  2
(%i4) f(2,a);
                                     3
(%o4)                               a  + 4
(%i5) apply ( f, [1,2] );
(%o5)                                 9
(%i6) functions;
(%o6) [qplot(exprlist, prange, [hvrange]), rtt(e), ts(e, v), to_atan(e, y, x), 
to_atan2(e, y, x), totan(e, v), totan2(e, v), mstate(), mattrib(), mclean(), 
fll(x)]
(%i7) map ( 'sin, [1,2,3] );
(%o7)                      [sin(1), sin(2), sin(3)]
(%i8) map ( lambda([x],x^2 + sin(x) ),[1,2,3] );
(%o8)                [sin(1) + 1, sin(2) + 4, sin(3) + 9]
(%i9) lambda ([x],x+1 )(3);
(%o9)                                 4
\end{myVerbatim}

\subsubsection{Recursive Functions; \textbf{factorial}, and \textbf{trace} }
Here is an example which does the same jub as the Maxima function \textbf{factorial}.
We turn on \textbf{trace} to watch the recursion process.
\begin{myVerbatim}
(%i1) myfac(n) := if n = 0 then 1 else n*myfac(n-1)$
(%i2) map ('myfac, [0,1,2,3,4] );
(%o2)                          [1, 1, 2, 6, 24]
(%i3) [0!,1!,2!,3!,4!];
(%o3)                          [1, 1, 2, 6, 24]
(%i4) map ('factorial, [0,1,2,3,4] );
(%o4)                          [1, 1, 2, 6, 24]
(%i5) trace (myfac);
(%o5)                               [myfac]
\end{myVerbatim}
\newpage
\begin{myVerbatim}
(%i6) myfac(4);
1 Enter myfac [4]
 2 Enter myfac [3]
  3 Enter myfac [2]
   4 Enter myfac [1]
    5 Enter myfac [0]
    5 Exit  myfac 1
   4 Exit  myfac 1
  3 Exit  myfac 2
 2 Exit  myfac 6
1 Exit  myfac 24
(%o6)                                 24
(%i7) untrace(myfac);
(%o7)                               [myfac]
(%i8) [myfac(4),factorial(4),4!];
(%o8)                            [24, 24, 24]
\end{myVerbatim}
\subsubsection*{Legendre Polynomial}
As a second example of a recursive definition of a Maxima function, one can define the Legendre
polynomial $\mathbf{P_{n}(x)}$ via a recursive relation that relates three different
  \mv|n| values (for a given \mv|x|).
We can check our design by using the Maxima function \textbf{legendre\_p (n,x)}, which is
  part of the orthogonal polynomial package \textbf{orthopoly.lisp}.
\begin{myVerbatim}
(%i1) p(n,x) := if n=0 then 1 elseif n=1 then x else
             expand( ( (2*n-1)/n )*x*p(n-1,x) - ((n-1)/n)*p(n-2,x) )$
(%i2) [p(0,x),p(1,x),p(2,x),p(3,x)];
                                   2         3
                                3 x    1  5 x    3 x
(%o2)                    [1, x, ---- - -, ---- - ---]
                                 2     2   2      2
(%i3) map ( lambda([nn], p (nn,x) ),[0,1,2] );
                                         2
                                      3 x    1
(%o3)                          [1, x, ---- - -]
                                       2     2
(%i4) map ( lambda([nn], expand (legendre_p (nn,x))),[0,1,2] );
                                         2
                                      3 x    1
(%o4)                          [1, x, ---- - -]
                                       2     2
\end{myVerbatim}
\subsubsection{Non-Recursive Subscripted Functions (Hashed Arrays)}
``Undeclared arrays'' are called ``hashed arrays'' and they grow dynamically as more ``array elements''
  are assigned values or otherwise defined.
Here is a non-recursive simple subscripted function having one subscript \mv|k|
  and one formal argument \mv|x|.
\begin{myVerbatim}
(%i1) f[k](x) := x^k + 1;
                                          k
(%o1)                           f (x) := x  + 1
                                 k
(%i2) arrays;
(%o2)                                 [f]
(%i3) makelist ( f[nn](y),nn,0,3 );
                                      2       3
(%o3)                     [2, y + 1, y  + 1, y  + 1]
(%i4) arrayinfo(f);
(%o4)                   [hashed, 1, [0], [1], [2], [3]]
\end{myVerbatim}
\newpage
\begin{myVerbatim}
(%i5) map ( lambda ( [m], f[m](y) ), [0,1,2,3] );
                                      2       3
(%o5)                     [2, y + 1, y  + 1, y  + 1]
\end{myVerbatim}
Here is a example which uses Maxima's anonymous \textbf{lambda} function to
   define a subscripted function which does the same job as above.
\begin{myVerbatim}
(%i1) h[nn] := lambda ([xx], xx^nn + 1);
                                               nn
(%o1)                    h   := lambda([xx], xx   + 1)
                          nn
(%i2) arrays;
(%o2)                                 [f, h]
(%i3) arrayinfo(h);
(%o3)                             [hashed, 1]
(%i4) h[2];
                                            2
(%o4)                        lambda([xx], xx  + 1)
(%i5) arrayinfo(h);
(%o5)                          [hashed, 1, [2]]
(%i6) h[2](y);
                                     2
(%o6)                               y  + 1
(%i7) map ( lambda ( [mm], h[mm](x) ),[0,1,2,3] );
                                      2       3
(%o7)                     [2, x + 1, x  + 1, x  + 1]
(%i8) arrayinfo(h);
(%o8)                   [hashed, 1, [0], [1], [2], [3]]
\end{myVerbatim}
\subsubsection{Recursive Hashed Arrays and Memoizing}
We again design a homemade factorial ``function'', this time using a ``hashed array''
  having one ``index'' \mv|n|.
\begin{myVerbatim}
(%i1) arrays;
(%o1)                                 []
(%i2) a[n] := n*a[n-1]$
(%i3) a[0] : 1$
(%i4) arrays;
(%o4)                                 [a]
(%i5) arrayinfo(a);
(%o5)                          [hashed, 1, [0]]
(%i6) a[4];
(%o6)                                 24
(%i7) arrayinfo(a);
(%o7)                [hashed, 1, [0], [1], [2], [3], [4]]
(%i8) a[n] := n/2$
(%i9) a[4];
(%o9)                                 24
(%i10) a[7];
                                       7
(%o10)                                 -
                                       2
(%i11) arrayinfo(a);
(%o11)             [hashed, 1, [0], [1], [2], [3], [4], [7]]
\end{myVerbatim}
We see in the above example that elements \mv|1| through \mv|4| were computed when
  the call \mv|a[4]| was first made. 
Once the array elements have been computed, they are not recomputed,
  even though our definition changes.
Note that after \mv|(%i8) a[n] := n/2$|, our interogation \mv|(%i9) a[4];|
  does not return our new definition, but the result of the original calculation.
If you compute \mv|a[50]| using our first recursive factorial definition,
  all values \mv|a[1]|  through \mv|a[50]| are remembered.
Computing \mv|a[51]| then takes only one step.
\subsubsection{Recursive Subscripted Maxima Functions}
We here use a ``subscripted function'' to achieve a recursive definition of a 
   Legendre polynomial $\mathbf{P_{n}(x)}$.
(The author thanks Maxima developer Richard Fateman for this example.)
\begin{myVerbatim}
(%i1) ( p[n](x) := expand(((2*n-1)/n)*x*p [n-1](x) - ((n-1)/n)*p [n-2](x) ),
           p [0](x) := 1, p [1](x) := x ) $
(%i2) makelist( p[m](x),m,0,4 );
                          2         3            4       2
                       3 x    1  5 x    3 x  35 x    15 x    3
(%o2)           [1, x, ---- - -, ---- - ---, ----- - ----- + -]
                        2     2   2      2     8       4     8
(%i3) arrays;
(%o3)                                 [p]
(%i4) arrayinfo(p);
(%o4)                [hashed, 1, [0], [1], [2], [3], [4]]
(%i5) [p[3](x),p[3](y)];
                               3           3
                            5 x    3 x  5 y    3 y
(%o5)                      [---- - ---, ---- - ---]
                             2      2    2      2
(%i6) map ( lambda([nn], p[nn](x)),[0,1,2,3] );
                                   2         3
                                3 x    1  5 x    3 x
(%o6)                    [1, x, ---- - -, ---- - ---]
                                 2     2   2      2
\end{myVerbatim}
Note that the input line \mv|%i1| has the structure \\
 \mv|   ( job1, job2, job3 )$|.
This is similar to using one input line to bind values to several variables.
\begin{myVerbatim}
(%i7) ( a:1, b:2/3, c : cos(4/3) )$
\end{myVerbatim}
If you want a visual reminder of the bindings, then the \mv|[job1,job2,..]|
  notation is more appropriate.
\begin{myVerbatim}
(%i8) [ a:1, b:2/3, c : cos(4/3) ];
                                    2      4
(%o8)                           [1, -, cos(-)]
                                    3      3
\end{myVerbatim}
\subsubsection{Floating Point Numbers from a Maxima Function}
Let's use our recursive subscripted function definition of $\mathbf{P_{n}(x)}$
  and compare the ``exact'' value with the default floating point number
  when we look for $\mathbf{P_{3}(8/10)}$.
Let \mv|p3t| hold the exact value \mv|2/25|.
Maxima's \textbf{float} function returns \mv|0.08| from \mv|float(2/25)|, but
  internally Maxima holds a binary representation with has a small error when 
  converted to a decimal number.
We can see the internally held decimal number by using \mv|?print| which accesses
  the lisp function \textbf{print} from Maxima.
\begin{myVerbatim}
(%i1) ( p[n](x) := expand(((2*n-1)/n)*x*p [n-1](x) - ((n-1)/n)*p [n-2](x) ),
           p [0](x) := 1, p [1](x) := x ) $
(%i2) p3 : p[3](x);
                                     3
                                  5 x    3 x
(%o2)                             ---- - ---
                                   2      2
(%i3) p3t : p3, x = 8/10;
                                      2
(%o3)                                 --
                                      25
\end{myVerbatim}
\newpage
\begin{myVerbatim}
(%i4) float(p3t);
(%o4)                                0.08
(%i5) ?print(%);
0.080000000000000002 
(%o5)                                0.08
(%i6) p3f : p3, x = 0.8;
(%o6)                                0.08
(%i7) ?print(%);
0.080000000000000071 
(%o7)                                0.08
(%i8) abs (p3f - p3t);
(%o8)                       6.9388939039072284E-17
(%i9) abs (%o4 - p3t);
(%o9)                                 0.0
(%i10) ?print(%);
0.0 
(%o10)                                0.0
\end{myVerbatim}
The input line \mv|(%i6) p3f : p3, x = 0.8;| replaces \mv|x| by the floating point decimal number
  \mv|0.8| (which has no exact binary representation) and then uses binary arithmetic to perform
  the indicated additions, divisions, and multiplications, and then returns the closest decimal
  number corresponding to the final binary value.
The result of all this hidden work is a number with some "floating point error", which
  we calculate in line \mv|%i8|.\\
  
We can come to the same conclusion in a slightly different way if we first
  define a Maxima function \mv|P3(x)| using \mv|p[3]|, say.
\begin{myVerbatim}
(%i11) define (P3(y), p[3](y))$
(%i12) P3(x);
                                     3
                                  5 x    3 x
(%o12)                            ---- - ---
                                   2      2
(%i13) P3f1 : float (P3(8/10));
(%o13)                               0.08
(%i14) ?print(%);
0.080000000000000002 
(%o14)                               0.08
(%i15) P3f2 : P3(0.8);
(%o15)                               0.08
(%i16) ?print(%);
0.080000000000000071 
(%o16)                               0.08
(%i17) abs (P3f2 - P3f1);
(%o17)                      6.9388939039072284E-17
\end{myVerbatim}
We have used \mv|?print(...)| above to look at the approximate decimal digit equivalent 
 of a binary representation held internally by Maxima for a floating point number.
We can sometimes learn new things by looking at Lisp translations of input and output
  using \mv|?print( Maxima stuff )|, \mv|:lisp #$ Maxima stuff $|, and \mv|:lisp $_|.
(See html Help Manual: Click on Right Panel, Click on Contents, Sec. 3.1, Lisp and Maxima.)
\begin{myVerbatim}
(%i18) ?print(a + b)$
((MPLUS SIMP) $A $B) 
(%i19) :lisp $_
((PRINT) ((MPLUS) $A $B))
(%i19) :lisp #$a+b$
((MPLUS SIMP) $A $B)
(%i19) ?print( a/b )$
((MTIMES SIMP) $A ((MEXPT SIMP) $B -1)) 
(%i20) :lisp $_
((PRINT) ((MQUOTIENT) $A $B))
\end{myVerbatim}
\newpage
\begin{myVerbatim}
(%i20) :lisp #$a / b$
((MTIMES SIMP) $A ((MEXPT SIMP) $B -1))
(%i21) bfloat(2/25),fpprec:30;
(%o21)                              8.0b-2
(%i22) ?print(%);
((BIGFLOAT SIMP 102) 3245185536584267267831560205762 -3) 
(%o22)                              8.0b-2
(%i23) p3, x = 8/10;
                                      2
(%o23)                                 --
                                      25
(%i24) :lisp $_
(($EV) $P3 ((MEQUAL) $X ((MQUOTIENT) 8 10)))
(%i24) :lisp #$ p3, x=8/10 $
((RAT SIMP) 2 25)
\end{myVerbatim}
\subsection{Pulling Out Overall Factors from an Expression}
If you have an expression of the form \mv|expr : u * ( fac*c1 + fac*c2 )|,
you can pull out the common factor \mv|fac| using \mv|   fac * ratsimp ( expr / fac ) |.
Of course there may also be other methods which will work
  for a particular expression.\\
  
\noindent Here is one example.
\begin{myVerbatim}
(%i1) e1 : x*(A*cos(d1)^3 - A*cos(d2)^3);
                                3            3
(%o1)                     x (cos (d1) A - cos (d2) A)
(%i2) A * ratsimp (e1/A);
                               3          3
(%o2)                      (cos (d1) - cos (d2)) x A
(%i3) rat (e1);
                                3          3
(%o3)/R/                  (- cos (d2) + cos (d1)) x A
\end{myVerbatim}
Here is a second example.
\begin{myVerbatim}
(%i4) e2 : x*(A*a*b1 - A*a*b2);
(%o4)                         x (a b1 A - a b2 A)
(%i5) A*a*ratsimp ( e2/( A*a));
(%o5)                           a (b1 - b2) x A
(%i6) factor(e2);
(%o6)                          - a (b2 - b1) x A
(%i7) rat(e2);
(%o7)/R/                      (- a b2 + a b1) x A
\end{myVerbatim}
We see above that \textbf{rat} does not pull out all the common factors in
  this second example.
We also find that \textbf{factor} does not simply pull out factors, but also
  rewrites the trig functions, when used with the first expression.
\begin{myVerbatim}
(%i8) factor (e1);
                                 2                            2
(%o8)  - (cos(d2) - cos(d1)) (cos (d2) + cos(d1) cos(d2) + cos (d1)) x A
\end{myVerbatim}
More discussion of display and simplification tools will appear in the
  upcoming new Chapter 4.
\newpage
\subsection{Construction and Use of a Test Suite File}
It is useful to construct a code test which can be run against new versions of
  your own personal code, as well as against new versions of Maxima.
We show here a very simple example of such a code test file and use.\\

\noindent \small The test of floating point division (below) is sensitive to the specific Lisp
 version used to compile the Maxima source code; different Lisp versions
 may have slightly different floating point behaviors and digits of precision.
 Here are the contents of \textbf{mytest.mac}, which contains the tests: addition of integers,
    multiplication of integers, integer division, and floating point division.
The convention is to include an input Maxima command followed by the expected
  Maxima answer (with the addition of a semicolon).
The command \mv|batch("mytest.mac",test)| will then use Maxima's test utilities.
(You will find much useful and practical information in the file \mv|README.how-to|
  in the \mv|...\maxima\5.19.0\tests| folder.)
\begin{myVerbatim2s}
/* this is mytest.mac */
1 + 2;
3;
2 * 3;
6;
2/3;
2/3;
2.0/3;
0.66666666666666663;
\end{myVerbatim2s}
Here is an example of running \textbf{mytest.mac} as a 
\textbf{batch} file in \textbf{test} mode using one dimensional output display.
\begin{myVerbatim1}  
(%i1) display2d:false$
(%i2) 2.0/3;
(%o2) 0.66666666666667
(%i3) ?print(%);
0.66666666666666663 
(%o3) 0.66666666666667
(%i4) batch ("mytest.mac",test )$
Error log on #<output stream mytest.ERR>
********************** Problem 1 ***************
Input:
2+1
Result:
3
... Which was correct.

********************** Problem 2 ***************
Input:
2*3
Result:
6
... Which was correct.

********************** Problem 3 ***************
Input:
2/3
Result:
2/3
... Which was correct.

********************** Problem 4 ***************
Input:
2.0/3
Result:
0.66666666666667
... Which was correct.
4/4 tests passed.
\end{myVerbatim1}
\Large
Note that to avoid an error return on the floating point division test, we needed to use for 
  the answer the result returned by the Lisp print function, which is available from 
  Maxima by preceding the Lisp function name with a question mark (?).  
\newpage
\subsection{History of Maxima's Development} 
From the Unix/Linux Maxima manual:
\begin{quote}
MACSYMA (Project MAC's SYmbolic MAnipulation System) was developed by
the Mathlab group of the MIT Laboratory for Computer Science
(originally known as Project MAC), during the years 1969-1972.  Their
work was supported by grants NSG 1323 of the National Aeronautics and
Space Administration, N00014-77-C-0641 of the Office of Naval
Research, ET-78-C-02-4687 of the U.S. Department of Energy, and
F49620-79-C-020 of the U.S. Air Force.  MACSYMA was further modified
for use under the UNIX operating system (for use on DEC VAX computers
and Sun workstations), by Richard Fateman and colleagues at the
University of California at Berkeley; this version of MACSYMA is known
as VAXIMA.  The present version stems from a re-working of the public
domain MIT MACSYMA for GNU Common Lisp, prepared by William Schelter, 
University of Texas at Austin until his passing away in 2001.  It contains
numerous additions, extensions and enhancements of the original.
\end{quote}
From the first page of the Maxima html help manual:
\begin{quote}
Maxima is derived from the Macsyma system, developed at MIT in the
 years 1968 through 1982 as part of Project MAC. MIT turned 
 over a copy of the Macsyma source code to the Department
 of Energy in 1982; that version is now known as DOE Macsyma.
 A copy of DOE Macsyma was maintained by Professor William F. 
 Schelter of the University of Texas from 1982 until his 
 death in 2001. In 1998, Schelter obtained permission 
 from the Department of Energy to release the DOE Macsyma 
 source code under the GNU Public License, and in 2000 
 he initiated the Maxima project at SourceForge to 
 maintain and develop DOE Macsyma, now called Maxima. 
 \end{quote}
Maxima is now developed and maintained by the Maxima project 
  at \\
  \mv|    http://maxima.sourceforge.net|.
\end{document}