% e. woollett
%
%   spring summer 09
% file ode1.tex
% maxima by example, ch.3 Ordinary Differential Equations
% april 30, 09, convert to new style
% C:\work2\mbe1-start.tex
% sec 1.4.5 used makelist to create xylist
% sec 1.5.10 and 1.5.12 used apply('matrix, datalist) to get matrix
% april 9, 08, sec -> subsec renumbering
% mar 23, 08: sec variables in maxima : quote from macrakis about don't need to declare x
%             using the maxima interface:  closing bott. window w keybd cmmds
%needs step.eps
% edit with Notepad++, then load into LED for latexing
\documentclass[11pt]{article}
\usepackage[dvips,top=1.5cm,left=1.5cm,right=1.5cm,foot=1cm,bottom=1.5cm]{geometry}
\usepackage{times,amsmath,amsbsy,graphicx,fancyvrb,url}
\usepackage[usenames]{color}
%\definecolor{MyDarkBlue}{rgb}{0,0.08,0.45}
\definecolor{mdb}{rgb}{0.1,0,0.55}
\newcommand{\tcdb}{\textcolor{mdb}}
\newcommand{\tcbr}{\textcolor{BrickRed}}
\newcommand{\tcb}{\textcolor{blue}}
\newcommand{\tcr}{\textcolor{red}}
\urldef\tedhome\url{ http://www.csulb.edu/~woollett/  }
\urldef\tedmail\url{ woollett@charter.net}
%1.  this is for maxima code: red framed bold, footnotesize 
\DefineVerbatimEnvironment%
   {myVerbatim}%
   {Verbatim}%
   {fontfamily=courier,fontseries=b,fontsize=\footnotesize ,frame=single,rulecolor=\color{BrickRed}}
\DefineVerbatimEnvironment%
   {myVerbatim1}%
   {Verbatim}%
   {fontfamily=courier,fontseries=b,fontsize=\scriptsize ,frame=single,rulecolor=\color{BrickRed}}
%2.  this is for blue framed bold 
\DefineVerbatimEnvironment%
   {myVerbatim2}%
   {Verbatim}%
   {fontfamily=courier,fontseries=b,frame=single,rulecolor=\color{blue}}
\DefineVerbatimEnvironment%
   {myVerbatim2s}%
   {Verbatim}%
   {fontfamily=courier,fontseries=b,fontsize=\small,frame=single,rulecolor=\color{blue}}
\DefineVerbatimEnvironment%
   {myVerbatim2f}%
   {Verbatim}%
   {fontfamily=courier,fontseries=b,fontsize=\footnotesize,frame=single,rulecolor=\color{blue}}
% 3.  this is for black framed  bold
\DefineVerbatimEnvironment%
   {myVerbatim3}%
   {Verbatim}%
   {fontfamily= courier, fontseries=b, frame=single}
% 4.  this is for no frame bold
\DefineVerbatimEnvironment%
   {myVerbatim4}%
   {Verbatim}%
   {fontfamily=courier, fontseries=b,fontsize=\small}
% 6.  for defaults use usual verbatim
\newcommand{\mv}{\Verb[fontfamily=courier,fontseries=b]}
\newcommand{\mvs}{\Verb[fontfamily=courier,fontseries=b,fontsize=\small]}
\newcommand{\mvf}{\Verb[fontfamily=courier,fontseries=b,fontsize=\footnotesize]}

\renewcommand{\thefootnote}{\ensuremath{\fnsymbol{footnote}}}
%%%%%%%%%%%%%%%%%%%%%%%%%%%%%%%%%%%%%%%%%%%%%%%%%%%%%%%%%%%%%%%%%%%%%%%%
%   title page
%%%%%%%%%%%%%%%%%%%%%%%%%%%%%%%%%%%%%%%%%%%%%%%%%%%%%%%%%%%%%%%%%%%%%%

\title{ Maxima by Example:\\ Ch. 3, Ordinary Differential Equation Tools 
             \thanks{This version uses \textbf{Maxima 5.18.1} \;
			 Check \; \textbf{ \tedhome } \; for the latest version of these notes. Send comments and
			 suggestions to \textbf{\tedmail} } }

%\author{Edwin L. Woollett \thanks{Dept. of Physics and Astron, Calif. State University at Long Beach,\\
%                    author's email: \tedmail} }
\author{ Edwin L. Woollett}
\date{\today}
%\date{}
%%%%%%%%%%%%%%%%%%%%%%%%%%%%%%%%%%%%%%%%%
%          document
%%%%%%%%%%%%%%%%%%%%%%%%%%%%%%%%%%%%%%%%%%
\begin{document}
%\SMALL
%\small
\maketitle
\tableofcontents
\numberwithin{equation}{section}
\newpage
%\normalsize
\subsubsection*{Preface}
\begin{myVerbatim2s} 
COPYING AND DISTRIBUTION POLICY    
This document is part of a series of notes titled
"Maxima by Example" and is made available
via the author's webpage http://www.csulb.edu/~woollett/
to aid new users of the Maxima computer algebra system.	
	
NON-PROFIT PRINTING AND DISTRIBUTION IS PERMITTED.
	
You may make copies of this document and distribute them
to others as long as you charge no more than the costs of printing.	

These notes (with some modifications) will be published in book form
eventually via Lulu.com in an arrangement which will continue
to allow unlimited free download of the pdf files as well as the option
of ordering a low cost paperbound version of these notes.
\end{myVerbatim2s}	
\smallskip
\noindent \tcbr{Feedback from readers is the best way for this series of notes
  to become more helpful to new users of Maxima}.
\tcdb{\emph{All} comments and suggestions for improvements will be appreciated and
  carefully considered}.
\smallskip
\begin{myVerbatim2s}
LOADING FILES
The defaults allow you to use the brief version load(fft) to load in the
Maxima file fft.lisp.
To load in your own file, such as qxxx.mac 
using the brief version load(qxxx), you either need to place 
qxxx.mac in one of the folders Maxima searches by default, or
else put a line like:
  
file_search_maxima : append(["c:/work2/###.{mac,mc}"],file_search_maxima )$
  
in your personal startup file maxima-init.mac (see later in this chapter
for more information about this).

Otherwise you need to provide a complete path in double quotes,
as in load("c:/work2/qxxx.mac"),
 
We always use the brief load version in our examples, which are generated 
using the XMaxima graphics interface on a Windows XP computer, and copied
into a fancy verbatim environment in a latex file which uses the fancyvrb
and color packages.
\end{myVerbatim2s} 
\smallskip
\begin{myVerbatim2s}
Maxima.sourceforge.net. Maxima, a Computer Algebra System. Version 5.18.1
 (2009). http://maxima.sourceforge.net/
\end{myVerbatim2s}   
\small
The homemade function \mv|fll(x)| (first, last, length) is used to return the first and last
  elements of lists (as well as the length), and is automatically loaded in with \mv|mbe1util.mac|
  from Ch. 1.
We will include a reference to this definition when working with lists.\\
  
\noindent This function has the definitions
\begin{myVerbatim2s}
fll(x) := [first(x),last(x),length(x)]$
declare(fll,evfun)$
\end{myVerbatim2s}
Some of the examples used in these notes are from the Maxima html help 
 manual or the Maxima mailing list:\\
  \mvs|http://maxima.sourceforge.net/maximalist.html|.\\  
  
\noindent The author would like to thank the Maxima developers for their friendly help via 
  the Maxima mailing list.
\normalsize
\newpage
\setcounter{section}{3}
\pagestyle{headings}
\subsection{Solving Ordinary Differential Equations}
\subsection{Solution of One First Order Ordinary Differential Equation (ODE)}
\subsubsection{Summary Table}
\begin{table}[h]
\begin{center}
  \renewcommand{\arraystretch}{1.25}
  \begin{tabular}{|c|}
    \hline
  \Large{\tcbr{ode2 and ic1}}\\ \hline
    \mv|gsoln : ode2 (de, u, t);|\\
	\mv|where de involves 'diff(u,t).|\\
	\mv|psoln : ic1 (gsoln, t = t0, u = u0);|\\ \hline
  \Large{\tcbr{desolve}}\\ \hline
    \mv|gsoln : desolve(de, u(t) );|\\
	\mv|where de includes the equal sign (=)|\\
	\mv|and 'diff(u(t),t) and possibly u(t).|\\
	\mv|psoln : ratsubst(u0val,u(o),gsoln)|\\ \hline
   \Large{\tcbr{plotdf}}\\ \hline
	\mv|plotdf ( dudt, [t,u], [trajectory_at, t0, u0],|\\
	\mv|[direction,forward], [t, tmin, tmax],|\\
	\mv|  [u, umin, umax] )$|\\ \hline
	\Large{\tcbr{rk}}\\ \hline
	\mv|points : rk ( dudt, u, u0, [t, t0, tlast, dt] )$|\\ \hline
	\mv|  where dudt is a function of t and u which|\\
    \mv|  determines diff(u,t).|\\ \hline	
	\end{tabular}  
\caption{Methods for One First Order ODE}
\end{center}
\end{table}
\noindent We will use these four different methods to solve the first order
  ordinary differential equation
\begin{equation}  \label{Eq:ode1}
\mathbf{\frac{d\,u}{d\,t} = e^{-t} + u }
\end{equation}
  subject to the condition that when $\mathbf{t = 2, \quad u = -0.1}$.
\subsubsection{Exact Solution with \textbf{ode2} and \textbf{ic1}}
Most ordinary differential equations have no known exact solution (or the exact
  solution is a complicated expression involving many terms with special
  functions) and one normally uses approximate methods.
However, some ordinary differential equations have simple exact solutions, and
  many of these can be found using \textbf{ode2}, \textbf{desolve},  or \textbf{contrib\_ode}.
The manual has the following information about \textbf{ode2}
\begin{quote}
Function: \textbf{ode2 (eqn, dvar, ivar)}\\
The function \textbf{ode2} solves an ordinary differential equation (ODE) of 
  \tcbr{first or second order}.
It takes three arguments: an ODE given by \textbf{eqn}, the dependent variable \textbf{dvar},
   and the independent variable \textbf{ivar}.
When successful, it returns either an explicit or implicit solution for the 
   dependent variable.
\mv|%c| is used to represent the integration constant in the case of first-order equations,
   and \mv|%k1| and \mv|%k2| the constants for second-order equations.
The dependence of the dependent variable on the independent variable does not
   have to be written explicitly, as in the case of \textbf{desolve}, 
   but the independent variable must always be given as the third argument. 
\end{quote}
\newpage
\noindent If the differential equation has the structure \textbf{Left(dudt,u,t) = Right(dudt,u,t)}
  ( here \textbf{u} is the dependent variable and \textbf{t} is the independent variable),
  we can always rewrite that differential equation as
  \textbf{de = Left(dudt,u,t) - Right(dudt,u,t) = 0}, or \textbf{de = 0}.\\

\noindent We can use the syntax \textbf{ode2(de,u,t)}, with the first argument  
  an expression which includes derivatives, instead of the complete equation including
  the \textbf{" = 0"} on the end, and \textbf{ode2} will assume we mean \textbf{de = 0}
  for the differential equation.
(Of course you can also use \textbf{ode2 ( de=0, u, t)}\\


\noindent We rewrite our example linear first order differential equation 
  Eq. \ref{Eq:ode1}
  in the way just described, using the \textbf{noun} form \textbf{'diff},
  which uses a single quote.
We then use \textbf{ode2}, and call the general solution \textbf{gsoln}.
\begin{myVerbatim}
(%i1) de : 'diff(u,t)- u - exp(-t);
                                du         - t
(%o1)                           -- - u - %e
                                dt
(%i2) gsoln : ode2(de,u,t);
                                        - 2 t
                                      %e         t
(%o2)                       u = (%c - -------) %e
                                         2
\end{myVerbatim}
The general solution returned by \textbf{ode2} contains one constant of
  integration \mvs|%c|, and is an explicit solution for \textbf{u} as a
  function of \textbf{t}, although the above does not bind the symbol \mvs|u|.\\
  
\noindent We next find the particular solution which has $\mathbf{t = 2, \; u = -0.1}$
  using \textbf{ic1}, and call this particular solution \textbf{psoln}.
We then check the returned solution in two ways: 1. does it satisfy the
  conditions given to \textbf{ic1}?, and 2. does it produce a zero value
  for our expression \textbf{de}?
\begin{myVerbatim}
(%i3) psoln : ic1(gsoln,t = 2, u = -0.1),ratprint:false;
                           - t - 4     2        2 t       4
                         %e        ((%e  - 5) %e    + 5 %e )
(%o3)              u = - -----------------------------------
                                         10
(%i4) rhs(psoln),t=2,ratsimp;
                                       1
(%o4)                                - --
                                       10
(%i5) de,psoln,diff,ratsimp;
(%o5)                                  0
\end{myVerbatim}
Both tests are passed by this particular solution.
We can now make a quick plot of this solution using \textbf{plot2d}.
\begin{myVerbatim}
(%i6) us : rhs(psoln);
                         - t - 4     2        2 t       4
                       %e        ((%e  - 5) %e    + 5 %e )
(%o6)                - -----------------------------------
                                       10
(%i7) plot2d(us,[t,0,7],
           [style,[lines,5]],[ylabel," "],
           [xlabel,"t0 = 2, u0 = -0.1, du/dt = exp(-t) + u"])$
\end{myVerbatim}
\newpage
\noindent which looks like
\smallskip
\begin{figure} [h]  
   \centerline{\includegraphics[scale=.7]{ch1p7.eps} }
	\caption{ Solution for which $\mathbf{t = 2, \;u = -0.1}$}
\end{figure}
 
%plot2d(us,[t,0,7],
%           [style,[lines,8]],[ylabel," "],
%           [xlabel,"t0 = 2, u0 = -0.1, du/dt = exp(-t) + u"],
%            [psfile,"ch1p7.eps"])$
\subsubsection{Exact Solution Using \textbf{desolve}}
\textbf{desolve} uses Laplace transform methods to find an exact solution.
To be able to use \textbf{desolve}, we need to write our example 
  differential equation Eq.\ref{Eq:ode1} in
  a more explicit form, with every \mv|u -> u(t)|, and include the
  \mv|=| sign in the definition of the differential equation.
\begin{myVerbatim}
(%i1) eqn : 'diff(u(t),t) - exp(-t) - u(t) = 0;
                         d                    - t
(%o1)                    -- (u(t)) - u(t) - %e    = 0
                         dt
(%i2) gsoln : desolve(eqn,u(t));
                                              t     - t
                               (2 u(0) + 1) %e    %e
(%o2)                   u(t) = ---------------- - -----
                                      2             2
(%i3) eqn,gsoln,diff,ratsimp;
(%o3)                                0 = 0
(%i4) bc : subst ( t=2, rhs(gsoln)) = - 0.1;
                         2                  - 2
                       %e  (2 u(0) + 1)   %e
(%o4)                  ---------------- - ----- = - 0.1
                              2             2
(%i5) solve ( eliminate ( [gsoln, bc],[u(0)]), u(t) ),ratprint:false;
                                 - t     t - 2       t - 4
                           - 5 %e    - %e      + 5 %e
(%o5)              [u(t) = -------------------------------]
                                         10
(%i6) us : rhs(%[1]);
                              - t     t - 2       t - 4
                        - 5 %e    - %e      + 5 %e
(%o6)                   -------------------------------
                                      10
\end{myVerbatim}
\newpage
\begin{myVerbatim}
(%i7) us, t=2, ratsimp;
                                       1
(%o7)                                - --
                                       10
(%i8) plot2d(us,[t,0,7],
           [style,[lines,5]],[ylabel," "],
           [xlabel,"t0 = 2, u0 = -0.1, du/dt = exp(-t) + u"])$
\end{myVerbatim}
  and we get the same plot as before.
The function \textbf{desolve} returns a solution in terms of the ``initial value'' \mv|u(0)|,
  which here means \mv|u(t = 0)|,  and
  we must go through an extra step to eliminate \mv|u(0)| in a way that
  assures our chosen boundary condition \mv|t = 2,  u - -0.1| is satisfied.\\
  
\noindent We have checked that the general solution satisfies the given
  differential equation in \mv|%i3|, and have checked that our particular
  solution satisfies the desired condition at \mv|t = 2| in \mv|%i7|.\\
  
\noindent If your problem requires that the value of the solution \mv|us| be
  specified at \mv|t = 0|, the route to the particular solution is much simpler than
    what we used above.
You simply use \mv|subst ( u(0) = -1, rhs (gsoln) )| if, for example, you wanted a
  particular solution to have the property that when \mv|t = 0|, \mv|u = -1|. 
\begin{myVerbatim}
(%i9) us : subst( u(0) = -1,rhs(gsoln) ),ratsimp;
                                  - t    2 t
                                %e    (%e    + 1)
(%o9)                         - -----------------
                                        2
(%i10) us,t=0,ratsimp;
(%o10)                                 - 1
\end{myVerbatim}
\subsubsection{Numerical Solution and Plot with \textbf{plotdf}}
We next use \textbf{plotdf} to numerically integrate the given first order
  ordinary differential equation, draw a direction field plot which governs any particular
  solution, and draw the particular solution we have chosen.\\
  
  
\noindent The default color choice of \textbf{plotdf} is to use small blue arrows give the local 
  direction of the trajectory of the particular solution passing though that point.
This direction can be defined by an angle $\boldsymbol{\alpha}$ such that if $\mathbf{u' = f(t,u)}$,
  then $\mathbf{\boldsymbol{\tan(\alpha}) = f(t,u)}$, and at the point $\mathbf{(t_{0}, \, u_{0})}$,
\begin{equation}
\mathbf{d\,u = f(t_{0},u_{0}) \times \,d\,t = d\,t \, \times  \left( \frac{d\,u}{d\,t} \right)_{t = t_{0},\;u = u_{0}} }
\end{equation}
 This equation determines the increase \textbf{d\,u} in the value of the dependent variable
 \textbf{u} induced by a small increase \textbf{d\,t} in the independent variable \textbf{t}
 at the point $\mathbf{(t_{0},\,u_{0})}$.
We then define a local vector with \textbf{t} component \textbf{d\,t} and
  \textbf{u} component \textbf{d\,u}, and draw a small arrow in that direction at a grid of
  chosen points to construct a direction field associated with the given first
  order differential equation.
The length of the small arrow can be increased some to reflect large values of the
  magnitude of $\mathbf{d\,u/d\,t}$.\\

\noindent For one first order ordinary differential equation, \textbf{plotdf},
  has the syntax
\begin{myVerbatim2}
       plotdf( dudt,[t,u], [trajectory_at, t0, u0], options ... )
\end{myVerbatim2}
  in which \mv|dudt| is the function of \mv|(t,u)| which determines the rate of change 
   $\mathbf{d\,u/d\,t}$.
\newpage
\begin{myVerbatim}
(%i1) plotdf(exp(-t) + u, [t, u], [trajectory_at,2,-0.1],
             [direction,forward], [t,0,7], [u, -6, 1] )$
\end{myVerbatim}
produces the plot
\smallskip
\begin{figure} [h]  
   \centerline{\includegraphics[scale=.6]{ch1p8.eps} }
	\caption{Direction Field for the Solution $\mathbf{t = 2, \;u = -0.1}$}
\end{figure}

\noindent (We have thickened the red curve using the \textbf{Config, Linewidth} menu option of
  \textbf{plotdf}, followed by \textbf{Replot}).\\

\noindent The help manual has an extensive discussion and examples  of the use of the direction
  field plot utility \textbf{plotdf}.
\subsubsection{Numerical Solution with 4th Order Runge-Kutta: \textbf{rk}}
Although the \textbf{plotdf} function is useful for orientation about the shapes and
  types of solutions possible, if you need a list with coordinate points to use for
  other purposes, you can use the fourth order Runge-Kutta function \textbf{rk}.\\
  
\noindent For one first order ordinary differential equation, the syntax has some
  faint resemblance to that of \textbf{plotdf}:
\begin{myVerbatim2}
   rk ( dudt, u, u0, [ t, t0, tlast, dt ] )
\end{myVerbatim2}  
  in which we are assuming that \textbf{u} is the dependent variable and \textbf{t} is the
   independent variable, and \mv|dudt| is that function of \textbf{(t, u)} which
   locally determines the value of $\mathbf{d\,u/d\,t}$.
This will numerically integrate the corresponding first order ordinary differential
  equation and return a list of pairs of \textbf{(t, u)} on the solution curve which
  has been requested:
\begin{myVerbatim2}
  [ [t0, u0], [t0 + dt, y(t0 + dt)], .... ,[tlast, y(tlast)] ]
\end{myVerbatim2}  
\newpage
For our example first order ordinary differential equation, choosing
  the same initial conditions as above, and choosing \mv|dt = 0.01|,
\begin{myVerbatim}
(%i1) fpprintprec:8$
(%i2) points : rk (exp(-t) + u, u, -0.1, [ t, 2, 7, 0.01 ] )$
(%i3) %, fll;
(%o3)                [[2, - 0.1], [7.0, - 4.7990034], 501]
(%i4) plot2d( [ discrete, points ], [ t, 0, 7],
           [style,[lines,5]],[ylabel," "],
           [xlabel,"t0 = 2, u0 = -0.1, du/dt = exp(-t) + u"])$
\end{myVerbatim}
(We have used our homemade function \mv|fll(x)|, loaded in at startup
  with the other functions defined in \mv|mbe1util.mac|, available with
  the Ch. 1 material.
We have provided the definition of \mv|fll| in the preface of this
  chapter.
Instead of \mv|%, fll ;|, you could use \mv|[%[1],last(%),length(%)];|
  to get the same information.)\\
  
\noindent The plot looks like
\smallskip
\begin{figure} [h]  
   \centerline{\includegraphics[scale=.6]{ch1p30.eps} }
	\caption{Runge-Kutta Solution with  $\mathbf{t = 2, \;u = -0.1}$ }
\end{figure}

%(%i7) plot2d([discrete,points],[t,0,7],
%           [style,[lines,8]],[ylabel," "],
%           [xlabel,"t0 = 2, u0 = -0.1, du/dt = exp(-t) + u"],
%            [psfile,"ch1p30.eps"])$
\newpage
\subsection{Solution of One Second Order ODE or Two First Order ODE's}
\subsubsection{Summary Table}
\begin{table}[h]
\begin{center}
  \renewcommand{\arraystretch}{1.25}
  \begin{tabular}{|c|}
    \hline
  \Large{\tcbr{ode2 and ic1}}\\ \hline
    \mv|gsoln : ode2 (de, u, t); where de involves 'diff(u,t,2)|\\
	\mv|and possibly 'diff(u,t).|\\
	\mv|psoln : ic2 (gsoln, t = t0, u = u0, 'diff(u,t) = up0);|\\ \hline
  \Large{\tcbr{desolve}}\\ \hline
    \mv| atvalue ( 'diff(u,t), t = 0, v(0) );|\\
    \mv|gsoln : desolve(de, u(t) );|\\
	\mv|where de includes the equal sign (=), 'diff(u(t),t,2),|\\
	\mv|and possibly 'diff(u(t),t) and  u(t).|\\
	\mv|One type of particular solution is returned by using |\\
	\mv|psoln : subst([ u(o) = u0, v(0) = v0] , gsoln)|\\ \hline
   \Large{\tcbr{plotdf}}\\ \hline
     \mv|plotdf ( [dudt, dvdt], [u, v], [trajectory_at, u0, v0],|\\
     \mv| [u, umin, umax],[v, vmin, vmax], [tinitial, t0],|\\
     \mv| [direction,forward], [versus_t, 1],[tstep, timestepval],|\\
	 \mv|[nsteps, nstepsvalue] )$|\\ \hline	
	\Large{\tcbr{rk}}\\ \hline
	\mv|points : rk ([dudt, dvdt ],[u, v],[u0, v0],[t, t0, tlast, dt] )$|\\ \hline
	\mv|  where dudt and dvdt are functions of t,u, and v which|\\
    \mv|  determine diff(u,t) and diff(v,t).|\\ \hline	
	\end{tabular}  
\caption{Methods for One Second Order or Two First Order ODE's}
\end{center}
\end{table}
We apply the above four methods to the simple second order
  ordinary differential equation:
\begin{equation}
\mathbf{\frac{d^{2}\,u}{d\,t^{2}} = 4\, u}
\end{equation}
  subject to the conditions that when $\mathbf{t = 2}$, $\mathbf{u = 1}$ 
  and $\mathbf{d\,u/d\,t = 0}$.
\subsubsection{Exact Solution with \textbf{ode2}, \textbf{ic2}, and \textbf{eliminate}}
The main difference here is the use of \textbf{ic2} rather than \textbf{ic1}.
\begin{myVerbatim}
(%i1) de : 'diff(u,t,2) - 4*u;
                                    2
                                   d u
(%o1)                              --- - 4 u
                                     2
                                   dt
(%i2) gsoln : ode2(de,u,t);
                                    2 t         - 2 t
(%o2)                     u = %k1 %e    + %k2 %e
(%i3) de,gsoln,diff,ratsimp;
(%o3)                                  0
\end{myVerbatim}
\newpage
\begin{myVerbatim}
(%i4) psoln : ic2(gsoln,t=2,u=1,'diff(u,t) = 0);
                                 2 t - 4     4 - 2 t
                               %e          %e
(%o4)                      u = --------- + ---------
                                   2           2
(%i5) us : rhs(psoln);
                               2 t - 4     4 - 2 t
                             %e          %e
(%o5)                        --------- + ---------
                                 2           2
(%i6) us, t=2, ratsimp;
(%o6)                                  1
(%i7) plot2d(us,[t,0,4],[y,0,10],
           [style,[lines,5]],[ylabel," "],
           [xlabel," U versus t, U''(t) = 4 U(t), U(2) = 1, U'(2) = 0 "])$
plot2d: expression evaluates to non-numeric value somewhere in plotting range.
\end{myVerbatim}
which produces the plot
\smallskip
\begin{figure} [h]  
   \centerline{\includegraphics[scale=.7]{ch1p31.eps} }
	\caption{Solution for which $\mathbf{t = 2, \;u = 1, \; u' = 0}$  }
\end{figure}

%plot2d(us,[t,0,4],[y,0,10],
%           [style,[lines,8]],[ylabel," "],
%           [xlabel," U versus t, U''(t) = 4 U(t), U(2) = 1, U'(2) = 0 "],
%           [psfile,"ch1p31.eps"])$
Next we make a ``phase space plot'' which is a plot of $\mathbf{v = d\,u/d\,t}$
  versus $\mathbf{u}$ over the range $\mathbf{ 1 \leq t \leq 3 }$.
\begin{myVerbatim}
(%i8) vs : diff(us,t),ratsimp;
                             - 2 t - 4    4 t     8
(%o8)                      %e          (%e    - %e )
(%i9) for i thru 3 do
        d[i]:[discrete,[float(subst(t=i,[us,vs]))]]$
(%i10) plot2d( [[parametric,us,vs,[t,1,3]],d[1],d[2],d[3] ],
            [x,0,8],[y,-12,12],
          [style, [lines,5,1],[points,4,2,1],
            [points,4,3,1],[points,4,6,1]],
          [ylabel," "],[xlabel," "],
           [legend," du/dt vs u "," t = 1 ","t = 2","t = 3"] )$
\end{myVerbatim}

\newpage

\noindent which produces the plot
\smallskip
\begin{figure} [h]  
   \centerline{\includegraphics[scale=.7]{ch1p32.eps} }
	\caption{ $\mathbf{t = 2, \;y = 1, \; y' = 0}$ Solution }
\end{figure}

%plot2d( [[parametric,us,vs,[t,1,3]],d[1],d[2],d[3] ],
%            [x,0,8],[y,-12,12],
%          [style, [lines,8,1],[points,6,2,1],
%            [points,6,3,1],[points,6,6,1]],
%          [ylabel," "],[xlabel," "],
%           [legend," du/dt vs u "," t = 1 ","t = 2","t = 3"],
%           [psfile,"ch1p32.eps"])$
%\newpage
If your boundary conditions are, instead, for \mv|t=0, u = 1|, and
  for \mv|t = 2, u = 4|, then one can eliminate the two constants ``by hand''
  instead of using \textbf{ic2} (see also next section).
\begin{myVerbatim}
(%i4) bc1 : subst(t=0,rhs(gsoln)) = 1$
(%i5) bc2 : subst(t = 2, rhs(gsoln)) = 4$
(%i6) solve(
        eliminate([gsoln,bc1,bc2],[%k1,%k2]), u ),
           ratsimp, ratprint:false;
                       - 2 t       4        4 t     8       4
                     %e      ((4 %e  - 1) %e    + %e  - 4 %e )
(%o6)           [u = -----------------------------------------]
                                        8
                                      %e  - 1
(%i7) us : rhs(%[1]);
                     - 2 t       4        4 t     8       4
                   %e      ((4 %e  - 1) %e    + %e  - 4 %e )
(%o7)              -----------------------------------------
                                      8
                                    %e  - 1
(%i8) us,t=0,ratsimp;
(%o8)                                  1
(%i9) us,t=2,ratsimp;
(%o9)                                  4
\end{myVerbatim}
\newpage
\subsubsection{Exact Solution with \textbf{desolve, atvalue,} and \textbf{eliminate}}
The function \textbf{desolve} uses Laplace transform methods which
  are set up to expect the use of initial values for dependent
  variables and their derivatives.
(However, we will show how you can impose more general boundary conditions.)
If the dependent variable is \mv|u(t)|, for example, the solution
  is returned in terms of a constant \mv|u(0)|, which refers to
  the value of \mv|u(t = 0)| (here we are assuming that the independent
  variable is \mv|t|).
To get a simple result from \textbf{desolve} which we can work with (for the case of a second
  order ordinary differential equation), we can use the \textbf{atvalue}
  function with the syntax (for example):
\begin{myVerbatim2}
  atvalue ( 'diff(u,t), t = 0, v(0) )
\end{myVerbatim2}
  which will allow \textbf{desolve} to return the solution to a 
  second order ODE in terms of the pair of constants \mv|( u(0), v(0) )|.
Of course, there is nothing sacred about using the symbol \mv|v(0)| here.
The function \textbf{atvalue} should be invoked before the use of
  \textbf{desolve}.\\

\noindent If the desired boundary conditions for a particular
  solution refer to \mv|t = 0|, then you can immediately find that particular
  solution using the syntax (if \mv|ug| is the general solution, say)
\begin{myVerbatim2}
  us : subst( [ u(0) = u0val, v(0) = v0val], ug ),
\end{myVerbatim2}
  or else by using \textbf{ratsubst} twice.\\
  
\noindent In our present example, the desired boundary conditions
  refer to \mv| t = 2 |, and impose conditions on the value of \textbf{u} and
  its first derivative at that value of \mv|t|.
This requires a little more work, and we use \textbf{eliminate} to get
  rid of the constants \mv|(u(0), v(0))| in a way that allows our desired
  conditions to be satisfied.  
\begin{myVerbatim}
(%i1) eqn : 'diff(u(t),t,2) - 4*u(t) = 0;
                             2
                            d
(%o1)                       --- (u(t)) - 4 u(t) = 0
                              2
                            dt
(%i2) atvalue ( 'diff(u(t),t), t=0, v(0) )$
(%i3) gsoln : desolve(eqn,u(t));
                                     2 t                     - 2 t
                   (v(0) + 2 u(0)) %e      (v(0) - 2 u(0)) %e
(%o3)       u(t) = --------------------- - -----------------------
                             4                        4
(%i4) eqn,gsoln,diff,ratsimp;
(%o4)                                0 = 0
(%i5) ug : rhs(gsoln);
                                  2 t                     - 2 t
                (v(0) + 2 u(0)) %e      (v(0) - 2 u(0)) %e
(%o5)           --------------------- - -----------------------
                          4                        4
(%i6) vg : diff(ug,t),ratsimp$
(%i7) ubc : subst(t = 2,ug) = 1$
(%i8) vbc : subst(t = 2,vg) = 0$
(%i9) solve (
          eliminate([gsoln, ubc, vbc],[u(0), v(0)]), u(t) ),
           ratsimp,ratprint:false;
                                - 2 t - 4    4 t     8
                              %e          (%e    + %e )
(%o9)                 [u(t) = -------------------------]
                                          2
\end{myVerbatim}
\newpage
\begin{myVerbatim}
(%i10) us : rhs(%[1]);
                             - 2 t - 4    4 t     8
                           %e          (%e    + %e )
(%o10)                     -------------------------
                                       2
(%i11) subst(t=2, us),ratsimp;
(%o11)                                 1
(%i12) vs : diff(us,t),ratsimp;
                             - 2 t - 4    4 t     8
(%o12)                     %e          (%e    - %e )
(%i13) subst(t = 2,vs),ratsimp;
(%o13)                                 0
(%i14) plot2d(us,[t,0,4],[y,0,10],
           [style,[lines,5]],[ylabel," "],
           [xlabel," U versus t, U''(t) = 4 U(t), U(2) = 1, U'(2) = 0 "])$
plot2d: expression evaluates to non-numeric value somewhere in plotting range.
(%i15) for i thru 3 do
        d[i]:[discrete,[float(subst(t=i,[us,vs]))]]$
(%i16) plot2d( [[parametric,us,vs,[t,1,3]],d[1],d[2],d[3] ],
            [x,0,8],[y,-12,12],
          [style, [lines,5,1],[points,4,2,1],
            [points,4,3,1],[points,4,6,1]],
          [ylabel," "],[xlabel," "],
           [legend," du/dt vs u "," t = 1 ","t = 2","t = 3"] )$
\end{myVerbatim}
  which generates the same plots found with the \textbf{ode2} method above.\\
  
\noindent If the desired boundary conditions are that \mv|u| have given values
  at \mv|t = 0| and \mv|t = 3|, then we can proceed from the same
  general solution above as follows with \mv|up| being a partially defined
  particular solution (assume \mv|u(0) = 1| and \mv|u(3) = 2|):
\begin{myVerbatim}
(%i17) up : subst(u(0) = 1, ug);
                                  2 t                - 2 t
                     (v(0) + 2) %e      (v(0) - 2) %e
(%o17)               ---------------- - ------------------
                            4                   4
(%i18) ubc : subst ( t=3, up) = 2;
                       6                - 6
                     %e  (v(0) + 2)   %e    (v(0) - 2)
(%o18)               -------------- - ---------------- = 2
                           4                 4
(%i19) solve(
          eliminate ( [ u(t) = up, ubc ],[v(0)] ), u(t) ),
             ratsimp, ratprint:false;
                        - 2 t       6        4 t     12       6
                      %e      ((2 %e  - 1) %e    + %e   - 2 %e )
(%o19)        [u(t) = ------------------------------------------]
                                         12
                                       %e   - 1
(%i20) us : rhs (%[1]);
                    - 2 t       6        4 t     12       6
                  %e      ((2 %e  - 1) %e    + %e   - 2 %e )
(%o20)            ------------------------------------------
                                     12
                                   %e   - 1
(%i21) subst(t = 0, us),ratsimp;
(%o21)                                 1
(%i22) subst (t = 3, us),ratsimp;
(%o22)                                 2
\end{myVerbatim}
\newpage
\begin{myVerbatim}
(%i23) plot2d(us,[t,0,4],[y,0,10],
           [style,[lines,5]],[ylabel," "],
           [xlabel," U versus t, U''(t) = 4 U(t), U(0) = 1, U(3) = 2 "])$
plot2d: expression evaluates to non-numeric value somewhere in plotting range.
\end{myVerbatim}
\noindent which produces the plot
\smallskip
\begin{figure} [h]  
   \centerline{\includegraphics[scale=.7]{ch1p33.eps} }
	\caption{Solution for $\mathbf{u(0) = 1, \; u(3) = 2 }$ }
\end{figure}

%plot2d ( us, [t,0,4], [y,0,10],
%           [style,[lines,8]],[ylabel," "],
%           [xlabel," U versus t, U''(t) = 4 U(t), U(0) = 1, U(3) = 2 "],
%           [psfile,"ch1p33.eps"])$

\noindent If instead, you need to satisfy \mv|u(1) = -1| and \mv|u(3) = 2|, you
  could proceed from \mv|gsoln| and \mv|ug| as follows:
\begin{myVerbatim}
(%i24) ubc1 : subst ( t=1, ug) = -1$
(%i25) ubc2 : subst ( t=3, ug) = 2$
(%i26) solve(
         eliminate ( [gsoln, ubc1, ubc2],[u(0),v(0)]), u(t) ),
          ratsimp, ratprint:false;
                        - 2 t       4        4 t     12       8
                      %e      ((2 %e  + 1) %e    - %e   - 2 %e )
(%o26)        [u(t) = ------------------------------------------]
                                        10     2
                                      %e   - %e
(%i27) us : rhs(%[1]);
                    - 2 t       4        4 t     12       8
                  %e      ((2 %e  + 1) %e    - %e   - 2 %e )
(%o27)            ------------------------------------------
                                    10     2
                                  %e   - %e
(%i28) subst ( t=1, us), ratsimp;
(%o28)                                - 1
(%i29) subst ( t=3, us), ratsimp;
(%o29)                                 2
(%i30) plot2d ( us, [t,0,4], [y,-2,8],
           [style,[lines,5]],[ylabel," "],
           [xlabel," U versus t, U''(t) = 4 U(t), U(1) = -1, U(3) = 2 "])$
\end{myVerbatim}
\newpage
\noindent which produces the plot
\smallskip
\begin{figure} [h]  
   \centerline{\includegraphics[scale=.7]{ch1p34.eps} }
	\caption{Solution for $\mathbf{u(1) = -1, \; u(3) = 2 }$ }
\end{figure}

%plot2d ( us, [t,0,4], [y,-2,8],
%           [style,[lines,8]],[ylabel," "],
%           [xlabel," U versus t, U''(t) = 4 U(t), U(1) = -1, U(3) = 2 "],
%           [psfile,"ch1p34.eps"])$
\noindent The simplest case of using \textbf{desolve} is the case in which you impose conditions
  on the solution and its first derivative at \mv|t = 0|, in which case you simply
  use:
\begin{myVerbatim}
(%i4) psoln : subst([u(0) = 1,v(0)=0],gsoln);
                                     2 t     - 2 t
                                   %e      %e
(%o4)                       u(t) = ----- + -------
                                     2        2
(%i5) us : rhs(psoln);
                                  2 t     - 2 t
                                %e      %e
(%o5)                           ----- + -------
                                  2        2
\end{myVerbatim}
  in which we have chosen the initial conditions \mv|u(0) = 1|, and
   \mv|v(0) = 0|.
\newpage
\subsubsection{Numerical Solution and Plot with \textbf{plotdf}}
Given a second order autonomous ODE, one needs to introduce a second
  dependent variable \mv|v(t)|, say, which is defined as the first
  derivative of the original single dependent variable \mv|u(t)|.
Then for our example, the starting ODE
\begin{equation}
\mathbf{\frac{d^{2}\,u}{d\,t^{2}} = 4\, u}
\end{equation}
  is converted into two first order ODE's
\begin{equation}
\mathbf{\frac{d\,u}{d\,t} = v, \quad \frac{d\,v}{d\,t} = 4\,u}
\end{equation}
and the \textbf{plotdf} syntax for two first order ODE's is
\begin{myVerbatim2}
    plotdf ( [dudt, dvdt], [u, v], [trajectory_at, u0, v0], [u, umin, umax],
	   [v, vmin, vmax], [tinitial, t0], [versus_t, 1],
	   [tstep, timestepval], [nsteps, nstepsvalue] )$
\end{myVerbatim2}
  in which at \mv|t = t0|, \mv|u = u0| and \mv|v = v0|.
If \mv|t0 = 0| you can omit the option \mv|[tinitial, t0]|.
The options \mv|[u, umin, umax]| and \mv|[v, vmin, vmax]| allow you to control
  the horizontal and vertical extent of the phase space plot (here \mv|v| versus \mv|u|) which will
  be produced. 
The option \mv|[versus_t,1]| tells \textbf{plotdf} to create a separate plot
  of both \mv|u| and \mv|v| versus the dependent variable.
The last two options are only needed if you are not satisfied with the
  plots and want to experiment with other than the default values of
  \mv|tstep| and \mv|nsteps|.\\
  
\noindent Another option you can add is \mv|[direction,forward]|, which will display the
  trajectory for \mv|t| greater than or equal to  \mv|t0|, rather than for a default interval
  around the value \mv|t0| which corresponds to \mv|[direction,both]|.  
  
\noindent Here we invoke \textbf{plotdf} for our example.
\begin{myVerbatim}
(%i1) plotdf ( [v, 4*u], [u, v], [trajectory_at, 1, 0],
            [u, 0, 8], [v, -10, 10], [versus_t, 1],
              [tinitial, 2])$
\end{myVerbatim}
\noindent The plot versus \mv|t| is
\smallskip
\begin{figure} [h]  
   \centerline{\includegraphics[scale=.4]{ch1p36.eps} }
	\caption{u(t) and u'(t) vs. $\mathbf{t}$ for $\mathbf{u(2) = 1, \; u'(2) = 0 }$ }
\end{figure}

\newpage
\noindent and the phase space plot is
\smallskip
\begin{figure} [h]  
   \centerline{\includegraphics[scale=.4]{ch1p35.eps} }
	\caption{u'(t) vs. u(t) for  $\mathbf{u(2) = 1, \; u'(2) = 0 }$ }
\end{figure}

\noindent In both of these plots we used the \mv|Config| menu to increase the linewidth,
  and then clicked on \mv|Replot|.
We also cut and pasted the color \mv|red| to be the second choice on the color cycle 
  (instead of green) used in the plot versus the independent variable \mv|t|.
Note that no matter what you call your independent variable, it will always be
  called \mv|t| on the plot of the dependent variables versus the independent
  variable.
\subsubsection{Numerical Solution with 4th Order Runge-Kutta: \textbf{rk}}
To use the fourth order Runge-Kutta numerical integrator \textbf{rk} for
  this example, we need to follow the procedure used in the previous
  section using \textbf{plotdf}, converting the second order ODE
  to a pair of first order ODE's.\\
  
\noindent The syntax for two first order ODE's with dependent variables
 \mv|[u,v]| and independent variable \mv|t| is
\begin{myVerbatim2}
   rk ( [ dudt, dvdt ], [u,v], [u0,v0], [t, t0, tmax, dt] )
\end{myVerbatim2}
  which will produce the list of lists:
\begin{myVerbatim2}
   [ [t0, u0,v0],[t0+dt, u(t0+dt),v(t0+dt)], ..., [tmax, u(tmax),v(tmax)] ]
\end{myVerbatim2}
For our example, following our discussion in the previous section with \textbf{plotdf},
  we use
\begin{myVerbatim2}
  points : rk ( [v, 4*u], [u, v], [1, 0], [t, 2, 3.6, 0.01] )
\end{myVerbatim2}
We again use the homemade function \mv|fll| (see the preface) to look at the first element, the last
  element, and the length of various lists.
\begin{myVerbatim}
(%i1) fpprintprec:8$
(%i2) points : rk([v,4*u],[u,v],[1,0],[t,2,3.6,0.01])$
(%i3) %, fll;
(%o3)            [[2, 1, 0], [3.6, 12.286646, 24.491768], 161]
(%i4) uL : makelist([points[i][1],points[i][2]],i,1,length(points))$
(%i5) %, fll;
(%o5)                   [[2, 1], [3.6, 12.286646], 161]
\end{myVerbatim}
\newpage
\begin{myVerbatim}
(%i6) vL : makelist([points[i][1],points[i][3]],i,1,length(points))$
(%i7) %, fll;
(%o7)                   [[2, 0], [3.6, 24.491768], 161]
(%i8) plot2d([ [discrete,uL],[discrete,vL]],[x,1,5],
           [style,[lines,5]],[y,-1,24],[ylabel," "],
           [xlabel,"t"],[legend,"u(t)","v(t)"])$
\end{myVerbatim}
%\newpage
\noindent which produces the plot
\smallskip
\begin{figure} [h]  
   \centerline{\includegraphics[scale=.5]{ch1p37.eps} }
	\caption{Runge-Kutta for  $\mathbf{u(2) = 1, \; u'(2) = 0 }$ }
\end{figure}

\noindent Next we make a phase space plot of \mv|v| versus \mv|u| from
  the result of the Runge-Kutta integration.
\begin{myVerbatim}
(%i9) uvL : makelist([points[i][2],points[i][3]],i,1,length(points))$
(%i10) %, fll;
(%o10)                [[1, 0], [12.286646, 24.491768], 161]
(%i11) plot2d( [ [discrete,uvL]],[x,0,13],[y,-1,25],
                [style,[lines,5]],[ylabel," "],
                [xlabel," v vs. u "])$
\end{myVerbatim}
%\newpage
\noindent which produces the phase space plot
\smallskip
\begin{figure} [h]  
   \centerline{\includegraphics[scale=.5]{ch1p38.eps} }
	\caption{R-K Phase Space Plot for  $\mathbf{u(2) = 1, \; u'(2) = 0 }$ }
\end{figure}

%plot2d( [ [discrete,uvL]],[x,0,13],[y,-1,25],
%                [style,[lines,8]],[ylabel," "],
%                [xlabel," v vs. u "],
%                [psfile,"ch1p38.eps"])$
  

\newpage
\subsection{Examples of ODE Solutions}  
\subsubsection{Ex. 1: Fall in Gravity with Air Friction: Terminal Velocity}
Let's explore a problem posed by Patrick T. Tam (A Physicist's Guide to Mathematica, Academic
  Press, 1997, page 349).
\begin{quote}
A small body falls downward with an initial velocity $\mathbf{v_{0}}$ from a height $\mathbf{h}$
  near the surface of the earth.
For low velocities (less than about $\mathbf{24 \,m/s}$), the effect of air resistance may be
  approximated by a frictional force proportional to the velocity.
Find the displacement and velocity of the body, and determine the terminal velocity.
Plot the speed as a function of time for several initial velocities.
\end{quote}
The net vector force $\mathbf{F}$ acting on the object is thus assumed to be the 
 (constant) force of gravity and the (variable) force due to air friction, 
  which is in a direction opposite to the direction of the velocity vector $\mathbf{v}$.
We can then write Newton's Law of motion in the form
\begin{equation}
\mathbf{F} = m \,\mathbf{g} - b \, \mathbf{v} = m\,\frac{d\,\mathbf{v}}{d\,t}
\end{equation}
In this vector equation, $m$ is the mass in kg., $\mathbf{g}$ is a vector pointing downward
   with magnitude $g$, and $b$ is a positive constant
	which depends on the size and shape of the object and on the viscosity of the air.
The velocity vector $\mathbf{v}$ points down during the fall.\\
	
\noindent If we choose the $y$ axis positive downward, with the point $y = 0$ the
  launch point, then  the net $y$ components of the  force and Newton's Law of motion are:
\begin{equation}
F_{y} =  m \,g - b \,v_{y}  = m\,\frac{d\,v_{y}}{d\,t}
\end{equation}
   where $g$ is the positive number $\;9.8 \,m/s^{2}$ and since the
   velocity component  $v_{y} > 0$ during the fall, the effects of gravity and 
   air resistance are in competition.\\
   
\noindent We see that the rate of change of velocity will become zero at the
  instant that $m \,g - b \,v_{y} = 0$, or $v_{y} = m \,g/b$, and the downward
  velocity stops increasing at that moment, the "\textbf{terminal velocity}" having been
  attained.\\
   
\noindent While working with Maxima, we can simplify our notation and
  let $v_{y} \rightarrow v$ and $(b/m) \rightarrow a$ so both $v$ and $a$ represent
  positive numbers.
We then use Maxima to solve the equation $d\,v/d\,t = g - a\,v$.  
The dimension of each term of this equation must evidently be the dimension of $v/t$,
  so $a$ has dimension $1/t$.
%\newpage
\begin{myVerbatim}
(%i1) de : 'diff(v,t) - g + a*v;
                                 dv
(%o1)                            -- + a v - g
                                 dt
(%i2) gsoln : ode2(de,v,t);
                                           a t
                                - a t  g %e
(%o2)                     v = %e      (------- + %c)
                                          a
(%i3) de, gsoln, diff,ratsimp;
(%o3)                                  0
\end{myVerbatim}
We then use \textbf{ic1} to get a solution such that $v = v0$ when $t = 0$.
\begin{myVerbatim}
(%i4) psoln : expand ( ic1 (gsoln,t = 0, v = v0 ) );
                                             - a t
                              - a t      g %e        g
(%o4)                   v = %e      v0 - --------- + -
                                             a       a
(%i5) vs : rhs(psoln);
                                          - a t
                           - a t      g %e        g
(%o5)                    %e      v0 - --------- + -
                                          a       a
\end{myVerbatim}
For consistency, we must get the correct \textbf{terminal speed} for large $t$:
\begin{myVerbatim}
(%i6) assume(a>0)$
(%i7) limit( vs, t, inf );
                                       g
(%o7)                                  -
                                       a
\end{myVerbatim}
which agrees with our analysis.\\

\noindent To make some plots, we can introduce a dimensionless time $u$ with the
  replacement $t \rightarrow u = a\,t$, and a dimensionless speed $w$ with the
  replacement $v \rightarrow w = a\,v/g$.
\begin{myVerbatim}
(%i8) expand(vs*a/g);
                             - a t
                         a %e      v0     - a t
(%o8)                    ------------ - %e      + 1
                              g
(%i9) %,[t=u/a,v0=w0*g/a];
                              - u        - u
(%o9)                       %e    w0 - %e    + 1
(%i10) ws : collectterms (%, exp (-u));
                                - u
(%o10)                        %e    (w0 - 1) + 1
\end{myVerbatim}
As our dimensionless time $u$ gets large, $\mathbf{ws \rightarrow 1}$, which is the value of the
  terminal speed in dimensionless units.\\
  
\noindent Let's now plot three cases, two cases with initial speed less than terminal speed
  and one case with initial speed greater than the terminal speed.
(The use of dimensionless units for plots generates what are called ``universal curves'', since
  they are generally valid, no matter what the actual numbers are).
\begin{myVerbatim}
(%i11) plot2d([[discrete,[[0,1],[5,1]]],subst(w0=0,ws),subst(w0=0.6,ws),
                    subst(w0=1.5,ws)],[u,0,5],[y,0,2],
           [style,[lines,2,7],[lines,4,1],[lines,4,2],[lines,4,3]],
           [legend,"terminal speed", "w0 = 0", "w0 = 0.6", "w0 = 1.5"],
         [ylabel, " "],
         [xlabel, " dimensionless speed w vs dimensionless time u"])$
\end{myVerbatim}  
%\newpage
\noindent which produces:
\smallskip
\begin{figure} [h]  
   \centerline{\includegraphics[scale=.5]{ch1p13.eps} }
	\caption{Dimensionless Speed Versus Dimensionless Time }
\end{figure}

%plot2d([[discrete,[[0,1],[5,1]]],subst(w0=0,ws),subst(w0=0.6,ws),
%           subst(w0=1.5,ws)],[u,0,5],[y,0,2],
%           [style,[lines,3,4],[lines,6,1],[lines,6,2],[lines,6,3]],
%           [legend,"terminal speed", "w0 = 0", "w0 = 0.6", "w0 = 1.5"],
%           [ylabel, " "],[xlabel, " dimensionless speed w vs dimensionless time u "],
%           [psfile,"ch1p13.eps"])$
\newpage
\noindent An object thrown down with an initial speed greater than the terminal
  speed (as in the top curve) slows down until its speed is the terminal speed.\\
  
\noindent Thus far we have been only concerned with
  the relation between velocity and time. 
We can now focus on the implications for distance versus time.
A dimensionless length $\mathbf{z}$ is $\mathbf{a^{2} \,y/g}$ and the relation
  $\mathbf{d\,y/d\,t = v}$ becomes $\mathbf{d\,z/d\,u = w}$, or $\mathbf{d\,z = w\,d\,u}$, which can
  be integrated over corresponding intervals: $\mathbf{z}$ over the interval $\mathbf{[0, z_{f}]}$,
  and $\mathbf{u}$ over the interval $\mathbf{[0, u_{f}]}$.
\begin{myVerbatim}
(%i12) integrate(1,z,0,zf) = integrate(ws,u,0,uf);
                            - uf            uf
(%o12)             zf = - %e     (w0 - uf %e   - 1) + w0 - 1
(%i13) zs : expand(rhs(%)),uf = u;
                            - u             - u
(%o13)                  - %e    w0 + w0 + %e    + u - 1
(%i14) zs, u=0;
(%o14)                                 0
\end{myVerbatim}
(Remember the object is launched at $\mathbf{y = 0}$ which means at $\mathbf{z = 0}$).
Let's make a plot of distance travelled vs time (dimensionless units)
  for the three cases considered above.
\begin{myVerbatim}
(%i15) plot2d([subst(w0=0,zs),subst(w0=0.6,zs),
          subst(w0=1.5,zs)],[u,0,1],[style,[lines,4,1],[lines,4,2],
		  [lines,4,3]], [legend,"w0 = 0", "w0 = 0.6", "w0 = 1.5"],
          [ylabel," "],
          [xlabel,"dimensionless distance z vs dimensionless time u"],
          [gnuplot_preamble,"set key top left;"])$
\end{myVerbatim}
%\newpage
\noindent which produces:
\smallskip
\begin{figure} [h]  
   \centerline{\includegraphics[scale=.8]{ch1p14.eps} }
	\caption{Dimensionless Distance Versus Dimensionless Time }
\end{figure}

%plot2d([subst(w0=0,zs),subst(w0=0.6,zs),
%          subst(w0=1.5,zs)],[u,0,1],[style,[lines,4,1],[lines,4,2],
%		  [lines,4,3]], [legend,"w0 = 0", "w0 = 0.6", "w0 = 1.5"],
%          [ylabel," "],
%          [xlabel,"dimensionless distance z vs dimensionless time u"],
%          [gnuplot_preamble,"set key top left;"],
%           [psfile,"ch1p14.eps"])$

\newpage
\subsubsection{Ex. 2: One Nonlinear First Order ODE}
Let's solve 
\begin{equation}
\mathbf{x^{2}\,y\,\frac{d\,y}{d\,x}  = x\,y^{2} + x^{3} - 1 }
\end{equation}
for a solution such that when $\mathbf{x = 1, \; y = 1}$.
\begin{myVerbatim}
(%i1) de : x^2*y*'diff(y,x) - x*y^2 - x^3 + 1;
                             2   dy      2    3
(%o1)                       x  y -- - x y  - x  + 1
                                 dx
(%i2) gsoln : ode2(de,y,x);
                              2      3
                         3 x y  - 6 x  log(x) - 2
(%o2)                    ------------------------ = %c
                                      3
                                   6 x
(%i3) psoln : ic1(gsoln,x=1,y=1);
                              2      3
                         3 x y  - 6 x  log(x) - 2   1
(%o3)                    ------------------------ = -
                                      3             6
                                   6 x
\end{myVerbatim}
This implicitly determines $\mathbf{y}$ as a function of the independent variable $\mathbf{x}$
By inspection, we see that $\mathbf{x = 0}$ is a singular point we should
  stay away from, so we assume from now on that $\mathbf{x \neq 0}$.  \\
  
\noindent To look at \tcbr{explicit} solutions \textbf{y(x)} we use \textbf{solve},
  which returns a list of two expressions depending on \textbf{x}.  
Since the \tcbr{implicit} solution is a quadratic in \textbf{y}, we will
  get two solutions from \textbf{solve}, which we call \textbf{y1} and \textbf{y2}.
\begin{myVerbatim}
(%i4) [y1,y2] : map('rhs, solve(psoln,y) );
                     2           2   2           2           2   2
             sqrt(6 x  log(x) + x  + -)  sqrt(6 x  log(x) + x  + -)
                                     x                           x
(%o4)     [- --------------------------, --------------------------]
                      sqrt(3)                     sqrt(3)
(%i5) [y1,y2], x = 1, ratsimp;
(%o5)                              [- 1, 1]
(%i6) de, diff, y= y2, ratsimp;
(%o6)                                  0
\end{myVerbatim}
We see from the values at $\mathbf{x=1}$ that $\mathbf{y2}$ is the particular solution
  we are looking for, and we have checked that $\mathbf{y2}$ satisfies the original
  differential equation.
From this example, we learn the lesson that \textbf{ic1} sometimes needs 
  some help in finding the particular solution we are looking for.\\ 

\noindent Let's make a plot of the two solutions found.
\begin{myVerbatim}
(%i7) plot2d([y1,y2],[x,0.01,5],
        [style,[lines,5]],[ylabel, " Y "],
          [xlabel," X "] , [legend,"Y1", "Y2"],
         [gnuplot_preamble,"set key bottom center;"])$
\end{myVerbatim}
\newpage
\noindent which produces:
\smallskip
\begin{figure} [h]  
   \centerline{\includegraphics[scale=.6]{ch1p20.eps} }
	\caption{ Positive X Solutions }
\end{figure}

%plot2d([y1,y2],[x,0.01,5],
%        [style,[lines,8]],[ylabel, " Y "],
%          [xlabel," X "] , [legend,"Y1", "Y2"],
%         [gnuplot_preamble,"set key bottom center;"],
%         [psfile,"ch1p20.eps"])$

\subsubsection{Ex. 3: One First Order ODE Which is Not Linear in Y'}
The differential equation to solve is
\begin{equation}
\mathbf{\left( \frac{d\,x}{d\,t} \right)^{2} + 5\,x^{2} = 8 }
\end{equation}
  with the initial conditions $\mathbf{t = 0, \quad x = 0}$.
\begin{myVerbatim}
(%i1) de: 'diff(x,t)^2 + 5*x^2 - 8;
                                dx 2      2
(%o1)                          (--)  + 5 x  - 8
                                dt
(%i2) ode2(de,x,t);
                                dx 2      2
(%t2)                          (--)  + 5 x  - 8
                                dt

                     first order equation not linear in y'

(%o2)                                false
\end{myVerbatim}
We see that direct use of \textbf{ode2} does not succeed. 
We can use \textbf{solve} to get equations which are linear in
  the first derivative, and then using \textbf{ode2} on each of the resulting
  linear ODE's.
\begin{myVerbatim}
(%i3) solve(de,'diff(x,t));
                  dx                 2   dx               2
(%o3)            [-- = - sqrt(8 - 5 x ), -- = sqrt(8 - 5 x )]
                  dt                     dt
(%i4) ode2 ( %[2], x, t );
                                  5 x
                          asin(----------)
                               2 sqrt(10)
(%o4)                     ---------------- = t + %c
                              sqrt(5)
\end{myVerbatim}
\newpage
\begin{myVerbatim}
(%i5) solve(%,x);
                     2 sqrt(10) sin(sqrt(5) t + sqrt(5) %c)
(%o5)           [x = --------------------------------------]
                                       5
(%i6) gsoln2 : %[1];
                     2 sqrt(10) sin(sqrt(5) t + sqrt(5) %c)
(%o6)            x = --------------------------------------
                                       5
(%i7) trigsimp ( ev (de,gsoln2,diff ) );
(%o7)                                 0
(%i8) psoln : ic1 (gsoln2, t=0, x=0);
solve: using arc-trig functions to get a solution.
Some solutions will be lost.
                            2 sqrt(10) sin(sqrt(5) t)
(%o8)                   x = -------------------------
                                        5
(%i9) xs : rhs(psoln);
                          2 sqrt(10) sin(sqrt(5) t)
(%o9)                     -------------------------
                                      5
(%i10) xs,t=0;
(%o10)                                 0
\end{myVerbatim}
We have selected only one of the linear ODE's to concentrate on here.
We have shown that the solution satisfies the original differential
  equation and the given boundary condition.
\subsubsection{Ex. 4: Linear Oscillator with Damping }
The equation of motion for a particle of mass \textbf{m} executing one dimensional
  motion which is subject to a linear restoring force proportional to \mv+|x|+
  and subject to a frictional force proportional to its speed is
\begin{equation}
\mathbf{m\,\frac{d^{2}\,x}{d\,t^{2}} + b\,\frac{d\,x}{d\,t} + k \,x = 0}
\end{equation}
Dividing by the mass \textbf{m}, we note that if there were no damping, this
  motion would reduce to a linear oscillator with the angular frequency 
\begin{equation}
\mathbf{\boldsymbol{\omega}_{0} = \left(\frac{k}{m}\right)^{1/2}.}
\end{equation}
In the presence of damping, we can define
\begin{equation}
\mathbf{\boldsymbol{\gamma} = \frac{b}{2\,m} }
\end{equation}
  and the equation of motion becomes
\begin{equation}
\mathbf{\frac{d^{2}\,x}{d\,t^{2}} + 2\,\boldsymbol{\gamma}\,\frac{d\,x}{d\,t} + \boldsymbol{\omega}_{0}^{2} \,x = 0}
\end{equation}
In the presence of damping, there are now two natural time scales
\begin{equation}
\mathbf{ t1 = \frac{1}{\boldsymbol{\omega}_{0}}, \qquad  t2 = \frac{1}{\boldsymbol{\gamma}} }
\end{equation}
  and we can introduce a dimensionless time $\mathbf{\boldsymbol{\theta} = \boldsymbol{\omega}_{0}\,t}$
  and the dimensionless positive constant $\mathbf{a = \boldsymbol{\gamma/\omega_{0}}}$, to get
\begin{equation}
\mathbf{\frac{d^{2}\,x}{d\,\boldsymbol{\theta}^{2}} + 2\,a\,\frac{d\,x}{d\,\boldsymbol{\theta}} + x = 0}
\end{equation}
\newpage
\noindent The ``underdamped'' case corresponds to $\mathbf{\boldsymbol{\gamma} < \boldsymbol{\omega}_{0}}$,
  or $\mathbf{a < 1}$ and results in damped oscillations around the final $\mathbf{x = 0}$.
The ``critically damped'' case corresponds to $\mathbf{a = 1}$, and the ``overdamped'' case
  corresponds to $\mathbf{a > 1}$.
We specialize to solutions which have the initial conditions
  $\mathbf{\boldsymbol{\theta} = 0, \qquad x = 1, \qquad d\,x/d\,t = 0 \; \Rightarrow d\,x/d\,\boldsymbol{\theta} = 0  }$.
\begin{myVerbatim}
(%i1) de : 'diff(x,th,2) + 2*a*'diff(x,th) + x ;
                               2
                              d x        dx
(%o1)                         ---- + 2 a --- + x
                                 2       dth
                              dth
(%i2) for i thru 3 do
          x[i] : rhs ( ic2 (ode2 (subst(a=i/2,de),x,th), th=0,x=1,diff(x,th)=0))$
(%i3) plot2d([x[1],x[2],x[3]],[th,0,10],
          [style,[lines,4]],[ylabel," "],
           [xlabel," Damped Linear Oscillator " ],
           [gnuplot_preamble,"set zeroaxis lw 2"],
           [legend,"a = 0.5","a = 1","a = 1.5"])$
\end{myVerbatim}
%\newpage
\noindent which produces
\smallskip
\begin{figure} [h]  
   \centerline{\includegraphics[scale=.8]{ch1p19.eps} }
	\caption{ Damped Linear Oscillator }
\end{figure}

%plot2d([x[1],x[2],x[3],[discrete,[[0,0],[10,0]]]],[th,0,10],
%          [style,[lines,5,1],[lines,5,2],[lines,5,3],[lines,3,4]],
%          [ylabel," "],[xlabel," Damped Linear Oscillator: x vs theta " ],
%           [psfile,"ch1p19.eps"],
%           [legend,"a = 0.5","a = 1","a = 1.5"," x = 0 "])$
\noindent and illustrates why engineers seek the critical damping case,
  which brings the system to $\mathbf{x = 0}$ most rapidly.
\newpage  
\noindent Now for a phase space plot with \mv|dx/dth| versus \mv|x|,
  drawn for the underdamped case:
\begin{myVerbatim}
(%i4) v1 : diff(x[1],th)$
(%i5) fpprintprec:8$
(%i6) [x5,v5] : [x[1],v1],th=5,numer;
(%o6)                      [- 0.0745906, 0.0879424]
(%i7) plot2d ( [ [parametric, x[1], v1, [th,0,10],[nticks,80]],
             [discrete,[[1,0]]], [discrete,[ [x5,v5] ] ] ],
         [x, -0.4, 1.2],[y,-0.8,0.2], [style,[lines,3,7],
                   [points,3,2,1],[points,3,6,1] ],
          [ylabel," "],[xlabel,"th = 0, x = 1, v = 0"],
          [legend," v vs x "," th = 0 "," th = 5 "])$
\end{myVerbatim}
%\newpage
\noindent which shows
\smallskip
\begin{figure} [h]  
   \centerline{\includegraphics[scale=.7]{ch1p21.eps} }
	\caption{Underdamped Phase Space Plot }
\end{figure}

%plot2d ( [ [parametric, x[1], v1, [th,0,10],[nticks,80]],
%             [discrete,[[1,0]]], [discrete,[ [x5,v5] ] ] ],
%         [x, -0.4, 1.2],[y,-0.8,0.2], [style,[lines,6,4],
%                   [points,5,2,1],[points,5,6,1] ],
%          [ylabel," "],[xlabel,"th = 0, x = 1, v = 0"],
%          [legend," v vs x "," th = 0 "," th = 5 "],
%          [psfile,"ch1p21.eps"])$
\subsubsection*{Using plotdf for the Damped Linear Oscillator}
Let's use \textbf{plotdf} to show the phase space plot of our underdamped
  linear oscillator, using the syntax
\begin{myVerbatim2}
   plotdf ( [dudt, dvdt],[u,v], options...)
\end{myVerbatim2}
  which requires that we convert our single second order ODE to an equivalent pair of
  first order ODE's.
If we let $\mathbf{d\,x/d\,\boldsymbol{\theta} = v}$, assume the dimensionless
  damping parameter $\mathbf{a = 1/2}$, we then have  $\mathbf{d\,v/d\,\boldsymbol{\theta} = -v -x}$,
  and we use the \textbf{plotdf} syntax
\begin{myVerbatim2}
   plotdf ( [dxdth, dvdth], [x, v], options... ).
\end{myVerbatim2}
One has to experiment with the number of steps, the step size, and
  the horizontal and vertical views.
The $\mathbf{v(\boldsymbol{\theta})}$ values determine the vertical position 
   and the $\mathbf{x(boldsymbol{\theta})}$ values determine
  the horizontal position of a point on the phase space plot curve.
The symbols used for the horizontal and vertical ranges should correspond
  to the symbols used in the second argument (here \mv|[x,v]|).
Since we want to get a phase space plot which agrees with our work above,
  we require the trajectory begin at $\mathbf{\boldsymbol{\theta} = 0, \quad x = 1, \quad v = 0}$,
   and we integrate forward in dimensionless time $\boldsymbol{\theta}$.
\begin{myVerbatim}
(%i8) plotdf([v,-v-x],[x,v],[trajectory_at,1,0],
           [direction,forward],[x,-0.4,1.2],[v,-0.6,0.2],
            [nsteps,400],[tstep,0.01])$
\end{myVerbatim}
This will bring up the phase space plot \textbf{v} vs. \textbf{x}, and you can thicken the
  red curve by clicking the \textbf{Config} button (which brings up the
  \textbf{Plot Setup} panel), increasing the \textbf{linewidth} to 3, and then
  clicking \mv|ok|.
To actually see the thicker line, you must then click on the \textbf{Replot} button.
This plot is
\smallskip
\begin{figure} [h]  
   \centerline{\includegraphics[scale=.4]{ch1p22.eps} }
	\caption{Underdamped Phase Space Plot Using plotdf }
\end{figure}

\noindent To see the separate curves $\mathbf{v(\boldsymbol{\theta})}$ and $\mathbf{x(\boldsymbol{\theta})}$,
  you can click on the \textbf{Plot Versus t} button. (The symbol \mv|t| is simply a placeholder
  for the independent variable, which in our case is $\boldsymbol{\theta}$.)
Again, you can change the linewidth and colors (we changed green to red) via the \textbf{Config}
  and \textbf{Replot} button process, which yields
\smallskip
\begin{figure} [h]  
   \centerline{\includegraphics[scale=.4]{ch1p23.eps} }
	\caption{$\mathbf{x(\boldsymbol{\theta})}$ and $\mathbf{v(\boldsymbol{\theta})}$ Using plotdf }
\end{figure}
\subsubsection{Ex. 5: Underdamped Linear Oscillator with Sinusoidal Driving Force}
We extend our previous oscillator example by adding a sinusoidal driving force.
The equation of motion is now
\begin{equation}
\mathbf{m\,\frac{d^{2}\,x}{d\,t^{2}} + b\,\frac{d\,x}{d\,t} + k \,x = A\,\boldsymbol{\cos(\omega}\,t)}
\end{equation}
We again divide by the mass \textbf{m} and let
\begin{equation}
\mathbf{\boldsymbol{\omega}_{0} = \left(\frac{k}{m}\right)^{1/2}.}
\end{equation}
As before, we define
\begin{equation}
\mathbf{\boldsymbol{\gamma} = \frac{b}{2\,m}. }
\end{equation}
Finally, let $\mathbf{B = A/m}$.
The equation of motion becomes
\begin{equation}
\mathbf{\frac{d^{2}\,x}{d\,t^{2}} + 2\,\boldsymbol{\gamma}\,\frac{d\,x}{d\,t} +
   \boldsymbol{\omega}_{0}^{2} \,x = B\,\boldsymbol{\cos(\omega}\,t)}
\end{equation}
There are now three natural time scales
\begin{equation}
\mathbf{ t1 = \frac{1}{\boldsymbol{\omega}_{0}}, \quad  t2 = \frac{1}{\boldsymbol{\gamma}},
  \quad t3 = \frac{1}{\boldsymbol{\omega}} }
\end{equation}
  and we can introduce a dimensionless time $\mathbf{\boldsymbol{\theta} = \boldsymbol{\omega}_{0}\,t}$,
  the dimensionless positive damping constant $\mathbf{a = \boldsymbol{\gamma/\omega_{0}}}$,
   the dimensionless oscillator displacement $\mathbf{y = x/B}$, and the dimensionless
   driving angular frequency $\mathbf{q = \boldsymbol{\omega}/\boldsymbol{\omega}_{0}}$   to get
\begin{equation}
\mathbf{\frac{d^{2}\,y}{d\,\boldsymbol{\theta}^{2}} +
   2\,a\,\frac{d\,y}{d\,\boldsymbol{\theta}} + y = \boldsymbol{\cos(q\,\theta})}
\end{equation}
The ``underdamped'' case corresponds to $\mathbf{\boldsymbol{\gamma} < \boldsymbol{\omega}_{0}}$,
  or $\mathbf{a < 1}$, and we specialize to the case $\mathbf{a = 1/2}$.
\begin{myVerbatim}
(%i1) de : 'diff(y,th,2) + 'diff(y,th) + y - cos(q*th);
                           2
                          d y    dy
(%o1)                     ---- + --- + y - cos(q th)
                             2   dth
                          dth
(%i2) gsoln : ode2(de,y,th);
                              2
          q sin(q th) + (1 - q ) cos(q th)
(%o2) y = --------------------------------
                     4    2
                    q  - q  + 1
                             - th/2          sqrt(3) th            sqrt(3) th
                         + %e       (%k1 sin(----------) + %k2 cos(----------))
                                                 2                     2
(%i3) psoln : ic2(gsoln,th=0,y=1,diff(y,th)=0);
                              2
          q sin(q th) + (1 - q ) cos(q th)
(%o3) y = --------------------------------
                     4    2
                    q  - q  + 1
                             4      2      sqrt(3) th        4     sqrt(3) th
                           (q  - 2 q ) sin(----------)      q  cos(----------)
                - th/2                         2                       2
            + %e       (--------------------------------- + ------------------)
                                 4            2                 4    2
                        sqrt(3) q  - sqrt(3) q  + sqrt(3)      q  - q  + 1
\end{myVerbatim}
\newpage
\noindent We now specialize to a high (dimensionless) driving angular frequency case,
  $\mathbf{q = 4}$, which means that we are assuming that the actual driving
  angular frequency is four times as large as the natural angular frequency of this oscillator.
\begin{myVerbatim}
(%i4) ys : subst(q=4,rhs(psoln));
                        sqrt(3) th            sqrt(3) th
                224 sin(----------)   256 cos(----------)
        - th/2              2                     2
(%o4) %e       (------------------- + -------------------)
                    241 sqrt(3)               241
                                                     4 sin(4 th) - 15 cos(4 th)
                                                   + --------------------------
                                                                241
(%i5) vs : diff(ys,th)$
\end{myVerbatim}
We now plot both the dimensionless oscillator amplitude and the dimensionless oscillator
  velocity on the same plot.
\begin{myVerbatim}
(%i6) plot2d([ys,vs],[th,0,12],
        [nticks,100],
        [style,[lines,5]],
        [legend," Y "," V "],
        [xlabel," dimensionless Y and V vs. theta"])$
\end{myVerbatim}
\smallskip
\begin{figure} [h]  
   \centerline{\includegraphics[scale=.6]{ch1p24.eps} }
	\caption{Dimensionless Y and V versus Dimensionless Time $\boldsymbol{\theta}$}
\end{figure}

%plot2d([ys,vs],[th,0,12],
%        [nticks,100],
%        [style,[lines,9]],
%        [legend," Y "," V "],
%        [xlabel," dimensionless Y and V vs. theta"],
%        [psfile,"ch1p24.eps"])$
We see that the driving force soon dominates the motion of the  underdamped linear 
  oscillator, which is forced to oscillate at the driving frequency.
This dominance evidently has nothing to do with the actual strength \textbf{A newtons}
  of the peak driving force, since we are solving for a dimensionless oscillator amplitude, and
  we get the same qualitative curve no matter what the size of \textbf{A} is.  
\newpage
\noindent We next make a phase space plot for the early ``capture'' part of the motion
  of this system. 
(Note that \textbf{plotdf} cannot numerically integrate this differential equation because
  of the explicit appearance of the dependent variable.)  
\begin{myVerbatim}
(%i7) plot2d([parametric,ys,vs,[th,0,8]],
           [style,[lines,5]],[nticks,100],
            [xlabel," V (vert) vs. Y (hor) "])$
\end{myVerbatim}
\smallskip
\begin{figure} [h]  
   \centerline{\includegraphics[scale=.6]{ch1p25.eps} }
	\caption{Dimensionless V Versus Dimensionless Y: Early History}
\end{figure}

%plot2d([parametric,ys,vs,[th,0,8]],
%           [style,[lines,9]],[nticks,100],
%            [xlabel," V (vert) vs. Y (hor) "],
%            [psfile,"ch1p25.eps"])$
We see the phase space plot being driven to regular oscillations about \textbf{y = 0}
  and \textbf{v = 0}.
\subsubsection{Ex. 6: Regular and Chaotic Motion of a Driven Damped Planar Pendulum}
The motion is pure rotation in a fixed plane (one degree of freedom), and if the pendulum is a simple
  pendulum with all the mass \textbf{m} concentrated at the end of a weightless support of length
  \textbf{L}, then the moment of inertia about the support point is
  $\mathbf{I = m\,L^{2}}$, and the angular acceleration is $\mathbf{\boldsymbol{\alpha}}$,
  and rotational dynamics implies the equation of motion
\begin{equation}
\mathbf{I\,\boldsymbol{\alpha} = m\,L^{2}\,\frac{d^{2}\boldsymbol{\theta}}{d\,t^{2}} = 
    \boldsymbol{\tau}_{z} = - m\,g\,L\,\boldsymbol{\sin \theta} - 
	c\, \frac{d\,\boldsymbol{\theta}}{d\,t} + A\,\boldsymbol{\cos}(\boldsymbol{\omega}_{d}\,t)}
\end{equation}
  We introduce a dimensionless time $\mathbf{\boldsymbol{\tau} = \boldsymbol{\omega}_{0}\,t}$
  and a dimensionless driving angular frequency
  $\mathbf{ \boldsymbol{\omega}  =  \boldsymbol{\omega}_{d} / \boldsymbol{\omega}_{0} }$,
  where $\mathbf{\boldsymbol{\omega}_{0}^{2} = g/L}$, to get the
equation of motion
\begin{equation}
\mathbf{  \frac{d^{2}\boldsymbol{\theta}}{d\,\boldsymbol{\tau}^{2}} = 
         - \boldsymbol{\sin \theta} - a\, \frac{d\,\boldsymbol{\theta}}{d\,\boldsymbol{\tau}} + 
		 b \, \boldsymbol{\cos}(\boldsymbol{\omega}\,\boldsymbol{\tau})}
\end{equation}
To simplify the notation for our exploration of this differential equation,
  we make the replacements $\mathbf{\boldsymbol{\theta} \rightarrow u}$,
  $\mathbf{\boldsymbol{\tau} \rightarrow t}$, and
  $\mathbf{\boldsymbol{\omega} \rightarrow w}$ 
  ( parameters \textbf{a}, \textbf{b}, and \textbf{w}
   are dimensionless) to work with the differential equation:
\begin{equation}
\mathbf{\frac{d^{2}\,u}{d\,t^{2}} = - \boldsymbol{\sin}\,u  -
   a\,\frac{d\,u}{d\,t} + b\,\boldsymbol{\cos}(w\,t)}
\end{equation}
  where now both \textbf{t} and \textbf{u} are dimensionless, with the measure
  of \textbf{u} being radians, and the physical values of the pendulum angle being
  limited to the range $\mathbf{ -\boldsymbol{\pi} \leq u \leq \boldsymbol{\pi}}$,
  both extremes being the ``flip-over-point'' at the top of the motion.\\
  
\noindent We will use both \textbf{plotdf} and \textbf{rk} to explore this
  system, with
\begin{equation}
\mathbf{  \frac{d\,u}{d\,t} = v, \quad \frac{d\,v}{d\,t} = - \boldsymbol{\sin}\,u -
         a\,v   +  b\,\boldsymbol{\cos}(w\,t) }
\end{equation}
\subsubsection{Free Oscillation Case}
Using \textbf{plotdf}, the phase space plot for NO friction and NO driving torque
is
\begin{myVerbatim}
(%i1) plotdf([v,-sin(u)],[u,v],[trajectory_at,float(2*%pi/3),0],
            [direction,forward],[u,-2.5,2.5],[v,-2.5,2.5],
             [tstep, 0.01],[nsteps,600])$
\end{myVerbatim}
%\newpage
\smallskip
\begin{figure} [h]  
   \centerline{\includegraphics[scale=.3]{ch1p26.eps} }
	\caption{No Friction, No Driving Torque:  V Versus Angle U}
\end{figure}

\noindent  and now we use the \textbf{Plot Versus t} button of \textbf{plotdf} 
 to show the angle \textbf{u radians}
  and the dimensionless rate of change of angle \textbf{v}
%\newpage
\smallskip
\begin{figure} [h]  
   \centerline{\includegraphics[scale=.4]{ch1p27.eps} }
	\caption{No Friction, No Driving Torque: Angle U [blue] and  V [red]}
\end{figure}

%\newpage
\newpage
\subsubsection{Damped Oscillation Case}
\noindent We now include some damping with $\mathbf{a = 1/2}$.
\begin{myVerbatim}
(%i2)  plotdf([v,-sin(u)-0.5*v],[u,v],[trajectory_at,float(2*%pi/3),0],
            [direction,forward],[u,-1,2.5],[v,-1.5,1],
             [tstep, 0.01],[nsteps,450])$
\end{myVerbatim}
%\newpage
\smallskip
\begin{figure} [h]  
   \centerline{\includegraphics[scale=.4]{ch1p28.eps} }
	\caption{With Friction, but No Driving Force: V Versus Angle U}
\end{figure}

\noindent  and now we use the \textbf{Plot Versus t} button of \textbf{plotdf} 
 to show the angle \textbf{u radians}
  and the dimensionless rate of change of angle \textbf{y} for the friction present
  case.
 % \newpage
  
\smallskip
\begin{figure} [h]  
   \centerline{\includegraphics[scale=.4]{ch1p29.eps} }
	\caption{With Friction, but No Driving Force: Angle U [blue] and  V [red]}
\end{figure}
  
\newpage
\subsubsection{Including a Sinusoidal Driving Torque}
We now use the Runge-Kutta function \textbf{rk} to integrate the differential equation set
  forward in time for \mv|ncycles|, which is the same as setting the
  final dimensionless \mv|tmax| equal to \mv|ncycles*2*%pi/w|, or \mv|ncycles*T|,
  where we can call \mv|T| the dimensionless period defined by the dimensionless
  angular frequency \mv|w|.
The physical meaning of \mv|T| is the ratio of the period of the driving torque
  to the period of unforced and undamped small oscillations of the free
  simple pendulum.\\
  
\noindent For simplicity of exposition, we will call \mv|t| the ``time'' and
  \mv|T| the ``period''.
We again use our homemade function \mv|fll| described in the preface.\\
  
\noindent One cycle (period) of time is divided into \mv|nsteps| subdivisions, so \mv|dt = T/nsteps|.\\

\noindent For both the regular and chaotic parameter cases, we have used the same parameters
 as used in \textbf{Mathematica in Theoretical Physics}, by Gerd Baumann, Springer/Telos, 1996,
  pages 46 - 53.
\subsubsection{Regular Motion Parameters Case}
We find regular motion of this driven system with
  \mv|a = 0.2, b = 0.52, and w = 0.694|, and with \mv|u0 = 0.8 rad|, and \mv|v0 = 0.8 rad/unit-time|.
\begin{myVerbatim}
(%i1) fpprintprec:8$
(%i2) (nsteps : 31, ncycles : 30, a : 0.2, b : 0.52, w : 0.694)$
(%i3) [dudt : v, dvdt : -sin(u) - a*v + b*cos(w*t),
           T : float(2*%pi/w ) ];
(%o3)        [v, - 0.2 v - sin(u) + 0.52 cos(0.694 t), 9.0535811]
(%i4) [dt : T/nsteps, tmax : ncycles*T ];
(%o4)                        [0.292051, 271.60743]
(%i5) tuvL : rk ([dudt,dvdt],[u,v],[0.8,0.8],[t,0,tmax, dt])$
(%i6) %, fll;
(%o6)      [[0, 0.8, 0.8], [271.60743, - 55.167003, 1.1281164], 931]
(%i7) 930*dt;
(%o7)                              271.60743
\end{myVerbatim}
\subsubsection*{Plot of u(t) and v(t)}
Plot of u(t) and v(t) against t
\begin{myVerbatim}
(%i8) tuL : makelist ([tuvL[i][1],tuvL[i][2]],i,1,length(tuvL))$
(%i9) %, fll;
(%o9)              [[0, 0.8], [271.60743, - 55.167003], 931]
(%i10) tvL : makelist ([tuvL[i][1],tuvL[i][3]],i,1,length(tuvL))$
(%i11) %, fll;
(%o11)              [[0, 0.8], [271.60743, 1.1281164], 931]
(%i12) plot2d([ [discrete,tuL], [discrete,tvL]],[x,0,280],
          [style,[lines,3]],[xlabel,"t"],
           [legend, "u", "v"],
             [gnuplot_preamble,"set key bottom left;"])$
\end{myVerbatim}
\newpage
\noindent which produces 
\smallskip
\begin{figure} [h]  
   \centerline{\includegraphics[scale=.5]{ch3p1.eps} }
	\caption{Angle u(t) and v(t)}
\end{figure}

%plot2d([ [discrete,tuL], [discrete,tvL]],[x,0,280],
%          [style,[lines,4]],[xlabel,"t"],
%           [legend, "u", "v"],
%             [gnuplot_preamble,"set key bottom left;"],
%             [psfile,"ch3p1.eps"])$

\noindent The above plot shows nine flips of the pendulum at the
  top:\\
the first passage over the top at  \mv|u = -3 pi/2 = -4.7 rad|,\\
the second passage over the top at  \mv|u = -7 pi/2 = -11 rad| ,\\
and so on.
\subsubsection*{Phase Space Plot}
\noindent We next construct a phase space plot.
\begin{myVerbatim}
(%i13) uvL : makelist ([tuvL[i][2],tuvL[i][3]],i,1,length(tuvL))$
(%i14) %, fll;
(%o14)            [[0.8, 0.8], [- 55.167003, 1.1281164], 931]
(%i15) plot2d ( [discrete,uvL],[x,-60,5],[y,-5,5],
                   [style,[lines,3]],
             [ylabel," "],[xlabel," v vs u "] )$
\end{myVerbatim}
%\newpage
\noindent which produces (note that we include the early points which are more
  heavily influenced by the initial conditions):
\smallskip
\begin{figure} [h]  
   \centerline{\includegraphics[scale=.5]{ch3p2.eps} }
	\caption{Non-Reduced Phase Space Plot}
\end{figure}

%plot2d ( [discrete,uvL],[x,-60,5],[y,-5,5],
%                   [style,[lines,4]],
%             [ylabel," "],[xlabel," v vs u "] ,
%              [psfile,"ch3p2.eps"] )$
\subsubsection*{Reduced Phase Space Plot}
Let's define a Maxima function \textbf{reduce} which brings \textbf{u} back to the
interval \textbf{(-pi, pi )} and then make a reduced phase space plot.
Since this is a strictly numerical task, we can simplify Maxima's efforts by defining
  a floating point number \mv|pi| once and for all, and simply work with that
  definition.
You can see the origin of our definition of \mv|reduce| in the manual's entry on
  Maxima's modulus function \textbf{mod}.
\begin{myVerbatim}
(%i16) pi : float(%pi);
(%o16)                             3.1415927
(%i17) reduce(yy) := pi - mod (pi - yy,2*pi)$
(%i18) float( [-7*%pi/2,-3*%pi/2 ,3*%pi/2, 7*%pi/2] );
(%o18)          [- 10.995574, - 4.712389, 4.712389, 10.995574]
(%i19) map('reduce, % );
(%o19)         [1.5707963, 1.5707963, - 1.5707963, - 1.5707963]
(%i20) uvL_red : makelist ( [ reduce( tuvL[i][2]),
                         tuvL[i][3]],i,1,length(tuvL))$
(%i21) %, fll;
(%o21)             [[0.8, 0.8], [1.3816647, 1.1281164], 931]
\end{myVerbatim}			  
To make a reduced phase space plot with our reduced regular motion points,
  we will only use the last two thirds  of the pairs \mv|(u,v)|.
This will then show the part of the motion which has been ``captured''
  by the driving torque and shows little influence of the initial 
  conditions.\\
  
\noindent We use the Maxima function \textbf{rest (list, n)} which returns \textbf{list} 
  with its first \textbf{n} elements removed if \textbf{n} is positive.
Thus we  use \mv|rest (list, num/3)| to get the last two thirds.
\begin{myVerbatim}
(%i22) uvL_regular : rest (uvL_red, round(length (uvL_red)/3) )$
(%i23) %, fll;
(%o23)      [[0.787059, - 1.2368529], [1.3816647, 1.1281164], 621]
(%i24) plot2d ( [discrete,uvL_regular],[x,-3.2,3.2],[y,-3.2,3.2],
                   [style,[lines,2]],
             [ylabel," "],[xlabel,"reduced phase space v vs u "] )$
\end{myVerbatim}
%\newpage
\noindent which produces
\smallskip
\begin{figure} [h]  
   \centerline{\includegraphics[scale=.6]{ch3p3.eps} }
	\caption{Reduced Phase Space Plot of Regular Motion Points}
\end{figure}

%plot2d ( [discrete,uvL_regular],[x,-3.2,3.2],[y,-3.2,3.2],
%                   [style,[lines,3]],
%             [ylabel," "],[xlabel,"reduced phase space v vs u "] ,
%             [psfile,"ch3p3.eps"])$
\subsubsection*{Poincare Plot}
We next construct a Poincare plot of the regular (reduced) phase space
  points by using a ``stroboscopic view'' of this phase space, displaying
  only phase space points which correspond to times separated by the driving
  period \textbf{T}.
We select \mv|(u,v)| pairs which correspond to intervals of time \mv|n*T|,
  where \mv|n = 10, 11, ..., 30| which will give us \mv|21| phase space points
  for our plot (this is roughly the same as taking the last two thirds of the
  points).\\

\noindent The time \mv|t = 30*T| corresponds to \mv|t = 30*31*dt| = \mv|930*dt| which is the
  time associated with element \mv|931|, the last element) \mv|of uvL_red|.
The value of \mv|j| used to select the last Poincare point is the solution
  of the equation \mv|1 + 10*nsteps + j*nsteps = 1 + ncycles*nsteps|, which for this
  case is equivalent to\\
  \mv|311 + j*31 = 931|.
\begin{myVerbatim}
(%i25) solve(311 + j*31 = 931);
(%o25)                             [j = 20]
(%i26) pL : makelist (1+10*nsteps + j*nsteps, j, 0, 20);
(%o26) [311, 342, 373, 404, 435, 466, 497, 528, 559, 590, 621, 652, 683, 714, 
                                             745, 776, 807, 838, 869, 900, 931]
(%i27) length(pL);
(%o27)                                21
(%i28) poincareL : makelist (uvL_red[i], i, pL)$
(%i29) %,fll;
(%o29)       [[0.787059, - 1.2368529], [1.3816647, 1.1281164], 21]
(%i30) plot2d ( [discrete,poincareL],[x,-0.5,2],[y,-1.5,1.5],
                   [style,[points,1,1,1 ]],
             [ylabel," "],[xlabel," Poincare Section v vs u "] )$
\end{myVerbatim}
%\newpage
\noindent which produces the plot
\smallskip
\begin{figure} [h]  
   \centerline{\includegraphics[scale=.8]{ch3p4.eps} }
	\caption{Reduced Phase Space Plot of Regular Motion Points}
\end{figure}

%plot2d ( [discrete,poincareL],[x,-0.5,2],[y,-1.5,1.5],
%                   [style,[points ]],
%             [ylabel," "],[xlabel," Poincare Section v vs u "],
%               [psfile,"ch3p4.eps"] )$


For this regular motion parameters case, the Poincare plot shows 
  the phase space point coming back to one of three general locations in phase
  space at times separated by the period \textbf{T}.    
\newpage
\subsubsection{Chaotic Motion Parameters Case.}
To exhibit an example of chaotic motion for this system, we use the same
  initial conditions for \mv|u| and \mv|v|, but use the parameter set 
  \mv|a = 1/2,  b = 1.15, w = 2/3|.
\begin{myVerbatim}
(%i1) fpprintprec:8$
(%i2) (nsteps : 31, ncycles : 240, a : 1/2, b : 1.15, w : 2/3)$
(%i3) [dudt : v, dvdt : -sin(u) - a*v + b*cos(w*t),
           T : float(2*%pi/w ) ];
                        v                     2 t
(%o3)             [v, - - - sin(u) + 1.15 cos(---), 9.424778]
                        2                      3
(%i4) [dt : T/nsteps, tmax : ncycles*T ];
(%o4)                        [0.304025, 2261.9467]
(%i5) tuvL : rk ([dudt,dvdt],[u,v],[0.8,0.8],[t,0,tmax, dt])$
(%i6) %, fll;
(%o6)       [[0, 0.8, 0.8], [2261.9467, 26.374502, 0.937008], 7441]
(%i7) dt*( last(%) - 1 );
(%o7)                              2261.9467
(%i8) tuL : makelist ([tuvL[i][1],tuvL[i][2]],i,1,length(tuvL))$
(%i9) %, fll;
(%o9)              [[0, 0.8], [2261.9467, 26.374502], 7441]
(%i10) tvL : makelist ([tuvL[i][1],tuvL[i][3]],i,1,length(tuvL))$
(%i11) %, fll;
(%o11)              [[0, 0.8], [2261.9467, 0.937008], 7441]
(%i12) plot2d([ [discrete,tuL], [discrete,tvL]],[x,0,2000],
               [y,-15,30],
          [style,[lines,2]],[xlabel,"t"], [ylabel, " "],
           [legend, "u","v" ] ,[gnuplot_preamble,"set key bottom;"])$
\end{myVerbatim}
%\newpage
\noindent which produces a plot of \mv|u(t)| and \mv|v(t)| over \mv|0 <= t <= 2000|:
\smallskip
\begin{figure} [h]  
   \centerline{\includegraphics[scale=.8]{ch3p5.eps} }
	\caption{Angle u(t), and v(t) for Chaotic Parameters Choice}
\end{figure}

%plot2d([ [discrete,tuL], [discrete,tvL]],[x,0,2000],
%               [y,-15,30],
%          [style,[lines,2]],[xlabel,"t"],[ylabel," "],
%           [legend, "u","v" ] ,[gnuplot_preamble,"set key bottom;"],
%           [psfile,"ch3p5.eps"])$
\newpage
\subsubsection*{Phase Space Plot}
We next construct a \tcbr{non-reduced} phase space plot, but
  show only the first \mv|2000| reduced phase space points.
\begin{myVerbatim}
(%i13) uvL : makelist ([tuvL[i][2],tuvL[i][3]],i,1,length(tuvL))$
(%i14) %, fll;
(%o14)             [[0.8, 0.8], [26.374502, 0.937008], 7441]
(%i15) uvL_first : rest(uvL, -5441)$
(%i16) %, fll;
(%o16)             [[0.8, 0.8], [23.492001, 0.299988], 2000]
(%i17) plot2d ( [discrete,uvL_first],[x,-12,30],[y,-3,3],
                   [style,[points,1,1,1]],
             [ylabel," "],[xlabel," v vs u "])$
\end{myVerbatim}
%\newpage
\noindent which produces
\smallskip
\begin{figure} [h]  
   \centerline{\includegraphics[scale=.45]{ch3p6.eps} }
	\caption{non-reduced phase space plot using first 2000 points}
\end{figure}

%plot2d ( [discrete,uvL_first],[x,-12,30],[y,-3,3],
%                   [style,[points,1,1,1]],
%             [ylabel," "],[xlabel," v vs u "],
%               [psfile,"ch3p6.eps"])$
If we use the \textbf{discrete} default \textbf{style} option \mv|lines| 
  instead of \mv|points|, 
\begin{myVerbatim}
(%i18) plot2d ( [discrete,uvL_first],[x,-12,30],[y,-3,3],
             [ylabel," "],[xlabel," v vs u "])$
\end{myVerbatim}
%\newpage
\noindent we get the non-reduced phase space plot drawn with
  lines between the points:
\smallskip
\begin{figure} [h]  
   \centerline{\includegraphics[scale=.45]{ch3p11.eps} }
	\caption{non-reduced phase space plot using first 2000 points}
\end{figure}

%plot2d ( [discrete,uvL_first],[x,-12,30],[y,-3,3],
%             [ylabel," "],[xlabel," v vs u "],
%              [psfile,"ch3p11.eps"])$
\newpage
\subsubsection*{Reduced Phase Space Plot}
We now construct the reduced phase space points as in the regular motion case
  and then omit the first \mv|400|.
\begin{myVerbatim}
(%i19) pi : float(%pi);
(%o19)                             3.1415927
(%i20) reduce(yy) := pi - mod (pi - yy,2*pi)$
(%i21) uvL_red : makelist ( [ reduce( first( uvL[i] )),
                        second( uvL[i] ) ],i,1,length(tuvL))$
(%i22) %, fll;
(%o22)             [[0.8, 0.8], [1.2417605, 0.937008], 7441]
(%i23) uvL_cut : rest(uvL_red, 400)$
(%i24) %, fll;
(%o24)        [[0.25464, 1.0166641], [1.2417605, 0.937008], 7041]
\end{myVerbatim}
We have discarded the first \mv|400| reduced phase space points in defining
  \mv|uvL_cut|.
If we now only plot the first \mv|1000| of the points retained in
  \mv|uvL_cut|:
\begin{myVerbatim}
(%i25) uvL_first : rest (uvL_cut, -6041)$
(%i26) %, fll;
(%o26)        [[0.25464, 1.0166641], [2.2678603, 0.608686], 1000]
(%i27) plot2d ( [discrete,uvL_first],[x,-3.5,3.5],[y,-3,3],
                 [style,[points,1,1,1]],
             [ylabel," "],[xlabel,"reduced phase space v vs u "])$
\end{myVerbatim}
%\newpage
\noindent we get the plot
\smallskip
\begin{figure} [h]  
   \centerline{\includegraphics[scale=.6]{ch3p7.eps} }
	\caption{1000 points reduced phase space plot}
\end{figure}

%plot2d ( [discrete,uvL_first],[x,-3.5,3.5],[y,-3,3],
%                 [style,[points,1,1,1]],
%             [ylabel," "],[xlabel,"reduced phase space v vs u "],
%              [psfile,"ch3p7.eps"])$

and the same set of points drawn with the default \mv|lines| option:
\begin{myVerbatim}
(%i28) plot2d ( [discrete,uvL_first],[x,-3.5,3.5],[y,-3,3],
             [ylabel," "],[xlabel,"reduced phase space v vs u "])$
\end{myVerbatim}
\newpage
\noindent produces the plot
\smallskip
\begin{figure} [h]  
   \centerline{\includegraphics[scale=.5]{ch3p8.eps} }
	\caption{1000 points reduced phase space plot}
\end{figure}

%plot2d ( [discrete,uvL_first],[x,-3.5,3.5],[y,-3,3],
%             [ylabel," "],[xlabel,"reduced phase space v vs u "],
%             [psfile,"ch3p8.eps"])$
\subsubsection*{3000 point phase space plot}
We next draw the same reduced phase space plot, but use the first \mv|3000|
  points of \mv|uvL_cut|.
\begin{myVerbatim}
(%i29) uvL_first : rest (uvL_cut, -4041)$
(%i30) %, fll;
(%o30)      [[0.25464, 1.0166641], [- 2.2822197, - 0.532184], 3000]
(%i31) plot2d ( [discrete,uvL_first],[x,-3.5,3.5],[y,-3,3],
                 [style,[points,1,1,1]],
             [ylabel," "],[xlabel,"reduced phase space v vs u "])$
\end{myVerbatim}
%\newpage
\noindent which produces
\smallskip
\begin{figure} [h]  
   \centerline{\includegraphics[scale=.6]{ch3p9.eps} }
	\caption{3000 points reduced phase space plot}
\end{figure}

%plot2d ( [discrete,uvL_first],[x,-3.5,3.5],[y,-3,3],
%                 [style,[points,1,1,1]],
%             [ylabel," "],[xlabel,"reduced phase space v vs u "],
%              [psfile,"ch3p9.eps"])$
\newpage
\noindent and again, the same set of points drawn with the default \mv|lines| option
\begin{myVerbatim}
(%i32) plot2d ( [discrete,uvL_first],[x,-3.5,3.5],[y,-3,3],
             [ylabel," "],[xlabel,"reduced phase space v vs u "])$
\end{myVerbatim}
%\newpage
\noindent which produces the plot
\smallskip
\begin{figure} [h]  
   \centerline{\includegraphics[scale=1.1]{ch3p11.eps} }
	\caption{3000 points reduced phase space plot}
\end{figure}

%plot2d ( [discrete,uvL_first],[x,-3.5,3.5],[y,-3,3],
%             [ylabel," "],[xlabel,"reduced phase space v vs u "],
%             [psfile,"ch3p11.eps"])$

\newpage
\subsubsection*{Poincare Plot}
We now construct the Poincare section plot as before, using all the
 points available in \mv|uvL_red|.
\begin{myVerbatim}
(%i33) pL : makelist(1+10*nsteps + j*nsteps, j, 0, ncycles - 10)$
(%i34) %, fll;
(%o34)                         [311, 7441, 231]
(%i35) poincareL : makelist(uvL_red[i], i, pL)$
(%i36) %, fll;
(%o36)      [[- 2.2070801, 1.3794391], [1.2417605, 0.937008], 231]
(%i37) plot2d ( [discrete,poincareL],[x,-3,3],[y,-4,4],
                   [style,[points,1,1,1]],
             [ylabel," "],[xlabel," Poincare Section v vs u "] )$
\end{myVerbatim}
%\newpage
\noindent which produces the plot
\smallskip
\begin{figure} [h]  
   \centerline{\includegraphics[scale=1.1]{ch3p10.eps} }
	\caption{231 poincare section points}
\end{figure}

%plot2d ( [discrete,poincareL],[x,-3,3],[y,-4,4],
%                   [style,[points,2,1,1]],
%             [ylabel," "],[xlabel," Poincare Section v vs u "],
%               [psfile,"ch3p10.eps"])$


\newpage
\subsection{Using \textbf{contrib\_ode} for ODE's}
The syntax of \textbf{contrib\_ode} is the same as \textbf{ode}.
Let's first solve the same first order ODE example used in the first sections.
\begin{myVerbatim}
(%i1) de : 'diff(u,t)- u - exp(-t);
                                du         - t
(%o1)                           -- - u - %e
                                dt
(%i2) gsoln : ode2(de,u,t);
                                        - 2 t
                                      %e         t
(%o2)                       u = (%c - -------) %e
                                         2
(%i3) contrib_ode(de,u,t);
                                   du         - t
(%o3)                  contrib_ode(-- - u - %e   , u, t)
                                   dt
(%i4) load('contrib_ode);
(%o4) C:/PROGRA~1/MAXIMA~3.1/share/maxima/5.18.1/share/contrib/diffequations/c\
ontrib_ode.mac
(%i5) contrib_ode(de,u,t);
                                        - 2 t
                                      %e         t
(%o5)                      [u = (%c - -------) %e ]
                                         2
(%i6) ode_check(de, %[1] );
(%o6)                                  0
\end{myVerbatim}
We see that \mv|contrib_ode|, with the same syntax as \mv|ode|, returns a list
  (here with one solution, but in general more than one solution) rather than
  simply one answer.\\
  
\noindent Moreover, the package includes the Maxima function \textbf{ode\_check}
  which can be used to confirm the general solution.\\
  
\noindent Here is a comparison for our second order ODE example.  
\begin{myVerbatim}
(%i7) de : 'diff(u,t,2) - 4*u;
                                    2
                                   d u
(%o7)                              --- - 4 u
                                     2
                                   dt
(%i8) gsoln : ode2(de,u,t);
                                    2 t         - 2 t
(%o8)                     u = %k1 %e    + %k2 %e
(%i9) contrib_ode(de,u,t);
                                    2 t         - 2 t
(%o9)                    [u = %k1 %e    + %k2 %e     ]
(%i10) ode_check(de, %[1] );
(%o10)                                 0
\end{myVerbatim}
Here is an example of an ODE which \textbf{ode2} cannot solve, but
  \textbf{contrib\_ode} can solve.
\begin{myVerbatim}
(%i11) de : 'diff(u,t,2) + 'diff(u,t) + t*u;
                                 2
                                d u   du
(%o11)                          --- + -- + t u
                                  2   dt
                                dt
(%i12) ode2(de,u,t);
(%o12)                               false
\end{myVerbatim}
\newpage
\begin{myVerbatim}
(%i13) gsoln : contrib_ode(de,u,t);
                                 3/2
                     1  (4 t - 1)                         - t/2
(%o13) [u = bessel_y(-, ------------) %k2 sqrt(4 t - 1) %e
                     3       12
                                                3/2
                                    1  (4 t - 1)                         - t/2
                         + bessel_j(-, ------------) %k1 sqrt(4 t - 1) %e     ]
                                    3       12
(%i14) ode_check(de, %[1] );
(%o14)                                 0
\end{myVerbatim}
This section will probably be augmented in the future with more examples of
  using \textbf{contrib\_ode}.   
\end{document}