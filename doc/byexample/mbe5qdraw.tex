% e. woollett
%   april 08 - jan. 09
% file qdraw.tex
% things to add: 

% edit with Notepad++, then load into LED for latexing
\documentclass[12pt]{article}
\usepackage[dvips,top=1.5cm,left=1.5cm,right=1.5cm,foot=1cm,bottom=1.5cm]{geometry}
\usepackage{times}
\usepackage{amsmath}
\usepackage{amsbsy}
\usepackage{graphicx} 
\usepackage{url}
\urldef\tedhome\url{ http://www.csulb.edu/~woollett/  }
\urldef\tedmail\url{ woollett@charter.net}

%%%%%%%%%%%%%%%%%%%%%%%%%%%%%%%%%%%%%%%%%%%%%%%%%%%%%%%%%%%%%%%%%%%%%%%%
%   title page
%%%%%%%%%%%%%%%%%%%%%%%%%%%%%%%%%%%%%%%%%%%%%%%%%%%%%%%%%%%%%%%%%%%%%%
\title{Maxima by Example: Ch.5: 2D Plots and Graphics using qdraw 
            \thanks{This version uses Maxima 5.17.1. This is a live
            document. Check \;  \tedhome \; for the latest version of these notes. Send comments and
			 suggestions to \tedmail } }


\author{Edwin L. Woollett}
\date{\today}
%%%%%%%%%%%%%%%%%%%%%%%%%%%%%%%%%%%%%%%%%
%          document
%%%%%%%%%%%%%%%%%%%%%%%%%%%%%%%%%%%%%%%%%%
\begin{document}
%\small
\maketitle
\tableofcontents
\numberwithin{equation}{section}
\newpage
\setcounter{section}{4}
%\normalsize
\newpage
COPYING AND DISTRIBUTION POLICY\\
\\    
	
    
    This document is part of a series of notes titled
    "Maxima by Example" and	is made available\\
	via the author's webpage 
	\verb|http://www.csulb.edu/~woollett/ | to aid
	new users of the Maxima computer algebra system.\\
	
	\smallskip
	NON-PROFIT PRINTING AND DISTRIBUTION IS PERMITTED.\\
	You may make copies of this document and distribute them to others
	as long as you charge no more than the costs of printing.\\
	
	\smallskip
	These notes (with some modifications) will be published in book form
	eventually via Lulu.com in an arrangement which will continue
	to allow unlimited free download of the pdf files as well as the option
	of ordering a low cost paperbound version of these notes.
\newpage
\section{2D Plots and Graphics using qdraw}


\subsection{Quick Plots for Explicit Functions: ex(...)  }
This chapter provides an introduction to a new graphics interface developed by
  the author of the Maxima by Example tutorial notes.
The \textbf{qdraw} package ( qdraw.mac: available for download on the
  Maxima by Example webpage ) is an interface to the \textbf{draw}
  package function \textbf{draw2d}; to obtain a plot you must load \textbf{draw}
  as well as \textbf{qdraw}. You can just use \verb|load(qdraw)| if you have the file
  in your work folder and have set up your file search as described in Chap. 1.
  Otherwise just put qdraw.mac into your \verb|...maxima...share\draw| folder where it
  will be found.
  
\smallskip
The primary motivation for the \textbf{qdraw} package is to provide
  "quick" (hence the "q" in "qdraw") plotting software which provides
  the kinds of plotting defaults which are of interest to students
  and researchers in the physical sciences and engineering. There are  two "quick"
  plotting functions you can use with \textbf{qdraw}: \textbf{ex(...)} and
  \textbf{imp(...)}.

  \smallskip

 An entry like
\small
\begin{verbatim}
(%i1) load(draw)$
(%i2) load(qdraw)$
(%i3)  qdraw( ex( [x,x^2,x^3],x,-3,3 ) )$
\end{verbatim}
\normalsize
will produce a plot of the three explicit functions of \verb|x| in the
  first argument list, using line width = 3, an automatic rotating series
  of default colors, clearly visible \verb|x| and \verb|y| axes, and
  also a "grid" as well. The \textbf{ex} function passes its arguments
  on to a series of calls to \textbf{draw2d}'s \textbf{explicit}
  function.
  
%\smallskip
%%\begin{figure}[!ht] 
\begin{figure} [h]
   \centerline{\includegraphics[scale=1.3]{ch5p1.eps} }
	\caption{ Using ex() for $x,x^2,x^3$ }
\end{figure}    
Since $3^3 = 27$, \textbf{draw2d} extends the vertical axis to $\pm \, 27$ by
  default. 
\newpage
You can control the vertical range of the "canvas" with the \textbf{yr(...)} function,
  which passes its arguments to a \textbf{draw2d} entry \verb|yrange = [y1,y2]|. 
\small
\begin{verbatim}
(%i4) qdraw( ex([x,x^2,x^3],x,-3,3 ),yr(-2,2) )$ 
\end{verbatim}
\normalsize
which produces the plot:
\begin{figure} [h]
   \centerline{\includegraphics[scale=1]{ch5p2.eps} }
	\caption{ Adding $yr(-2,2)$ }
\end{figure}    

\smallskip
You can make the lines thinner or thicker than the default line
  width (3) by using the \textbf{lw(n)} option, which only affects the
  quick plotting functions \textbf{ex(...)} and \textbf{imp(...)}, as in
\small
\begin{verbatim}
(%i5) qdraw(ex([x,x^2,x^3],x,-3,3 ),yr(-2,2), lw(6))$
\end{verbatim}
\normalsize
to get:
\begin{figure} [h]
   \centerline{\includegraphics[scale=1]{ch5p3.eps} }
	\caption{ Adding $lw(6)$ }
\end{figure}    
\newpage
You can place the plot "key" (legend) at the bottom right by using
  the \textbf{key(bottom)} option, as in:
\small
\begin{verbatim}
(%i6) qdraw( ex( [x,x^2,x^3],x,-3,3 ),yr(-2,2),lw(6),key(bottom) )$
\end{verbatim}
\normalsize
to get:

\begin{figure} [h]
   \centerline{\includegraphics[scale=1]{ch5p4.eps} }
	\caption{ Adding $key(bottom)$ }
\end{figure}    

You can remove the default grid and xy axes by adding \textbf{cut(grid,xyaxes)}
as in:
\small
\begin{verbatim}
(%i7) qdraw( ex( [x,x^2,x^3],x,-3,3 ),yr(-2,2),
                 lw(6),key(bottom), cut(grid, xyaxes) )$
\end{verbatim}
\normalsize

\begin{figure} [h]
   \centerline{\includegraphics[scale=1]{ch5p5.eps} }
	\caption{ Adding $cut(grid, xyaxes)$ }
\end{figure}    
\newpage
You can remove the grid, axes, the key, and all the borders using \textbf{cut( all )},
  as in:
\small
\begin{verbatim}
(%i8) qdraw( ex( [x,x^2,x^3],x,-3,3 ),yr(-2,2),
                 lw(6), cut( all ) )$
\end{verbatim}
\normalsize
which results in a "clean" canvas:
\begin{figure} [h]
   \centerline{\includegraphics[scale=.9]{ch5p6.eps} }
	\caption{ Adding $cut( all )$ }
\end{figure}
\smallskip
    
Restoring the (default) grid and axes, we can place points (default size 3 and color black)
  at the intersection points using the \textbf{pts(...)} option, which passes
  a points list to \textbf{draw2d}'s \textbf{points} function:
\small
\begin{verbatim}
(%i9)  qdraw( ex([x,x^2,x^3],x,-3,3 ),yr(-2,2),lw(6),
              key(bottom),pts( [ [-1,-1],[0,0],[1,1] ] ) )$
\end{verbatim}
\normalsize
which produces:
\begin{figure} [h]
   \centerline{\includegraphics[scale=.9]{ch5p7.eps} }
	\caption{ Adding $pts( ptlist )$ }
\end{figure}    
\newpage
We can overide the default size and color of those points by including inside
  the \textbf{pts} function the optional \textbf{ps(n)} and \textbf{pc(c)} 
  arguments, as in:
\small
\begin{verbatim}
(%i10) qdraw( ex( [x,x^2,x^3],x,-3,3 ),yr(-2,2), lw(6),key(bottom), 
              pts([ [-1,-1],[0,0],[1,1] ],ps(2),pc(magenta) ) )$
\end{verbatim}
\normalsize

\smallskip
which produces:
\begin{figure} [h]
   \centerline{\includegraphics[scale=.9]{ch5p8.eps} }
	\caption{ Adding $pts(ptlist,ps(2),pc(magenta) )$ }
\end{figure}    
 
\smallskip
We can include a key entry for the points using the \textbf{pk(string)} option
  for the \textbf{pts} function, as in:
\small
\begin{verbatim}
(%i11) qdraw( ex( [x,x^2,x^3],x,-3,3 ),yr(-2,2), lw(6),key(bottom), 
           pts([ [-1,-1],[0,0],[1,1] ],ps(2),pc(magenta),pk("intersections") ) )$
\end{verbatim}
\normalsize
which produces:
\begin{figure} [h]
   \centerline{\includegraphics[scale=.9]{ch5p9.eps} }
	\caption{ $pts(ptlist,ps(2),pc(magenta),pk("intersections") )$ }
\end{figure}      
\newpage
The "eps" file ch5p9.eps used to get the last figure in the Tex file
  which is the source of this pdf file was produced using the \textbf{pic}(type, filename)
  option to \textbf{qdraw}, as in:
\small
\begin{verbatim}
(%i12) qdraw( ex( [x,x^2,x^3],x,-3,3 ),yr(-2,2), lw(6),key(bottom), 
        line(-3,0,3,0,lw(2)),line(0,-2,0,2,lw(2)),
    pts([ [-1,-1],[0,0],[1,1] ],ps(2),pc(magenta),pk("intersections")),
            pic(eps,"ch5p9") )$
\end{verbatim}
\normalsize
We have discussed, at the end of Chapter 1, Getting Started, how we insert such
  an "eps" file into our Tex file in order to get the figures you see here.  

\smallskip
The extra, optional, arguments we have included inside \textbf{qdraw} can be
  entered in any order; in fact, all arguments to \textbf{qdraw} are optional
  and can be entered in any order.
For example
\small
\begin{verbatim}
(%i13) qdraw( yr(-2,2),lw(6), ex( [x,x^2,x^3],x,-3,3 ),
              key(bottom), ex(sin(3*x)*exp(x/3),x,-3,3),
              pts([ [-1,-1],[0,0],[1,1] ]) )$
\end{verbatim}
\normalsize
which adds $sin(3\,x)\, e^{x/3}$ with a separate \textbf{ex(...)} argument to
  \textbf{qdraw}, and produces

\begin{figure} [h]
   \centerline{\includegraphics[scale=.9]{ch5p10.eps} }
	\caption{ Adding $sin(3\,x)\, e^{x/3}$  }
\end{figure}        
  
  \smallskip
We next add a label "qdraw at work" to our plot. 

\smallskip

Using Windows, the font size must be adjusted only after getting
  the plot drawn in a Gnuplot window by right clicking the icon in
  the upper left hand corner, and selecting Options, Choose Font,...
If you increase the windows graphics font from the default value of 10
  to 20, say, you will see a dramatic increase in the size of the
  label, but also in the size of the x and y axis coordinate numbers, and
  also a large increase in size of any features of the graphics which
  used a call to draw2d's \textbf{points} function (such as \textbf{qdraw}'s
  \textbf{pts} function).

\smallskip
This behavior seems to be related to the limitations of the present incarnation of
  the adaptation of Gnuplot to the Windows system, and hopefully will be addressed
  in the future by the volunteers who work on Gnuplot software.

\smallskip
Our illustration of the use of labels will simply be what one gets by
  sending the graphics object to a graphics file "ch5p11.eps" and including
  that file in our Tex/pdf file.
In Maxima, we use the code:
\small
\begin{verbatim}
(%i14) qdraw( yr(-2,2),lw(6), ex( [x,x^2,x^3],x,-3,3 ),
              key(bottom), ex(sin(3*x)*exp(x/3),x,-3,3),
              pts([ [-1,-1],[0,0],[1,1] ]) ,
               label(["qdraw at work",-2.9,1.5]),
              pic(eps,"ch5p11",font("Times-Bold",20) ) );
\end{verbatim}
\normalsize
\newpage
  
The resulting plot is then

\begin{figure} [h]
   \centerline{\includegraphics[scale=.9]{ch5p11.eps} }
	\caption{ Adding a label at (-2.9,1.5) }
\end{figure}        
 
If you look in the html Maxima manual under the index item "font", which
  takes you to a subsection of the draw package documentation, you will find
  a listing of the available postscript fonts, which one can use with the
  \textbf{pic(eps, filename, font( name, size ) ) } function call.
These options have the names Times-Roman, Times-Bold, Helvetica, Helvetica-Bold,
  Courier, Courier-Bold, also -Italic options.

\smallskip
If you want to save the graphics as a jpeg file, the font name should be
  a string containing the path to the desired font file.
Using the Windows XP operating system, the available windows fonts are in
  the folder \verb|c:\windows\fonts\|.
Here is Maxima code to get a jpeg graphics file based on our present drawing:
\small
\begin{verbatim}
(%i15) qdraw( yr(-2,2),lw(6), ex( [x,x^2,x^3],x,-3,3 ),
              key(bottom), ex(sin(3*x)*exp(x/3),x,-3,3),
              pts([ [-1,-1],[0,0],[1,1] ]) ,
               label(["qdraw at work",-2.9,1.5]),
              pic(jpg,"ch5p11",font("c:/windows/fonts/timesbd.ttf",20) ) );
\end{verbatim}
\normalsize
The resulting jpeg file has thicker lines and bolder labels, so some experimentation
  may be called for to get the desired result.
The font file requested corresponds to times roman bold.
The font file extension "ttf" stands for "true type fonts".
If you look in the windows, fonts folder you can find other interesting
  choices.
\newpage

\subsection{Quick Plots for Implicit Functions: imp(...) }
The quick plotting function \textbf{imp(...)} has the syntax\\
\verb|     imp( eqnlist, x,x1,x2,y,y1,y2 ) | \\
 or \verb|     imp( eqn,     x,x1,x2,y,y1,y2 ) |.\\
If the equation(s) are actually functions of (u,v) then $x \rightarrow u$ and
  $y \rightarrow v$.
The numbers (x1,x2) determine the horizontal canvas extent, and the
  numbers (y1,y2) determine the vertical canvas extent.
\smallskip
Here is an example using the single equation form:
\small
\begin{verbatim}
(%i16) qdraw( imp( sin(2*x)*cos(y)=0.4,x,-3,3,y,-3,3 ) ,
             cut(key) );
\end{verbatim}
\normalsize
which produces the "implicit plot":

\begin{figure} [h]
   \centerline{\includegraphics[scale=.9]{ch5p12.eps} }
	\caption{ Implicit plot of $sin(2\,x)\,cos(y)$ }
\end{figure}      
which uses the default line width = 3, the first of the default rotating colors (blue),
  and, of course, the default axes and grid.
To remove the default key, we have used the \textbf{cut} function.
Since the left hand side of this equation will periodically return to the
  same numerical value in both the x and the y directions, there is no "limit" to
  the solutions obtained by setting the left hand side equal to some numerical
  value between zero and one.

\smallskip
This looks like one piece of a contour plot for the given function.
We can add more contour lines using the \textbf{imp} function by using
  the list\_of\_equations form:
\small
\begin{verbatim}
(%i17) qdraw( imp( [sin(2*x)*cos(y)=0.4,
             sin(2*x)*cos(y)=0.7,
             sin(2*x)*cos(y)=0.9] ,x,-3,3,y,-3,3 ) ,
             cut(key) );
\end{verbatim}
\normalsize
The resulting plot with the default rotating color set is shown on the top
  of the next page.
\newpage


\begin{figure} [h]
   \centerline{\includegraphics[scale=.7]{ch5p13.eps} }
	\caption{contour plot of $sin(2\,x)\,cos(y)$ using imp() }
\end{figure}      

Of course if we define g, say, to be the expression $sin(2\,x)\,cos(y)$ first,
  we can use that binding to simplify our call to \textbf{imp(...) }:
\small
\begin{verbatim}
(%i18) g : sin(2*x)*cos(y)$
(%i19) qdraw( imp( [g = 0.4,g = 0.7,g = 0.9] ,x,-3,3,y,-3,3 ) ,
             cut(key) );
\end{verbatim}
\normalsize
to achieve the same plot.

\smallskip
We can also use symbols like \verb|%pi|, which will evaluate to a real number,
  in our horizontal and vertical limit slots, as in:
\small
\begin{verbatim}
(%i20) qdraw( imp( [g = 0.4,g = 0.7,g = 0.9] ,x,-%pi,%pi,y,-%pi,%pi ) ,
             cut(key) );
\end{verbatim}
\normalsize
We need to arrange that the horizontal canvas width is about $1.4$ time the vertical
  canvas height in order that geometrical shapes look closer to reality.
For the present plot we simply change the numerical values of the \textbf{imp(...)}
  function (x1,x2) parameters:
\small
\begin{verbatim}
(%i21) qdraw( imp( [g = 0.4,g = 0.7,g = 0.9] ,x,-4.2,4.2,y,-3,3 ) ,
             cut(key) );
\end{verbatim}
\normalsize
which produces a slightly different looking plot:

\begin{figure} [h]
   \centerline{\includegraphics[scale=.7]{ch5p14.eps} }
	\caption{using $(x1,x2) = (-4.2,4.2)$ }
\end{figure}      


\newpage
\subsection{Contour Plots with contour(...)}
Since we are talking about contour plots, this is a natural place to
give some examples of the \textbf{qdraw} package's \textbf{contour(...)} function
 which has two forms:\\
 \verb| contour( expr,x,x1,x2,y,y1,y2,cvals( v1,v2,...),options )| \\
 \verb| contour( expr,x,x1,x2,y,y1,y2, crange(n,min,max), options ) |.\\
where \verb|expr| is assumed to be a function of (x,y) and the first 
  form allows the setting of \verb|expr| to the supplied numerical values,
  while the second form allows one to supply the number of contours (n),
  the minimum value for a contour (min) and the maximum value for a
  contour (max).
If we use the most basic \verb|cvals(...)| form (ignoring options):
\small
\begin{verbatim}
(%i22) qdraw( contour(g, x,-4.2,4.2, y,-3,3, cvals(0.4,0.7,0.9) ) );
\end{verbatim}
\normalsize
we get a "plain jane" contour plot having line width 1, the key, grid, and xy-axes removed,
  in "black":
\begin{figure} [h]
   \centerline{\includegraphics[scale=.6]{ch5p15.eps} }
	\caption{simplest default contour example }
\end{figure}      
Since the quick plot functions \textbf{ex} and \textbf{imp} both use the rotating default
  colors which cannot be turned off, we would have to use the \textbf{imp1} function (which we
  have not yet discussed) with some of its options, to get the same results as the default
  use of \textbf{contour} produces. 
 The available "options" which can be used in any order but after the  required
   first eight arguments, are \textbf{lw(n)}, \textbf{lc(color)}, and \textbf{add(options)},
   where the "add options" are any or all of the set \verb|[grid,xaxis,yaxis,xyaxes]|.

\smallskip
Thus the following invocation of \textbf{contour}:
\small
\begin{verbatim}
(%i23) qdraw( contour(g,x,-4.2,4.2,y,-3,3,cvals(0.4,0.7,0.9),
        lw(2),lc(brown) ), ipgrid(15) );
\end{verbatim}
\normalsize
produces:
\begin{figure} [h]
   \centerline{\includegraphics[scale=.6]{ch5p16.eps} }
	\caption{adding lw(2), lc(brown) }
\end{figure}      
   
\newpage
We also added the separate \textbf{qdraw} function \textbf{ipgrid} with
  argument $15$ to over-ride the \textbf{qdraw} default value of the
  \textbf{draw2d} parameter \verb|ip_grid_in |. The \textbf{draw2d} default
  for this parameter is $5$, which results in some "jaggies" in implicit plots.
The default value inside the \textbf{qdraw} package is $10$, which generally
  produces smoother plots, but the drawing process takes more time, of course.
For our example here, we increased this parameter from $10$ to $15$ to get
  a smoother plot at the price of increased drawing time.

\smallskip
Here is an example of using the second, "crange", form of \textbf{contour}:
\small
\begin{verbatim}
(%i24) qdraw( contour(g,x,-4.2,4.2,y,-3,3,crange(4,0.2,0.9),
        lw(2),lc(brown) ), ipgrid(15)  )$
\end{verbatim}
\normalsize
which produces the plot:
\begin{figure} [h]
   \centerline{\includegraphics[scale=.8]{ch5p17.eps} }
	\caption{using $crange(4,.2,.9)$ }
\end{figure} 

\smallskip
     
A final example illustrates the \textbf{contour} option \textbf{add}:
\small
\begin{verbatim}
(%i25) qdraw( contour(sin(x)*sin(y),x,-2,2,y,-2,2,crange(4,0.2,0.9),
              lw(3),lc(blue),add(xyaxes) ), ipgrid(15) )$
\end{verbatim}
\normalsize
with the plot:
\begin{figure} [h]
   \centerline{\includegraphics[scale=.8]{ch5p18.eps} }
	\caption{using add( xyaxes ) }
\end{figure}      
\newpage

\subsection{Density Plots with qdensity(...)}
A type of plot closely related to the contour plot is the density plot,
  which paints small regions of the graphics window with a variable color
  chosen to highlight regions where the function of two variables takes
  on large values.
A completely separate density plotting function, \textbf{qdensity}, is
  supplied in the qdraw package.
The \textbf{qdensity} function in completely independent of the default
  conventions and syntax associated with the function \textbf{qdraw}.

\smallskip
  
The syntax of \textbf{qdensity} is:\\
\verb|qdensity(expr,[x,x1,x2,dx],[y,y1,y2,dy], options palette(p),pic(..)) |,
  where the two optional arguments are palette(p) and pic(type,filename).
The \verb|x| interval \verb|(x1,x2)| is divided into subintervals of size \verb|dx|,
  and likewise the \verb|y| interval \verb|(y1,y2)| is divided into subintervals
  of size \verb|dy|.  
\smallskip

If the palette(p) option is not present, a default "shades of blue"
   density plot is drawn (which corresponds to \verb|palette = [1,3,8]|).
To use the paletter option, the argument "p" can be either blue, gray,
   color, or a three element list \verb|[n1,n2,n3]|, where (n1,n2,n3) are positive integers which
   select functions to apply respectively to red, green, and blue.

\smallskip

To use the pic(...) option, the type is eps, eps\_color, jpg, or png,
and the filename is a string like "case5a".
As usual, use "x" and "y" if \verb|expr| depends explicitly on x and y,
or use "u" and "v" if \verb|expr| depends explicitly on u and v, etc.

\smallskip
A simple function of two variables to try is $f(x,y) = x\,y$, which increases from
  zero at the origin to $1$ at (1,1).
\small
\begin{verbatim}
(%i26) qdensity(x*y,[x,0,1,0.2],[y,0,1,0.2] )$
\end{verbatim}
\normalsize

This produces the density plot:
% this is an eps file converted from a jpeg using
%cygwin's convert function:
%  convert case6.jpg case6.eps
\begin{figure} [h]
   \centerline{\includegraphics[scale=.65]{ch5p19.eps} }
%      \centerline{\includegraphics[scale=.65]{case4.eps} }
	\caption{default palette density plot }
\end{figure}      
\newpage
If we use the gray palette opton
\small
\begin{verbatim}
(%i27) qdensity(x*y,[x,0,1,0.2],[y,0,1,0.2],
                 palette(gray) )$
\end{verbatim}
\normalsize
we get
\begin{figure} [h]
   \centerline{\includegraphics[scale=.45]{ch5p20.eps} }
	\caption{palette(gray) option }
\end{figure}      


\smallskip
while if we use palette(color), we get
\begin{figure} [h]
   \centerline{\includegraphics[scale=.45]{ch5p21.eps} }
	\caption{palette(color) option }
\end{figure}      



\newpage

To get a finer sampling of the function, you should decrease the values of
  \verb|dx| and \verb|dy| to \verb|0.05| or less.
Using the default palette choice with the interval choice \verb|0.05|, 
\small
\begin{verbatim}
(%i28) qdensity(x*y,[x,0,1,0.05],[y,0,1,0.05] )$
\end{verbatim}
\normalsize
yields a refined density plot with \verb|20 x 20 = 400| painted  rectangular panels.
\begin{figure} [h]
   \centerline{\includegraphics[scale=.35]{ch5p22.eps} }
	\caption{interval set to $0.05$ }
\end{figure}      

\smallskip
A more interesting function to look at is $f(x,y) = sin(x)\,sin(y)$.
\small
\begin{verbatim}
(%i29) qdensity(sin(x)*sin(y),[x,-2,2,0.05],[y,-2,2,0.05] )$
\end{verbatim}
\normalsize
which yields
\begin{figure} [h]
   \centerline{\includegraphics[scale=.6]{ch5p23.eps} }
	\caption{$sin(x)\,sin(y)$}
\end{figure}  

\smallskip
\newpage
    

\newpage

\subsection{Explicit Plots with Greater Control: ex1(...) }
If we are willing to deal with one explicit function or expression at a time,
  we get more control over the plot elements if we use the \textbf{qdraw} function
  \textbf{ex1(...)}, which has the syntax:\\
\verb|ex1( expr, x, x1,x2, lc(c), lw(n),lk(string) ) |.\\

\smallskip
As usual, if the expression \verb|expr| is actually a function of \verb|u|, then
  $x \rightarrow u$.
The first four arguments are required and must be in the first four slots.
The last three arguments are all optional and can be in any order.

\smallskip
Let's illustrate the use of \textbf{ex1(...)} by displaying a simple curve
  and the tangent and normal at one point of the curve. 
We will use the curve $y = x^2$ with the "slope" $dy/dx = 2\,x$, and
  construct the tangent line tangent at the point $(x_0,y_0)$:
  $$  (y - y_0) = m\,(x - x_0) $$
 where $m$ is the slope at $(x_0,y_0)$.
As we discuss in the next chapter, the normal line through the same point is
$$ (y - y_0) = (-1/m) \, (x - x_0). $$
For the point $x_0 = 1, y_0 = 1, m = 2$, the tangent line is
  $y = 2\,x -1$ and the normal line is $y = -x/2 + 3/2$.
\small
\begin{verbatim}
(%i30) qdraw( xr(-1.4,2.8),yr(-1,2),
        ex1(x^2,x,-2,2,lw(5),lc(brown),lk("X^2")),
        ex1(2*x-1,x,-2,2,lw(2),lc(blue),lk("TANGENT")),
        ex1(-x/2 + 3/2,x,-2,2,lw(2),lc(magenta),lk("NORMAL") ) ,
         pts( [ [1,1] ],ps(2),pc(red) ) )$
\end{verbatim}
\normalsize
Note that we were careful to force the x-range to be about $1.4$ times as
  great as the y-range (to get the correct geometry of the tangent and
  normal lines).
The resulting plot is:
\begin{figure} [h]
   \centerline{\includegraphics[scale=1.2]{ch5p25.eps} }
	\caption{plot using $ex1(...)$ }
\end{figure}      


\newpage
Here we use \textbf{ex1} to plot the first few Bessel functions
  of the first kind $J_{n}(x)$ for integral $n$ and real $x$,
\small
\begin{verbatim}
(%i31) qdraw( ex1(bessel_j(0,x),x,0,20,lc(red),lw(6),lk("bessel_j ( 0, x)") ),
          ex1(bessel_j(1,x),x,0,20,lc(blue),lw(5),lk("bessel_j ( 1, x)")),
          ex1(bessel_j(2,x),x,0,20,lc(brown),lw(4),lk("bessel_j ( 2, x)") ),
          ex1(bessel_j(3,x),x,0,20,lc(green),lw(3),lk("bessel_j ( 3, x)") ) )$
\end{verbatim}
\normalsize
which produces the plot:
\begin{figure} [h]
   \centerline{\includegraphics[scale=1]{ch5p26.eps} }
	\caption{$J_{n}(x)$ }
\end{figure}      

\smallskip
Here is a plot of $J_{0}( \sqrt{x} )$:
\small
\begin{verbatim}
(%i32) qdraw(line(0,0,50,0,lc(red),lw(2) ),
               ex1(bessel_j(0, sqrt(x)),x,0,50 ,lc(blue),
                lw(7),lk("J0( sqrt(x) )") ) )$
\end{verbatim}
\normalsize

\begin{figure} [h]
   \centerline{\includegraphics[scale=.7]{ch5p27.eps} }
	\caption{$J_{0}(\sqrt{x} )$ }
\end{figure}      


%\newpage
We chose to emphasize the axis $y = 0$ with a red line supplied by
  another of the \textbf{qdraw} functions, \textbf{line}, which we will
  discuss later in the section on geometric figures.
Placing the \textbf{line} element before \textbf{ex1(..)} causes the
  curve to write "over" the line, rather than the reverse.
  
\newpage
\subsection{Explicit Plots with ex1(...) and Log Scaled Axes}
The name "log plot" usually refers to a plot of $\ln(y)$ vs $x$ using linear
  graph paper, which is equivalent to a plot of $y$ vs $x$ on graph paper
  which uses a "logarithmic scale" on the vertical axis.
Given an expression $g$ depending on x, you can either use   
  the syntax \verb|qdraw( ex1( log(g),x,x1,x2 ), other options )| to generate such a "log plot"
  or \verb|qdraw( ex1(g, x, x1, x2), log(y) , other options )|.

\smallskip
  
Let's show the differences using the function $f(x) = x\,e^{-x}$, but
  using an expression called $g$ rather than a Maxima function.  
\small
\begin{verbatim}
(%i33) g : x*exp(-x)$
(%i34) qdraw( ex1( log(g),x,0.001,10, lc(red) ),yr(-8,0)  )$
\end{verbatim}
\normalsize
which displays the plot
\begin{figure} [h]
   \centerline{\includegraphics[scale=.6]{ch5p27a.eps} }
	\caption{Linear Graph Paper Plot of $\ln(g)$ }
\end{figure}      

\smallskip
The numbers on the vertical axis correspond to values of $\ln(g)$.
Since $g$ is singular at $x = 0$, we have avoided that region by using
  $x_{1} = 0.001$.

\smallskip
The second way to get a "log plot" of $g$ is to request "semi-log" graph paper
  which has the vertical axis marked using a logarithmic scale for the values of $g$.
Using the \textbf{log(y)} option of the \textbf{qdraw} function, we use:
\small
\begin{verbatim}
(%i35) qdraw( ex1(g, x, 0.001,10,lc(red) ),
                 yr(0.0001, 1), log(y) )$
\end{verbatim}
\normalsize
The \textbf{yr(y1,y2)} option takes into account the numerical limits of
  $g$ over the $x$ interval requested.
The minimum value of $g$ is $0.005$ which occurs at $x = 10$.
The maximum value of $g$ is about $0.37$.
The resulting plot is:
\begin{figure} [h]
   \centerline{\includegraphics[scale=.68]{ch5p27b.eps} }
	\caption{Log Paper Plot of $g$ }
\end{figure}      
  
\newpage
The name "log-linear plot" can be used to mean "x axis marked with a log scale, y axis
  marked with a linear scale".
Using the same function, we generate this plot by using the \textbf{log(x)}
  option to \textbf{qdraw}:
\small
\begin{verbatim}
(%i36) qdraw( ex1(g, x, 0.001,10,lc(red),lw(7) ),
                 yr(0,0.4), log(x) )$
\end{verbatim}
\normalsize
% eps file generated by
%(%i35) qdraw( ex1(g, x, 0.001,10,lc(red),lw(7) ),
%                 yr(0,0.4), log(x),
%                 pic(eps,"ch5p27c") );
%  
This generates the plot
\begin{figure} [h]
   \centerline{\includegraphics[scale=.85]{ch5p27c.eps} }
	\caption{Log-Linear Plot of $g$}
\end{figure}      

\smallskip
Scientists and engineers normally like to use a log scaled axis for
  a variable which varies over many powers of ten, which is not the case
  for our example.

\smallskip

Finally, we can request "log-log paper" which has both axes marked with a log scale,
  by using the \textbf{log(xy)} option to \textbf{qdraw}.
\small
\begin{verbatim}
(%i37) qdraw( ex1(g, x, 0.001,10,lc(red) ),
                  yr(0.0001,1), log(xy) )$
\end{verbatim}
\normalsize
% eps file generation
%(%i42) qdraw( ex1(g, x, 0.001,10,lc(red),lw(7) ),
%               yr(0.0001,1), log(xy) ,
%              pic(eps,"ch5p27d") );
%
which produces
\begin{figure} [h]
   \centerline{\includegraphics[scale=.85]{ch5p27d.eps} }
	\caption{Log-Log Plot of $g$}
\end{figure}      

\newpage

\subsection{Data Plots with Error Bars: pts(...) and errorbars(...)}
In Chapter One of Maxima by Example, Section 1.5.8, we created a data file
  called "fit1.dat", which can be downloaded from the author's webpage.
We will use that data file, together with "fit2.dat", also available, to illustrate
  making simple data plots using the \textbf{qdraw} functions \textbf{pts(...)} and
  \textbf{errorbars(...)}.
The syntax of \textbf{pts(...)} is:\\
\verb|pts( pointlist, pc(c), ps(s), pt(t), pj(lw), pk(string) ) |\\
The only required argument is the first argument "pointlist" which 
  has the form: \\      \verb|[ [x1,y1], [x2,y2], [x3,y3],...] |.\\
The remaining arguments are all optional and may be entered in any
  order following the first required argument.\\
The optional argument \verb|pc(c)| overrides the default color (black), 
  for example, \verb|pc(red)|.\\
The optional argument \verb|ps(s)| overrides the default size (3), and
  an example is \verb|ps(2)|.\\
The optional argument \verb|pt(t)| overrides the default type (7, which is
  the integer used for filled\_circle: see the Maxima manual index entry
  for "point\_type"); an example would be \verb|pt(8)|, which would use
  an open "up\_triangle" instead of a filled circle.\\
The optional argument \verb|pj(lw)|, if present, will cause the points
  provided by the nested list "pointlist" to be joined using a line
  whose width is given by the argument of \verb|pj|; an example is
  \verb|pj(2)| which would set the line width to the value 2.\\
The optional argument \verb|pk(string)| provides text for a key
  entry for the set of points represented by pointlist; as example
  is \verb|pk("case x^2")|.\\

\smallskip
Before making the data plot, let's look at the data file contents from inside
  Maxima:
\small
\begin{verbatim}
(%i38) printfile("fit1.dat")$
1 1.8904 
2 3.0708 
3 3.9215 
4 5.1813 
5 5.9443 
6 7.0156 
7 7.8441 
8 8.8806 
9 9.8132 
10 11.129 
\end{verbatim}
\normalsize
We next use Maxima's \textbf{read\_nested\_list} function to create a list of
  data points from the data file.
\small
\begin{verbatim}
(%i39) plist : read_nested_list("fit1.dat");
(%o39) [[1, 1.8904], [2, 3.0708], [3, 3.9215], [4, 5.1813], [5, 5.9443], 
   [6, 7.0156], [7, 7.8441], [8, 8.880599999999999], [9, 9.8132], [10, 11.129]]
\end{verbatim}
\normalsize
\newpage

The most basic plot of this data uses the \textbf{pts(...)} function defaults:
\small
\begin{verbatim}
(%i40) qdraw( pts(plist) )$
\end{verbatim}
\normalsize
which produces:
\begin{figure} [h]
   \centerline{\includegraphics[scale=1]{ch5p27e.eps} }
	\caption{Using pts(...) Defaults}
\end{figure}      

\smallskip
We can use the \textbf{qdraw} functions \textbf{xr(...)} and \textbf{yr(...)}
  to override the default range selected by \textbf{draw2d}, and decrease the
  point size:
\small
\begin{verbatim}
(%i41) qdraw( pts(plist, ps(2)), xr(0,12),yr(0,15) )$
\end{verbatim}
\normalsize
with the result:
\begin{figure} [h]
   \centerline{\includegraphics[scale=1]{ch5p27f.eps} }
	\caption{Adding ps(2), xr(..), yr(..)}
\end{figure}      

\newpage
Now we add color and a key string, as well as simple error bars corresponding to
  an assumed uncertainty of the \verb|y| value of plus or minus \verb|1| for all
  the data points.
\small
\begin{verbatim}
(%i42) qdraw( pts(plist,pc(blue),pk("fit1"), ps(2)), xr(0,12),yr(0,15),
          key(bottom), errorbars( plist, 1) )$
\end{verbatim}
\normalsize
% eps file generator
%(%i15) qdraw( pts(plist,pc(blue),pk("fit1"), ps(2)), xr(0,12),yr(0,15),
%          key(bottom), errorbars( plist, 1),
%           pic(eps, "ch5p27g") );
%
which looks like:
\begin{figure} [h]
   \centerline{\includegraphics[scale=0.8]{ch5p27g.eps} }
	\caption{Adding pc(blue) and Simple Error Bars }
\end{figure}      

\smallskip
The default error bar line width of \verb|1| is almost too small to see, so we thicken
  the error bars and add color
\small
\begin{verbatim}
(%i43) qdraw( pts(plist,pc(blue),pk("fit1"), ps(2)), xr(0,12),yr(0,15),
          key(bottom), errorbars( plist, 1, lw(3),lc(red) )  )$
\end{verbatim}
\normalsize
% eps file
%(%i17) qdraw( pts(plist,pc(blue),pk("fit1"), ps(2)), xr(0,12),yr(0,15),
%          key(bottom), errorbars( plist, 1, lw(2),lc(red) ),
%           pic(eps, "ch5p27h" ) );
%
with the result:
\begin{figure} [h]
   \centerline{\includegraphics[scale=1]{ch5p27h.eps} }
	\caption{Adding lw(3), lc(red) to errorbars(...) }
\end{figure} 
     
\smallskip
The difference in the fonts is due to my using
  \verb|pic(eps, "ch5p27h", font("Times-Roman",18) )| to create the eps graphic instead
  of just \verb|pic(eps, "ch5p27h"  )| as another argument  to \textbf{qdraw}.
\newpage
If the data set has individual uncertainties in the \verb|y| value, we create a
  list \verb|dyl|, say, \\
  of the values \verb|dy1, dy2, dy3,...| and use the syntax:\\
  \verb|   errorbars( pointlist, dylist, lw(n), lc(c) )| \\
Here is an example:
\small
\begin{verbatim}
(%i44) dyl : [0.2,0.3,0.5,1.5,0.8,1,1.4,1.8,2,2];
(%o44)           [0.2, 0.3, 0.5, 1.5, 0.8, 1, 1.4, 1.8, 2, 2]
(%i45) map(length,[plist,dyl] );
(%o45)                             [10, 10]
(%i46) qdraw( pts(plist,pc(blue),pk("fit1"), ps(2)), xr(0,12),yr(0,15),
          key(bottom), errorbars( plist, dyl, lw(3),lc(red) )  )$
\end{verbatim}
\normalsize
% eps file
%(%i24) qdraw( pts(plist,pc(blue),pk("fit1"), ps(2)), xr(0,12),yr(0,15),
%          key(bottom), errorbars( plist, dyl, lw(3),lc(red) ),
%         pic(eps, "ch5p27i", font("Times-Roman",18) ) );
%  
with the result
\begin{figure} [h]
   \centerline{\includegraphics[scale=1]{ch5p27i.eps} }
	\caption{Using a list of dy values with errorbars(..) }
\end{figure} 

\smallskip
We now repeat the least squares fit of this data which we carried out in Chapter 1.
See our discussion there for an explanation of what we are doing here.
\small
\begin{verbatim}
(%i47) display2d:false$
(%i48) pmatrix : apply( 'matrix, plist );
(%o48) matrix([1,1.8904],[2,3.0708],[3,3.9215],[4,5.1813],[5,5.9443],
              [6,7.0156],[7,7.8441],[8,8.880599999999999],[9,9.8132],
              [10,11.129])
(%i49) load(lsquares);
(%o49) "C:/PROGRA~1/MAXIMA~4.0/share/maxima/5.15.0/share/contrib/lsquares.mac"
(%i50) soln : (lsquares_estimates(pmatrix,[x,y],y=a*x+b,
                   [a,b]), float(%%) );
(%o50) [[a = 0.99514787679748,b = 0.99576667381004]]
(%i51) [a,b] : (fpprintprec:5, map( 'rhs, soln[1] ) )$
(%i52) [a,b];
(%o52) [0.995,0.996]
(%i53) qdraw( pts(plist,pc(blue),pk("fit1"), ps(2)), xr(0,12),yr(0,15),
          key(bottom), errorbars( plist, dyl, lw(3),lc(red) ),
         ex1( a*x + b,x,0,12, lc(brown),lk("linear fit") )  )$
\end{verbatim}
\normalsize
% eps file gen
%(%i32) qdraw( pts(plist,pc(blue),pk("fit1"), ps(2)), xr(0,12),yr(0,15),
%          key(bottom), errorbars( plist, dyl, lw(3),lc(red) ),
%         ex1( a*x + b,x,0,12, lc(brown),lk("linear fit") ),
%         pic(eps,"ch5p27j",font("Times-Roman",18) ) );
%
\newpage
We use the \textbf{qdraw} function \textbf{ex1(...)} to add the line
 $f(x) = a\,x + b$ to the data plot.
The resulting plot with the least squares fit added is then:
\begin{figure} [h]
   \centerline{\includegraphics[scale=1.5]{ch5p27j.eps} }
	\caption{Adding the Linear Fit Line }
\end{figure} 

\smallskip
Now we add the data in the file "fit2.dat":
\small
\begin{verbatim}
(%i54) printfile("fit2.dat");
1 0.9452 
2 1.5354 
3 1.9608
4 2.5907
5 2.9722
6 3.5078
7 3.9221
8 4.4403
9 4.9066
10 5.5645
(%o54) "fit2.dat"
(%i55) p2list: read_nested_list("fit2.dat");
(%o55) [[1,0.945],[2,1.5354],[3,1.9608],[4,2.5907],[5,2.9722],[6,3.5078],
        [7,3.9221],[8,4.4403],[9,4.9066],[10,5.5645]]
(%i56) qdraw( pts(plist,pc(blue),pk("fit1"), ps(2)), xr(0,12),yr(0,15),
          key(bottom), errorbars( plist, dyl, lw(3),lc(red) ),
         ex1( a*x + b,x,0,12, lc(brown),lk("linear fit 1") ),
         pts(p2list, pc(magenta),pk("fit2"),ps(2)),
         errorbars( p2list,0.5,lw(3) ) )$		
\end{verbatim}
\normalsize
% eps file gen
%(%i38) qdraw( pts(plist,pc(blue),pk("fit1"), ps(2)), xr(0,12),yr(0,15),
%          key(bottom), errorbars( plist, dyl, lw(3),lc(red) ),
%         ex1( a*x + b,x,0,12, lc(brown),lk("linear fit 1") ),
%         pts(p2list, pc(magenta),pk("fit2"),ps(2)),
%         errorbars( p2list,0.5,lw(3) ) ,
%         pic(eps,"ch5p27k",font("Times-Roman",18) ) );
%
\newpage

Here is the plot with the second data set:
\begin{figure} [h]
   \centerline{\includegraphics[scale=1.5]{ch5p27k.eps} }
	\caption{Adding the Second Set of Data }
\end{figure} 

\smallskip
We could then find the least squares fit to the data set 2 and again use
  the function \textbf{ex1(...)} to add that fit to our plot, and add any
  other features desired.



\newpage
  
\subsection{Implicit Plots with Greater Control: imp1(...) }  
If we are willing to deal with one implicit equation of two
  variables at a time, we get more control over the plot elements if we use
  the \textbf{qdraw} function \textbf{imp1(...)}, which has the syntax:\\
\verb|imp1( eqn, x, x1,x2, y, y1,y2, lc(c), lw(n),lk(string) ) |.\\

\smallskip
As usual, if the equation \verb|eqn| is actually a function of  the pair
  of variables \verb|u| and \verb|v|, then let $x \rightarrow u$,
  and $y \rightarrow v$.
The first seven arguments are required and must be in the first seven slots.
The last three arguments are all optional and can be in any order.

\smallskip
Let's illustrate the use of \textbf{imp1(...)} by displaying a translated
  and rotated ellipse, together with the rotated $x$ and $y$ axes.
In the following, \verb|eqn1| describes the rotated ellipse, \verb|eqn2|
  describes the rotated x axis, and \verb|eqn3| describes the
  rotated y axis.  
The angle of rotation is about $63.4 \deg$ (counter clockwise), which corresponds
  to $\tan{\phi} = 2$.
Notice that we take care to get the x-axis range about $1.4$ times 
  the y-axis range, in order to get the geometry approximately
  right.
\small
\begin{verbatim}
(%i1) eqn1 : 5*x^2 + 4*x*y + 8*y^2 - 16*x + 8*y - 16 = 0$
(%i2) eqn2 : y+1 = 2*(x-2)$
(%i3) eqn3 : y+1 = -(x-2)/2$
(%i4) qdraw( imp1(eqn1,x,-2,6.4,y,-4,2,lc(red),lw(6),lk("ELLIPSE")),
              imp1(eqn2,x,-2,6.4,y,-4,2,lc(blue),lw(4),lk("ROT X AXIS")),
              imp1(eqn3,x,-2,6.4,y,-4,2,lc(brown),lw(4),lk("ROT Y AXIS") ),
          pts([ [2,-1] ],ps(2),pc(magenta),pk("TRANSLATED ORIGIN") ) )$
\end{verbatim}
\normalsize
We get the following figure, if we increase the font size,

\begin{figure} [h]
   \centerline{\includegraphics[scale=1.4]{ch5p28.eps} }
	\caption{Rotated and Translated Ellipse }
\end{figure}        
 \newpage
 
As a second example with \textbf{imp1} we make a simple plot 
  based on the equation $y^3 = x^2$.
\small
\begin{verbatim}
(%i5) qdraw( imp1(y^3=x^2,x,-3,3,y,-1,3,lw(10),lc(dark-blue)) )$
\end{verbatim}
\normalsize
%\newpage

which produces the plot:
\begin{figure} [h]
   \centerline{\includegraphics[scale=1.5]{ch5p29.eps} }
	\caption{Implicit Plot of $y^3 = x^2$ }
\end{figure}      

\smallskip
Notice that you can use hyphenated color choices (see Maxima color index) without the double
  quotes, or with the double quotes.
  
\newpage

  
\subsection{Parametric Plots with para(...) }
The \textbf{qdraw} function \textbf{para} can be used to draw parametric plots
and has the syntax\\
\verb|para(xofu,yofu,u,u1,u2,lw(n),lc(c),lk(string) )|\\
where, as usual, the line width, line color, and key string
  are optional and can be in any order.
The parameter "u" can, of course, be any symbol.

\smallskip
A simple  example, in which we use "t" for the parameter,
  and let the x coordinate corresponding to some value of
  t be $sin(t)$, and let the y coordinate corresponding
  to that same value of t be $sin(2\,t)$ is:
\small
\begin{verbatim}
(%i6) qdraw(xr(-1.5,2),yr(-2,2), 
         para(sin(t),sin(2*t),t,0,2*%pi ) ,
         pts( [ [sin(%pi/8),sin(%pi/4)] ],ps(2),pc(blue),pk("t = pi/8")),
         pts( [ [1,0] ],ps(2),pc(red),pk("t = pi/2")) )$
\end{verbatim}
\normalsize
% eps file generated by:
%(%i34) qdraw(xr(-1.5,2),yr(-2,2),
%         line(-1.5,0,2,0,lw(2)),
%          line(0,-2,0,2,lw(2)),
%         para(sin(t),sin(2*t),t,0,2*%pi ) ,
%         pts( [ [sin(%pi/8),sin(%pi/4)] ],ps(2),pc(blue),pk("t = pi/8")),
%         pts( [ [1,0] ],ps(2),pc(red),pk("t = pi/2")),
%         pic(eps,"ch5p30",font("Times-Roman",18) ) );
%  
produces the plot:
\begin{figure} [h]
   \centerline{\includegraphics[scale=1.5]{ch5p30.eps} }
	\caption{$x = \sin(t), y = \sin(2\,t) $ }
\end{figure}      

\newpage

A second example of a parametric plot has $u$ as the parameter,
  $x = 2\,cos(u)$, and $y = u^2$:
\small
\begin{verbatim}
(%i7) qdraw(xr(-3,4),yr(-1,40), para(2*cos(u),u^2,u,0,2*%pi) ,
         pts([ [2,0] ],ps(2),pc(blue),pk("u = 0")),
         pts( [ [0,(%pi/2)^2] ],ps(2), pc(red), pk("u = pi/2")),
         pts([ [-2,%pi^2]],ps(2),pc(green),pk("u = pi")),
         pts( [[0,(3*%pi/2)^2]],ps(2),pc(magenta),pk("u = 3*pi/2")) )$
\end{verbatim}
\normalsize
% eps file created with
%(%i43) qdraw(xr(-3,4),yr(-1,40),
%          line(-3,0,4,0,lw(2)),
%          line(0,-1,0,40,lw(2)),
%             para(2*cos(u),u^2,u,0,2*%pi) ,
%         pts([ [2,0] ],ps(2),pc(blue),pk("u = 0")),
%         pts( [ [0,(%pi/2)^2] ],ps(2), pc(red), pk("u = pi/2")),
%         pts([ [-2,%pi^2]],ps(2),pc(green),pk("u = pi")),
%         pts( [[0,(3*%pi/2)^2]],ps(2),pc(magenta),pk("u = 3*pi/2")),
%           pic(eps,"ch5p31",font("Times-Roman",18) ) ); 
%


which yields the plot:

\begin{figure} [h]
   \centerline{\includegraphics[scale=1.5]{ch5p31.eps} }
	\caption{$x = 2\,cos(u), y = u^2 $ }
\end{figure}      

\newpage

\subsection{Polar Plots with polar(...) }
A "polar plot" plots the points $(x = r(\theta)\,\cos (\theta) ,y = r(\theta)\,\sin(\theta) )$,
  where the function $r(\theta)$ is supplied.

\smallskip
The \textbf{qdraw} function \textbf{polar} has the syntax:\\
  \verb|polar( roftheta, theta, th1,th2, lc(c), lw(n), lk(string) )|\\
  where theta, th1, and th2 are in radians, and the last three arguments
  are optional.
  
\smallskip
A simple example is provided by the hyperbolic spiral $r(\theta) = 10/\theta $.
\small
\begin{verbatim}
(%i8) qdraw( polar(10/t,t,1,3*%pi,lc(brown),lw(5)),nticks(200),
            xr(-4,6),yr(-3,9),key(bottom) ,
       pts( [[10*cos(1),10*sin(1)]],ps(3),pc(red),pk("t = 1 rad")),         
     pts([[5*cos(2),5*sin(2)]],ps(3),pc(blue),pk("t = 2 rad") ),
              line(0,0,5*cos(2),5*sin(2)) )$
\end{verbatim}
\normalsize
% eps file generation
%(%i56) qdraw( polar(10/t,t,1,3*%pi,lc(brown),lw(7)),nticks(200),
%            xr(-4,6),yr(-3,9),
%       pts( [[10*cos(1),10*sin(1)]],ps(3),pc(red),pk("t = 1 rad")),
%         key(bottom) ,
%     pts([[5*cos(2),5*sin(2)]],ps(3),pc(blue),pk("t = 2 rad") ),
%       line(0,0,5*cos(2),5*sin(2)),
%     line(-4,0,6,0,lw(2)),line(0,-3,0,9,lw(2)),
%      pic(eps,"ch5p32") ); 
%
which looks like:
\begin{figure} [h]
   \centerline{\includegraphics[scale=1.5]{ch5p32.eps} }
	\caption{Polar Plot with $r = 10/\theta $ }
\end{figure}      
\newpage

\subsection{Geometric Figures: line(...) }
The \textbf{qdraw} function \textbf{line} has the syntax:\\
  \verb|  line( x1,y1,x2,y2, lc(c), lw(n), lk(string) ) |\\
  which draws a line from \verb|(x1,y1)| to \verb|(x2,y2)|.
The last three arguments are optional and can be in any order after
  the first four arguments.

\smallskip
For example, \verb|line(0,0,1,1,lc(blue),lw(6),lk("radius") ) | will draw
  a line from $(0,0)$ to $(1,1)$ in blue with line width $6$ and with
   a key entry with the text 'radius'. 
The defaults are color black, line width $3$, and no key entry.
\small
\begin{verbatim}
(%i9) qdraw( line(0,0,1,1) )$
\end{verbatim}
\normalsize
produces the default line with \textbf{draw2d}'s default range:
\begin{figure} [h]
   \centerline{\includegraphics[scale=.6]{ch5p33.eps} }
	\caption{Default line(..) }
\end{figure}      

\smallskip
Adding some options and extending the canvas range in both directions
\small
\begin{verbatim}
(%i10) qdraw( line(0,0,1,1,lc(blue),lw(6),lk("radius") ),
              xr(0,2),yr(0,2),key(bottom),
              pts([ [1,1] ] ,pc(red),pk("point")) )$
\end{verbatim}
\normalsize
produces a line to a point marked in red:
\begin{figure} [h]
   \centerline{\includegraphics[scale=.8]{ch5p34.eps} }
	\caption{Adding options to line(..) }
\end{figure}      
\newpage
Here we define a Maxima function "doplot1(n)" in a file "doplot1.mac" which has the
  following contents:
\small
\begin{verbatim}
/* file doplot1.mac  */

disp("doplot1(nlw)")$

doplot1(nlw) := block([cc,qlist,x,val,i ],
     /* list of 20 single name colors  */
     cc : [aquamarine,beige,blue,brown,cyan,gold,goldenrod,green,khaki,
            magenta,orange,pink,plum,purple,red,salmon,skyblue,turquoise,
            violet,yellow ],
     qlist : [ xr(-3.3,3) ],     
     for i thru length(cc) do (
       x : -3.3 + 0.3*i,
       val : line( x,-1,x,1, lc( cc[i] ),lw(nlw) ),
       qlist : append(qlist, [val] )
     ),
     qlist : append( qlist,[ cut(all) ] ),
     apply('qdraw, qlist)
    )$                            
\end{verbatim}
\normalsize
Here is a record of loading and using the function defined to
  produce a series of vertical colored lines.
\small
\begin{verbatim}
(%i11) load(doplot1);
                                  doplot1(nlw)
(%o11)                         c:/work2/doplot1.mac
(%i12) doplot1(20);
\end{verbatim}
\normalsize
which produces the graphic (note use of \textbf{cut}(all) to get a
  blank canvas):
\begin{figure} [h]
   \centerline{\includegraphics[scale=1.25]{ch5p35.eps} }
	\caption{Using line(...) to Display Colors }
\end{figure}      
\newpage
\subsection{Geometric Figures: rect(...) }
The \textbf{qdraw} function \textbf{rect} has the syntax:\\
\verb|   rect( x1,y1,x2,y2, lc(c), lw(n), fill(c)  ) |\\
  which will draw a rectangle with opposite corners \verb|(x1,y1)| and \verb|(x2,y2)|.
The last three arguments are optional; without them the rectangle is drawn in
  black with line thickness 3 and with no fill color.
An example is \verb|rect(0,0,1,1,lc(brown),lw(2),fill(khaki) ) |.
We start with the basic rectangle call:
\small
\begin{verbatim}
(%i13) qdraw( xr(-1,2),yr(-1,2),rect(0,0,1,1) )$
\end{verbatim}
\normalsize
% eps file code
%(%i23) qdraw( xr(-1,2),yr(-1,2),rect(0,0,1,1,lw(5)),
%             line(-1,0,2,0,lw(2) ),
%             line(0,-1,0,2,lw(2)), pic(eps,"ch5p36") );
%  
with the result
\begin{figure} [h]
   \centerline{\includegraphics[scale=.8]{ch5p36.eps} }
	\caption{Default rect(0,0,1,1) }
\end{figure}      

\smallskip
We now add some color, thickness and fill:
\small
\begin{verbatim}
(%i14) qdraw( xr(-1,2),yr(-1,2),
            rect(0,0,1,1,lw(5),lc(brown),fill(khaki) ) )$
\end{verbatim}
\normalsize
% eps file code
%(%i29) qdraw( xr(-1,2),yr(-1,2),
%            rect(0,0,1,1,lw(8),lc(brown),fill(khaki) ),
%             pic(eps,"ch5p37" ),
%   line(-1,0,0,0,lw(2)),line(1,0,2,0,lw(2)),
%   line(0,-1,0,0,lw(2)),line(0,1,0,2,lw(2)) );
%
with the output:
\begin{figure} [h]
   \centerline{\includegraphics[scale=.9]{ch5p37.eps} }
	\caption{rect(0,0,1,1,lc(brown),lw(5),fill(khaki) ) }
\end{figure}      

\newpage
Finally, we use \textbf{rect} for a set of nested rectangles.
\small
\begin{verbatim}
(%i15) qdraw( xr(-3,3),yr(-3,3), rect( -2.5,-2.5,2.5,2.5,lw(4),lc(blue) ),
         rect( -2,-2,2,2,lw(4),lc(red) ),
         rect( -1.5,-1.5,1.5,1.5,lw(4),lc(green) ),
         rect( -1,-1,1,1,lw(4),lc(brown) ),
         rect( -.5,-.5,.5,.5,lw(4),lc(magenta) ),
           cut(all) )$
\end{verbatim}
\normalsize

% eps file code
%  (%i32) qdraw( xr(-3,3),yr(-3,3), rect( -2.5,-2.5,2.5,2.5,lw(8),lc(blue) ),
%         rect( -2,-2,2,2,lw(8),lc(red) ),
%         rect( -1.5,-1.5,1.5,1.5,lw(8),lc(green) ),
%         rect( -1,-1,1,1,lw(8),lc(brown) ),
%         rect( -.5,-.5,.5,.5,lw(8),lc(magenta) ),
%           cut(all), 
%          pic(eps,"ch5p38") );
%
which produces:
\begin{figure} [h]
   \centerline{\includegraphics[scale=1.3]{ch5p38.eps} }
	\caption{Nested Rectangles using rect(..) }
\end{figure}      
%\newpage
 
\subsection{Geometric Figures: poly(...) }
The \textbf{qdraw} function \textbf{poly} has the syntax:\\
\verb|   poly( pointlist, lc(c), lw(n), fill(c) ) |\\
  in which "pointlist" has the same form as when used with \textbf{pts}:\\
  \verb|  [ [x1,y1], [x2,y2], ... [xn,yn] ] | ,\\
  and the arguments lc, lw, and fill are optional and can be
  in any order after pointlist.
The last point in the list will be automatically connectd to the first.
  
\smallskip
The default call to \textbf{poly} has color black, line width 3
  and no fill color.
\small
\begin{verbatim}
(%i16) qdraw( xr(-2,2),yr(-1,2),cut(all),
            poly([ [-1,-1],[1,-1], [2,2] ]  ) )$
\end{verbatim}
\normalsize
% eps file code
% (%i34) qdraw( xr(-2,2),yr(-1,2),cut(all),
%            poly([ [-1,-1],[1,-1], [2,2] ],lw(5) ),
%            pic(eps, "ch5p39") );
%
\newpage

This default use of \textbf{poly} produces a "plain jane" triangle:
\begin{figure} [h]
   \centerline{\includegraphics[scale=.9]{ch5p39.eps} }
	\caption{Default use of poly(...) }
\end{figure}      

\smallskip
Next we create the work file "doplot2.mac" which contains the following Maxima function 
  which will draw eighteen right triangles in various colors:
\small
\begin{verbatim}
/*  eighteen triangles  */
disp("doplot2()")$
print("eighteen colored triangles")$
doplot2() := 
  block([cc, qlist,x1,x2,y1,y2,i,val ],
     cc : [aquamarine,beige,blue,brown,cyan,gold,goldenrod,green,khaki,
            magenta,orange,pink,plum,purple,red,salmon,skyblue,turquoise,
            violet,yellow ],            
     qlist : [ xr(-3.3,3.3), yr(-3.3,3.3) ],     
  /* top row of triangles  */
     y1  : 1,
     y2 : 3,
     for i:0 thru 5 do ( 
       x1 : -3 + i,
       x2 : x1 + 1,
       val : poly( [ [x1,y1],[x2,y1],[x1,y2]], fill( cc[i+1] ) ),
       qlist : append(qlist, [val ] ) 
     ),
  /* middle row of triangles  */
     y1 : -1,
     y2 : 1,
     for i:0 thru 5 do (
        x1 : -3 + i,
        x2 : x1 + 1,
        val : poly( [ [x1,y1],[x1,y2],[x2,y2]], fill( cc[i+7] ) ),
        qlist : append(qlist, [val ] ) 
     ),                
\end{verbatim}
\newpage
\begin{verbatim}
 /* bottom row of triangles  */
     y1 : -3,
     y2 : -1,
     for i:0 thru 5 do (
       x1 : -3 + i,
       x2 : x1 + 1,
       val : poly( [ [x1,y1],[x2,y1],[x1,y2]], fill( cc[i+13] ) ),
       qlist : append(qlist, [val ] )        
     ),          
     qlist : append(qlist,[ cut(all) ] ),                    
     apply( 'qdraw, qlist )                                   
  )$     
\end{verbatim}
\normalsize
Here is a record of use of this work file:
\small
\begin{verbatim}
(%i17) load(doplot2);
                                   doplot2()
eighteen colored triangles 
(%o17)                        c:/work2/doplot2.mac
(%i18) doplot2();
\end{verbatim}
\normalsize
and the resulting figure:
\begin{figure} [h]
   \centerline{\includegraphics[scale=.7]{ch5p40.eps} }
	\caption{Using poly(...) with Color}
\end{figure}      

\smallskip
For "homework", use \textbf{poly} and \textbf{pts} to draw the following figure.
(Hint: you should also use \textbf{xr(...)} ).
\begin{figure} [h]
   \centerline{\includegraphics[scale=.6]{ch5p41.eps} }
	\caption{Homework Problem}
\end{figure}      
% eps code
% (%i54) qdraw( poly([ [-.5,-1],[0,1], [.5,-1],[0,-2]  ],fill(khaki) ),
%               xr(-2,2),yr(-2,1),cut(all),
%               pts( [ [0,0],[0,-.5],[0,-1] ] ),
%                pic(eps,"ch5p41") );
%
\newpage

\subsection{Geometric Figures: circle(...) and ellipse(...) }
The \textbf{qdraw} function \textbf{circle} has the syntax:\\
  \verb|  circle( xc,yc, r, lc(c), lw(n), fill(c) ) | \\
  to draw a circle centered at \verb|(xc,yc)| and having radius \verb|r|.
The last three arguments are optional and may be entered in any order after
  the required first three arguments.
 \smallskip
This object will not "look" like a circle unless you take care to make the
  horizontal extent of the "canvas" about $1.4$ times the vertical extent
  (by using \textbf{xr(...)} and \textbf{yr(...)} ).

\smallskip
Here is the default circle in black, with line width 3, and no fill color.
\small
\begin{verbatim}
(%i19) qdraw( xr(-2, 2), yr(-2, 2), circle( 0, 0, 1 ) )$
\end{verbatim}
\normalsize
% eps file code
%(%i59) qdraw(xr(-2,2),yr(-2,2),circle(0,0,1,lw(5) ),
%          line(-2,0,2,0,lw(2)),line(0,-2,0,2,lw(2)),
%             pic(eps,"ch5p42") );
%
which looks like:
\begin{figure} [h]
   \centerline{\includegraphics[scale=.6]{ch5p42.eps} }
	\caption{Default "circle"}
\end{figure}      

\smallskip
By using \textbf{xr(...)} and \textbf{yr(...)} we try for a "round" circle and also add
  what should be a 45 degree line.
\small
\begin{verbatim}
(%i20) qdraw(xr(-2.1,2.1),yr(-1.5,1.5),cut(all),
            circle(0,0,1,lw(5),lc(brown),fill(khaki) ),
              line(-1.5,-1.5,1.5,1.5,lw(8), lc(red) ),
                key(bottom) )$
\end{verbatim}
\normalsize
% eps file code
% (%i68) qdraw(xr(-2.1,2.1),yr(-1.5,1.5),cut(all),
%            circle(0,0,1,lw(8),lc(brown),fill(khaki) ),
%              line(-1.5,-1.5,1.5,1.5,lw(12), lc(red) ),
%                key(bottom),pic(eps,"ch5p43" ,font("Times-Roman",18)) );
%
with the result:
\begin{figure} [h]
   \centerline{\includegraphics[scale=.85]{ch5p43.eps} }
	\caption{line over "round" circle}
\end{figure}      
\newpage
The line painted over the circle because of the order of the arguments to
  \textbf{qdraw}.
If we reverse the order, drawing the line before the circle:
\small
\begin{verbatim}
(%i21) qdraw(xr(-2.1,2.1),yr(-1.5,1.5),cut(all),
            line(-1.5,-1.5,1.5,1.5,lw(8),lc(red) ),
            circle(0,0,1,lw(8),lc(brown),fill(khaki) ),
                key(bottom) )$
\end{verbatim}
\normalsize
% eps file code
% (%i70) qdraw(xr(-2.1,2.1),yr(-1.5,1.5),cut(all),
%            line(-1.5,-1.5,1.5,1.5,lw(12),lc(red) ),
%            circle(0,0,1,lw(5),lc(brown),fill(khaki) ),
%                key(bottom),
%             pic(eps,"ch5p44" ,font("Times-Roman",18)) );
%
then the circle color will hide the line:
\begin{figure} [h]
   \centerline{\includegraphics[scale=.8]{ch5p44.eps} }
	\caption{circle over line}
\end{figure}      

\smallskip
The \textbf{qdraw} function \textbf{ellipse} has the syntax:\\
\verb|   ellipse( xc,yc,xsma,ysma,th0deg,dthdeg, lw(n), lc(c), fill(c) ) |\\
  which will plot a partial or whole ellipse centered at \verb|(xc,yc)|,
  oriented with ellipse axes aligned along the \verb|x| and \verb|y| axes,
  having horizontal semi-axis xsma, vertical semi-axis ysma, beginning
  at th0deg degrees measured counter clockwise from the  positive \verb|x| axis,
  and drawn for dthdeg degrees in the counter clockwise direction.
  
\smallskip
The last three arguments are optional.
The default is the outline of an ellipse for the specified angular range in
  color black, line width 3, and no fill color.
  
\smallskip
Here is the default behavior:
\small
\begin{verbatim}
(%i22) qdraw( xr(-4.2,4.2),yr(-3,3),
         ellipse(0,0,3,2,90,270) )$
\end{verbatim}
\normalsize
% eps file code
%  (%i6) qdraw( xr(-4.2,4.2),yr(-3,3),
%         ellipse(0,0,3,2,90,270),
%          pic(eps, "ch5p45",font("Times-Roman",18) ),
%           line(-4.2,0,4.2,0,lw(2)),
%           line(0,-3,0,3,lw(2)) );
\begin{figure} [h]
   \centerline{\includegraphics[scale=.6]{ch5p45.eps} }
	\caption{ellipse(0,0,3,2,90,270)}
\end{figure}      

\newpage
If we add color, fill, and some curvy background, as in
\small
\begin{verbatim}
(%i23) qdraw( xr(-5.6,5.6),yr(-4,4),ex1(x,x,-4,4,lc(blue),lw(5)),
                  ex1(4*cos(x),x,-4,4,lc(red),lw(5) ),
         ellipse(0,0,3,1,90,270,lc(brown),lw(5),fill(khaki)) ,
             cut(all) )$
\end{verbatim}
\normalsize
% eps file code
%(%i15) qdraw( xr(-5.6,5.6),yr(-4,4),ex1(x,x,-4,4,lc(blue),lw(10)),
%                  ex1(4*cos(x),x,-4,4,lc(red),lw(10) ),
%         ellipse(0,0,3,1,90,270,lc(brown),lw(8),fill(khaki)),
%             cut(all),pic(eps,"ch5p46",font("Times-Roman",18) ) );
%

we get
\begin{figure} [h]
   \centerline{\includegraphics[scale=1.3]{ch5p46.eps} }
	\caption{Filled Ellipse}
\end{figure}      

\smallskip
\subsection{Geometric Figures: vector(..)  }
The \textbf{qdraw} function \textbf{vector} has the syntax:\\
\verb|vector([x,y],[dx,dy],ha(thdeg),hb(v),hl(v),ht(t),lw(n),lc(c),lk(string) )|\\
   which draws a vector with components [dx,dy] starting at [x,y].\\
The first two list arguments are required, all others are optional and can be entered in
   any order after the first two required arguments.\\
The default "head angle" is 30 deg; change to 45 deg using ha(45) for example.\\
The default "head both" value is f for false; use hb(t) to set it to true,
       and hb(f) to return to false.\\
The default "head length" is 0.5; use hl(0.7) to change to 0.7.\\
The default "head type" is "nofilled"; use ht(e) for "empty", ht(f) for "filled",
       and ht(n) to change back to "nofilled".           \\
Once one of the "head properties" has been changed in a call to \textbf{vector},
   the next calls to vector assume the change is still in force. \\	   
The default line width is 3; use lw(5) to change to 5.\\
The default line color is black; use lc(brown) to change to brown.\\
The default is no key string; use lk("A1"), for example, to create a text string for the key.\\

\newpage

Here is a use of \textbf{vector} which accepts all defaults:
\small
\begin{verbatim}
(%i24) qdraw( xr(-2,2), yr(-2,2), vector( [-1,-1], [2,2] ) )$
\end{verbatim}
\normalsize
% eps file code
% (%i21) qdraw(xr(-2,2),yr(-2,2),vector([-1,-1],[2,2],lw(5)),
%          line(-2,0,2,0,lw(2)),line(0,-2,0,2,lw(2)),
%           pic(eps,"ch5p47",font("Times-Roman",18) ) );
%
with the result:
\begin{figure} [h]
   \centerline{\includegraphics[scale= 1]{ch5p47.eps} }
	\caption{Default Vector}
\end{figure}      

\smallskip
We can thicken and apply color to this basic vector with
\small
\begin{verbatim}
(%i25) qdraw(xr(-2,2),yr(-2,2),
           vector([-1,-1],[2,2],lw(5),lc(brown),lk("vec 1")),
            key(bottom) )$
\end{verbatim}
\normalsize
% eps file code
% (%i24) qdraw(xr(-2,2),yr(-2,2),
%           vector([-1,-1],[2,2],lw(8),lc(brown),lk("vec 1")),
%            key(bottom),
%             line(-2,0,2,0,lw(2)),line(0,-2,0,2,lw(2)),
%           pic(eps,"ch5p48",font("Times-Roman",18) )  );
%
which looks like:
\begin{figure} [h]
   \centerline{\includegraphics[scale=1]{ch5p48.eps} }
	\caption{Adding Color, etc.}
\end{figure}      

\newpage
Next we can alter the "head" properties, but let's also make this vector
  shorter.
We use \verb|ht(e)| to set head\_type to "empty", \verb|hb(t)| to
  set head\_both to "true", and \verb|ha(45)| to set head\_angle
  to $45$ degrees.
\small
\begin{verbatim}
(%i26) qdraw(xr(-2,2),yr(-2,2),
           vector([0,0],[1,1],lw(5),lc(blue),lk("vec 1"),
                  ht(e),hb(t),ha(45) ), key(bottom) )$
\end{verbatim}
\normalsize
% eps file code
%  (%i26) qdraw(xr(-2,2),yr(-2,2),
%           vector([0,0],[1,1],lw(8),lc(blue),lk("vec 1"),
%                       ht(e),hb(t),ha(45) ), 
%            key(bottom),
%                   line(-2,0,2,0,lw(2)),line(0,-2,0,2,lw(2)),
%           pic(eps,"ch5p49",font("Times-Roman",18) ) );
%
which produces:
\begin{figure} [h]
   \centerline{\includegraphics[scale=.8]{ch5p49.eps} }
	\caption{Changing Head Properties}
\end{figure}      

\smallskip
Once you invoke the head properties options, the new settings are
  used on your next calls to \textbf{vector} (unless you deliberately
  change them).
Here is an example of that memory feature at work:
\small
\begin{verbatim}
(%i27) qdraw(xr(-2.8,2.8),yr(-2,2),
           vector([0,0],[1,1],lw(5),lc(blue),lk("vec 1"),
                       ht(e),hb(t),ha(45) ),
           vector([0,0],[-1,-1],lw(5),lc(red),lk("vec 2")),
            key(bottom) )$
\end{verbatim}
\normalsize
% eps file code
% (%i28) qdraw(xr(-2.8,2.8),yr(-2,2),
%           vector([0,0],[1,1],lw(8),lc(blue),lk("vec 1"),
%                       ht(e),hb(t),ha(45) ),
%           vector([0,0],[-1,-1],lw(8),lc(red),lk("vec 2")),
%            key(bottom),
%                   line(-2,0,2,0,lw(2)),line(0,-2,0,2,lw(2)),
%           pic(eps,"ch5p50",font("Times-Roman",18) ) );
%
and we also used the x-range setting to get the geometry closer to reality,
with the result:
\begin{figure} [h]
   \centerline{\includegraphics[scale=.8]{ch5p50.eps} }
	\caption{Head Properties Memory at Work}
\end{figure}      		   
\newpage
\subsection{Geometric Figures: arrowhead(..) }
The syntax of the \textbf{qdraw} function \textbf{arrowhead} is:\\
\verb|  arrowhead( x, y, theta-degrees, s, lc(c), lw(n) )|\\
  which will draw an arrow head with the vertex at \verb|(x,y)|.\\
The first four arguments are required and must be numbers.\\
The third argument theta is an angle in degrees and is the direction
  the arrowhead is to point relative to the positive x axis,
  ccw from x axis taken as a positive angle.\\
The fourth argument s is the length of the sides of the arrowhead.\\
The arguments lc(c) and lw(n) are optional, and are used to
  alter the default color (black) and line width (3).\\
The opening half angle is hardwired to be phi = 25 deg = 0.44 radians.\\
The geometry will look better if the x-range is about 1.4 times the y-range.

\smallskip
Here are four arrow heads drawn with the default line widths and color
  and "size" $0.3$, which show the use of the direction argument in degrees.
\small
\begin{verbatim}
(%i28) qdraw(xr(-2.8,2.8),yr(-2,2),
            arrowhead(1.5,0,180,.3),arrowhead(0,1,270,.3),
            arrowhead(-1.5,0,0,.3),arrowhead(0,-1,90,.3) )$
\end{verbatim}
\normalsize
% eps file code
%(%i6) qdraw(xr(-2.8,2.8),yr(-2,2),
%            arrowhead(1.5,0,180,.3),arrowhead(0,1,270,.3),
%            arrowhead(-1.5,0,0,.3),arrowhead(0,-1,90,.3),
%            line(-2.8,0,2.8,0,lw(2)),line(0,-2,0,2,lw(2)),
%             pic(eps,"ch5p51",font("Times-Roman",18) ) );
%			 
\begin{figure} [h]
   \centerline{\includegraphics[scale=.8]{ch5p51.eps} }
	\caption{Using arrowhead(..)}
\end{figure}      

\smallskip

\subsection{Labels with Greek Letters}
\subsubsection{Enhanced Postscript Methods}
Here we combine \textbf{line(..)}, \textbf{ellipse},  \textbf{arrowhead}
  and \textbf{label}, using the enhanced postscript abilities of 
  \textbf{draw2d}'s \verb|terminal = eps| option, which became
  effective with the June 11, 2008 draw package update.
This update includes an extension of the  \verb|terminal = eps| abilities to
 include local conversion of font properties and the use of Greek and some math characters.
As of the writing of this section, it was necessary to download this update
 from the webpage\\
 \verb|http://maxima.cvs.sourceforge.net/maxima/maxima/share/draw/draw.lisp|.
On that webpage you will see \verb|Log of /maxima/share/draw/draw.lisp|, and
  the top entry is\\
  \verb|Revision 1.31 - (view) (download) (annotate) - [select for diffs] |\\
  \verb|Wed Jun 11 18:19:55 2008 UTC|.
Click on the "download" link, and the text of draw.lisp will appear in your
  browser with the top looking like:
\small
\begin{verbatim}
;;;                 COPYRIGHT NOTICE
;;;  
;;;  Copyright (C) 2007 Mario Rodriguez Riotorto
;;;  
;;;  This program is free software; you can redistribute
;;;  it and/or modify it under the terms of the
;;;   ........................ etc
\end{verbatim}
\normalsize
This is a program written in the Lisp language, and down near the bottom is a
series of lines which provide for the enhanced postscript behavior:
\small
\begin{verbatim}
      ($eps (format cmdstorage "set terminal postscript eps enhanced ~a
                            	  size ~acm, ~acm~%set out '~a.eps'"
                           (write-font-type) ; other alternatives are Arial, Courier
                           (get-option '$eps_width)
                           (get-option '$eps_height)
                           (get-option '$file_name)))
\end{verbatim}
\normalsize
If you save this text file with the name "draw.lisp" and place it in the draw package
  folder (on my Windows computer, the rather complicated path:\\
\verb|drive c, program files, maxima-5.15.0, share, maxima, 5.15.0, share, draw | )
to replace the old "draw.lisp" (which you could rename prior to the save as ),
  then you can use the label syntax inside strings as shown in the following.
  
\smallskip
\small
\begin{verbatim}
(%i29) qdraw(xr(0,2.8),yr(0,2),
             line(0,0,2.8,0,lw(2)),
             line(0,0,2,2,lc(blue),lw(8) ),
           ellipse(0,0,1,1,0,45 ),
           arrowhead(0.707,0.707,135,0.15),
   label(["{/=36 {/Symbol q   \\254 }  The Incline Angle}",1,0.4]),
            cut(all),
           pic(eps,"ch5p52" ) );
\end{verbatim}
\normalsize
%\newpage

The result looks like:
\begin{figure} [h]
   \centerline{\includegraphics[scale=.7]{ch5p52.eps} }
	\caption{line(..), ellipse(..), arrowhead(..), label(..)}
\end{figure}      

%\newpage
A summary of the enhanced postscript syntax can be downloaded:\\
\verb|http://www.telefonica.net/web2/biomates/maxima/gpdraw/ps/ps_guide.ps |
although the examples of using the "characters" works for me only if
I use two back slashes, as in the example just shown, where the
  entry \verb|\\254| inside the \verb|{/Symbol }| structure produces the leftward
  pointing arrow (thanks to the draw package developer, Mario Rodriguez Riotorto, 
  for the enhanced postscript abilities, and for aiding my understanding of 
  how to use these features).
An example of creating text for a label which includes the integral sign
  and Greek letters is given on the webpage:\\
\verb|http://www.telefonica.net/web2/biomates/maxima/gpdraw/ps/|.  
\smallskip

The entry \verb|{/Symbol q }| by itself would produce just the Greek
  letter $\theta$.
Wrapping the text entry in the structure \verb|{/=36  }| accepts the
  default font type and sets the font size to 36 for the text inside the
  pair of braces.
 
\smallskip
Here we make a lower case Latin to Greek conversion reminder using four instances of
  \textbf{label}, although we could alternatively have used
  the syntax \verb| label( [s1,x1,x2], [s2,x2,y2],... )|.
\small
\begin{verbatim}
(%i30) qdraw(xr(-3,3),yr(-2,2),label_align(c),
         label( ["{/=48 a b c d e f g h i j k l m}",0,1.5] ),
        label( ["{/Symbol=48 a b c d e f g h i j k l m}",0,0.5] ),
        label( ["{/=48 n o p q r s t u v w x y z}",0,-.5] ),
         label( ["{/Symbol=48 n o p q r s t u v w x y z}",0,-1.5] ),   
        cut(all), pic(eps,"ch5p53") )$
\end{verbatim}
\normalsize
Note how we increase the font size of the latin alphabet \verb|a,b,c...|.
Here is the resulting eps figure:
\begin{figure} [h]
   \centerline{\includegraphics[scale=.7]{ch5p53.eps} }
	\caption{Lower Case Latin to Greek}
\end{figure}      

\smallskip

We can repeat that label figure using upper case Latin letters:
\small
\begin{verbatim}
(%i31) qdraw(xr(-3,3),yr(-2,2),label_align(c),
         label( ["{/=48 A B C D E F G H I J K L M}",0,1.5] ),
        label( ["{/Symbol=48 A B C D E F G H I J K L M}",0,0.5] ),
        label( ["{/=48 N O P Q R S T U V W X Y Z}",0,-.5] ),
         label( ["{/Symbol=48 N O P Q R S T U V W X Y Z}",0,-1.5] ),   
        cut(all), pic(eps,"ch5p54") )$
\end{verbatim}
\normalsize
You can see the resulting figure on the next page.

\smallskip
\smallskip
Useful character codes, used as \verb|{/Symbol \\abc  \\rst  etc }|\\
  or as   \verb|{/Symbol=36 \\abc  \\rst  etc }| are:\\
\verb|\\243| (less than or equal )\\
\verb|\\245| (infinity symbol)\\
\verb|\\253| (double ended arrow)\\
\verb|\\254| (left arrow)\\
\verb|\\256| (right arrow)\\
\verb|\\261| (plus or minus)\\
\verb|\\263| (greater than or equal)\\
\verb|\\264| (times)\\
\verb|\\271| (not equal)\\
\verb|\\273| (approx equal)\\
\verb|\\345| (summation sign)\\
\verb|\\362| (integral sign)\\

\begin{figure} [h]
   \centerline{\includegraphics[scale=.7]{ch5p54.eps} }
	\caption{Upper Case Latin to Greek}
\end{figure}      

\smallskip

Here we use \textbf{label} to illustrate these possible symbols you can use:
\small
\begin{verbatim}
(%i32) s1 : "{/Symbol=48 \\243   \\245   \\253   \\254   \\256}"$
(%i33) s2 : "{/Symbol=48 \\261   \\263   \\264   \\271   \\273   \\345   \\362}"$
(%i34) qdraw(xr(-3,3),yr(-2,2),label_align(c),
        label( [s1,0,1] ), label( [s2,0,-1] ), cut(all), pic(eps,"ch5p55") )$
\end{verbatim}
\normalsize
which produces the figure:
\begin{figure} [h]
   \centerline{\includegraphics[scale=.7]{ch5p55.eps} }
	\caption{Useful Character Code Symbols}
\end{figure}      



\newpage
\subsubsection{Windows Fonts Methods with jpeg Files}
We can produce the Greek letter $\theta$ in a jpeg file or a png file by using
  the Greek font files in the\\
  \verb|c:\windows\fonts| folder as follows:  
\small
\begin{verbatim}
(%i35) qdraw(xr(0,2.8),yr(0,2),
             line(0,0,2.8,0),
             line(0,0,2,2,lc(blue),lw(5) ),
           ellipse(0,0,1,1,0,45 ),
           arrowhead(0.707,0.707,135,0.15),
            label(["q",1,0.4]), cut(all),
           pic(jpg,"ch5p52p",font("c:/windows/fonts/grii.ttf",36)) );  
\end{verbatim}
\normalsize
which produces the file \verb|ch5p52.jpg|, which we converted to an eps
  file using cygwin's convert function:\\
  \verb|convert name.jpg name.eps| which will also work to convert a png graphics file.

\begin{figure} [h]
   \centerline{\includegraphics[scale=.5]{ch5p52p.eps} }
	\caption{Greek in jpg converted to eps}
\end{figure}      

%\newpage
Here we use the label function to show all the greek letters
 available via the windows, fonts folder:
\small
\begin{verbatim}
(%i36) qdraw(xr(-3,3),yr(-2,2), 
     label(["a b c d e f g h i j k l m",-2.5,1.5],
             ["n o p q r s t u v w x y z",-2.5,0.5],
              ["A B C D E F G H I J K L M",-2.5,-0.5],
                ["N O P Q R S T U V W X Y Z",-2.5,-1.5]  ),     
            cut(all),
           pic(png,"ch5p60",font("c:/windows/fonts/grii.ttf",24)) );
\end{verbatim}
\normalsize
\newpage

After conversion of the png to an eps graphics file using Cygwin's convert function,
we get:
\begin{figure} [h]
   \centerline{\includegraphics[scale=.4]{ch5p60.eps} }
	\caption{windows fonts conv. of a - Z to Greek}
\end{figure}      

\smallskip

\subsubsection{Using Windows Fonts with the Gnuplot Console Window}
In the default Windows Gnuplot console mode, you can convert some
  Latin letters to Greek as follows:  
  
\small
\begin{verbatim}
(%i37) qdraw(xr(0,2.8),yr(0,2),
             line(0,0,2.8,0),
             line(0,0,2,2,lc(blue),lw(5) ),
           ellipse(0,0,1,1,0,45 ),
           arrowhead(0.707,0.707,135,0.15),
            label(["q",1,0.4]), cut(all) );
\end{verbatim}
\normalsize
When the console graphics window appears, right click on the upper left corner
 icon and select Options, Choose Font.
In the Font panels, choose Graecall font, "regular" from the
  middle panel, and size 36 from the right panel and click "ok".
The English letter "q" (lower case) is then converted to the Greek
  lower case theta.
Use the Gnuplot window menu again to save the resulting image to
  the clipboard, and open an image viewer.
I use the freely available Infanview.
If you use View, Paste, the clipboard image appears inside Infanview,
  and you can save the image as a jpeg file in your choice of folder.
Since I am using eps graphics files for this latex file, I converted
  the jpg to eps using Cygwin's convert function:\\
\verb|convert name.jpg  name.eps|.
%\newpage

Here is the result:
\begin{figure} [h]
   \centerline{\includegraphics[scale=.2]{ch5p52w.eps} }
	\caption{Greek via Windows Clipboard}
\end{figure}      
  
\newpage

Unfortunately, saving the Gnuplot window image to the Window's clipboard
  also saves the current cursor position, which is not desirable.

\subsection{Even More with more(...) }
You can use the \textbf{qdraw} function \textbf{more(...)}, containing any legal \textbf{draw2d}
  elements, as we illustrate by adding a label to the x-axis and a title.
We focus here on producing an eps graphics file to display the enhanced ability to
  show subscripts and superscripts.
\small
\begin{verbatim}
(%i38)  qdraw( lw(8), ex([x,x^2,x^3],x,-2,2),
          more(xlabel = "X AXIS", title="intersections of x, x^2, x^3" ),
		   cut(key),line(-2,0,2,0,lw(2)),line(0,-8,0,8,lw(2)),
               vector([-1,5],[-0.4,-2.7],lc(red),hl(0.1) ),
               label(["x^2",-0.9,6]),
               vector([-1.2,-6],[-0.5,0],lc(turquoise),lw(8)),
               label( ["x^3", -1,-5.5] ),
              pts( [[-1,-1],[0,0],[1,1]],ps(2),pc(magenta)  ),
              pic(eps,"ch5p56",font("Times-Roman",28)) )$
\end{verbatim}
\normalsize
The lines for the x and y axes need special emphasis to show up clearly with
  an eps file, so we have used \textbf{qdraw}'s \textbf{line} function for
  that task.
We also need to increase the line width setting for the eps file case, which
  we have done with the \textbf{qdraw} top level function \textbf{lw}, which
  only affects the "quick plotting" functions \textbf{ex} and \textbf{imp}.
We have used \textbf{qdraw}'s \textbf{more} function to provide an x-axis
  label and a title.
The font setting in the \textbf{pic} function supplies an overall drawing
  font type and size which affects all elements unless locally over-ridden
  with the special enhanced postscript features.
In the title and labels, \verb|x^n| is converted automatically to $x^n$.
\smallskip

Here is the resulting plot:
\begin{figure} [h]
   \centerline{\includegraphics[scale=1.3]{ch5p56.eps} }
	\caption{Using more(...) for Title and X Axis Label}
\end{figure}      

\newpage
\subsection{Programming Homework Exercises}
\subsubsection{General Comments}
The file qdraw.mac is a text file which you can modify with a good text editor
  such as notepad2.
This Maxima code is heavily commented as an aid to passing on some Maxima
  language programming examples.
You can get some experience with the Maxima programming language elements by
  copying the file qdraw.mac to another name, say myqdraw.mac, and use
  that copy to make modifications to the code which might interest you.
By frequently loading in the modified file with \verb|load(myqdraw)|, you can
  let Maxima check for syntax errors, which it does immediately.
  
  \smallskip
  
The most common syntax errors involve parentheses and commas, with strange
  error messages such as " BLANK IS NOT AN INFIX OPERATOR", or "TOO MANY PARENTHESES", etc.
Placing a comma just before a closing parenthsis is a fatal error which
  can nevertheless creep in.
This sounds obvious, but you may find it useful to insert some special debug
  printouts, such as \verb|print("in blank, a = ",a)| or \verb|display(a)|, perhaps
  at the end of a do loop, so you are working with the structure:\\
\small
\begin{verbatim}
  for i thru n do (
    job1,
    job2,
    job3,
	print(" i = ",i,"  blank = ",blank)
	  /* end do loop */
	),
	...program continues...

\end{verbatim}
\normalsize
When you are finished debugging a section, you either will comment out the
  debug printout or simply delete it to clean up the code.
If you are not fully awake, you might then load into Maxima
\small
\begin{verbatim}
  for i thru n do (
    job1,
    job2,
    job3,
      /* end do loop */
	),
	...program continues...

\end{verbatim}
\normalsize
and, of course, Maxima will object, since that extra comma no longer
  makes sense.
  
 \smallskip
 
It is crucial to use a good text editor which will "balance" parentheses, brackets,
  and braces to minimize parentheses etc errors.

\smallskip
If you look at the general structure of \textbf{qdraw}, you will see that most of the
  real work is done by \textbf{qdraw1}. 
If you call \textbf{qdraw1} instead of \textbf{qdraw}, you will be presented with
  a rather long list of elements which are understood by \textbf{draw2d}.
Even if you use \textbf{qdraw}, you will see the same long list wrapped by "draw2d"
  if you have not loaded the \textbf{draw} package.

\smallskip
One feature you should look at is how a function which takes an arbitrary number
  of arguments, depending on the user (as does the function \textbf{draw2d}), is defined.
If this seems strange to you, experiment with a toy function having a variable
  number of arguments, and use printouts inside the function to see what Maxima is
  doing.

\newpage
\subsubsection{XMaxima Tips}
It is useful to first try out a small code idea directly in XMaxima, even if the code is
  ten or fifteen lines long, since the XMaxima interface has been greatly improved.
When you want to edit your previous "try", use \verb|Alt-p| to enter your previous
  code, and immediately backspace over the final \verb|);| or \verb|)$|.
You can then cursor up to an area where you want to add a new line of code,
  and with the cursor placed just after a comma, press \verb|ENTER| to create
  a new (blank) line.
Since the block of code has not been properly concluded with either a \verb|);|
  or \verb|)$|, Maxima will not try to "run" with the version you are working
  on when you press \verb|ENTER|.
Once you have made the changes you want, cursor your way to the end and put back
  the correct ending and then pressing \verb|ENTER| will send your code to the 
   Maxima engine.  

\smallskip
The use of \verb|HOME|, \verb|END|, \verb|PAGEUP|, \verb|PAGEDOWN|, \verb|CNTRL-HOME|, and
  \verb|CNTRL-END| greatly speeds up working with XMaxima.
For example to copy a code entry up near the top of your current workspace,
  first enter \verb|HOME| to put the cursor at the beginning of the current
  line, then \verb|PAGEUP| or \verb|CNTRL-HOME| to get up to the top fast,
  then drag over the code (don't include the \verb|(%i5)| part) to the
  end but not to the concluding \verb|);| or \verb|)$|.
You can hold down the \verb|SHIFT| key and use the right (and left) cursor key to
  help you select a region to copy.
Then press \verb|CNTRL-C| to copy the selected code to Window's clipboard.
Then press \verb|CTRL-END| to have the cursor move to the bottom of your workspace
  where XMaxima is waiting for your next input.
Press \verb|CNTRL-V| to paste your selection.
If the selection extends over multiple lines, use the down cursor key to find
  the end of the selection which should be without the proper code ending \verb|);|
  or \verb|)$|.
You are then in the driver's seat and can cursor your way around the code and make
  any changes without danger of XMaxima pre-emptively sending your work to the
  computing engine until you go to that end and provide the proper ending.  

\subsubsection{Suggested Projects}
You will have noticed that we used the \textbf{qdraw} function \textbf{more}
  in order to insert axis labels and a title into our plot.
Design  \textbf{qdraw} functions \verb|xlabel(string)|, \verb|ylabel(string)|,
  and \verb|title(string)|.
Place them in the "scan 3" section of \textbf{qdraw} and try them out.
You will need to pay attention to how new elements get passed to \textbf{draw2d}.
In particular, look at the list \verb|drlist|, using your text editor search
  function ( in notepad2, Ctrl-f ) to see how that list is constructed based
  on the user input.
  
  \smallskip
 A second small project would be to add a "line type" option for the
   \textbf{qdraw} function \textbf{line}. 
My experience is that setting \verb|line_type = dots| in \textbf{draw2d} produces
  no immediate change in the Windows Gnuplot console window, produces a finely
  dotted line for jpeg and png image files, and produces a dashed line with eps image files.
Your addition to \textbf{qdraw} should follow the present style, so the user
  would use the syntax \verb|line(x1,y1,x2,y2,lc(c),lw(n),lk(string),lt(type) )|,
  where type is either s or d (for solid or dots).
  
\smallskip
A third small project would be to design a function \textbf{triangle} for \textbf{qdraw},
  including the options which are presently in \textbf{poly}.


\smallskip
A fourth small project would be to include the option cbox(..) in the
  \textbf{qdensity} function. The present default is to include the
  colorbox key next to the density plot, but if the user entered
  \verb|qdensity(....,cbox(f) ) |, the colorbox would be removed.

\smallskip


A more challenging project would be to write a \textbf{qdraw} function
  which would directly access the creation of bar charts.
These notes are written with the needs of the typical physical science or
  engineering user in mind, so no attention has been paid to bar charts here.
Naturally, if you frequently construct bar charts, this project would be
  interesting for you.
Start this project by first working with \textbf{draw2d} directly, to get
  familiar with what is already available, and to avoid "re-creating the wheel".  

\smallskip
One general principle to keep in mind is that the Maxima language is an "interpretive
  language"; Maxima does not make multiple passes over your code to reconcile function
  references, such as a compiler does. 
This means that if a part of your code "calls" some user defined function, Maxima
  needs to have already read about that function definition in your code.

\subsection{Acknowledgements}
The author would like to thank Mario Rodriguez Riotorto, the creator of Maxima's \textbf{draw}
 graphics interface to Gnuplot, for his encouragement and advice at crucial stages
  in the development of the \textbf{qdraw} interface to \textbf{draw2d}.
The serious graphics user should spend time with the many powerful features of the
  \textbf{draw} package, and the examples provided on the \textbf{draw} webpages,\\
  \verb|http://www.telefonica.net/web2/biomates/maxima/gpdraw/|.\\
These examples go far beyond the simple graphics in this chapter.
The recent updating of the \textbf{draw} package to allow use of
  Gnuplot's enhanced postscript features makes Maxima a more attractive tool
  for the creation of educational diagrams.




\end{document}