% e. woollett
%   april 08
% file calculus1.tex
% things to add: 
%  section on directional derivatives in three coordinate systems

% edit with Notepad++, then load into LED for latexing
\documentclass[12pt]{article}
\usepackage[dvips,top=1.5cm,left=1.5cm,right=1.5cm,foot=1cm,bottom=1.5cm]{geometry}
\usepackage{times}
\usepackage{amsmath}
\usepackage{amsbsy}
\usepackage{graphicx} 
\usepackage{url}
\urldef\tedhome\url{ http://www.csulb.edu/~woollett/  }
\urldef\tedmail\url{ woollett@charter.net}

%%%%%%%%%%%%%%%%%%%%%%%%%%%%%%%%%%%%%%%%%%%%%%%%%%%%%%%%%%%%%%%%%%%%%%%%
%   title page
%%%%%%%%%%%%%%%%%%%%%%%%%%%%%%%%%%%%%%%%%%%%%%%%%%%%%%%%%%%%%%%%%%%%%%
\title{Maxima by Example: Ch.6: Differential Calculus 
            \thanks{This version uses Maxima 5.15. This is a live
            document. Check \;  \tedhome \; for the latest version of these notes. Send comments and
			 suggestions to \tedmail } }


\author{Edwin L. Woollett}
\date{\today}
%%%%%%%%%%%%%%%%%%%%%%%%%%%%%%%%%%%%%%%%%
%          document
%%%%%%%%%%%%%%%%%%%%%%%%%%%%%%%%%%%%%%%%%%
\begin{document}
\small
\maketitle
\tableofcontents
\numberwithin{equation}{section}
\newpage
\normalsize
COPYING AND DISTRIBUTION POLICY\\
\\    
	
    
    This document is part of a series of notes titled
    "Maxima by Example" and	is made available\\
	via the author's webpage 
	\verb|http://www.csulb.edu/~woollett/ | to aid
	new users of the Maxima computer algebra system.\\
	
	\smallskip
	NON-PROFIT PRINTING AND DISTRIBUTION IS PERMITTED.\\
	You may make copies of this document and distribute them to others
	as long as you charge no more than the costs of printing.\\
	
	\smallskip
	These notes (with some modifications) will be published in book form
	eventually via Lulu.com in an arrangement which will continue
	to allow unlimited free download of the pdf files as well as the option
	of ordering a low cost paperbound version of these notes.
\newpage

\setcounter{section}{5}

\section{Differential Calculus}
The methods of calculus lie at the heart of the physical sciences and engineering.

\smallskip
The student of calculus needs to take charge of his or her own understanding, taking the
  subject apart and putting it back together again in a way that makes logical sense.
The best way to learn any subject is to work through a large collection of problems.
The more problems you work on your own, the more you "own" the subject.
Maxima can help you make faster progress, if you are just learning calculus.

\smallskip
For those who are already calculus savvy, the examples in this chapter will offer
  an opportunity to see some Maxima tools in the context of very simple
  examples, but you will likely be thinking about much harder problems you want to
  solve as you see these tools used here.

\smallskip
After examples of using \textbf{diff} and \textbf{gradef}, we present examples of
  finding the critical and inflection points of plane curves defined by an explicit
  function.

\smallskip
We then present examples of calculating and plotting the tangent and normal to a
  point on a curve, first for explicit functions and then for implicit functions.  

\smallskip
We also have two examples of finding the maximum or minimum of a function of two 
  variables.  

\smallskip
We then present several examples of using Maxima's powerful \textbf{limit} function,
 followed by several examples of using \textbf{taylor}(..).

\smallskip
The next section shows examples of using the vector calculus functions (as well as
  cross product) in the package vcalc.mac, developed by the author and available on
  his webpage with this chapter, to calculate the gradient, divergence, curl,
  and Laplacian in cartesian, cylindrical, and spherical polar coordinate systems.
An example of using this package would be \verb|curl([r*cos(theta),0,0]);| (if the
  current coordinate system has already been changed to spherical polar) or
  \verb|curl([r*cos(theta),0,0], s(r,theta,phi) );| if the coordinate system needs
  to be shifted to spherical polar from either cartesian (the starting default) or
  cylindrical.
The order of list vector components corresponds to the order of the arguments 
  in \verb|s(r,theta,phi)|.  
The Maxima output is the list of the vector curl components in the current coordinate system,
  in this case \verb|[0, 0, sin(theta)]| plus a reminder to the user of what the current
   coordinate system is and what symbols are currently being used for the independent 
   variables.

\smallskip
Thus the syntax is based on lists and is similar (although better!) than Mathematica's
  syntax.
There is a separate function to change the current coordinate system.
An example of its use would be\\
 \verb|setcoord( cy(rho,phi,z) );| to set the use
  of cylindrical coordinates (rho,phi,z), or\\
  \verb|setcoord( cy(r,t,z) );| to set
  cylindrical coordinates (r,t,z).

\smallskip
The package vcalc.mac also contains the plotting function \verb|plotderiv(..)| which
  is useful for "automating" the plotting of a function and its first \verb|n| derivatives.


\smallskip
The next two sections dicuss the use of the batch file  mode of problem solving
  by presenting Maxima based derivations of the form of the gradient, divergence,
   curl, and Laplacian by starting with the cartesian forms and changing 
   variables to (separately)  cylindrical and spherical polar coordinates.
   These two sections show Maxima's implementation of the calculus chain rule at work
   with use of both \textbf{depends} and \textbf{gradef}.
   
\subsection{Differentiation of Explicit Functions}
We begin with explicit functions of a single variable.
After giving a few examples of the use of Maxima's \textbf{diff(...)} function,
  we will discuss critical and inflection points of curves defined by
  explicit functions, and the construction and plotting of the tangent and
  normal of a point of such curves.


\subsubsection{diff}
The command \textbf{diff}\verb|(expr,var,num)| will differentiate the expresssion 
  in slot one with respect to the variable entered in slot two a number of times
  determined by a positive integer in slot three.
Unless a dependency has been established, all parameters and "variables" in the
  expression are treated as constants when taking the derivative.  
  
Thus \textbf{diff}\verb|(expr,x,2)| will yield the second derivative of \verb|expr|
  with respect to the variable \verb|x|.\\

\smallskip
The simple form \textbf{diff}\verb|(expr, var)| is equivalent to \textbf{diff}\verb|(expr,var,1)|.\\  

If the expression depends on more than one variable, we can use commands such as
  \textbf{diff}\verb|(expr,x,2,y,1)| to find the result of taking the second
  derivative with respect to \verb|x| (holding \verb|y| fixed) followed by the first derivative with
  respect to \verb|y| (holding \verb|x| fixed).
Here are some simple examples.\\
\smallskip

We first calculate the derivative of $x^{n}$.
\small
\begin{verbatim}
(%i1) diff(x^n,x);
                                      n - 1
(%o1)                              n x
\end{verbatim}
\normalsize

Next we calculate the third derivative of $x^{n}$.
\small
\begin{verbatim}
(%i2) diff(x^n,x,3);
                                              n - 3
(%o2)                      (n - 2) (n - 1) n x
\end{verbatim}
\normalsize

Here we take the derivative (with respect to x) of an expression
  depending on both $x$ and $y$.
\small
\begin{verbatim}
(%i3) diff(x^2 + y^2,x);
(%o3)                                 2 x
\end{verbatim}
\normalsize

You can differentiate with respect to any expression that does not
  involve explicit mathematical operations.
\small
\begin{verbatim}
(%i4) diff( x[1]^2 + x[2]^2, x[1] );
(%o4)                                2 x
                                        1
\end{verbatim}
\normalsize
Note that $x_{1}$ is Maxima's way of "pretty printing" \verb|x[1]|.
We can use the \textbf{grind} function to display the output \verb|%o4|
  in the "non-pretty print mode" (what would have been returned if we
  had set the \textbf{display2d} switch to \verb|false|).
\small
\begin{verbatim}
(%i5) grind(%)$
2*x[1]$
(%i6) display2d;
(%o6)                                true
\end{verbatim}
\normalsize
Note the dollar sign \textbf{grind} adds to the end of \textbf{its} output.  

\smallskip
Finally, an example of using one invocation of \textbf{diff}  to differentiate 
  with respect to more than one variable:
\small
\begin{verbatim}
(%i7) diff(x^2*y^3,x,1,y,2);
(%o7)                               12 x y
\end{verbatim}
\normalsize

\subsubsection{Total Differential}
If you use the \textbf{diff} function without a symbol in slot two, Maxima returns the
  "total differential" of  the expression in slot one, and by default assumes every
  parameter is a variable.

  \smallskip
If an expression contains a single parameter, say $x$, then \verb|diff(expr)| will
  generate
\begin{equation}
d\,f(x) = \left(\frac{d\, f(x)}{d\, x}  \right) \; d\,x
\end{equation}  
\smallskip
In the first example, we calculate the differential of the expression $x^2$, that is,
 the derivative of the expression (with respect to the variable $x$)
  multiplied by "the differential of the independent variable $x$", $d\,x$.
Maxima uses \textbf{del(x)} for the differential of $x$, a small increment of $x$.  
\small
\begin{verbatim}
(%i1) diff(x^2);
(%o1)                             2 x del(x)
\end{verbatim}
\normalsize
If the expression contains two parameters $x$ and $y$, then the total 
  differential is equivalent to the Maxima expression\\
\verb|  diff(expr,x)*del(x) + diff(expr,y)*del(y) |, \\
which generates (using conventional notation):
\begin{equation}
 d\,f(x,y) = \left(\frac{\partial f(x,y)}{\partial x}  \right)_{y} \; d\,x  +
    \left(\frac{\partial f(x,y)}{\partial y}  \right)_{x} \; d\,y.
\end{equation}  
Each additional parameter induces an additional term of the same form.  
\small
\begin{verbatim}
(%i2) diff(x^2*y^3);
                           2  2               3
(%o2)                   3 x  y  del(y) + 2 x y  del(x)
(%i3) diff(a*x^2*y^3);
                    2  2                 3           2  3
(%o3)          3 a x  y  del(y) + 2 a x y  del(x) + x  y  del(a)
\end{verbatim}
\normalsize

%\newpage
We can use the \textbf{subst} function to replace, say \verb|del(x)|, by anything else:
\small
\begin{verbatim}
(%i4) subst(del(x) = dx, %o1);
(%o4)                               2 dx x
(%i5) subst([del(x)= dx, del(y) = dy, del(a) = da],%o3 );
                         2  3             3           2  2
(%o5)                da x  y  + 2 a dx x y  + 3 a dy x  y
\end{verbatim}
\normalsize
%\newpage
You can use the \textbf{declare} function to prevent some of the symbols from
  being treated as variables when calculating the total differential:
\small
\begin{verbatim}
(%i6) declare (a, constant)$
(%i7) diff(a*x^2*y^3);
                           2  2                 3
(%o7)                 3 a x  y  del(y) + 2 a x y  del(x)
(%i8) declare([b,c],constant)$
(%i9) diff( a*x^3 + b*x^2 + c*x );
                                2
(%o9)                     (3 a x  + 2 b x + c) del(x)
(%i10) properties(a);
(%o10)                            [constant]
(%i11) propvars(constant);
(%o11)                             [a, b, c]
(%i12) kill(a,b,c)$
(%i13) propvars(constant);
(%o13)                                []
(%i14) diff(a*x);
(%o14)                        a del(x) + x del(a)
\end{verbatim}
\normalsize
We can "map" \textbf{diff} on to a list of functions of $x$, say, and divide by \textbf{del(x)},
  to generate a list of derivatives, as in
\small
\begin{verbatim}
(%i15) map('diff,[sin(x),cos(x),tan(x) ] )/del(x);
                                                2
(%o15)                     [cos(x), - sin(x), sec (x)]
(%i16) map('diff,%)/del(x);
                                              2
(%o16)               [- sin(x), - cos(x), 2 sec (x) tan(x)]
\end{verbatim}
\normalsize

  
\subsubsection{Controlling the Form of a Derivative with gradef(..)}

We can use \textbf{gradef} is to select one from among a number of
  alternative ways of writing the result of differentiation.
The large number of "trigonometric identities" means that any given expression
  containing trig functions can be written in terms of a different set of
  trig functions.

\smallskip
As an example, consider trig identities which express $\sin(x)$,  $\cos(x)$,
  and $\sec(x)^2$ in terms of $\tan(x)$ and $\tan(x/2)$:
\begin{equation}
\sec(x)^2 = 1 + \tan(x)^2,
\end{equation}
\begin{equation}
\sin(x) = 2 \, \tan(x/2) / (1 + \tan(x/2)^2 ) ,
\end{equation}
\begin{equation}
\cos(x) = ( 1 - \tan(x/2)^2 ) / ( 1 + \tan(x/2)^2 ) .
\end{equation}
The default Maxima result for the first derivatives of $\sin(x)$, $\cos(x)$,
  and $\tan(x)$ is:
\small
\begin{verbatim}
(%i1) map('diff,[sin(x),cos(x),tan(x)] )/del(x);
                                                2
(%o1)                    [cos(x), - sin(x), sec (x)]
\end{verbatim}
\normalsize
Before using \textbf{gradef} to alter the return value of \textbf{diff} for these
  three functions, let's check that Maxima agrees with the trig "identities" displayed
  above.
Variable amounts of expression simplification are needed to get what we want.
\small
\begin{verbatim}
(%i2) trigsimp( 1 + tan(x)^2 );
                                       1
(%o2)                              -------
                                       2
                                    cos (x)
(%i3) ( trigsimp( 2*tan(x/2)/(1 + tan(x/2)^2) ), trigreduce(%%) );
(%o3)                              sin(x)
(%i4) ( trigsimp( (1-tan(x/2)^2)/(1 + tan(x/2)^2) ),
          trigreduce(%%), expand(%%) );
(%o4)                              cos(x)									
\end{verbatim}
\normalsize
Recall that \textbf{trigsimp} will convert \verb|tan|, \verb|sec|, etc to
  \verb|sin| and \verb|cos| of the same argument.
Recall also that \textbf{trigreduce} will convert an expression containing
  $\cos(x/2)$  and $\sin(x/2)$ into an expression containing $\cos(x)$
  and $\sin(x)$.
Finally, remembering that $\sec(x) = 1/\cos(x)$, we see that Maxima has "passed
  the test".
  
\newpage

Now let's use \textbf{gradef} to express the derivatives in terms of
  $\tan(x)$ and $\tan(x/2)$:
\small
\begin{verbatim}
(%i5) gradef(tan(x), 1 + tan(x)^2 );
(%o5)                              tan(x)
(%i6) gradef(sin(x),(1-tan(x/2)^2 )/(1 + tan(x/2)^2 ) );
(%o6)                              sin(x)
(%i7) gradef(cos(x), -2*tan(x/2)/(1 + tan(x/2)^2 ) );
(%o7)                              cos(x)
\end{verbatim}
\normalsize
Here we check the behavior:
\small
\begin{verbatim}
(%i8) map('diff,[sin(x),cos(x),tan(x)] )/del(x);
                           2 x            x
                    1 - tan (-)     2 tan(-)
                             2            2        2
(%o8)             [-----------, - -----------, tan (x) + 1]
                       2 x            2 x
                    tan (-) + 1    tan (-) + 1
                         2              2
\end{verbatim}
\normalsize


\subsection{Critical and Inflection Points of a Curve Defined by an Explicit Function}
The critical points of a function $f(x)$ are the points $x_{j}$ such that $f'(x_{j})=0$.
We will use $f'$ to indicate the first derivative, and $f''$ to indicate the second
  derivative.
The "extrema" (ie., maxima and minima) are the values of the function at the
  critical points, provided the "slope" $f'$ actually has a different sign on
  the opposite sides of the critical point.
\begin{center}
\fbox{\parbox{4.9in}{
Maximum: $f'(a) = 0$, \qquad $f'(x)$ \quad changes from $+$ to $-$ \qquad $f''(a)<0 $\\
Minimum: $f'(a) = 0$, \qquad $f'(x)$ \quad changes from $-$ to $+$ \qquad $f''(a)>0 $
  } }
\end{center}

\smallskip
If $f(x)$ is a function with a continuous second derivative, and if, as $x$ increases
  through the value $a$, $f\,''(x)$  changes sign, then the plot of $f(x)$ has an
  inflection point at $x=a$ and $f\,''(a) = 0$.
The inflection point requirement that $f\,''(x)$ changes sign at $x = a$ is equivalent
  to $f\,'''(a) \ne 0 $.
We consider some simple examples taken from Sect. 126 of Analytic Geometry and Calculus,
  by Lloyd L. Smail, Appleton-Century-Crofts, N.Y., 1953 (a reference which "dates" the writer
    of these notes!). 
  
\subsubsection{Example 1: A Polynomial}
To find the maxima and minima of the function $f(x)=2\,x^{3} - 3\,x^{2} - 12 \, x + 13$,
  we use an "expression" (called "g") rather than a Maxima function.
We use \verb|gp| (g prime) for the first derivative, and \verb|gpp| (g double prime)
  as notation for the second derivative.
Since we will want to make some simple plots, we use the package \textbf{qdraw}, available
  on the author's webpage, and discussed in detail in chapter five of these notes.  
\small
\begin{verbatim}
(%i1) load(draw)$
(%i2) load(qdraw);
               qdraw(...), qdensity(...), syntax: type qdraw(); 
(%o2)                         c:/work2/qdraw.mac
(%i3) g : 2*x^3 - 3*x^2 - 12*x + 13$
(%i4) gf : factor(g);
                                        2
(%o4)                       (x - 1) (2 x  - x - 13)
\end{verbatim}
\newpage
\begin{verbatim}
(%i5) gp : diff(g,x);
                                   2
(%o5)                           6 x  - 6 x - 12
(%i6) gpf : factor(gp);
(%o6)                          6 (x - 2) (x + 1)
(%i7) gpp : diff(gp,x);
(%o7)                              12 x - 6
(%i8) qdraw( ex(g,x,-4,4), key(bottom)  )$
\end{verbatim}
\normalsize
Since the coefficients of the given polynomial are integral, we have tried out
  \textbf{factor} on both the given expression and its first derivative.
We see from output \verb|%o6| that the first derivative vanishes when \verb|x = -1, 2 |.
We can confirm these critical points of the given function by using \textbf{qdraw} with the
  quick plotting argument \textbf{ex(..)}, which
  gives us the plot:
  
\smallskip
%%\begin{figure}[!ht] 
\begin{figure} [h]
   \centerline{\includegraphics[scale=1]{ch6p1.eps} }
	\caption{ Plot of g over the range $[-4,4]$}
\end{figure}               
% eps file code
% (%i11) qdraw( line(-4,0,4,0,lw(2)),
%                line(0,-115,0,45,lw(2)), lw(7),
%               ex(g,x,-4,4) ,pic(eps,"ch6p1",font("Times-Roman",18)),
%               key(bottom) ); 
%
 \smallskip
We can use the cursor on the plot to find that \verb|g| takes on the value of roughly
\verb|-7| when \verb|x = 2| and the value of about \verb|20| when \verb|x = -1|.
%\newpage
We can use \textbf{solve} to find the roots of the equation \verb|gp = 0|, and
  then evaluate the given expression \verb|g|, as well as the second derivative \verb|gpp|
  at the critical points found:
\small
\begin{verbatim}
(%i9) solve(gp=0);
(%o9)                         [x = 2, x = - 1]
(%i10) [g,gpp],x=-1;
(%o10)                            [20, - 18]
(%i11) [g,gpp],x=2;
(%o11)                             [- 7, 18]
\end{verbatim}
\normalsize
%\newpage
We next look for the inflection points, the points where the curvature changes sign, which
  is equivalent to the points where the second derivative vanishes.
\small
\begin{verbatim}
(%i12) solve(gpp=0);
                                         1
(%o12)                              [x = -]
                                         2
(%i13) g,x=0.5;
(%o13)                                6.5										 
\end{verbatim}
\normalsize
We have found one inflection point, that is a point on a smooth curve where the
  curve changes from concave downward to concave upward, or visa versa (in this case
  the former).\\
We next plot \verb|g, gp, gpp| together.
\small
\begin{verbatim}
(%i14) qdraw2( ex([g,gp,gpp],x,-3,3), key(bottom),
                pts([[-1,20]],ps(2),pc(magenta),pk("MAX") ),
             pts([[2,-7]],ps(2),pc(green),pk("MIN") ),
            pts([[0.5, 6.5]],ps(2),pk("INFLECTION POINT") ) )$
\end{verbatim}
\normalsize
% eps code 
% (%i13) qdraw( line( -3,0,4,0,lw(2) ),
%              line( 0,-42,0,60,lw(2) ),  
%             ex([g,gp,gpp],x,-3,4) ,key(bottom),
%             pts([[-1,20]],ps(2),pc(magenta),pk("MAX") ),
%             pts([[2,-7]],ps(2),pc(green),pk("MIN") ) ,
%              pts([[0.5, 6.5]],ps(2),pk("INFLECTION POINT") ),  
%             pic(eps,"ch6p2",font("Times-Roman",18) ) );
%
with the resulting plot:
\smallskip
%%\begin{figure}[!ht] 
\begin{figure} [h]
   \centerline{\includegraphics[scale=1.3]{ch6p2.eps} }
	\caption{ Plot of 1: g, 2: gp, and 3: gpp }
\end{figure}               

\smallskip
The curve \verb|g| is concave downward until \verb|x = 0.5| and then is concave upward.
Note that each successive differentiation tends to flatten out the kinks in the previous
  expression (function). In other words, \verb|gp| is "flatter" than \verb|g|, and
  \verb|gpp| is "flatter" than \verb|gp|.
This behavior under \textbf{diff} is simply because each successive differentiation
  reduces by one the degree of the polynomial.
%\newpage
\subsubsection{Automating Derivative Plots with plotderiv(..)}
In a separate text file "vcalc.mac" (available on the author's webpage)
  is a small homemade Maxima function called \verb|plotderiv(...)| which
  will plot a given expression together with as many derivatives as you
  want, using \textbf{qdraw} from Ch.5.
  

\small
\begin{verbatim}
/* vcalc.mac */
plotderiv_syntax() :=
  disp("plotderiv(expr,x,x1,x2,y1,y2,numderiv)  constructs a list of
       the submitted expression expr and its first numderiv derivatives
       with respect to the independent variable, and then passes
       this list to qdraw(..). You need to have used load(draw)
       and load(qdraw) before using this function ")$

/*  version 1 commented out
plotderiv(expr,x,x1,x2,y1,y2,numderiv) :=
   block([plist],
     plist : [],
     for i thru numderiv do
        plist : cons(diff(expr,x,i), plist),
     plist : reverse(plist),
     plist : cons(expr, plist),
     display(plist),
     apply( 'qdraw, [ ex( plist,x,x1,x2 ),yr(y1,y2), key(bottom) ]  )
   )$   
*/
/* version 2  slightly more efficient  */ 
plotderiv(expr,x,x1,x2,y1,y2,numderiv) :=
   block([plist,aa],
     plist : [],
     aa[0] : expr,
     for i thru numderiv do (
        aa[i] : diff(aa[i-1],x),
        plist : cons(aa[i], plist)
         ),
     plist : reverse(plist),
     plist : cons(expr, plist),
     display(plist),
     apply( 'qdraw, [ ex( plist,x,x1,x2 ),yr(y1,y2), key(bottom) ]  )
   )$
\end{verbatim}
\normalsize
We have provided two versions of this function, the first version commented out.
If the expression to be plotted is a very complicated function and you are
  concerned with computing time, you can somewhat improve the efficiency of
  \textbf{plotderiv} by introducing a "hashed array" called \verb|aa[j]|, say,
  to be able to simply differentiate the previous derivative expression.
That is the point of version 2 which is not commented out.  

\smallskip

We have added the line \verb|display(plist)| to print a list containing the
  expression as well as all the derivatives requested.
  
  
We test \verb|plotderiv| on the expression \verb|u^3|.
\small
\begin{verbatim}
(%i15) load(vcalc)$
                vcalc.mac:   for syntax, type: vcalc_syntax(); 
 CAUTION: global variables set and used in this package:
       hhh1, hhh2, hhh3, uuu1, uuu2, uuu3, nnnsys 
(%i16) plotderiv(u^3,u,-3,3,-27,27,4)$
                                   3     2
                         plist = [u , 3 u , 6 u, 6, 0]
\end{verbatim}
\normalsize
\newpage

which produces the plot:
\begin{figure} [h]
   \centerline{\includegraphics[scale=.9]{ch6p3.eps} }
	\caption{ 1: $u^3$,\; 2: $3\,u^2$,\; 3: $6\,u$,\; 4: $6$,\; 5: $0$ }
\end{figure}      

%\newpage

% eps file code
% (%i5) qdraw( line(-3,0,3,0,lw(2) ), line(0,-27,0,27,lw(2)),
%             lw(5),key(bottom),
%             ex([x^3,3*x^2,6*x,6,0],x,-3,3),
%             pic(eps,"ch6p3",font("Times-Roman",18) ) );
%


\subsubsection{Example 2: Another Polynomial}
To find the critical points, inflection points, and extrema of the expression
  $ 3\,x^{4} - 4\, x^{3}$.
We first look at the expression and its first two derivatives using
  \verb|plotderiv|.
\small
\begin{verbatim}
(%i17) plotderiv(3*x^4 - 4*x^3,x,-1,3,-5,5,2)$
                          4      3      3       2      2
              plist = [3 x  - 4 x , 12 x  - 12 x , 36 x  - 24 x]
\end{verbatim}
\normalsize
which produces the plot:
\begin{figure} [h]
   \centerline{\includegraphics[scale=1]{ch6p4.eps} }
	\caption{ 1: $f = 3\,x^4 - 4\, x^3$,\; 2: $f\,'$,\; 3: $f\,''$  }
\end{figure}      

\smallskip
We have inflection points at $x = 0, \; 2/3$ since the second derivative is zero at
  both points and changes sign as we follow the curve through those points.
The curve changes from concave up to concave down passing through $x = 0$
  and changes from concave down to concave up passing through $x = 2/3$.
The following provides confirmation of our inferences from the plot:
\small
\begin{verbatim}
(%i18) g:3*x^4 - 4*x^3$
(%i19) g1: diff(g,x)$
(%i20) g2 : diff(g,x,2)$
(%i21) g3 : diff(g,x,3)$
(%i22) x1 : solve(g1=0);
(%o22)                           [x = 0, x = 1]
(%i23) gcrit : makelist(subst(x1[i],g),i,1,2);
(%o23)                              [0, - 1]
(%i24) x2 : solve(g2=0);
                                     2
(%o24)                           [x = -, x = 0]
                                     3
(%i25) ginflec : makelist( subst(x2[i],g ),i,1,2 );
                                      16
(%o25)                              [- --, 0]
                                      27
(%i26) g3inflec : makelist( subst(x2[i],g3),i,1,2 );
(%o26)                             [24, - 24]
\end{verbatim}
\normalsize
We have critical points where the first derivative vanishes at $x = 0, \; 1$.
Since the first derivative does not change sign as the curve passes through
  the point $x = 0$, that point is neither a maximum nor a mimimum point.
The point $x = 1$ is a minimum point since the first derivative changes sign
  from negative to positive as the curve passes through that point.


\subsubsection{Example 3: $x^{2/3}$, \;  Singular Derivative, \; Use of limit(..)}
Searching for points where a function has a minimum or maximum by looking at
  points where the first derivative is zero is useful as long as the first
  derivative is well behaved.
Here is an example in which the given function of $x$ has a minimum at $x=0$,
  but the first derivative is singular at that point.
Using the expression \verb|g = x^(2/3)| with \verb|plotderiv(..)|:
\small
\begin{verbatim}
(%i27) plotderiv(x^(2/3),x,-2,2,-2,2,1);									
\end{verbatim}
\normalsize

%\begin{figure}[!ht] 
\begin{figure} [h]
   \centerline{\includegraphics[scale=.7]{ch6p5.eps} }
	\caption{ Plot of $1:\;f(x) = x^{2/3},\;2:\;f\,'$ }
\end{figure}               

\newpage
% eps file code
% (%i9) qdraw(key(bottom),xr(-2,2),yr(-2,2),
%         line(-2,0,2,0,lw(2)),line(0,-2,0,2,lw(2)),
%         lw(5),ex([x^(2/3),2/(3*x^(1/3))],x,-2,2),
%           pic(eps,"ch6p5",font("Times-Roman",18) ) );
%



We see that the first derivative is singular at $x=0$ and approaches either
  positive or negative infinity, depending on your direction of approaching
  the point $x=0$.
We can see from the plot of $f(x) = x^{2/3}$ that the tangent line (the "slope") of the curve
  is negative for $x<0$ and becomes increasingly negative (approaching minus infinity)
  as we approach the point $x=0$ from the negative side.
We also see from the plot of $f(x)$ that the tangent line ("slope") is positive for
  positive values of $x$ and as we pass $x=0$, moving from smaller $x$ to larger $x$
  that the sign of the first derivative $f\,'(x)$ changes from $-$ to $+$, which
  is a signal of a local minimum of a function.

\smallskip
We can practice using the Maxima function \textbf{limit} with this example:  
\small
\begin{verbatim}
(%i28) gp : diff(x^(2/3),x);
                                      2
(%o28)                              ------
                                       1/3
                                    3 x
(%i29) limit(gp,x,0,plus);
(%o29)                                inf
(%i30) limit(gp,x,0,minus);
(%o30)                               minf
(%i31) limit(gp,x,0);
(%o31)                                und
\end{verbatim}
\normalsize
The most frequent use of the Maxima \textbf{limit} function has the syntax  
\small
\begin{quote}
Function: \textbf{limit}\verb|(expr, x, val, dir)|\\
Function: \textbf{limit}\verb|(expr, x, val) |\\

The first form computes the limit of \verb|expr| as the real variable \verb|x| approaches
 the value \verb|val| from the direction \verb|dir|.
\verb|dir| may have the value \verb|plus| for a limit from above,
  \verb|minus| for a limit from below.\\
In the second form, \verb|dir| is omitted,  implying
   a "two-sided limit" is to be computed. \\
\verb|limit| uses the following special symbols: \verb|inf| (positive infinity)
   and \verb|minf| (negative infinity).
On output it may also use \verb|und| (undefined), \verb|ind| (indefinite but bounded)
  and \verb|infinity| (complex infinity). \\
\end{quote}
\normalsize


Returning to our example, if we ignore the point $x=0$, the slope is always
 decreasing, so the second derivative is always negative.

\subsection{Tangent and Normal of a Point of a Curve Defined by an Explicit Function} \label{explicit}

The equation of a line with the form $y = m\, (x - x_0) + y_0$, where $m$, $x_0$, and $y_0$ 
  are constants, is identically satisfied if we consider the point $(x,y) = (x_0,y_0)$.
Hence this line passes through the point $(x_0, y_0)$.
The "slope" of this line (the first derivative) is the constant $m$.

\smallskip

Now we also assume that the point $(x_0, y_0)$ is a point of a curve given by some
  explicit function of $x$, say $y = f(x)$; hence $y_0 = f(x_0)$.
The first derivative of this function, evaluated at the point $x_0$ is the local
  "slope", which defines the \textbf{local tangent line} $y = m \,(x - x_0) + y_0$ if
  we use for $m$ the value of $f\,'(x_0)$, where the notation means first take the derivative for arbitrary
  $x$ and then evaluate the derivative at $x = x_0$.
\smallskip
Let the two dimensional vector \verb|tvec| be a vector parallel to the tangent line (at the
  point of interest) with components \verb|(tx,ty)|, such that the vector makes an angle
  $ \theta$ with the positive $x$ axis.
Then $\tan(\theta) = t_{y}/t_{x} = m = d\,y/d\,x $ is the slope of the tangent (line) at this point.
Let the two dimensional vector \verb|nvec| with components \verb|(nx,ny)| be parallel to the
line perpendicular to the tangent at the given point.
This \textbf{normal} line will be perpendicular to the tangent line if the vectors \verb|nvec| and
  \verb|tvec| are "orthogonal".
\newpage
  
In Maxima, we can represent vectors by lists:
\small
\begin{verbatim}
(%i1) tvec : [tx,ty];
(%o1)                              [tx, ty]
(%i2) nvec : [nx, ny];
(%o2)                              [nx, ny]
\end{verbatim}
\normalsize
Two vectors are "orthogonal" if the "dot product" is zero.
A simple example will illustrate this general property.
Consider the pair of unit vectors \verb|ivec = [1,0]| parallel to the positive
 $x$ axis and \verb|jvec = [0,1]| parallel to the positive $y$ axis.
\small
\begin{verbatim}
(%i3) ivec : [1,0]$
(%i4) jvec : [0,1]$
(%i5) ivec . jvec;
(%o5)                                  0
\end{verbatim}
\normalsize

Since Maxima allows us to use a period to find the dot product (inner product) of two
  lists (two vectors), we will ensure that the normal vector is "at right angles" to
  the tangent vector by requiring \verb|nvec . tvec = 0|
\small
\begin{verbatim}
(%i6) eqn : nvec . tvec = 0;
(%o6)                          ny ty + nx tx = 0
(%i7) eqn : ( eqn/(nx*ty), expand(%%) );
                                  tx   ny
(%o7)                             -- + -- = 0
                                  ty   nx
\end{verbatim}
\normalsize
We represent the normal  (the line perpendicular to the tangent line) 
  at the point $(x_{0},y_{0})$ as
  $y  = m_{n}\,(x - x_{0}) +  y_{0}$, where $m_{n}$ is the slope of the
  normal line: \verb|mn = ny/nx = - (1/(ty/tx)) = -1/m |

\smallskip
In words, \textbf{the slope of the normal line is the negative reciprocal of the slope of
  the tangent line}.
\smallskip
 Thus the equation of the \textbf{local normal} to the point $(x_0,y_0)$ can be written
   as $y =  -(x - x_0)/m + y_0$, where $m$ is the slope of the local tangent.

%\newpage
\subsubsection{Example 1: $x^2$}
As an easy first example, let's use the function $f(x) = x^2$ and construct the
  tangent line and normal line to this plane curve at the point \verb|x=1, y=f=1|.
(We have discussed this same example in Ch. 5).
The first derivative is $2\,x$, and its value at $x=1$ is $2$.
 Thus the equation of the local tangent to the curve at \verb|(x,y) = (1,1)| is
   $y = 2\,x - 1$.
The equation of the local normal at the same point is $y =  -x/2 + 3/2$.

\smallskip
We will plot the curve $x^2$, the local tangent, and the local normal together.
Special care is needed to get the horizontal "canvas" width about \verb|1.4| times
  the vertical height to achieve a geometrically correct plot; otherwise the "normal" line
  would not cross the tangent line at right angles.
\small
\begin{verbatim}
(%i8) qdraw( ex([x^2,2*x-1,-x/2+3/2],x,-1.4,2.8), yr(-1,2) ,
               pts( [ [1,1]],ps(2) ) )$
\end{verbatim}
\normalsize
% eps file code
% (%i16) qdraw( lw(5),line(-1.4,0,2.8,0,lw(2) ),
%                line(0,-1,0,2,lw(2) ),
% ex([x^2,2*x-1,-x/2+3/2],x,-1.4,2.8), yr(-1,2) ,
%               pts( [ [1,1]],ps(2) ),
%             pic(eps,"ch6p6",font("Times-Roman",18) ) ); 
\newpage

The plot looks like:

\smallskip
%%\begin{figure}[!ht] 
\begin{figure} [h]
   \centerline{\includegraphics[scale=1]{ch6p6.eps} }
	\caption{ Tangent and Normal to $x^2$ at point (1,1)  }
\end{figure}                 

\smallskip

We can see directly from the plot that the slope of the tangent line has the value $2$.
Remember that we can think of "slope" as being the number derived from dividing the "rise"
  by the "run".
Thus if we choose a "run" to be the $x$ interval \verb|(0,1)|, from the plot this
  corresponds to the "rise" of $1 - (-1) = 2$.
Hence "rise/run" is $2$.

\smallskip


\subsubsection{Example 2: $\ln(x)$}
Our next example repeats the analysis of Example 1 for the function $\ln(x)$, 
  using the point $(x = 2, y = \ln(2) )$.
Maxima's natural log function is written \textbf{log(...)}.
Maxima does not have a core log to the base 10 function, although you can 
  "roll your own" with a definition like \verb|log10(x) := log(x)/log(10)|,
  which is equivalent to \verb|log(x)/2.303|.
Thus in Maxima, \verb|log(10) = 2.303, log(2) = 0.693|, etc.
We are constructing the local tangent and normal at the point
  $(x = 2, y = 0.693)$.
The derivative of the natural log is
\small
\begin{verbatim}
(%i9) diff(log(x),x);
                                       1
(%o9)                                  -
                                       x
\end{verbatim}
\normalsize
so the "slope" of our curve (and the tangent) at the chosen point is $1/2$, 
  and the slope of the local normal is $-2$.
The equation of the tangent is then $(y - \ln(2) ) = (1/2) \, (x - 2)$, or
  $y = x/2 - 0.307$.
The equation of the normal is $(y - \ln(2) ) = -2 \,(x - 2)$, or
  $y = -2\,x + 4.693$.
% eps file code
% (%i8) qdraw(key(bottom), yr(-1,1.786),lw(5),
%         line(0.1,0,4,0,lw(2) ),line(0,-1,0,1.786,lw(2)),
%         ex([log(x), x/2 - 0.307, -2*x + 4.693],x,0.1,4),
%          pts( [ [2,0.693]],ps(2) ),
%         pic(eps,"ch6p7",font("Times-Roman",18) ) );
%
In order to get a decent plot, we need to stay away from the singular point $x=0$, 
  so we start the plot with $x$ a small positive number. 
We choose to use the x-axis range $(0.1, 4)$, then $\Delta x = 3.9$.
Since we want $\Delta x \approx 1.4 \, \Delta y$, we need $\Delta y = 2.786$,
  which is satisfied if we choose the y-axis range to be $(-1, 1.786)$.
\small
\begin{verbatim}
(%i10) qdraw(key(bottom), yr(-1,1.786),
          ex([log(x), x/2 - 0.307, -2*x + 4.693],x,0.1,4),
          pts( [ [2,0.693]],ps(2) )  );
\end{verbatim}
\normalsize

\newpage
which produces the plot:
\begin{figure} [h]
   \centerline{\includegraphics[scale=1.3]{ch6p7.eps} }
	\caption{ Tangent and Normal to $\ln(x)$ at point $(2,\ln(2))$  }
\end{figure}      

\smallskip
\subsection{Maxima and Minima of a Function of Two Variables}
You can find proofs of the following criteria in calculus texts.

\small
\begin{quote}
If $f(x,y)$ and its first derivatives are continuous in a region including
 the point $(a,b)$, a necessary condition that $f(a,b)$ shall be an extreme (maximum
 or minimum) value of the function $f(x,y)$ is that (I):
\begin{equation}
 \frac{\partial f(a,b)}{\partial x} = 0, \quad  \frac{\partial f(a,b)}{\partial y} = 0
\end{equation}
in which the notation means first take the partial derivatives for arbitrary $x$ and $y$
  and then evaluate the resulting derivatives at the point $(x,y) = (a,b)$.
\smallskip
Examples show that these two conditions (I) do not guarantee that the function actually
  takes on an extreme value at the point $(a,b)$, although in many practical problems
  the existence and nature of an extreme value is often evident from the problem
  itself and no extra test is needed; all that is needed is the location of the
  extreme value.
\smallskip
If condition (I) is satisfied and if, in addition, at the point $(a,b)$ we have (II):
\begin{equation}
 \Delta = \left(\frac{\partial^2\,f}{\partial \, x^2} \right)\, \left(\frac{\partial^2\,f}{\partial \, y^2} \right) -
            \left(\frac{\partial^2\,f}{\partial x \,\partial y} \right)^2  > 0
\end{equation}  
then $f(x,y)$ will have a maximum value or a minimum value given by $f(a,b)$ according as 
  $\partial^2\,f/\partial\,x^2$ ( or $\partial^2\,f/\partial\,y^2$ ) is negative or positive
  for $x=a, y=b$.
\newpage
If condition (I) holds and $\Delta < 0$ then $f(a,b)$ is neither a maximum nor a minimum;
if $\Delta = 0$ the test fails to give any information.  
\end{quote}
\normalsize
\subsubsection{Example 1: Minimize the Area of a Rectangular Box of Fixed Volume}
Let the sides of a "rectangular box" (rectangular prism, or cuboid, if you wish) of
  fixed volume $v$ be $x$, $y$, and $z$.
We wish to determine the shape of this box (for fixed $v$) which minimizes the surface
  area $s = 2\,(x\,y + y\,z + z\,x)$.
We can use the fixed volume relation to express $z$ as a function of $x$ and $y$
  and then achieve an expression for the surface area which only depends of the 
   two variables $x$ and $y$.
We then know that a necessary condition that the surface area be an extremum
  is that the two equations of condition I above be satisfied.
We need to expose the hidden dependence of $z$ on $x$ and $y$ via the volume constraint
  before we can correctly derive the two equations of condition I.  
  
You probably already know the answer to this shape problem is a cube, in which all sides are equal in
  length, and a side is the same as the cube root of the volume.  
We require three equations to be satisfied, the first being the constant volume
  expressed as $v = x\,y\,z$,  and the other two being the equations of condition I above.  
There are multiple paths to such a solution in Maxima.
Perhaps the simplest is to use \textbf{solve} to generate a solution of the
  three equations for the variables $(x,y,z)$, although the solutions returned
  will include non-physical solutions and we must select the physically realizable
  solution.
\small
\begin{verbatim}
(%i1) eq1 : v = x*y*z;
(%o1)                              v = x y z
(%i2) solz : solve(eq1,z);
                                         v
(%o2)                              [z = ---]
                                        x y
(%i3) s : ( subst(solz,2*(x*y  + y*z + z*x) ), expand(%%) );
                                       2 v   2 v
(%o3)                          2 x y + --- + ---
                                        y     x
(%i4) eq2 : diff(s,x)=0;
                                       2 v
(%o4)                            2 y - --- = 0
                                        2
                                       x
(%i5) eq3 : diff(s,y) = 0;
                                       2 v
(%o5)                            2 x - --- = 0
                                        2
                                       y
(%i6) solxyz: solve([eq1,eq2,eq3],[x,y,z]);
\end{verbatim}
\newpage
\begin{verbatim}
             1/3       1/3       1/3
(%o6) [[x = v   , y = v   , z = v   ], 
            1/3                              1/3
         2 v               (sqrt(3) %i + 1) v
[x = --------------, y = - ---------------------, 
     sqrt(3) %i - 1                  2
                        1/3                 1/3
      (sqrt(3) %i + 1) v                 2 v
z = - ---------------------], [x = - --------------, 
                2                    sqrt(3) %i + 1
                      1/3                        1/3
    (sqrt(3) %i - 1) v         (sqrt(3) %i - 1) v
y = ---------------------, z = ---------------------]]
              2                          2
\end{verbatim}
\normalsize
Since the physical solutions must be real, the first sublist is the physical
  solution which implies the cubical shape.
%\newpage
  
\smallskip

An alternative solution path is to solve \verb|eq2| for $y$ as a function of $x$ and $v$
  and use this to eliminate $y$ from \verb|eqn3|
\small
\begin{verbatim}
(%i7) soly : solve(eq2,y);
                                        v
(%o7)                              [y = --]
                                         2
                                        x
(%i8) eq3x : ( subst(soly, eq3),factor(%%) );
                                      3
                                2 x (x  - v)
(%o8)                         - ------------ = 0
                                     v
\end{verbatim}
\normalsize
The factored form \verb|%o7| shows that, since $x \neq 0$, the solution
  must have $x = v^{1/3}$.
We can then use this solution for $x$ to generate the solution for $y$
  and then for $z$.
\small
\begin{verbatim}
(%i9) solx : x = v^(1/3);
                                        1/3
(%o9)                              x = v
(%i10) soly : subst(solx, soly);
                                        1/3
(%o10)                             [y = v   ]
(%i11) subst( [solx,soly[1] ], solz );
                                        1/3
(%o11)                            [z = v   ]
\end{verbatim}
\normalsize
We find that $\Delta > 0$ and that  the second derivative of $s(x,y)$ with
  respect to $x$ is positive, as needed for the cubical solution to
  define a minimum surface solution.
\small
\begin{verbatim}
(%i12) delta : ( diff(s,x,2)*diff(s,y,2) - diff(s,x,1,y,1), expand(%%) );
                                       2
                                   16 v
(%o12)                             ----- - 2
                                    3  3
                                   x  y
(%i13) delta : subst([x^3=v,y^3=v], delta );
(%o13)                                14
\end{verbatim}
\newpage
\begin{verbatim}
(%i14) dsdx2 : diff(s,x,2);
                                      4 v
(%o14)                                ---
                                       3
                                      x
(%i15) dsdy2 : diff(s,y,2);
                                      4 v
(%o15)                                ---
                                       3
                                      y
\end{verbatim}
\normalsize
%\newpage

\subsubsection{Example 2: Maximize the Cross Sectional Area of a Trough}
A long rectangular sheet of aluminum of width $L$ is to be formed into a flat bottom
  trough by bending both a left and right hand length $x$ up by an angle $\theta$
  with the horizontal.
A cross section view is then:
\begin{figure} [h]
   \centerline{\includegraphics[scale=1]{ch6p18.eps} }
	\caption{ Flat Bottomed Trough }
\end{figure}        
% eps file code : need to crop this 
%(%i29) qdraw(xr(-2.8,2.8),yr(-2,2),
%            line(-1.5,0,1.5,0,lw(10) ),
%             line(-1.5,0,-2,0.866,lw(10)),
%             line(1.5,0,2,0.866,lw(10) ),
%             ellipse(1.5,0,0.5,0.5,0,60 ),
%             ellipse(-1.5,0,0.5,0.5,120,60 ),
%             label(["{/Symbol=36 q}",2.1,0.3],
%              ["{/Symbol=36 q}",-2.2,0.3],
%               ["{/=36 x}",1.5,0.5],["{/=36 x}",-1.6,0.5],
%                 ["{/=36 L - 2 x}",-0.5,-0.3]   ),
%             vector([-0.8,-0.3],[-0.7,0],hl(0.2) ),
%              vector([0.8,-0.3],[0.7,0],hl(0.2) ),
%               line(-1.5,0,-1.5,-0.7,lw(2) ),
%               line(1.5,0,1.5,-0.7,lw(2) ),
%                line(-2.5,0,-1.5,0,lw(2) ),
%                line( 1.5,0,2.5,0,lw(2) ),
%             pic(eps,"ch6p18"), cut(all) );
%

\smallskip
The lower base is $l= L - 2\,x$, the upper base is $u =l + 2\,x\,\cos(\theta)$.
The altitude is $h= x\,\sin(\theta)$.
The area of a trapezoid whose parallel sides are $l$ and $u$ and whose altitude is $h$
  is $(h/2)\,(l + u)$ (ie., the height times the average width: Exercise: check this
  by calculating the area of a rectange $h\,u$, where $u$ is the larger base, and
  then subtracting the area of the two similar right triangles on the left
  and right hand sides.)
Hence the area of the cross section (called \verb|s| in our Maxima work)
 is $A = L\,x\,\sin(\theta) - 2\,x^2\,\sin(\theta) + x^2\,\sin(\theta)\,\cos(\theta)$
 
\smallskip
Using condition I equations, we proceed to find the extremum solutions for $x$ and $\theta$
  (which is called \verb|th| in our Maxima work).
Our method is merely one possible path to a solution.  
\small
\begin{verbatim}
(%i1) s : L*x*sin(th) - 2*x^2*sin(th) + x^2*sin(th)*cos(th);
                                               2              2
(%o1)           sin(th) x L + cos(th) sin(th) x  - 2 sin(th) x
(%i2) dsdx : ( diff(s,x), factor(%%) );
(%o2)                   sin(th) (L + 2 cos(th) x - 4 x)
\end{verbatim}
\normalsize
We see that one way we can get \verb|dsdx| to be zero is to set $\sin(\theta) = 0$,
  however this would mean the angle of bend was zero degrees, which is not the solution
  of interest.
Hence our first equation comes from setting the second factor of \verb|dsdx| equal to zero.
\small
\begin{verbatim}
(%i3) eq1 : dsdx/sin(th) = 0;
(%o3)                      L + 2 cos(th) x - 4 x = 0
(%i4) solx : solve(eq1,x);
                                          L
(%o4)                        [x = - -------------]
                                    2 cos(th) - 4
(%i5) dsdth : ( diff(s,th), factor(%%) );
                               2            2
(%o5)        x (cos(th) L - sin (th) x + cos (th) x - 2 cos(th) x)									
\end{verbatim}
\normalsize
We see that we can get \verb|dsdth| to be zero if we set $x = 0$, but this
  is not the physical situation we are examining.
We assume (of course) that $x \neq 0$, and arrive at our second equation of
  condition I by setting the second factor of \verb|dsdth| equal to zero.
\small
\begin{verbatim}
(%i6) eq2 : dsdth/x = 0;
                            2            2
(%o6)        cos(th) L - sin (th) x + cos (th) x - 2 cos(th) x = 0
(%i7) eq2 : ( subst(solx,eq2), ratsimp(%%) );
                        2          2
                    (sin (th) + cos (th) - 2 cos(th)) L
(%o7)               ----------------------------------- = 0
                               2 cos(th) - 4
\end{verbatim}
\normalsize
The only way we can satisfy \verb|eq2| is to set the numerator of the left hand
  side equal to zero, and divide out the factor \verb|L| which is positive:
\small
\begin{verbatim}
(%i8) eq2 : num(lhs(eq2) )/L = 0;
                         2          2
(%o8)                 sin (th) + cos (th) - 2 cos(th) = 0
(%i9) eq2 : trigsimp(eq2);
(%o9)                          1 - 2 cos(th) = 0
(%i10) solcos : solve( eq2, cos(th) );
                                            1
(%o10)                           [cos(th) = -]
                                            2
(%i11) solx : subst(solcos, solx);
                                         L
(%o11)                              [x = -]
                                         3
(%i12) solth : solve(solcos,th);
`solve' is using arc-trig functions to get a solution.
Some solutions will be lost.
                                        %pi
(%o12)                            [th = ---]
                                         3
\end{verbatim}
\normalsize
  
\subsection{Tangent and Normal of a Point of a Curve Defined by an Implicit Function}
\textbf{diff} does "partial differentiation". 
In the following, Maxima assumes that $y$ is independent of x:
\small
\begin{verbatim}
(%i1) diff(x^2 + y^2,x);
(%o1)                                 2 x
\end{verbatim}
\normalsize
\newpage
We can indicate explicitly that $y$ depends on $x$ for purposes of taking
  this derivative by replacing $y$ by $y(x)$.
\small
\begin{verbatim}
(%i2) diff(x^2 + y(x)^2,x);
                                   d
(%o2)                      2 y(x) (-- (y(x))) + 2 x
                                   dx
\end{verbatim}
\normalsize
Instead of using the functional notation to indicate dependency, we can
  use the \textbf{depends} function before taking the derivative.
\small
\begin{verbatim}
(%i3) depends(y,x);
(%o3)                               [y(x)]
(%i4) g : diff(x^2 + y^2, x);
                                     dy
(%o4)                            2 y -- + 2 x
                                     dx
(%i5) grind(g)$
2*y*'diff(y,x,1)+2*x$
(%i6) gs1 : subst('diff(y,x) = x^3, g );
                                    3
(%o6)                            2 x  y + 2 x	
\end{verbatim}
\normalsize
In \verb|%i9| we defined \verb|g| as the derivative with respect to \verb|x|
  of the expression \verb|x^2 + y^2| after telling Maxima that the symbol \verb|y|
  is to be considered dependent on the value of \verb|x|.
Since Maxima, as yet, has no specific information about the nature of the
  dependence of \verb|y| on \verb|x|, the output is expressed in terms of the
  "noun form" of \textbf{diff}, which Maxima's pretty print shows as $\frac{d\,y}{d\,x}$.
To see the "internal" form, we again use the \textbf{grind} function, which shows
  the explicit noun form \verb|'diff(y,x,1)|.
This is useful to know if we want to later replace that unknown derivative with
  a known result, as we do in input \verb|%i11|.

\smallskip
However, Maxima doesn't do anything creative with the derivative substitution
  done in \verb|%i11|, like working backwards to what the explicit function of \verb|x|
  the symbol \verb|y| might stand for (different answers differ by a constant).
The output \verb|%o11| still contains the symbol \verb|y|.

\smallskip


Two ways to later implement our knowledge of \verb|y(x)| are shown in steps
\verb|%i12| and \verb|%i13|, which use the \textbf{ev} function.
(In Chapter 2 we discussed using \textbf{ev} for making substitutions, although
  the use of \textbf{subst} is more generally recommended for that job.)
\small
\begin{verbatim}
(%i7) ev(g,y=x^4/4,diff);
                                    7
                                   x
(%o7)                             -- + 2 x
                                   2
(%i8) ev(g,y=x^4/4,nouns);
                                    7
                                   x
(%o8)                             -- + 2 x
                                   2
(%i9) y;
(%o9)                                 y								   
(%i10) g;
                                     dy
(%o10)                           2 y -- + 2 x
                                     dx
\end{verbatim}
\normalsize
We see that Maxima does not bind the symbol \verb|y| to anything when we
  call \textbf{ev} with an equation like \verb|y = x^4/4|,
  and that the binding of \verb|g| has not changed.

\smallskip
  
A list which reminds you of all dependencies in force is \textbf{dependencies}.
The Maxima function \textbf{diff} is the only core function which makes use of 
  the \textbf{dependencies} list.
The functions \textbf{integrate} and \textbf{laplace} do not use the \textbf{depends}
  assignments; one must indicate the dependence explicitly by using functional notation.  

\smallskip
  
In the following, we first ask for the contents of the \textbf{dependencies} list
  and then ask Maxima to remove the above dependency of $y$ on $x$, using
  \textbf{remove}, then check the list contents again, and carry out the previous
  differentiation with Maxima no longer assuming that $y$ depends on $x$.
\small
\begin{verbatim}
(%i11) dependencies;
(%o11)                              [y(x)]
(%i12) remove(y, dependency);
(%o12)                               done
(%i13) dependencies;
(%o13)                                []
(%i14) diff(x^2 + y^2,x);
(%o14)                                2 x
\end{verbatim}
\normalsize
On can also remove the properties associated with the symbol $y$ by using
  \textbf{kill}\verb|(y)|, although this is more drastic than using \textbf{remove}.
\small
\begin{verbatim}
(%i15) depends(y,x);
(%o15)                               [y(x)]
(%i16) dependencies;
(%o16)                               [y(x)]
(%i17) kill(y);
(%o17)                               done
(%i18) dependencies;
(%o18)                                []
(%i19) diff(x^2+y^2,x);
(%o19)                                2 x
\end{verbatim}
\normalsize
There are many varieties of using the \textbf{kill} function.
The way we are using it here corresponds to the syntax:\\
\small
\begin{quote}
Function: \textbf{kill}\verb|(a_1, ..., a_n) |\\
Removes all bindings (value, function, array, or rule) from the
  arguments \verb|a_1, ..., a_n|.
An argument \verb|a_k| may be a symbol or a single array element.
\end{quote}
\normalsize
The list \textbf{dependencies} is one of the lists Maxima uses to hold information
  introduced during a work session. 
You can use the command \textbf{infolists} to obtain a list of the names of all
 of the information lists in Maxima.
\small
\begin{verbatim}
(%i1) infolists;
(%o2) [labels, values, functions, macros, arrays, myoptions, props, aliases, 
                   rules, gradefs, dependencies, let_rule_packages, structures]
\end{verbatim}
\normalsize
When you first start up Maxima, each of the above named lists is empty.
\small
\begin{verbatim}
(%i2) functions;
(%o2)                                 []
\end{verbatim}
\normalsize


\subsubsection{Tangent of a Point of a Curve Defined by $f(x,y) = 0$} \label{gradef}
Suppose some plane curve is defined by the equation $f(x,y) = 0$.
Every point $(x,y)$ belonging to the curve must satisfy that equation.
For such pairs of numbers, changing $x$ forces a change in $y$ so that
  the equation of the curve is still satisfied. 
When we start plotting tangent lines to plane curves, we will need to
  evaluate the change in $y$, given a change in $x$, such that the numbers
  $(x,y)$ are always points of the curve.
Let's then regard $y$ as depending on $x$ via the equation of the curve.
Given that the equation $f(x,y) = 0$ is valid, we obtain another valid
  equation by taking the derivative of both sides of the equation with
  respect to $x$.
Let's work just on the left hand side of the resulting equation for
  now:
\small
\begin{verbatim}
(%i1) depends(y,x);
(%o1)                               [y(x)]
(%i2) diff(f(x,y),x);
                                 d
(%o2)                            -- (f(x, y))
                                 dx
\end{verbatim}
\normalsize
We can make progress by assigning values to the partial derivative of $f(x,y)$ with respect to
  the first argument $x$ and also to the partial derivative of $f(x,y)$ with respect to the
  second argument $y$, using the \textbf{gradef} function.
The assigned values can be explicit mathematical expressions or symbols.
We are going to adopt the symbol \verb|dfdx| to stand for the
  partial derivative 
$$  dfdx  =  \left(\frac{\partial f(x,y)}{\partial x}  \right)_{y}\, ,  $$
where the $y$ subscript means "treat $y$ as a constant when evaluating this 
  derivative".
  
\smallskip
Likewise, we will use the symbol \verb|dfdy| to stand for the partial derivative of
  $f(x,y)$ with respect to $y$ (regarding $x$ to be independent of $y$):
$$   dfdy  =  \left(\frac{\partial f(x,y)}{\partial y}  \right)_{x}\, .  $$



\small
\begin{verbatim}
(%i3) gradef(f(x,y), dfdx, dfdy );
(%o3)                               f(x, y)
(%i4) g : diff( f(x,y), x );
                                     dy
(%o4)                           dfdy -- + dfdx
                                     dx
(%i5) grind(g)$
dfdy*'diff(y,x,1)+dfdx$
(%i6) g1 : subst('diff(y,x) = dydx, g);
(%o6)                          dfdy dydx + dfdx
\end{verbatim}
\normalsize
We have adopted the symbol \verb|dydx| for the "rate of change" of $y$ as $x$ varies,
  subject to the constraint that the numbers $(x,y)$ always satisfy the equation of
  the curve $f(x,y) = 0$, which implies that \verb|g1 = 0|.
\small
\begin{verbatim}
(%i7) solns : solve(g1=0, dydx);
                                          dfdx
(%o7)                           [dydx = - ----]
                                          dfdy
\end{verbatim}
\normalsize
We see that Maxima knows enough about the rules for differentiation to
  allow us to get to a famous calculus formula.
If $x$ and $y$ are constrained by the equation $f(x,y) = 0$, then the "slope" of
  the local tangent to the point $(x,y)$ is 
\begin{equation} \label{Eq:1}
 \frac{d\,y}{d\,x} = - \frac{\left(\frac{\partial f(x,y)}{\partial x}  \right)_{y} }{
                        \left(\frac{\partial f(x,y)}{\partial y}  \right)_{x} }.  
\end{equation}						
Of course, this formal expression has no meaning at a point where the denominator
  is zero.

\smallskip
In your calculus book you will find the derivation of the differentiation 
  rule embodied in our output \verb|%o4| above:
\begin{equation}  \label{Eq:2}
\frac{d}{d\,x} f(x,y(x)) = \left(\frac{\partial f(x,y)}{\partial x}  \right)_{y} +
        \left(\frac{\partial f(x,y)}{\partial y}  \right)_{x} \frac{d\,y(x)}{d\,x}
\end{equation}  
Remember that Maxima knows only what the code writers put in; we normally assume
  that the correct laws of mathematics are encoded as an aid to calculation.
If Maxima were not consistent with the differentiation rule Eq. (\ref{Eq:2}),
  then a bug would be present and would have to be removed.
\smallskip
  

\smallskip
The result Eq.(\ref{Eq:1}) provides a way to find the slope (which we call $m$)
 at a general point $(x,y)$.
 
This "slope" should be evaluated at the curve point $(x_0,y_0)$ of interest
  to get the tangent and normal in numerical form.  
Recall our discussion in the first part of Section(\ref{explicit}) where we
  derived the relation between the slopes of the tangent and normal.
If the numerical value of the slope is denoted $m_0$, then the tangent (line) is the
  equation $y  = m_0 \, (x - x_{0}) + y_{0}$, and the normal (line) is the
  equation $y  = -(x - x_{0})/m_0 + y_{0}$

\subsubsection*{A Function which Solves for $d\,y/d\,x$ given that $f(x,y) = 0$}
Given an equation relating $x$ and $y$, we assume we can rewrite that equation in
  the form $f(x,y) = 0$, and then define the following function:
\small
\begin{verbatim}
(%i1) dydx(expr,x,y) := -diff(expr,x)/diff(expr,y);
                                          - diff(expr, x)
(%o1)                dydx(expr, x, y) := ---------------
                                           diff(expr, y)
\end{verbatim}
\normalsize
As a first example, consider finding $d\,y/d\,x$ given that $x^3 + y^3 = 1$.
\small
\begin{verbatim}
(%i2) dydx( x^3+y^3-1,x,y);
                                        2
                                       x
(%o2)                               - --
                                        2
                                       y
\end{verbatim}
\normalsize


Next find $d\,y/d\,x$ given that $\cos(x^2 - y^2) = y \, cos(x)$.
\small
\begin{verbatim}
(%i3) r1 : dydx( cos(x^2 - y^2) - y*cos(x),x,y );
                                    2    2
                         - 2 x sin(y  - x ) - sin(x) y
(%o3)                   -----------------------------
                                     2    2
                          - 2 y sin(y  - x ) - cos(x)
(%i4) r2 : (-num(r1))/(-denom(r1));
                                   2    2
                          2 x sin(y  - x ) + sin(x) y
(%o4)                    ---------------------------
                                    2    2
                           2 y sin(y  - x ) + cos(x)
\end{verbatim}
\normalsize
For some reason Maxima 5.15 does not simplify \verb|r1| by cancelling the minus signs,
  and we have resorted to dividing the negative of the numerator by the negative of
  the denominator(!) to get the form of \verb|r2|.
Notice also that Maxima has chosen to write $\sin(y^2 - x^2)$ instead of $\sin(x^2-y^2)$.

\smallskip
Finally, find $d\,y/d\,x$ given that $3\,y^4 + 4\,x -x^2\,\sin(y) - 4 = 0$.
\small
\begin{verbatim}
(%i5) dydx(3*y^4 +4*x -x^2*sin(y) - 4, x, y );
                                2 x sin(y) - 4
(%o5)                         -----------------
                                   3    2
                               12 y  - x  cos(y)
\end{verbatim}
\normalsize






\subsubsection{Example 1: Tangent and Normal of a Point of a Circle}
As an easy first example, lets consider a circle of radius $r$ defined
  by the equation $x^2 + y^2 = r^2$.
Let's choose $r = 1$.
Then $f(x,y) = x^2 + y^2 -1 = 0$ defines the circle of interest, and we use
  the "slope" function above to calculate the slope "m" for a general point $(x,y)$.
\small
\begin{verbatim}
(%i1) dydx(expr,x,y) := -diff(expr,x)/diff(expr,y)$
(%i2) f:x^2 + y^2-1$
(%i3) m : dydx(f,x,y);
                                        x
(%o3)                                 - -
                                        y
\end{verbatim}
\normalsize
We then choose a point of the circle, evaluate the slope at that point,
  and construct the tangent and normal at that point.
\small
\begin{verbatim}
(%i4) fpprintprec:8$
(%i5) [x0,y0] : [0.5,0.866];
(%o5)                            [0.5, 0.866]
(%i6) m : subst([x=x0,y=y0],m );
(%o6)                             - 0.577367
(%i7) tangent : y = m*(x-x0) + y0;
(%o7)                   y = 0.866 - 0.577367 (x - 0.5)
(%i8) normal : y = -(x-x0)/m + y0;
(%o8)                     y = 0.866 - 1.732 (0.5 - x)
\end{verbatim}
\normalsize
We can then use \textbf{qdraw} to show the circle with both the
  tangent and normal.
\small
\begin{verbatim}
(%i9) load(draw);
(%o9)   C:/PROGRA~1/MAXIMA~4.0/share/maxima/5.15.0/share/draw/draw.lisp
(%i10) load(qdraw);
               qdraw(...), qdensity(...), syntax: type qdraw(); 

(%o10)                        c:/work2/qdraw.mac
(%i11) ratprint:false$
(%i12) qdraw(key(bottom),ipgrid(15),
          imp([ f = 0,tangent,normal],x,-2.8,2.8,y,-2,2 ),
           pts([ [0.5,0.866]],ps(2) ) )$
\end{verbatim}
\normalsize

% eps file code
% (%i16) qdraw(key(bottom),lw(5),line(-2.8,0,2.8,0,lw(2)),
%           line(0,-2,0,2,lw(2) ),ipgrid(15),
%          imp([x^2 + y^2 = 1,(y - 0.866) = - 0.577*(x - 0.5),
%                    (y - 0.866) = 1.732*(x - 0.5)],x,-2.8,2.8,
%                   y,-2,2 ),
%           pts([ [0.5,0.866]],ps(2) ),
%           pic(eps,"ch6p8",font("Times-Roman",18)) );   
%
\newpage
The result is
\begin{figure} [h]
   \centerline{\includegraphics[scale=1.3]{ch6p8.eps} }
	\caption{ Tangent and Normal to $x^2+y^2=1$ at point $(0.5,0.866)$  }
\end{figure}      


\subsubsection{Example 2: Tangent and Normal of a Point of the Curve $sin(2\,x)\,cos(y)=0.5$}
A curve is defined by $f(x,y) = \sin(2\,x)\,\cos(y) - 0.5 = 0$,
  with  $0 \leq x \leq 2$ and $0 \leq y \leq 2$.
Using Eq. (\ref{Eq:1}) we calculate the slope of the tangent to this curve at some
  point $(x,y)$ and then specialize to the point $(x=1, y=y_0)$ with $y_0$ to be found:
\small
\begin{verbatim}
(%i13) f : sin(2*x)*cos(y) - 0.5$
(%i14) m : dydx(f,x,y);
                               2 cos(2 x) cos(y)
(%o14)                         -----------------
                                sin(2 x) sin(y)
(%i15) s1 : solve(subst(x=1,f),y);
`solve' is using arc-trig functions to get a solution.
Some solutions will be lost.
                                          1
(%o15)                       [y = acos(--------)]
                                       2 sin(2)
(%i16) fpprintprec:8$
(%i17) s1 : float(s1);
(%o17)                          [y = 0.988582]
(%i18) m : subst([ x=1, s1[1] ], m);
                               1.3166767 cos(2)
(%o18)                         ----------------
                                    sin(2)
\end{verbatim}
\newpage
\begin{verbatim}
(%i19) m : float(m);
(%o19)                            - 0.602587
(%i20) mnorm : -1/m;
(%o20)                             1.6595113
(%i21) y0 : rhs( s1[1] );
(%o21)                             0.988582
(%i22) qdraw( imp([f = 0, y - y0 = m*(x - 1), 
               y - y0 = mnorm*(x - 1) ],x,0,2,y,0,1.429),
              pts( [ [1,y0] ], ps(2) ), ipgrid(15))$
\end{verbatim}
\normalsize
which produces the plot
\begin{figure} [h]
   \centerline{\includegraphics[scale=.8]{ch6p9.eps} }
	\caption{ Tangent and Normal to $\sin(2\,x)\,\cos(y)=0.5$ at point $(1,0.988)$  }
\end{figure} 



% eps file code
% (%i59) qdraw(lw(5), imp([f=0,y-y0=m*(x-1),
%                    y-y0=mnorm*(x-1)],x,0,2,y,0,1.429),
%             ipgrid(15),pts( [ [1,y0] ], ps(2) ) , 
%              pic(eps,"ch6p9",font("Times-Roman",18)) );
%  

\subsubsection{Example 3: Tangent and Normal of a Point of the Curve: $x = \sin(t), y = \sin(2\,t)$ }
This example is the parametric curve example we plotted in Ch. 5.
If we divide the differential of $y$ by the differential of $x$,
  the common factor of $d\,t$ will cancel out and we will have
  an expression for the slope of the tangent to the curve at a
   point determined by the value the parameter t, which in this
  example must be an angle expressed in radians (as usual in
  calculus).
  
\smallskip
We then specialize to a point on the curve corresponding to
  $x = 0.8$, with $0 \leq t \leq \pi/2$, which we solve for:  
\small
\begin{verbatim}
(%i23) m : diff(sin(2*t))/diff(sin(t));
                                  2 cos(2 t)
(%o23)                            ----------
                                    cos(t)
(%i24) x0 : 0.8;
(%o24)                                0.8
(%i25) tsoln : solve(x0 = sin(t), t);
                                           4
(%o25)                           [t = asin(-)]
                                           5
(%i26) tsoln : float(tsoln);
(%o26)                          [t = 0.927295]
(%i27) t0 : rhs( tsoln[1] );
(%o27)                             0.927295
(%i28) m : subst( t = t0, m);
(%o28)                            - 0.933333
(%i29) mnorm : -1/m;
(%o29)                             1.0714286
(%i30) y0 : sin(2*t0);
(%o30)                               0.96
(%i31) qdraw( xr(0, 2.1), yr(0,1.5), ipgrid(15),nticks(200),
              para(sin(t),sin(2*t),t,0,%pi/2, lc(brown) ),                 
             ex([y0+m*(x-x0),y0+mnorm*(x-x0)],x,0,2.1 ),
            pts( [ [0.8, 0.96]],ps(2) ) )$
\end{verbatim}
\normalsize
% eps file code
%(%i77) qdraw( xr(0, 2.1), yr(0,1.5), ipgrid(15),nticks(200),
%              para(sin(t),sin(2*t),t,0,%pi/2, lc(brown) ),lw(5),
%             ex([y0+m*(x-x0),y0+mnorm*(x-x0)],x,0,2.1 ),
%            pts( [ [0.8, 0.96]],ps(2) ) ,
%             pic(eps,"ch6p10",font("Times-Roman",18)) );
%
with the resulting plot
\begin{figure} [h]
   \centerline{\includegraphics[scale=.8]{ch6p10.eps} }
	\caption{ Tangent and Normal to $x=\sin(t),\,y=\sin(2\,t)$ at point $t=0.927$ radians  }
\end{figure} 


\subsubsection{Example 4: Polar Plot:   $x = r(t)\, \cos(t), y = r(t)\, \sin(t)$ }
We use the polar plot example from ch. 5, in which we took $r(t) = 10/t$, and
  again $t$ is in radians, and we consider the interval $ 1 \leq t \leq \pi$ and
  find the tangent and normal at the curve point corresponding to $t = 2$ radians.
We find the general slope of the tangent by again forming the ratio of the
  differential $dy$ to the differential $dx$.
\small
\begin{verbatim}
(%i32) r : 10/t;
                                      10
(%o32)                                --
                                      t
(%i33) xx : r * cos(t);
                                   10 cos(t)
(%o33)                             ---------
                                       t
(%i34) yy : r * sin(t);
                                   10 sin(t)
(%o34)                             ---------
                                       t
\end{verbatim}
\newpage
\begin{verbatim}
(%i35) m : diff(yy)/diff(xx);
                             10 cos(t)   10 sin(t)
                             --------- - ---------
                                 t           2
                                            t
(%o35)                      -----------------------
                              10 sin(t)   10 cos(t)
                            - --------- - ---------
                                  t           2
                                             t
(%i36) m : ratsimp(m);
                               sin(t) - t cos(t)
(%o36)                         -----------------
                               t sin(t) + cos(t)
(%i37) m : subst(t = 2.0, m);
(%o37)                             1.2418222
(%i38) mnorm : -1/m;
(%o38)                            - 0.805268
(%i39) x0 : subst(t = 2.0, xx);
(%o39)                            - 2.0807342
(%i40) y0 : subst(t = 2.0, yy);
(%o40)                             4.5464871
(%i41) qdraw(polar(10/t,t,1,3*%pi,lc(brown) ),
           xr(-6.8,10),yr(-3,9),
           ex([y0 + m*(x-x0),y0 + mnorm*(x-x0)],x,-6.8,10 ),
           pts( [ [x0,y0] ], ps(2),pk("t = 2 rad") ) );
\end{verbatim}
\normalsize
% eps file code
% (%i93) qdraw( line(-6.8,0,10,0,lw(2)),
%           line(0,-3,0,9,lw(2) ),
%           polar(10/t,t,1,3*%pi,lc(brown),lw(5) ),
%           xr(-6.8,10),yr(-3,9),lw(5),
%           ex([y0 + m*(x-x0),y0 + mnorm*(x-x0)],x,-6.8,10 ),
%           pts( [ [x0,y0] ], ps(2),pk("t = 2 rad") ),
%             pic(eps,"ch6p11",font("Times-Roman",18) ) );
%
which produces the plot:


\begin{figure} [h]
   \centerline{\includegraphics[scale=1.2]{ch6p11.eps} }
	\caption{ Tangent and Normal to $x=10\,\cos(t)/t,\,y=10\,\sin(t)/t$ at point $t=2$ radians  }
\end{figure} 

%\newpage

\subsection{Limit Examples Using Maxima's limit(...) Function}
Maxima has a powerful \textbf{limit(...)} function which uses l'Hospital's rule and Taylor
  series expansions to investigate the limit of a univariate function as the variable
  approaches some point.
The Maxima manual has the following description the Maxima function \textbf{limit}:
\small
\begin{quote}
Function: \textbf{limit}\verb|(expr, x, val, dir)|\\
Function: \textbf{limit}\verb|(expr, x, val) |\\
Function: \textbf{limit}\verb|(expr)|\\ 
Computes the limit of \verb|expr| as the real variable \verb|x| approaches
 the value \verb|val| from the direction \verb|dir|.
\verb|dir| may have the value \verb|plus| for a limit from above,
  \verb|minus| for a limit from below, or may be omitted (implying
   a two-sided limit is to be computed). \\
\verb|limit| uses the following special symbols: \verb|inf| (positive infinity)
   and \verb|minf| (negative infinity).
On output it may also use \verb|und| (undefined), \verb|ind| (indefinite but bounded)
  and \verb|infinity| (complex infinity). \\
\verb|lhospitallim| is the maximum number of times L'Hospital's rule is used in \verb|limit|.
This prevents infinite looping in cases like \verb|limit (cot(x)/csc(x), x, 0)|.\\ 
\verb|tlimswitch| when \verb|true| will allow the \verb|limit| command
   to use Taylor series expansion when necessary. \\
\verb|limsubst| prevents \verb|limit| from attempting substitutions on unknown forms.
This is to avoid bugs like \verb|limit (f(n)/f(n+1), n, inf)| giving 1.
Setting \verb|limsubst| to \verb|true| will allow such substitutions. \\
\verb|limit| with one argument is often called upon to simplify constant
  expressions, for example, \verb|limit (inf-1)|. \\
\verb|example (limit)| displays some examples. 
\end{quote}
\normalsize
Here is the result of calling Maxima's \textbf{example} function:
\small
\begin{verbatim}
(%i1) example(limit)$
(%i2) limit(x*log(x),x,0,plus)
(%o2)                                  0
(%i3) limit((x+1)^(1/x),x,0)
(%o3)                                 %e
(%i4) limit(%e^x/x,x,inf)
(%o4)                                 inf
(%i5) limit(sin(1/x),x,0)
(%o5)                                 ind
\end{verbatim}
\normalsize
Most use of \textbf{limit} will use the first two ways to call \textbf{limit}.
The "direction" argument is optional.
The default values of the option switches mentioned above are:
\small
\begin{verbatim}
(%i6) [lhospitallim,tlimswitch,limsubst];
(%o6)                          [4, true, false]
\end{verbatim}
\normalsize
Thus the default Maxima behavior is to allow the use of a Taylor series expansion
  in finding the correct limit.
(We will discuss Taylor series expansions soon in this chapter.)
The default is also to prevent "substitutions" on unknown (formal) functions.
The third (single argument) syntax is illustrated by
\small
\begin{verbatim}
(%i7) limit(inf - 1);
(%o7)                                 inf
\end{verbatim}
\normalsize
The expression presented to the \textbf{limit} function in input \verb|%i7| contains only known
  constants, so there are no unbound (formal) parameters like \verb|x| for
  \textbf{limit} to worry about.

  \smallskip
Here is a use of \textbf{limit} which mimics the calculus definition of
  a derivative of a power of $x$.
\small
\begin{verbatim}
(%i8) limit( ( (x+eps)^3 - x^3 )/eps, eps, 0 );
                                        2
(%o8)                               3 x
\end{verbatim}
\normalsize
%\newpage

And a similar use of \textbf{limit} with $\ln(x)$:
\small
\begin{verbatim}
(%i9) limit( (log(x+eps) - log(x))/eps, eps,0 );
                                       1
(%o9)                                 -
                                       x
\end{verbatim}
\normalsize
What does Maxima do with a typical calculus definition of a derivative of a
  trig function?
\small
\begin{verbatim}
(%i10) limit((sin(x+eps)-sin(x))/eps, eps,0 );
Is  sin(x)  positive, negative, or zero?
p;
Is  cos(x)  positive, negative, or zero?
p;
(%o10)                              cos(x)
\end{verbatim}
\normalsize
We see above a typical Maxima query before producing an answer.
Using \verb|p;| instead of \verb|positive;| is allowed.  
Likewise one can use \verb|n;| instead of \verb|negative;|.
\subsubsection{Discontinuous Functions}
A simple example of a discontinuous function can be created using Maxima's
  \textbf{abs} function.\\
\verb|abs(expr)| returns either the absolute value \verb|expr|,
  or (if \verb|expr| is complex) the complex modulus of \verb|expr|.

\smallskip
We first plot the function $\vert x \vert /x$.  
\small
\begin{verbatim}
(%i11)  load(draw)$
(%i12)  load(qdraw)$
(%i13) qdraw( yr(-2,2),lw(8),ex(abs(x)/x,x,-1,1 ) )$
\end{verbatim}
\normalsize
% eps file code
% (%i31) qdraw( yr(-2,2), line(-1,0,1,0,lw(2)),
%            line(0,-2,0,2,lw(2) ),
%           lw(10),ex(abs(x)/x,x,-1,1 ) ,
%        pic(eps,"ch6p13",font("Times-Roman",18) ) );

Here is that plot of $\vert x \vert /x$:
\begin{figure} [h]
   \centerline{\includegraphics[scale=1.2]{ch6p13.eps} }
	\caption{ $\vert x \vert /x$  }
\end{figure} 

\newpage

Maxima correctly evaluates the one-sided limits:
\small
\begin{verbatim}
(%i14) limit(abs(x)/x,x,0,plus);
(%o14)                                 1
(%i15) limit(abs(x)/x,x,0,minus);
(%o15)                                - 1
(%i16) limit(abs(x)/x,x,0);
(%o16)                                und
\end{verbatim}
\normalsize
and Maxima also computes a derivative:
\small
\begin{verbatim}
(%i17) g : diff(abs(x)/x,x);
                                  1      abs(x)
(%o17)                          ------ - ------
                                abs(x)      2
                                           x
(%i18) g, x = 0.5;
(%o18)                                0.0
(%i19) g, x = - 0.5;
(%o19)                                0.0
(%i20) g,x=0;
Division by 0
 -- an error.  To debug this try debugmode(true);
(%i21) limit(g,x,0,plus);
(%o21)                                 0
(%i22) limit(g,x,0,minus);
(%o22)                                 0
(%i23) load(vcalc)$
(%i24) plotderiv(abs(x)/x,x,-2,2,-2,2,1)$
                                abs(x)    1      abs(x)
                       plist = [------, ------ - ------]
                                  x     abs(x)      2
                                                   x
\end{verbatim}
\normalsize
The derivative does not simplify to $0$ since the derivative is
  undefined at $x = 0$.
The plot of the step function and its derivative, as returned
  by \textbf{plotderiv(..)} is
\begin{figure} [h]
   \centerline{\includegraphics[scale=.7]{ch6p14.eps} }
	\caption{ $\vert x \vert /x$ and its Maxima derivative }
\end{figure} 

\newpage

A homemade unit step function can now be defined by adding $1$ to lift the
  function up to the value of $0$ for $ x < 0$ and then dividing the result
  by $2$ to get a "unit step function".
\small
\begin{verbatim}
(%i25) mystep : ( (1 + abs(x)/x)/2 , ratsimp(%%) );
                                  abs(x) + x
(%o25)                            ----------
                                     2 x
\end{verbatim}
\normalsize
We then use \textbf{qdraw} to plot the definition
\small
\begin{verbatim}
(%i26) qdraw(yr(-1,2),lw(5),ex(mystep,x,-1,1) )$
\end{verbatim}
\normalsize
% eps file code
% (%i12) qdraw(yr(-1,2),line(-1,0,1,0,lw(2)),
%            line(0,-1,0,2,lw(2) ),
%              lw(8),ex(mystep,x,-1,1) ,
%        pic(eps,"ch6p15",font("Times-Roman",18) ) );
with the result:
\begin{figure} [h]
   \centerline{\includegraphics[scale=.7]{ch6p15.eps} }
	\caption{ mystep }
\end{figure} 


\smallskip
Maxima has a function called \textbf{unit\_step(x)} available if you load
  the \textbf{orthopoly} package.
This function is "left continuous", since it has the value $0$ for $x \leq 1$.
However, no derivative is defined, although you can use \textbf{gradef}.
\small
\begin{verbatim}
(%i27) load(orthopoly)$
(%i28) map(unit_step,[-1/10,0,1/10] );
(%o28)                             [0, 0, 1]
(%i29) diff(unit_step(x),x);
                               d
(%o29)                         -- (unit_step(x))
                               dx
(%i30) gradef(unit_step(x),0);
(%o30)                           unit_step(x)
(%i31) diff(unit_step(x),x);
(%o31)                                 0							   
\end{verbatim}
\normalsize
Of course, defining the derivative to be $0$ everywhere can be dangerous in
  some circumstances.
You can use \textbf{unit\_step} in a plot using, say,
  \verb|qdraw(yr(-1,2),lw(5),ex(unit_step(x),x,-1,1) );|.  
%\newpage
Here we use \textbf{unit\_step} to define a "unit pulse" function
  \verb|upulse(x,x0,w)| which is a function of x which becomes equal to $1$
  when $x = x_0$ and has width $w$.
\small
\begin{verbatim}
(%i32) upulse(x,x0,w) := unit_step(x-x0) - unit_step(x - (x0+w))$
\end{verbatim}
\normalsize
and then make a plot of three black pulses of width $0.5$.
\small
\begin{verbatim}
(%i33) qdraw(yr(-1,2), xr(-3,3), 
           ex1( upulse(x,-3,0.5),x,-3,-2.49,lw(5)),  
          ex1( upulse(x,-1,0.5),x,-1,-.49,lw(5)),
           ex1( upulse(x,1,0.5),x,1,1.51,lw(5)) )$
\end{verbatim}
\normalsize
% eps file code
% (%i36) qdraw(yr(-1,2), xr(-3,3),line(0,-1,0,2 ), 
%           ex1( upulse(x,-3,0.5),x,-3,-2.49,lw(8)),  
%          ex1( upulse(x,-1,0.5),x,-1,-.49,lw(8)),
%           ex1( upulse(x,1,0.5),x,1,1.51,lw(8)),
%           pic(eps,"ch6p16",font("Times-Roman",18) ) );
with the result:
\begin{figure} [h]
   \centerline{\includegraphics[scale=.9]{ch6p16.eps} }
	\caption{ Using upulse(x,x0,w) }
\end{figure} 

\smallskip
Of course, the above drawing can be done more easily with the \textbf{poly(..)}
  function in \textbf{qdraw}, which uses \textbf{draw2d}'s \textbf{polygon(..)}
  function.


\subsubsection{Indefinite Limits}
The \textbf{example} run showed that $\sin(1/x)$ has no definite limit as
  the variable $x$ approaches zero.
This function oscillates increasingly rapidly between $\pm \,1$ as $x \rightarrow 0$.
It is instructive to make a plot of this function near the origin 
  using the smallest line width:
\small
\begin{verbatim}
(%i34) qdraw(lw(1),ex(sin(1/x),x,0.001,0.01));
\end{verbatim}
\normalsize
% eps file code
%(%i21) qdraw( line(0.001,0,0.01,0,lw(2) ),
%             lw(1),ex(sin(1/x),x,0.001,0.01),
%          pic(eps,"ch6p12",font("Times-Roman",18) ) );
%
%\newpage

The eps image reproduced here actually uses a finer line width than
  the Windows Gnuplot console window:
\begin{figure} [h]
   \centerline{\includegraphics[scale=.6]{ch6p12.eps} }
	\caption{ $\sin(1/x)$ Behavior Near $x = 0$  }
\end{figure} 

\newpage

The Maxima \textbf{limit} function correctly returns \textbf{ind} for "indefinite 
  but bounded" when asked to find the limit as $x \rightarrow 0^{+}$.
\small
\begin{verbatim}
(%i35) limit(sin(1/x),x,0,plus);
(%o35)                                ind
\end{verbatim}
\normalsize

\smallskip
An example of a function which is well behaved at $x=0$ but whose derivative is
  indefinite but bounded is $x^2\,\sin(1/x)$, which approaches the value $0$ at
  $x = 0$.
\small
\begin{verbatim}
(%i36) g : x^2*sin(1/x)$
(%i37) limit(g,x,0);
(%o37)                                 0
(%i38) dgdx : diff(g,x);
                                    1          1
(%o38)                        2 sin(-) x - cos(-)
                                    x          x
(%i39) limit(dgdx,x,0);
(%o39)                                ind
\end{verbatim}
\normalsize
In the first term of the derivative, $x\,\sin(1/x)$ is driven to $0$ by the factor $x$,
  but the second term oscillates increasingly rapidly between $\pm 1$ as $x \rightarrow 0$.
For a plot, we use the smallest line width and color blue for the derivative, and
  use a thicker red curve for the original function $x^2\,\sin(1/x)$.
\small
\begin{verbatim}
(%i40) qdraw( yr(-1.5,1.5),ex1(2*x*sin(1/x)-cos(1/x),x,-1,1,lw(1),lc(blue)),
               ex1(x^2*sin(1/x),x,-1,1,lc(red) ) )$
\end{verbatim}
\normalsize

\smallskip
Here is the plot:  



\begin{figure} [h]
   \centerline{\includegraphics[scale=1.2]{ch6p17.eps} }
	\caption{ Indefinite but Bounded Behavior of Derivative }
\end{figure} 

\subsection{Taylor Series Expansions using taylor(...) }
A Taylor (or Laurent) series expansion of a univariate expression has the syntax\\
\verb|  taylor( expr, x, a, n ) |\\
  which will return a Taylor or Laurent series expansion of \verb|expr| in the
  variable \verb|x| about the point \verb|x = a|, through terms proportional
  to \verb|(x - a)^n |.
Here are some examples of first getting the expansion, then evaluating the
  expansion at some other point, using both \textbf{at} and \textbf{subst}.
\small
\begin{verbatim}
(%i1) t1 : taylor(sqrt(1+x),x,0,5);
                              2    3      4      5
                         x   x    x    5 x    7 x
(%o1)/T/             1 + - - -- + -- - ---- + ---- + . . .
                         2   8    16   128    256
(%i2) [ at( t1, x=1 ),subst(x=1, t1 ) ];
                                   365  365
(%o2)                             [---, ---]
                                   256  256
(%i3) float(%);
(%o3)                      [1.42578125, 1.42578125]
(%i4) t2: taylor (cos(x) - sec(x), x, 0, 5);
                                       4
                                  2   x
(%o4)/T/                       - x  - -- + . . .
                                      6
(%i5) [ at( t2, x=1 ),subst(x=1, t2 ) ];
                                     7    7
(%o5)                             [- -, - -]
                                     6    6
(%i6) t3 : taylor(cos(x),x,%pi/4,4);
                                 %pi                 %pi 2                %pi 3
                    sqrt(2) (x - ---)   sqrt(2) (x - ---)    sqrt(2) (x - ---)
          sqrt(2)                 4                   4                    4
(%o6)/T/ ------- - ----------------- - ------------------ + ------------------
             2              2                   4                    12
                                                                  %pi 4
                                                     sqrt(2) (x - ---)
                                                                   4
                                                   + ------------------ + . . .
                                                             48
(%i7) ( at(t3, x = %pi/3), factor(%%) );
                       4         3           2
           sqrt(2) (%pi  + 48 %pi  - 1728 %pi  - 41472 %pi + 497664)
(%o7)     ---------------------------------------------------------
                                    995328
\end{verbatim}
\normalsize
To expand an expression depending on two variables, say \verb|(x,y)|, there
  are two essentially different forms.
The first form is\\
\verb|  taylor(expr, [x,y], [a,b],n )|\\
  which will expand in \verb|x| about \verb|x = a|  and expand
  in the variable \verb|y| about \verb|y = b|, up through combined powers
   of order \verb|n|.
If \verb|a = b|, then the simpler syntax:\\
\verb|  taylor(expr, [x,y], a, n ) |\\
will suffice. 
\newpage
  
\small
\begin{verbatim}
(%i8) t4 : taylor(sin(x+y),[x,y],0,3);
                            3        2      2      3
                           x  + 3 y x  + 3 y  x + y
(%o8)/T/           y + x - ------------------------- + . . .
                                       6
(%i9) (subst( [x=%pi/2, y=%pi/2], t4), ratsimp(%%) );
                                     3
                                  %pi  - 6 %pi
(%o9)                           - ------------
                                       6
(%i10) ( at( t4, [x=%pi/2,y=%pi/2]), ratsimp(%%) );
                                     3
                                  %pi  - 6 %pi
(%o10)                           - ------------
                                       6
\end{verbatim}
\normalsize
Note the crucial difference between this last example and the next.
\small
\begin{verbatim}
(%i11) t5 : taylor(sin(x+y),[x,0,3],[y,0,3] );
              3                 2                      3
             y                 y                  y   y            2
(%o11)/T/ y - -- + . . . + (1 - -- + . . .) x + (- - + -- + . . .) x
             6                 2                  2   12
                                                          2
                                                     1   y            3
                                                + (- - + -- + . . .) x  + . . .
                                                     6   12
(%i12) (subst([x=%pi/2,y=%pi/2],t5),ratsimp(%%) );
                              5         3
                           %pi  - 32 %pi  + 192 %pi
(%o12)                     ------------------------
                                     192
\end{verbatim}
\normalsize
Thus the syntax\\
\verb| taylor( expr, [x,a,nx], [y,b,ny] )|\\
will expand to higher combined powers in \verb|x| and \verb|y|.

\smallskip
We can differentiate and integrate a taylor series returned expression:
\small
\begin{verbatim}
(%i13) t6 : taylor(sin(x),x,0,7 );
                               3    5      7
                              x    x      x
(%o13)/T/                 x - -- + --- - ---- + . . .
                              6    120   5040
(%i14) diff(t6,x);
                                2    4    6
                               x    x    x
(%o14)/T/                  1 - -- + -- - --- + . . .
                               2    24   720
(%i15) integrate(%,x);
                                 7     5     3
                                x     x     x
(%o15)                       - ---- + --- - -- + x
                               5040   120   6
(%i16) integrate(t6,x);
                                8      6     4    2
                               x      x     x    x
(%o16)                      - ----- + --- - -- + --
                              40320   720   24   2
\end{verbatim}
\normalsize
A common use of the Taylor series is to expand a function of \verb|x| in the form
  $f(x + dx)$, for $dx$ small.
\small
\begin{verbatim}
(%i17) taylor(f(x+dx),dx,0,3);
                                                               !
                                                 2             !
                                                d              !          2
                                              (---- (f(x + dx))!      ) dx
                                                  2            !
                                !              ddx             !
                  d             !                              !dx = 0
(%o17)/T/ f(x) + (--- (f(x + dx))!      ) dx + -----------------------------
                 ddx            !                           2
                                !dx = 0
                                                           !
                                             3             !
                                            d              !          3
                                          (---- (f(x + dx))!      ) dx
                                              3            !
                                           ddx             !
                                                           !dx = 0
                                        + ----------------------------- + . . .
                                                        6
(%i18) taylor(cos(x+dx),dx,0,4);
                                       2            3            4
                              cos(x) dx    sin(x) dx    cos(x) dx
(%o18)/T/ cos(x) - sin(x) dx - ---------- + ---------- + ---------- + . . .
                                  2            6            24
\end{verbatim}
\normalsize
  
Another frequent use of the Taylor series is the expansion of a function which contains
  a small parameter which we will symbolize by \verb|e|.
\small
\begin{verbatim}
(%i19) g : log(a/e + sqrt(1+(a/e)^2 ) );
                                           2
                                 a        a
(%o19)                        log(- + sqrt(-- + 1))
                                 e         2
                                          e
(%i20) taylor(g,e,0,2);
                                                  2
                                                 e
(%o20)/T/          - log(e) + log(2 a) + . . . + ---- + . . .
                                                   2
                                                4 a
\end{verbatim}
\normalsize

More examples of Taylor and Laurent series expansions can be found in the
  Maxima manual.


\subsection{Vector Calculus Calculations and Derivations using vcalc.mac}
The package vcalc.mac has been mentioned in the introduction to this chapter.
Here we give a few examples of using this package for vector calculus
  work.

\subsubsection*{General Comments: Working with the Batch File vcalcdem.mac}
A separate file "vcalcdem.mac" is available with this chapter on the author's webpage.
This file is designed to be introduced into Maxima using the command \verb|batch(vcalcdem)|
 (if vcalcdem.mac is in your work directory, say, and you have set up your file search
  paths as suggested in Ch. 1).
You should first load in the vcalc.mac file which contains the definitions of
  the vcalc vector calculus functions.  

\smallskip
There are three points to emphasize when using batch files in this mode.
The first feature is the use of lines like \verb|"this is a comment"$| in the file, 
  which allow you to place readable comments between the usual Maxima input
  lines.
Secondly, a Maxima input which simply defines something and for which there
  is no need to see the "output" line should end with the dollar sign \verb|$|.
Finally, when you "batch in" the file, you see neither the dollar signs, nor the semi-colons
  which end the other inputs.
However, input and output numbers of the form \verb|(%i12)| and \verb|(%o12)| will
  be seen for inputs which end with a semi-colon, whereas there are no output 
  numbers for inputs which end with the dollar sign.
Once you get a little practice reading the text form of the batch file and
  comparing that to the Maxima display of the "batched in" file, you should have
  no problem understanding what is going on.  
The first lines of vcalcdem.mac are:
\small
\begin{verbatim}
" vcalcdem.mac: sample calculations and derivations"$
" default coordinates are cartesian (x,y,z)"$
" gradient and laplacian of a scalar field "$
depends(f,[x,y,z]);
grad(f);
lap(f);
\end{verbatim}
\normalsize
Here is what you will see in your Maxima session:
\small
\begin{verbatim}
(%i1) load(vcalc);
                vcalc.mac:   for syntax, type: vcalc_syntax(); 

 CAUTION: global variables set and used in this package:

       hhh1, hhh2, hhh3, uuu1, uuu2, uuu3, nnnsys, nnnprint, tttsimp 

(%o1)                         c:/work2/vcalc.mac
(%i2) batch(vcalcdem);
batching #pc:/work2/vcalcdem.mac
(%i3)          vcalcdem.mac: sample calculations and derivations
(%i4)              default coordinates are cartesian (x,y,z)
(%i5)              gradient and laplacian of a scalar field 
(%i6)                        depends(f, [x, y, z])
(%o6)                            [f(x, y, z)]
(%i7)                               grad(f)
cartesian  [x, y, z] 
                                  df  df  df
(%o7)                            [--, --, --]
                                  dx  dy  dz
\end{verbatim}
\newpage
\begin{verbatim}
(%i8)                               lap(f)
cartesian  [x, y, z] 
                                 2     2     2
                                d f   d f   d f
(%o8)                           --- + --- + ---
                                  2     2     2
                                dz    dy    dx
\end{verbatim}
\normalsize
The default coordinates are cartesian $(x,y,z)$, and by telling Maxima that the
  otherwise undefined symbol \verb|f| is to be treated as an explicit function
  of $(x,y,z)$, we get symbolic output from \verb|grad(f)| and \verb|lap(f)| which
  respectively produce the gradient and laplacian in the current coordinate system.

\smallskip
For each function used, a reminder is printed to your screen concerning the current
 coordinate system and the current choice of independent variables.

\smallskip
Three dimensional vectors (the only kind allowed by this package) are represented
  by lists with three elements.
In the default cartesian coordinate system $(x,y,z)$, the first slot is the
  $x$ component of the vector, the second slot is the $y$ component, and the
  third slot is the $z$ component.

\smallskip
Now that you have seen the difference between the file vcalcdem.mac and the
  Maxima response when using "batch", we will just show the latter for brevity.
You can, of course, look at the file vcalcdem.mac with a text editor.

\smallskip
Here the batch file displays the divergence and curl of a general 3-vector in a cartesian
  coordinate system.  
\small
\begin{verbatim}
(%i9)                divergence and curl of a vector field
(%i10)                        avec : [ax, ay, az]
(%o10)                           [ax, ay, az]
(%i11)                     depends(avec, [x, y, z])
(%o11)              [ax(x, y, z), ay(x, y, z), az(x, y, z)]
(%i12)                             div(avec)
cartesian  [x, y, z] 
                                daz   day   dax
(%o12)                          --- + --- + ---
                                dz    dy    dx
(%i13)                            curl(avec)
cartesian  [x, y, z] 
                        daz   day  dax   daz  day   dax
(%o13)                 [--- - ---, --- - ---, --- - ---]
                        dy    dz   dz    dx   dx    dy
\end{verbatim}
\normalsize
We next have two vector calculus identities:
\small
\begin{verbatim}
(%i14)          vector identities true in any coordinate system
(%i15)                           curl(grad(f))
cartesian  [x, y, z] 
cartesian  [x, y, z] 
(%o15)                             [0, 0, 0]
(%i16)                          div(curl(avec))
cartesian  [x, y, z] 
cartesian  [x, y, z] 
(%o16)                                 0
\end{verbatim}
\normalsize
\newpage
and an example of finding the Laplacian of a vector field (rather than a scalar field)
  with an explicit example:
\small
\begin{verbatim}
(%i17)                    laplacian of a vector field 
                                3       3     2    3
(%i18)                   aa : [x  y, x y  z, x  y z ]
                              3       3     2    3
(%o18)                      [x  y, x y  z, x  y z ]
(%i19)                              lap(aa)
cartesian  [x, y, z] 
                                            3      2
(%o19)                [6 x y, 6 x y z, 2 y z  + 6 x  y z]
\end{verbatim}
\normalsize
and an example of the use of the vector cross product which is included with this
  package:
\small
\begin{verbatim}
(%i20)                       vector cross product 
(%i21)                        bvec : [bx, by, bz]
(%o21)                           [bx, by, bz]
(%i22)                        lcross(avec, bvec)
(%o22)           [ay bz - az by, az bx - ax bz, ax by - ay bx]
\end{verbatim}
\normalsize
We next change the current coordinate system to cylindrical with a choice of
  the symbols \verb|rho|, \verb|phi|, and \verb|z| as the independent variables.
We again start with some general expressions
\small
\begin{verbatim}
(%i23)             cylindrical coordinates using (rho,phi,z) 
(%i24)                     setcoord(cy(rho, phi, z))
(%o24)                               true
(%i25)             gradient and laplacian of a scalar field 
(%i26)                     depends(g, [rho, phi, z])
(%o26)                         [g(rho, phi, z)]
(%i27)                              grad(g)
cylindrical  [rho, phi, z] 
                                       dg
                                      ----
                                 dg   dphi  dg
(%o27)                         [----, ----, --]
                                drho  rho   dz
(%i28)                              lap(g)
cylindrical  [rho, phi, z] 
                                   2
                                  d g
                           dg    -----
                          ----       2    2      2
                          drho   dphi    d g    d g
(%o28)                    ---- + ----- + --- + -----
                          rho       2      2       2
                                 rho     dz    drho
(%i29)               divergence and curl of a vector field
(%i30)                       bvec : [brh, bp, bz]
(%o30)                           [brh, bp, bz]
(%i31)                   depends(bvec, [rho, phi, z])
(%o31)       [brh(rho, phi, z), bp(rho, phi, z), bz(rho, phi, z)]
\end{verbatim}
\newpage
\begin{verbatim}
(%i32)                             div(bvec)
cylindrical  [rho, phi, z] 
                                  dbp
                                  ----
                            brh   dphi   dbz   dbrh
(%o32)                      --- + ---- + --- + ----
                            rho   rho    dz    drho
(%i33)                            curl(bvec)
cylindrical  [rho, phi, z] 
                 dbz                        dbrh
                 ----                       ----
                 dphi   dbp  dbrh   dbz     dphi   bp    dbp
(%o33)          [---- - ---, ---- - ----, - ---- + --- + ----]
                 rho    dz    dz    drho    rho    rho   drho
\end{verbatim}
\normalsize
Instead of using \verb|setcoord(...)| to change the current coordinate system,
  we can insert an extra argument in the vector calculus function we are using.
Here is an example of not changing the coordinate system, but telling Maxima to
  use \verb|r| instead of \verb|rho| with the currently operative cylindrical
  coordinates.
\small
\begin{verbatim}
(%i34)   change from cylindrical coordinate label rho to r on the fly: 
                                          1
(%i35)                         bvec : [0, -, 0]
                                          r
(%i36)                     div(bvec, cy(r, phi, z))
cylindrical  [r, phi, z] 
(%o36)                                 0
\end{verbatim}
\normalsize
Here is an example of using this method to switch to spherical polar
  coordinates:
\small
\begin{verbatim}
(%i37)         change to spherical polar coordinates on the fly 
                                         sin(theta)
(%i38)                     cvec : [0, 0, ----------]
                                              2
                                             r
(%i39)                    div(cvec, s(r, theta, phi))
spherical polar  [r, theta, phi] 
(%o39)                                 0
(%i40)   coordinate system remains spherical unless explicitly changed 
(%i41)                    cvec : [0, 0, r sin(theta)]
(%i42)                             div(cvec)
spherical polar  [r, theta, phi] 
(%o42)                                 0
(%i43)          example of div(vec) = 0 everywhere except r = 0 
                                     1
(%i44)                          div([--, 0, 0])
                                      2
                                     r
spherical polar  [r, theta, phi] 
(%o44)                                 0
\end{verbatim}
\normalsize
The best way to get familiar with vcalc.mac is just to play with it.
Some syntax descriptions are built into the package.
  
\subsection{Maxima Derivation of Vector Calculus Formulas in Cylindrical Coordinates   }
In this section we start with the cartesian (rectangular) coordinate expressions for the
  divergence, the gradient, the curl, and the Laplacian of a scalar expression, and
  use the power of Maxima to find the correct forms of these vector calculus formulas
  in a cylindrical coordinate system.
  
\begin{figure} [h]
   \centerline{\includegraphics[scale=1]{cylinder2.eps} }
	\caption{ $\rho$ and $\phi$ unit vectors   }
\end{figure}   

\smallskip
  
Consider a change of variable
  from cartesian coordinates $(x,y,z)$ to cylindrical
  coordinates $(\rho,\varphi, z)$.  
Given $(\rho,\varphi)$, we obtain $(x,y)$ from the pair of equations
  $x = \rho \,\cos(\varphi)$ and $y = \rho \,\sin(\varphi)$.
Given $(x,y)$, we obtain $(\rho,\varphi)$ from the equations
  $\rho = \sqrt{x^2 + y^2}$ (or $\rho^2 = x^2 + y^2$) and
  $\tan(\varphi) = y/x$ ( or $\varphi = \arctan(y/x)$ ).

\smallskip
  
We consider the derivatives of a general scalar function $f(\rho,\varphi,z)$ which
  is an explicit function of the cylindrical coordinates $(\rho, \varphi, z)$
  and an implicit function of the cartesian coordinates $(x,y,z)$ via the dependence of
  $\rho$ and $\varphi$ on $x$ and $y$.

  \smallskip
Likewise we consider a general three dimensional vector $\mathbf{B}(\rho,\varphi,z)$
 which is an explicit function of the cylindrical coordinates and an implicit function
  of the cartesian coordinates.
  
Since many of the fundamental equations governing processes in the physical 
  sciences and engineering can be written down in terms of the Laplacian, the
  divergence, the gradient, and the curl, we concentrate on using Maxima
  to do the "heavy lifting" (ie., the tedious algebra) to work out
  formulas for these operations in cylindrical coordinates.

Our approach to the use of vectors here is based on choosing an explicit
  set of three element lists to represent the cartesian unit vectors.
We then use the built-in dot product of lists to implement the scalar
  product of three-vectors, used to check orthonormality, and we also provide
  a simple cross product function which can be used to check that the unit
  vectors we write down are related by the vector cross product to each other
  in the conventional way (the "right hand rule").
  
\subsubsection{The Calculus Chain Rule in Maxima}
Let $g$ be some scalar function which depends implicitly on $(x,y,z)$
  via an explicit dependence on $(u,v,w)$.
We want to express the partial derivative of $g$ with respect
  to $x$, $y$, or $z$ in terms of derivatives of $g$ with
  respect to $u$, $v$, and $w$.

\smallskip
We first tell Maxima to treat $g$ as an explicit function of $(u,v,w)$.  
\small
\begin{verbatim}
(%i1) depends(g,[u,v,w]);
(%o1)                            [g(u, v, w)]
\end{verbatim}
\normalsize
If, at this point, we ask for the derivative of $g$ with respect to $x$,
  we get zero, since Maxima has no information yet about dependence of
  $u$, $v$, and $w$ on $x$.
\small
\begin{verbatim}
(%i2) diff(g,x);
(%o2)                                  0
\end{verbatim}
\normalsize
One way to make progress is to use another \textbf{depends} statement, as in
\small
\begin{verbatim}
(%i3) depends([u,v,w],[x,y,z]);
(%o3)                [u(x, y, z), v(x, y, z), w(x, y, z)]
\end{verbatim}
\normalsize
which now allows Maxima to demonstrate its knowledge of the calculus "chain rule",
  which Maxima writes as:
\small
\begin{verbatim}
(%i4) diff(g,x);
                             dg dw   dg dv   dg du
(%o4)                        -- -- + -- -- + -- --
                             dw dx   dv dx   du dx
\end{verbatim}
\normalsize
Note how Maxima writes the "chain rule" in this example, which would
  be written in a calculus text as
\begin{equation}
\left(\frac{\partial g(u,v,w)}{\partial x}  \right)_{(y,z)} = 
    \frac{\partial g(u,v,w)}{\partial u} \, \frac{\partial\,u(x,y,z)}{\partial\,x}  +
    \frac{\partial g(u,v,w)}{\partial v}  \, \frac{\partial\,v(x,y,z)}{\partial\,x} +
	\frac{\partial g(u,v,w)}{\partial w}  \, \frac{\partial\,w(x,y,z)}{\partial\,x}
\end{equation}
Instead of using a \textbf{depends} statement to tell Maxima about the $(x,y,z)$ 
  dependence of $(u,v,w)$ , we can use the \textbf{gradef} function in a syntax
  we have not used before.

\small
\begin{verbatim}
(%i5) (gradef(u,x,dudx),gradef(u,y,dudy),gradef(u,z,dudz),
        gradef(v,x,dvdx),gradef(v,y,dvdy),gradef(v,z,dvdz),
         gradef(w,x,dwdx),gradef(w,y,dwdy),gradef(w,z,dwdz) )$
\end{verbatim}
\normalsize
This now produces the partial derivative
\small
\begin{verbatim}
(%i6) diff(g,x);
                               dg        dg        dg
(%o6)                     dwdx -- + dvdx -- + dudx --
                               dw        dv        du
(%i7) grind(%)$
dwdx*'diff(g,w,1)+dvdx*'diff(g,v,1)+dudx*'diff(g,u,1)$							   
\end{verbatim}
\normalsize
This is the method we will use in this section on cylindrical coordinates and in
  the next section on spherical polar coordinates.
%\newpage

\smallskip

In the following sections we discuss successive parts of a batch file called
  "cylinder.mac", available with this chapter on the author's webpage.
This file is designed to be introduced into Maxima with the input: \verb|batch(cylinder)| 
  (if cylinder.mac is in your work directory, say, and you have set up your file search
  paths as suggested in Ch. 1).
We have given some orientation concerning batch files in the previous section.

\subsubsection*{Relating $(x,y,z)$ to $(\rho,\varphi,z)$}
In cylinder.mac we use \verb|rh| to represent $\rho$ and \verb|p| to represent
 the angle $\varphi$  expressed in radians.
The ranges of the independent variables are $ 0 < \rho < \infty$,
  $0 \leq \varphi \leq 2\,\pi$, and $-\infty \leq z \leq +\infty$. 
We first define \verb|c3rule| as a list of replacement "rules"
  in the form of equations which can be used later by the \textbf{subst} function.
To get automatic simplification of \verb|rh/abs(rh)| we use the \textbf{assume}
 function.  
\small
\begin{verbatim}
"  cylindrical coordinates (rho, phi, z ) = (rh, p, z)  "$   
" replacement rules x,y,z to rh, p, z  "$
c3rule : [x = rh*cos(p), y = rh*sin(p)  ]$
c3sub(expr) := (subst(c3rule,expr),trigsimp(%%) )$
rhxy : sqrt(x^2 + y^2)$
assume(rh > 0)$
\end{verbatim}
\normalsize
which Maxima displays as:
\small
\begin{verbatim}
(%i1) batch(cylinder);
batching #pc:/work2/cylinder.mac
(%i2)           cylindrical coordinates (rho,phi,z ) = (rh,p,z)  
(%i3)                 replacement rules x,y,z to rh,p,z  
(%i4)               c3rule : [x = rh cos(p), y = rh sin(p)]
(%i5)         c3sub(expr) := (subst(c3rule, expr), trigsimp(%%))
                                          2    2
(%i6)                        rhxy : sqrt(y  + x )
(%i7)                           assume(rh > 0)
\end{verbatim}
\normalsize
Now that you have seen what Maxima does with a file introduced into Maxima
 via \verb|batch(cylinder)|, we will stop displaying the contents of cylinder.mac
 but just show you Maxima's response.
You can, of course, look at cylinder.mac with a text editor if you are not sure
  what was the input.

\smallskip
We next tell Maxima how to work with the cylindrical coordinates and their derivatives:
We let the symbol \verb|drhdx|, for example, hold (for later use) $\partial\, \rho / \partial\,x$.
\small
\begin{verbatim}
(%i8)           partial derivatives of rho and phi wrt x and y 
(%i9)                 drhdx : (diff(rhxy, x), c3sub(%%))
(%o9)                               cos(p)
(%i10)                drhdy : (diff(rhxy, y), c3sub(%%))
(%o10)                              sin(p)
                                       y
(%i11)               dpdx : (diff(atan(-), x), c3sub(%%))
                                       x
                                     sin(p)
(%o11)                             - ------
                                       rh
                                       y
(%i12)               dpdy : (diff(atan(-), y), c3sub(%%))
                                       x
                                    cos(p)
(%o12)                              ------
                                      rh
(%i13)               tell Maxima rh=rh(x,y) and p = p(x,y) 
(%i14)           (gradef(rh, x, drhdx), gradef(rh, y, drhdy))
(%i15)             (gradef(p, x, dpdx), gradef(p, y, dpdy))
\end{verbatim}
\normalsize
We thus have established the derivatives $\partial \rho(x,y) / \partial x = \cos(\varphi)$,
  $\partial \rho(x,y) / \partial y = \sin(\varphi)$, 
  $\partial \varphi(x,y) / \partial x = -\sin(\varphi)/\rho$, 
  and $\partial \varphi(x,y) / \partial y = \cos(\varphi)/\rho$.
  

\subsubsection{Laplacian \quad $\boldsymbol{\nabla}^2 \, f(\rho,\varphi,z)$ }
The Laplacian of a scalar function $f$ depending on $(x,y,z)$,is defined
  as
\begin{equation}
 \boldsymbol{\nabla}^2\,f(x,y,z) \equiv \frac{\partial^2 \, f(x,y,z)}{\partial \, x^2}  +
                          \frac{\partial^2 \, f(x,y,z)}{\partial \, y^2} +
						  \frac{\partial^2 \, f(x,y,z)}{\partial \, z^2}
\end{equation}
Let's calculate the Laplacian of a scalar function $f$ when
  the function depends implicitly on $(x,y,z)$ via an explicit dependence on
  the cylindrical coordinates $(\rho,\varphi,z)$.
Because of our Maxima input\\
 "\verb|depends(f,[rh,p,z])|", the cartesian form of
  the Laplacian of $f$ is expressed in terms of the fundamental set of
  derivatives ($\partial\,\rho/\partial\,x$, etc. ) which have been established above.

\newpage

The next section of cylinder.mac then produces the response:  

\small
\begin{verbatim}
(%i16)           tell Maxima to treat scalar function f as an 
(%i17)                    explicit function of (rh,p,z) 
(%i18)                      depends(f, [rh, p, z])
(%o18)                           [f(rh, p, z)]
(%i19)      calculate the Laplacian of the scalar function f(rh,p,z) 
(%i20)                   using the cartesian definition 
(%i21) (diff(f, z, 2) + diff(f, y, 2) + diff(f, x, 2), trigsimp(%%), 
                                     multthru(%%) )
                                   2
                                  d f
                            df    ---
                            ---     2    2     2
                            drh   dp    d f   d f
(%o21)                      --- + --- + --- + ----
                            rh      2     2      2
                                  rh    dz    drh
(%i22)                             grind(%)
'diff(f,rh,1)/rh+'diff(f,p,2)/rh^2+'diff(f,z,2)+'diff(f,rh,2)$
\end{verbatim}
\normalsize
We then have the result:

\begin{equation}
\boldsymbol{\nabla}^2\,f(\rho,\varphi,z) =  \frac{1}{\rho} \frac{\partial \, f}{\partial \, \rho} +
       \frac{\partial^2 \, f}{\partial \, \rho^2} +
	   \frac{1}{\rho^2} \frac{\partial^2 \, f}{\partial \, \varphi^2} +
	   \frac{\partial^2 \, f}{\partial \, z^2}
\end{equation}
The two terms involving derivatives with respect to $\rho$ can be combined:
\begin{equation}
\frac{1}{\rho} \frac{\partial \, f}{\partial \, \rho} + 
\frac{\partial^2 \, f}{\partial \, \rho^2}  =  
\frac{1}{\rho} \, \frac{\partial}{\partial \, \rho} \left( \rho \,\frac{\partial f}{\partial \rho} \right)
\end{equation}
We clearly need to avoid points for which $\rho = 0$.

\smallskip

It is one thing to use Maxima to help derive the correct form of an operation
  like the Laplacian operator in cylindrical coordinates, and another to build
  a usable function which can be used (without repeating a derivation each time!)
  to calculate the Laplacian when we have some function we are interested in.

\smallskip
The file "vcalc.mac", available with this chapter on the author's webpage, contains
  usable functions for the Laplacian, gradient, divergence, and curl for the
  three coordinate systems: cartesian, cylindrical, and spherical polar.

  \smallskip
  However, here is a Maxima function specifically designed for cylindrical coordinates
  which will calculate the Laplacian of a scalar expression in cylindrical coordinates.
No effort at simplification is made inside this function; the result
  can be massaged by the alert user which Maxima always assumes is at the ready.  
(Note: the laplacian and other functions in vcalc.mac start with the general 
  expressions valid in any orthonormal coordinate system, using the appropriate
  "scale factors" (h1,h2,h3), and some simplification is done.
Also, the laplacian in vcalc.mac will correctly compute the Laplacian of
  a vector field as well as a scalar field.)  
\small
\begin{verbatim}
(%i1) cylaplacian(expr,rho,phi,z) :=                
             (diff(expr,rho)/rho + diff(expr,phi,2)/rho^2 +
                diff(expr,rho,2) + diff(expr,z,2)) $   
\end{verbatim}
\normalsize

\newpage
We can show that the combinations $\rho^n\,\cos(n\,\varphi)$ and $\rho^n\,\sin(n\,\varphi)$
  are each solutions of Laplace's equation $\nabla^2\,u = 0$.
\small
\begin{verbatim}
(%i2) ( cylaplacian(rh^n*cos(n*p),rh,p,z), factor(%%) );
(%o2)                                 0
(%i3) ( cylaplacian(rh^n*sin(n*p),rh,p,z), factor(%%) );
(%o3)                                 0
(%i4) ( cylaplacian(rh^(-n)*cos(n*p),rh,p,z), factor(%%) );
(%o4)                                 0
(%i5) ( cylaplacian(rh^(-n)*sin(n*p),rh,p,z), factor(%%) );
(%o5)                                 0
\end{verbatim}
\normalsize
Another general solution is $\ln(\rho)$:
\small
\begin{verbatim}
(%i6) cylaplacian(log(rh),rh,p,z);
(%o6)                                 0
\end{verbatim}
\normalsize


Here is another example of the use of this Maxima function.
The expression $u = (2/3)\,(\rho - 1/\rho)\,\sin(\varphi)$ is proposed as the solution to
  a problem defined by: (partial differential equation: pde)
  \; $\boldsymbol{\nabla}^2\,u = 0$ for $1 \le \rho \le 2$,
  and (boundary conditions: bc) $u(1,\varphi) = 0$, and $u(2,\varphi) = \sin(\varphi)$.
Here we use Maxima to check the three required solution properties.  
\small
\begin{verbatim}
(%i7) u : 2*(rh - 1/rh)*sin(p)/3$
(%i8) (cylaplacian(u,rh,p,z), ratsimp(%%) );
(%o8)                                 0
(%i9) [subst(rh=1,u),subst(rh=2,u) ];
(%o9)                            [0, sin(p)]
\end{verbatim}
\normalsize
    
\subsubsection{Gradient \quad $\boldsymbol{\nabla} \, f(\rho,\varphi,z)$ }
There are several ways one can work with vector calculus problems.
One method is to use symbols for a set of orthogonal unit vectors, and
  assume a set of properties for these unit vectors without connecting
  the set to any specific list basis representation.
One can then construct the dot product, the cross product and the derivative operations
 in terms of the coefficients of these symbolic unit vectors (using \textbf{ratcoeff}, for example).
 
 
\smallskip
We have chosen to define our 3-vectors in terms of three element lists,
 in which the first element of the list contains the $x$ axis component of
 the vector, the second element of the list contains the $y$ axis component,
   and the third element contains the $z$ component.
   
\smallskip
This method is less abstract and closer to the beginning student's experience of
  vectors, and provides a straightforward path which can be used with any 
  orthonormal coordinate system
  
  
\smallskip
The unit vector along the $x$ axis ($\mathbf{\hat{x}}$) is represented by 
  \verb|xu = [1,0,0]| (for " x - unit vector"), the unit vector along the $y$ axis
  ($\mathbf{\hat{y}}$) is represented by \verb|yu = [0,1,0]|, and the unit vector along the
  $z$ axis ($\mathbf{\hat{z}}$) is represented by \verb|zu = [0,0,1]|.
 

 \smallskip
Here we return to the batch file cylinder.mac, where we define \verb|lcross(u,v)|
  to calculate the vector cross product when we are using three element lists
  to represent vectors.

\smallskip
Note that our "orthonormality checks" use the built-in "dot product" of
  lists provided by the period \verb|.| (which is also used for non-commutative
  matrix multiplication). 
The dot product of two vectors represented by Maxima lists is obtained by placing a 
  period between the lists.
We have checked that the cartesian vectors defined are "unit vectors" (dot product yields unity)
  and are also mutually orthogonal (ie., at right angles to each other) which is 
  equivalent to the dot product being zero.
  
\smallskip
  
\small
\begin{verbatim}
(%i23)        prepare to find derivatives which involve vectors. 
(%i24)          cross product rule when using lists for vectors 
(%i25) lcross(u, v) := (u  v  - u  v ) zu + (u  v  - u  v ) yu
                         1  2    2  1         3  1    1  3
                                                           + (u  v  - u  v ) xu
                                                               2  3    3  2
(%i26)             cross product of parallel vectors is zero 
(%i27)                lcross([a, b, c], [n a, n b, n c])
(%o27)                                 0
(%i28)                  a function we can use with map 
(%i29)                  apcr(ll) := apply('lcross, ll)
(%i30)               3d cartesian unit vectors using lists 
(%i31)         (xu : [1, 0, 0], yu : [0, 1, 0], zu : [0, 0, 1])
(%i32)          orthonormality checks on cartesian unit vectors 
(%i33)      [xu . xu, yu . yu, zu . zu, xu . yu, xu . zu, yu . zu]
(%o33)                        [1, 1, 1, 0, 0, 0]
(%i34)     low tech check of cross products of cartesian unit vectors
(%i35)                        lcross(xu, yu) - zu
(%o35)                             [0, 0, 0]
(%i36)                        lcross(yu, zu) - xu
(%o36)                             [0, 0, 0]
(%i37)                        lcross(zu, xu) - yu
(%o37)                             [0, 0, 0]
(%i38)         [lcross(xu, xu), lcross(yu, yu), lcross(zu, zu)]
(%o38)                 [[0, 0, 0], [0, 0, 0], [0, 0, 0]]
(%i39)    high tech checks of cross products of cartesian unit vectors
(%i40)     map('apcr, [[xu, yu], [yu, zu], [zu, xu]]) - [zu, xu, yu]
(%o40)                 [[0, 0, 0], [0, 0, 0], [0, 0, 0]]
(%i41)            map('apcr, [[xu, xu], [yu, yu], [zu, zu]])
(%o41)                 [[0, 0, 0], [0, 0, 0], [0, 0, 0]]
\end{verbatim}
\normalsize

Thus we have cartesian unit vector relations such as
  $\mathbf{\hat{x}} \boldsymbol{\cdot} \mathbf{\hat{x}} = 1$, and
  $\mathbf{\hat{x}} \boldsymbol{\cdot} \mathbf{\hat{y}} = 0$, and
  $\mathbf{\hat{x}} \boldsymbol{\times} \mathbf{\hat{y}} = \mathbf{\hat{z}}$.

\smallskip
  
The unit vector $\boldsymbol{\hat{\rho}}$  along the direction  of increasing $\rho$ 
  at the point $(\rho,\varphi,z)$ is defined in terms of the cartesian unit
  vectors $\mathbf{\hat{x}}$ and $\mathbf{\hat{y}}$ via
\begin{equation}  
  \boldsymbol{\hat{\rho}} = \mathbf{\hat{x}}\,\cos \varphi +   \mathbf{\hat{y}} \,\sin \varphi
\end{equation}
Because the direction of $\boldsymbol{\hat{\rho}}$ depends on $\varphi$,
  a more explicit notation would be $\boldsymbol{\hat{\rho}}(\varphi)$, but following 
  convention, we surpress that dependence in the following.

\smallskip
  
The unit vector $\boldsymbol{\hat{\varphi}}$  along the direction of increasing $\varphi$
  at the point $(\rho,\varphi,z)$ is
\begin{equation}  
  \boldsymbol{\hat{\varphi}} = -\mathbf{\hat{x}} \,\sin \varphi +   \mathbf{\hat{y}} \,\cos \varphi
\end{equation}  
Again, because the direction of $\boldsymbol{\hat{\varphi}}$ depends on $\varphi$,
  a more explicit notation would be $\boldsymbol{\hat{\varphi}}(\varphi)$, but following
  conventional use we surpress that dependence in the following.
We use the symbol \verb|rhu| (rh-unit-vec) for $\boldsymbol{\hat{\rho}}$, and the
  symbol \verb|pu| for $\boldsymbol{\hat{\varphi}}$.

\small
\begin{verbatim}
(%i42)       cylindrical coordinate unit vectors rho-hat, phi-hat  
(%i43)                    rhu : sin(p) yu + cos(p) xu
(%o43)                        [cos(p), sin(p), 0]
(%i44)                    pu : cos(p) yu - sin(p) xu
(%o44)                       [- sin(p), cos(p), 0]
\end{verbatim}
\newpage
\begin{verbatim}
(%i45)               orthonormality checks on unit vectors 
(%i46) ([rhu . rhu, pu . pu, zu . zu, rhu . pu, rhu . zu, pu . zu], 
                                                                  trigsimp(%%))
(%o46)                        [1, 1, 1, 0, 0, 0]
(%i47)                 low tech check of cross products 
(%i48)               (lcross(rhu, pu), trigsimp(%%)) - zu
(%o48)                             [0, 0, 0]
(%i49)               (lcross(pu, zu), trigsimp(%%)) - rhu
(%o49)                             [0, 0, 0]
(%i50)               (lcross(zu, rhu), trigsimp(%%)) - pu
(%o50)                             [0, 0, 0]
(%i51)                [lcross(rhu, rhu), lcross(pu, pu)]
(%o51)                      [[0, 0, 0], [0, 0, 0]]
(%i52)                high tech checks of cross products 
(%i53) (map('apcr, [[rhu, pu], [pu, zu], [zu, rhu]]), trigsimp(%%))
                                                                - [zu, rhu, pu]
(%o53)                 [[0, 0, 0], [0, 0, 0], [0, 0, 0]]
(%i54)                map('apcr, [[rhu, rhu], [pu, pu]])
(%o54)                      [[0, 0, 0], [0, 0, 0]]
\end{verbatim}
\normalsize
Thus we have cylindrical unit vector relations such as 
  $\boldsymbol{\hat{\rho}} \boldsymbol{\cdot} \boldsymbol{\hat{\rho}} = 1$,
  $\boldsymbol{\hat{\rho}} \boldsymbol{\cdot} \boldsymbol{\hat{\varphi}} = 0$,
  and $\boldsymbol{\hat{\rho}} \boldsymbol{\times} \boldsymbol{\hat{\varphi}} = \mathbf{\hat{z}}$.

\smallskip

The gradient of a scalar function $f(x,y,z)$ is defined by
\begin{equation}
\boldsymbol{\nabla} \, f(x,y,z) =
           \mathbf{\hat{x}}\,\frac{\partial f}{\partial x} +
          \mathbf{\hat{y}}\,\frac{\partial f}{\partial y} +
		  \mathbf{\hat{z}}\,\frac{\partial f}{\partial z}
\end{equation}  
For a function $f$ which depends implicitly on $(x,y,z)$ and explicitly on $(\rho,\varphi,z)$,
  we can use the relations established to express
  the gradient of $f$ (called here \verb|fgradient|) in terms of the cartesian unit vectors and in
  terms of derivatives with respect to $(\rho,\varphi,z)$.

\smallskip
Again returning to cylinder.mac:  

\small
\begin{verbatim}
(%i55)              cartesian def. of gradient of scalar f 
(%i56)     fgradient : diff(f, z) zu + diff(f, y) yu + diff(f, x) xu
\end{verbatim}
\normalsize


The $\rho$ component of a vector $\mathbf{A}$ (at the point $(\rho,\varphi,z)$) is given
  by 
\begin{equation}
 A_{\rho} = \boldsymbol{\hat{\rho}} \boldsymbol{\cdot} \mathbf{A},
\end{equation}  
and the $\varphi$ component of a vector $\mathbf{A}$ 
  at the point $(\rho,\varphi,z)$ is given by

\begin{equation}
 A_{\varphi} = \boldsymbol{\hat{\varphi}} \, \boldsymbol{\cdot} \, \mathbf{A}.
\end{equation} 
Our batch file follows this pattern to get the $(\rho,\varphi,z)$
  components of the vector $\boldsymbol{\nabla}f$:

\small
\begin{verbatim}
(%i57)               rho, phi, and z components of grad(f) 
(%i58)          fgradient_rh : (rhu . fgradient, trigsimp(%%))
                                      df
(%o58)                                ---
                                      drh
\end{verbatim}
\newpage
\begin{verbatim}
(%i59)           fgradient_p : (pu . fgradient, trigsimp(%%))
                                      df
                                      --
                                      dp
(%o59)                                --
                                      rh
(%i60)           fgradient_z : (zu . fgradient, trigsimp(%%))
                                      df
(%o60)                                --
                                      dz
\end{verbatim}
\normalsize

Hence we have derived
\begin{equation}
\boldsymbol{\nabla} \, f(\rho,\varphi,z) = 
  \boldsymbol{\hat{\rho}}\, \frac{\partial f}{\partial \rho} +
   \boldsymbol{\hat{\varphi}} \, \frac{1}{\rho} \, \frac{\partial f}{\partial \varphi} +
   \mathbf{\hat{z}}\, \frac{\partial f}{\partial z}.
\end{equation}


\subsubsection{Divergence \quad $\boldsymbol{\nabla} \boldsymbol{\cdot} \mathbf{B}(\rho,\varphi,z)$  }
In cartesian coordinates the divergence of a three dimensional vector field 
  $\mathbf{B}(x,y,z)$ can be calculated with the equation
\begin{equation}
\boldsymbol{\nabla} \boldsymbol{\cdot} \mathbf{B}(x,y,z) = \frac{\partial B_{x}}{\partial x} +
                        \frac{\partial B_{y}}{\partial y} +
						\frac{\partial B_{z}}{\partial z}
\end{equation}  
Consider a vector field $\mathbf{B}(\rho,\varphi,z)$ which is an
  implicit function of $(x,y,z)$ and an explicit function of the 
  cylindrical coordinates $\rho,\varphi,z)$.
We will use the symbol \verb|bvec| to represent $\mathbf{B}(\rho,\varphi,z)$, and use the
  symbols \verb|bx|, \verb|by|, and \verb|bz| to represent the $x$, $y$, and $z$ components of
  $\mathbf{B}(\rho,\varphi,z)$.

\smallskip
The $\rho$ component (the component in the direction of increasing $\rho$ with 
  constant $\varphi$ and constant $z$)
 of $\mathbf{B}(\rho,\varphi,z)$ at the point $(\rho,\varphi,z)$ is given by
  
\begin{equation}
 B_{\rho}(\rho,\varphi,z) = \boldsymbol{\hat{\rho}} \boldsymbol{\cdot} \mathbf{B}(\rho,\varphi,z).
\end{equation} 
We use the symbol \verb|brh| for $B_{\rho}(\rho,\varphi,z)$ .
The component of $\mathbf{B}(\rho,\varphi,z)$ at the point $(\rho,\varphi,z)$ in the
  direction of increasing $\varphi$ (with constant $\rho$ and constant $z$) is given by the equation
  
% something bad here  
\begin{equation}
B_{\varphi}(\rho,\varphi,z) = \boldsymbol{\hat{\varphi}} \boldsymbol{\cdot} \mathbf{B}(\rho,\varphi,z).
\end{equation}
We use the symbol \verb|bp| for $B_{\varphi}(\rho,\varphi,z)$.    
Returning to the cylinder.mac batch file:  

\small
\begin{verbatim}
(%i61)                        divergence of bvec 
(%i62)                   bvec : bz zu + by yu + bx xu
(%o62)                           [bx, by, bz]
(%i63)         two equations which relate cylindrical components
(%i64)                 of bvec to the cartesian components 
(%i65)                      eq1 : brh = rhu . bvec
(%o65)                    brh = by sin(p) + bx cos(p)
(%i66)                       eq2 : bp = pu . bvec
(%o66)                    bp = by cos(p) - bx sin(p)
(%i67)                      invert these equations 
(%i68)       sol : (linsolve([eq1, eq2], [bx, by]), trigsimp(%%))
(%o68)    [bx = brh cos(p) - bp sin(p), by = brh sin(p) + bp cos(p)]
(%i69)                     [bx, by] : map('rhs, sol)
(%i70)          tell Maxima to treat cylindrical components as 
(%i71)                   explicit functions of (rh,p,z) 
\end{verbatim}
\newpage
\begin{verbatim}
(%i72)                depends([brh, bp, bz], [rh, p, z])
(%o72)            [brh(rh, p, z), bp(rh, p, z), bz(rh, p, z)]
(%i73)                 calculate the divergence of bvec 
(%i74) bdivergence : (diff(bz, z) + diff(by, y) + diff(bx, x), trigsimp(%%), 
                                                                  multthru(%%))
                                  dbp
                                  ---
                            brh   dp    dbz   dbrh
(%o74)                      --- + --- + --- + ----
                            rh    rh    dz    drh
\end{verbatim}
\normalsize

Hence we have derived the result:
\begin{equation}
\boldsymbol{\nabla} \boldsymbol{\cdot} \mathbf{B}(\rho,\varphi,z) = 
 \frac{1}{\rho}\,\frac{\partial}{\partial \,\rho} \,\left( \rho \, B_{\rho} \right) +
       \frac{1}{\rho} \, \frac{\partial}{\partial \, \varphi} \, B_{\varphi} +
	   \frac{\partial \, B_{z} }{\partial \, z},
\end{equation}  
in which we have used the identity
\begin{equation}
\frac{1}{\rho}\,\frac{\partial}{\partial \,\rho} \,\left( \rho \, B_{\rho} \right) = 
  \frac{B_{\rho}}{\rho} + \frac{\partial}{\partial \, \rho} \, B_{\rho} 
\end{equation}

\subsubsection{Curl \quad $\boldsymbol{\nabla} \boldsymbol{\times}  \mathbf{B}(\rho,\varphi,z)$  }
In cartesian coordinates the curl of a three dimensional vector field 
  $\mathbf{B}(x,y,z)$ can be calculated with the equation
\begin{equation}
\boldsymbol{\nabla} \boldsymbol{\times} \mathbf{B}(x,y,z) =
 \mathbf{\hat{x}}\,\left( \frac{\partial B_z}{\partial y} - \frac{\partial B_y}{\partial z} \right) +
 \mathbf{\hat{y}}\,\left( \frac{\partial B_x}{\partial z} - \frac{\partial B_z}{\partial x} \right) +
 \mathbf{\hat{z}}\,\left( \frac{\partial B_y}{\partial x} - \frac{\partial B_x}{\partial y} \right)
\end{equation}  
Remember that we have already bound the symbols \verb|bx| and \verb|by| to
  linear combinations of \verb|brh| and \verb|bp|, and have told Maxima (using \textbf{depends})
  to treat \verb|brh| and \verb|bp| as explicit functions of \verb|(rh,p,z)|.
Hence our assignment of the symbol \verb|bcurl| in cylinder.mac will result in Maxima
  using the calculus chain rule.
We can then extract the cylindrical components of $\mathbf{B}$ by taking the
  dot product of the cylindrical unit vectors with $\mathbf{B}$, just as we did
  to get the cylindrical components of $\boldsymbol{\nabla}f$.
\smallskip
  
Returning to cylinder.mac:

\small
\begin{verbatim}
(%i75)                cartesian definition of curl(vec)   
(%i76) bcurl : (diff(by, x) - diff(bx, y)) zu + (diff(bx, z) - diff(bz, x)) yu
                                               + (diff(bz, y) - diff(by, z)) xu
(%i77)             find cylindrical components of curl(bvec) 
(%i78)       bcurl_rh : (rhu . bcurl, trigsimp(%%), multthru(%%))
                                   dbz
                                   ---
                                   dp    dbp
(%o78)                             --- - ---
                                   rh    dz
(%i79)        bcurl_p : (pu . bcurl, trigsimp(%%), multthru(%%))
                                  dbrh   dbz
(%o79)                            ---- - ---
                                   dz    drh
(%i80)        bcurl_z : (zu . bcurl, trigsimp(%%), multthru(%%))
                                 dbrh
                                 ----
                                  dp    bp   dbp
(%o80)                         - ---- + -- + ---
                                  rh    rh   drh
\end{verbatim}
\normalsize
Hence we have derived
\begin{equation}
\boldsymbol{\nabla} \boldsymbol{\times} \mathbf{B}(\rho,\varphi,z) =
  \boldsymbol{\hat{\rho}}\, \left( \frac{1}{\rho}\, \frac{\partial \, B_{z}}{\partial \,\varphi} -
     \frac{\partial \, B_{\varphi}}{\partial \, z} \right) +
   \boldsymbol{\hat{\varphi}} \, \left( \frac{\partial \, B_{\rho}}{\partial \,z} -
       \frac{\partial \, B_{z}}{\partial \, \rho} \right)    +
   \mathbf{\hat{z}}\, \left( \frac{1}{\rho}\, \frac{\partial }{\partial \,\rho} \,(\rho \, B_{\varphi} ) -
        \frac{1}{\rho}\, \frac{\partial \, B_{\rho}}{\partial \, \varphi} \right)  
\end{equation}
in which we have used 
\begin{equation}  \label{Eq:cycurl}
\frac{B_{\varphi}}{\rho} + \frac{\partial \, B_{\phi}}{\partial \, \rho} = 
        \frac{1}{\rho}\, \frac{\partial }{\partial \,\rho} \,(\rho \, B_{\varphi} ).
\end{equation}

\subsection{Maxima Derivation of Vector Calculus Formulas in Spherical Polar Coordinates   }
The batch file sphere.mac, available on the author's webpage with this chapter, uses Maxima
  to apply the same method we used in the previous section to derive spherical polar
  coordinate expressions for the gradient, divergence, curl and Laplacian.
We will include less commentary in this section since the method is the same.


We consider a change of variable
  from cartesian coordinates $(x,y,z)$ to spherical polar
  coordinates $(r,\theta,\varphi )$.  
Given $(r,\theta,\varphi)$, we obtain $(x,y,z)$ from the  equations
 $x =  r \, \sin(\theta) \, \cos(\varphi)$,  $y = r \, \sin(\theta) \, \sin(\varphi)$,
   and $z = r \, \cos(\theta)$.
Given $(x,y,z)$, we obtain $(r,\theta,\varphi)$ from the equations
  $r = \sqrt{x^2 + y^2 + z^2}$,  $\cos(\theta) = z/\sqrt{x^2+y^2+z^2}$,
  and $\tan(\varphi) = y/x$.
  
\smallskip
In sphere.mac we use \verb|t| to represent $\theta$ and \verb|p| to represent
 the angle $\varphi$  expressed in radians.
The ranges of the independent variables are $ 0 < r < \infty$,
  $0 < \theta <  \pi$, and $0 \leq \varphi < 2\,\pi$.
  
We first define \verb|cs3rule| as a list of replacement "rules"
  in the form of equations which can be used later by the \textbf{subst} function.
To get automatic simplification of \verb|r/abs(r)| and \verb|sin(t)/abs(sin(t)|we
 use the \textbf{assume} function.  

\smallskip
Here is the top part of sphere.mac:
\small
\begin{verbatim}
(%i1) batch(sphere);
batching #pc:/work2/sphere.mac
(%i2)        spherical polar coordinates (r,theta,phi ) = (r,t,p)  
(%i3)                  replacement rules x,y,z to r,t,p  
(%i4)  s3rule : [x = r sin(t) cos(p), y = r sin(t) sin(p), z = r cos(t)]
(%i5)         s3sub(expr) := (subst(s3rule, expr), trigsimp(%%))
(%i6)                      assume(r > 0, sin(t) > 0)
                                        2    2    2
(%i7)                      rxyz : sqrt(z  + y  + x )
(%i8)      partial derivatives of r, theta, and phi wrt x, y, and z 
(%i9)                  drdx : (diff(rxyz, x), s3sub(%%))
(%o9)                            cos(p) sin(t)
(%i10)                 drdy : (diff(rxyz, y), s3sub(%%))
(%o10)                           sin(p) sin(t)
(%i11)                 drdz : (diff(rxyz, z), s3sub(%%))
(%o11)                              cos(t)
                                       z
(%i12)              dtdx : (diff(acos(----), x), s3sub(%%))
                                      rxyz
                                 cos(p) cos(t)
(%o12)                           -------------
                                       r
\end{verbatim}
\newpage
\begin{verbatim}
                                       z
(%i13)              dtdy : (diff(acos(----), y), s3sub(%%))
                                      rxyz
                                 sin(p) cos(t)
(%o13)                           -------------
                                       r
                                       z
(%i14)              dtdz : (diff(acos(----), z), s3sub(%%))
                                      rxyz
                                     sin(t)
(%o14)                             - ------
                                       r
                                       y
(%i15)               dpdx : (diff(atan(-), x), s3sub(%%))
                                       x
                                     sin(p)
(%o15)                            - --------
                                    r sin(t)
                                       y
(%i16)               dpdy : (diff(atan(-), y), s3sub(%%))
                                       x
                                    cos(p)
(%o16)                             --------
                                   r sin(t)
(%i17)   (gradef(r, x, drdx), gradef(r, y, drdy), gradef(r, z, drdz))
(%i18)   (gradef(t, x, dtdx), gradef(t, y, dtdy), gradef(t, z, dtdz))
(%i19)             (gradef(p, x, dpdx), gradef(p, y, dpdy))
\end{verbatim}
\normalsize
The batch file next calculates the Laplacian of a scalar function $f$.
\small
\begin{verbatim}
(%i23)      calculate the Laplacian of the scalar function f(r,t,p) 
(%i24) (diff(f, z, 2) + diff(f, y, 2) + diff(f, x, 2), trigsimp(%%), 
                                                        scanmap('multthru, %%))
                                   2                2
                                  d f              d f
                   df             ---         df   ---
                   -- cos(t)        2       2 --     2    2
                   dt             dp          dr   dt    d f
(%o24)             --------- + ---------- + ---- + --- + ---
                    2           2    2       r      2      2
                   r  sin(t)   r  sin (t)          r     dr
(%i25)                             grind(%)
'diff(f,t,1)*cos(t)/(r^2*sin(t))+'diff(f,p,2)/(r^2*sin(t)^2)+2*'diff(f,r,1)/r
                                +'diff(f,t,2)/r^2+'diff(f,r,2)$
\end{verbatim}
\normalsize
Hence the form of the Laplacian of a scalar field in spherical polar coordinates:
\begin{equation}
\boldsymbol{\nabla}^{2} \, f(r,\theta,\varphi ) =
 \frac{1}{r^{2}} \frac{\partial }{\partial \, r} \left( r^{2} \, \frac{\partial f}{\partial r} \right) +
 \frac{1}{r^{2} \, \sin \theta} \, \frac{\partial}{\partial \, \theta} 
       \left( \sin \theta \, \frac{\partial f}{\partial \theta} \right) +
 \frac{1}{r^{2} \, \sin^{2} \theta} \frac{\partial^{2} \, f}{\partial \, \varphi^{2}}
\end{equation} 
in which we have used the identities
\begin{equation}
\frac{\partial^{2} f}{\partial \, r^{2}} + \frac{2}{r} \, \frac{\partial f}{\partial \, r} =
\frac{1}{r^{2}} \frac{\partial }{\partial \, r} \left( r^{2} \, \frac{\partial f}{\partial \, r} \right)
\end{equation}
and
\begin{equation}
\frac{\partial^{2} f}{\partial \, \theta^{2}}
 + \frac{\cos \theta}{\sin \theta} \frac{\partial f}{\partial \, \theta} =
 \frac{1}{\sin \theta} \frac{\partial}{\partial \, \theta} \left(\sin \theta \, \frac{\partial f}{\partial \, \theta} \right)
\end{equation}
We need to avoid $r = 0$ and $\sin \theta = 0$.
The latter condition means avoiding $\theta = 0$ and $\theta = \pi$.
\subsubsection*{Gradient of a Scalar Field}

\begin{figure} [h]
   \centerline{\includegraphics[scale=1]{sphere2.eps} }
	\caption{ $r$ and $\theta$ unit vectors   }
\end{figure}   

\smallskip
The batch file sphere.mac introduces the cartesian and spherical polar unit vectors
  corresponding to the coordinates $(r,\theta,\varphi)$.
The unit vector $\boldsymbol{\hat{\varphi}}$ is the same as in cylindrical coordinates:
\begin{equation}  
  \boldsymbol{\hat{\varphi}} = -\mathbf{\hat{x}} \,\sin \varphi +   \mathbf{\hat{y}} \,\cos \varphi
\end{equation}  
We can express $\mathbf{\hat{r}}$ and $\boldsymbol{\hat{\theta}}$ in terms of the
  cylindrical $\boldsymbol{\hat{\rho}}$ and $\mathbf{\hat{z}}$ (from the figure above):
\begin{equation}
 \mathbf{\hat{r}} = \mathbf{\hat{z}} \, \cos \theta +  \boldsymbol{\hat{\rho}} \, \sin \theta
\end{equation}  
and
\begin{equation}
\boldsymbol{\hat{\theta}} = - \mathbf{\hat{z}} \, \sin \theta +  \boldsymbol{\hat{\rho}} \, \cos \theta.
\end{equation}
Hence we have
\begin{equation}  
  \mathbf{\hat{r}} = \mathbf{\hat{x}}\, \sin \theta \, \cos \varphi + 
  \mathbf{\hat{y}}\, \sin \theta \sin \varphi + \mathbf{\hat{z}} \, \cos \theta
\end{equation}
and
\begin{equation}  
  \boldsymbol{\hat{\theta}} = \mathbf{\hat{x}} \, \cos \theta \cos \varphi +
   \mathbf{\hat{y}} \, \cos \theta \,\sin \varphi - \mathbf{\hat{z}} \, \sin \theta
\end{equation}  


Maxima has already been told to treat $f$ as an explicit function of \verb|(r,t,p)| which we are
  using for $(r,\theta,\varphi)$.
\newpage
The vector components of $\boldsymbol{\nabla} f$ are isolated by using the dot product of
  the unit vectors with $\boldsymbol{\nabla} f$.
For example,
\begin{equation}
 \left( \boldsymbol{\nabla} f \right)_{r} = \mathbf{\hat{r}} \boldsymbol{\cdot} \boldsymbol{\nabla} f
\end{equation}  

\small
\begin{verbatim}
(%i26)         prepare to find derivatives which involve vectors 
(%i27)                      cartesian unit vectors 
(%i28)         (xu : [1, 0, 0], yu : [0, 1, 0], zu : [0, 0, 1])
(%i29)             spherical polar coordinate unit vectors  
(%i30)       ru : cos(t) zu + sin(t) sin(p) yu + sin(t) cos(p) xu
(%o30)              [cos(p) sin(t), sin(p) sin(t), cos(t)]
(%i31)      tu : - sin(t) zu + cos(t) sin(p) yu + cos(t) cos(p) xu
(%o31)             [cos(p) cos(t), sin(p) cos(t), - sin(t)]
(%i32)                    pu : cos(p) yu - sin(p) xu
(%o32)                       [- sin(p), cos(p), 0]
(%i33)              cartesian def. of gradient of scalar f 
(%i34)     fgradient : diff(f, z) zu + diff(f, y) yu + diff(f, x) xu
(%i35)              r, theta, and phi components of grad(f) 
(%i36)           fgradient_r : (ru . fgradient, trigsimp(%%))
                                      df
(%o36)                                --
                                      dr
(%i37)           fgradient_t : (tu . fgradient, trigsimp(%%))
                                      df
                                      --
                                      dt
(%o37)                                --
                                      r
(%i38)           fgradient_p : (pu . fgradient, trigsimp(%%))
                                      df
                                      --
                                      dp
(%o38)                             --------
                                   r sin(t)
\end{verbatim}
\normalsize
Thus we have the gradient of a scalar field in spherical polar coordinates:
\begin{equation}
\boldsymbol{\nabla} \, f(r,\theta,\varphi ) = 
  \mathbf{\hat{r}}\, \frac{\partial f}{\partial r} +
   \boldsymbol{\hat{\theta}} \, \frac{1}{r} \, \frac{\partial f}{\partial \theta} +
   \boldsymbol{\hat{\varphi}} \, \frac{1}{r \,\sin\theta} \, \frac{\partial f}{\partial \varphi}   
\end{equation}

\subsubsection*{Divergence of a Vector Field}
The path here is the same as in the cylindrical case.
For example, we define the spherical polar components of the vector $\mathbf{B}$ via
\begin{equation}
B_{r} = \mathbf{\hat{r}} \boldsymbol{\cdot} \mathbf{B}, B_{\theta} = 
  \boldsymbol{\hat{\theta}} \boldsymbol{\cdot} \mathbf{B},  B_{\varphi} = 
  \boldsymbol{\hat{\varphi}} \boldsymbol{\cdot} \mathbf{B}.
\end{equation}
\small
\begin{verbatim}
(%i39)                        divergence of bvec 
(%i40)                   bvec : bz zu + by yu + bx xu
(%o40)                           [bx, by, bz]
(%i41)      three equations which relate spherical polar components
(%i42)                 of bvec to the cartesian components 
(%i43)                       eq1 : br = ru . bvec
(%o43)       br = by sin(p) sin(t) + bx cos(p) sin(t) + bz cos(t)
\end{verbatim}
\newpage
\begin{verbatim}
(%i44)                       eq2 : bt = tu . bvec
(%o44)      bt = - bz sin(t) + by sin(p) cos(t) + bx cos(p) cos(t)
(%i45)                       eq3 : bp = pu . bvec
(%o45)                    bp = by cos(p) - bx sin(p)
(%i46)                      invert these equations 
(%i47)   sol : (linsolve([eq1, eq2, eq3], [bx, by, bz]), trigsimp(%%))
(%o47) [bx = br cos(p) sin(t) + bt cos(p) cos(t) - bp sin(p), 
by = br sin(p) sin(t) + bt sin(p) cos(t) + bp cos(p), 
bz = br cos(t) - bt sin(t)]
(%i48)                   [bx, by, bz] : map('rhs, sol)
(%i49)        tell Maxima to treat spherical polar components as 
(%i50)                    explicit functions of (r,t,p) 
(%i51)                 depends([br, bt, bp], [r, t, p])
(%o51)              [br(r, t, p), bt(r, t, p), bp(r, t, p)]
(%i52)                 calculate the divergence of bvec 
(%i53) bdivergence : (diff(bz, z) + diff(by, y) + diff(bx, x), trigsimp(%%), 
                                                        scanmap('multthru, %%))
                                  dbp      dbt
                                  ---      ---
                    bt cos(t)     dp       dt    2 br   dbr
(%o53)              --------- + -------- + --- + ---- + ---
                    r sin(t)    r sin(t)    r     r     dr
\end{verbatim}
\normalsize
Hence we have the spherical polar coordinate version of the divergence of a vector field:
\begin{equation}
\boldsymbol{\nabla} \boldsymbol{\cdot} \mathbf{B}(r,\theta,\varphi) = 
 \frac{1}{r^2}\,\frac{\partial}{\partial \, r} \,\left( r^2 \, B_{r} \right) +
       \frac{1}{r \, \sin \theta} \, \frac{\partial}{\partial \, \theta} \,\left(\sin \theta \, B_{\theta} \right) +
	   \frac{1}{r \, \sin \theta} \, \frac{\partial \, B_{\varphi} }{\partial \, \varphi}
\end{equation}  
using the identities
\begin{equation}
\frac{2}{r}\,B_{r} + \frac{\partial \, B_{r}}{\partial \, r} = 
 \frac{1}{r^2}\,\frac{\partial}{\partial \,r} \left(r^2\,B_{r} \right)
\end{equation}
and
\begin{equation} \label{Eq:spdiv}
\frac{\cos \theta}{\sin \theta} \,B_{\theta} + \frac{\partial \,B_{\theta}}{\partial \,\theta} = 
\frac{1}{\sin \theta} \,\frac{\partial}{\partial \, \theta} \left(\sin \theta \,B_{\theta} \right)
\end{equation}
\subsubsection*{Curl of a Vector Field}
Again the path is essentially the same as in the cylindrical case:
\small
\begin{verbatim}
(%i54)                  cartesian curl(vec) definition  
(%i55) bcurl : (diff(by, x) - diff(bx, y)) zu + (diff(bx, z) - diff(bz, x)) yu
                                               + (diff(bz, y) - diff(by, z)) xu
(%i56)           find spherical polar components of curl(bvec) 
(%i57)   bcurl_r : (ru . bcurl, trigsimp(%%), scanmap('multthru, %%))
                                        dbt      dbp
                                        ---      ---
                          bp cos(t)     dp       dt
(%o57)                    --------- - -------- + ---
                          r sin(t)    r sin(t)    r
\end{verbatim}
\newpage
\begin{verbatim}
(%i58)   bcurl_t : (tu . bcurl, trigsimp(%%), scanmap('multthru, %%))
                                dbr
                                ---
                                dp       bp   dbp
(%o58)                        -------- - -- - ---
                              r sin(t)   r    dr
(%i59)   bcurl_p : (pu . bcurl, trigsimp(%%), scanmap('multthru, %%))
                                     dbr
                                     ---
                                bt   dt    dbt
(%o59)                          -- - --- + ---
                                r     r    dr
(%o60)                        c:/work2/sphere.mac
\end{verbatim}
\normalsize
which produces the spherical polar coordinate version of the curl of a vector field:
\begin{equation}
\left( \boldsymbol{\nabla} \boldsymbol{\times}  \mathbf{B} \right)_{r} =
  \frac{1}{r \, \sin \theta} \left[ \frac{\partial }{\partial \theta} \left( \sin \theta\,B_{\varphi} \right)-
    \frac{\partial \,B_{\theta}}{\partial \,\varphi} \right]
\end{equation}
\begin{equation}
\left( \boldsymbol{\nabla} \boldsymbol{\times}  \mathbf{B} \right)_{\theta} =
  \frac{1}{r} \left[ \frac{1}{\sin \theta} \,\frac{\partial \,B_{r} }{\partial \varphi}   -
    \frac{\partial }{\partial \, r}  \, \left( r\, B_{\varphi} \right) \right]
\end{equation}
\begin{equation}
\left( \boldsymbol{\nabla} \boldsymbol{\times}  \mathbf{B} \right)_{\varphi} =
  \frac{1}{r} \left[ \frac{\partial }{\partial r} \, \left(r\, B_{\theta} \right)   -
    \frac{\partial \, B_{r} }{\partial \, \theta}   \right]
\end{equation}
in which we have used Eq.(\ref{Eq:spdiv}) with $B_{\theta}$ replaced by $B_{\varphi}$
and also Eq.(\ref{Eq:cycurl}) with $\rho$ replaced by $r$.

\end{document}