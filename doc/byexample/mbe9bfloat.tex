% e. woollett
%   july 08 - april 09
% updated 2-3-11, sect. on bfallroots
% file bfloat.tex this is  ch. 9
% ch. 8 nint.tex is just the quadpack romberg stuff nint.tex
% and ch. 9 bigfloats and arb prec integration
% then make ch 10 fourier integrals and laplace transforms
% and ch. 11 fast fourier transforms
% check with new version of maxima
% after cleaning up line numbers, check
% sec 9.2.1 "we also see from output %o8 ...
% 1. bfloat erf(z) in changevar example 4
% 2. ex. 3 of tanh-sinh meth bfloat of gamma line 1500
% things to add: 
% new version barton 8-23 mailing list as well as below
%Also, there are functions float_approx_equal and bfloat_approx_equal.
%The default for float_approx_equal_tolerance is about 8 * unit roundoff.
%
%(%i1) float_approx_equal_tolerance;
%(%o1) 8.881784197001254E-16
%
%(%i2) float_approx_equal(float(2/3), 0.66666666666666666666666666666);
%(%o2) true
%
%(%i4) float_approx_equal(float(2/3), 0.66666666666);
%(%o4) false
%
%Source code comment:
%
%;; Return true if $float converts both a and b to floats and
%;; |a - b| <= float_approx_equal_tolerance * min(|a|, |b|).
%;; In all other cases, return false.
%
%
%Barton
% bffac (expr, n) is Bigfloat version of the factorial (shifted gamma) 
% function. The second argument is how many digits to 
% retain and return, it's a good idea to request a couple of extra. 
% bfallroots is bigfloat version of allroots
% bfallroots works just like allroots, except it
% produces bigfloats for the roots.
% rjf ml 6-21-09
%Figuring out stuff like this often requires that you look at the numbers 
%in the radix representation in which the calculations
%are done.
%fpprec is a number which is converted to an (over) estimate of the 
%number of BINARY bits required in the fraction part of a bfloat.
%thus  fpprec:32   (decimal places)   is really 109  bits.    (try 
%this:   fpprec:32;    ?fpprec;)
%the maxima variable $fpprec is what is set to 32.   the lisp variable 
%fpprec is set to 109.
%The rounding for bfloats is done according to the round-to-nearest rule 
%for IEEE floats. Maybe that is the answer you want.
%This rounding probably doubles or even quadruples the time taken for the 
%basic arithmetic operation.
%RJF
% from barton
%As an alternative to abs((((4.0/3.0 rounded) - 1.0)*3.0 - 1.0), you
%can use the functions in floatproperties:
% (%i44) load("floatproperties")$
% (%i48) bigfloat_eps(), fpprec : 42;
% (%o48) 1.79366203433576585078237386661109264803874b-43 
%Also
%(%i97) fpprec : 17;
%(%o97) 17
%(%i98) ?fpprec;
%(%o98) 59
%I'm not positive, but I think bigfloat_eps() returns the least
%positive number such that 1 + x # 1:
%(%i102) rationalize(1 + bigfloat_eps());
%(%o102) 288230376151711745/288230376151711744
%(%i103) rationalize(1 + bigfloat_eps() * (9/10));
%(%o103) 1
%The number of bits in the fractional part of a big float is ?fpprec - 1;
%and bigfloat_eps() is a tad larger than 1/2^?fpprec:
%(%i109) is(rationalize(0.5b0^59 < bigfloat_eps()));
%(%o109) true
%For round to even, bigfloat_eps() might be a tad larger than needed
%for y fop z = (y op z)/(1-a), |a| < bigfloat_eps(). But I've never
%been certain about that.
%If you are trying to prove theorems using float point numbers, such
%things matter, but for most work, I think the different definitions of
%the machine epsilon (or unit roundoff) don't matter all that much.  I
%have a hunk of code that uses the bounds for the relative difference
%between y fop z and y op z 100s or 1000s times. After a few 100 over
%estimations, the bounds are all way too big (well almost always)
%anyway.
%Barton
% edit with Notepad++, then load into LED for latexing
\documentclass[12pt]{article}
\usepackage[dvips,top=1.5cm,left=1.5cm,right=1.5cm,foot=1cm,bottom=1.5cm]{geometry}
\usepackage{times,amsmath,amsbsy,graphicx,fancyvrb,url}
\usepackage[usenames]{color}
%\definecolor{MyDarkBlue}{rgb}{0,0.08,0.45}
\definecolor{mdb}{rgb}{0.1,0,0.55}
\newcommand{\tcdb}{\textcolor{mdb}}
\newcommand{\tcbr}{\textcolor{BrickRed}}
\newcommand{\tcb}{\textcolor{blue}}
\newcommand{\tcr}{\textcolor{red}}
\urldef\tedhome\url{ http://www.csulb.edu/~woollett/  }
\urldef\tedmail\url{ woollett@charter.net}
%1.  this is for maxima code: red framed bold, footnotesize 
\DefineVerbatimEnvironment%
   {myVerbatim}%
   {Verbatim}%
   {fontfamily=courier,fontseries=b,fontsize=\footnotesize ,frame=single,rulecolor=\color{BrickRed}}
\DefineVerbatimEnvironment%
   {myVerbatim1}%
   {Verbatim}%
   {fontfamily=courier,fontseries=b,fontsize=\scriptsize ,frame=single,rulecolor=\color{BrickRed}}
%2.  this is for blue framed bold 
\DefineVerbatimEnvironment%
   {myVerbatim2}%
   {Verbatim}%
   {fontfamily=courier,fontseries=b,frame=single,rulecolor=\color{blue}}
\DefineVerbatimEnvironment%
   {myVerbatim2s}%
   {Verbatim}%
   {fontfamily=courier,fontseries=b,fontsize=\small,frame=single,rulecolor=\color{blue}}
\DefineVerbatimEnvironment%
   {myVerbatim2f}%
   {Verbatim}%
   {fontfamily=courier,fontseries=b,fontsize=\footnotesize,frame=single,rulecolor=\color{blue}}
% 3.  this is for black framed  bold
\DefineVerbatimEnvironment%
   {myVerbatim3}%
   {Verbatim}%
   {fontfamily= courier, fontseries=b, frame=single}
% 4.  this is for no frame bold
\DefineVerbatimEnvironment%
   {myVerbatim4}%
   {Verbatim}%
   {fontfamily=courier, fontseries=b}
% 6.  for defaults use usual verbatim
\newcommand{\mv}{\Verb[fontfamily=courier,fontseries=b]}
\newcommand{\mvs}{\Verb[fontfamily=courier,fontseries=b,fontsize=\small]}
\newcommand{\mvf}{\Verb[fontfamily=courier,fontseries=b,fontsize=\footnotesize]}

\renewcommand{\thefootnote}{\ensuremath{\fnsymbol{footnote}}}
%%%%%%%%%%%%%%%%%%%%%%%%%%%%%%%%%%%%%%%%%%%%%%%%%%%%%%%%%%%%%%%%%%%%%%%%
%   title page
%%%%%%%%%%%%%%%%%%%%%%%%%%%%%%%%%%%%%%%%%%%%%%%%%%%%%%%%%%%%%%%%%%%%%%

\title{  Maxima by Example:\\ Ch.9: Bigfloats and Arbitrary Precision Quadrature
            \thanks{This version uses \textbf{Maxima 5.18.1} except for 
			the revised section on bfallroots, which uses \textbf{Maxima 5.23.2}.
			The author would like to thank the Maxima developers for their friendly help via 
            the Maxima mailing list, and Michel Talon for feedback 
			about \textbf{bfallroots} behavior.This is a live
            document. Check \; \textbf{ \tedhome } \; for the latest version of these notes. Send comments and
			 suggestions to \textbf{\tedmail} } }


\author{ Edwin L. Woollett}
\date{\today}
%%%%%%%%%%%%%%%%%%%%%%%%%%%%%%%%%%%%%%%%%
%          document
%%%%%%%%%%%%%%%%%%%%%%%%%%%%%%%%%%%%%%%%%%
\begin{document}
%\footnotesize
%\small
\maketitle
\tableofcontents
\numberwithin{equation}{section}
\newpage
\subsubsection*{Preface}
\begin{myVerbatim2} 
COPYING AND DISTRIBUTION POLICY    
This document is part of a series of notes titled
"Maxima by Example" and is made available
via the author's webpage http://www.csulb.edu/~woollett/
to aid new users of the Maxima computer algebra system.	
	
NON-PROFIT PRINTING AND DISTRIBUTION IS PERMITTED.
	
You may make copies of this document and distribute them
to others as long as you charge no more than the costs of printing.	

These notes (with some modifications) will be published in book form
eventually via Lulu.com in an arrangement which will continue
to allow unlimited free download of the pdf files as well as the option
of ordering a low cost paperbound version of these notes.
\end{myVerbatim2}	
\smallskip
\noindent \tcbr{Feedback from readers is the best way for this series of notes
  to become more helpful to new users of Maxima}.
\tcdb{\emph{All} comments and suggestions for improvements will be appreciated and
  carefully considered}.
\smallskip
\begin{myVerbatim2s}
LOADING FILES
The defaults allow you to use the brief version load(brmbrg) to load in the
Maxima file brmbrg.lisp.
To load in your own file, such as qbromberg.mac (used in this chapter),
using the brief version load(qbromberg), you either need to place 
qbromberg.mac in one of the folders Maxima searches by default, or
else put a line like:
  
file_search_maxima : append(["c:/work3/###.{mac,mc}"],file_search_maxima )$
  
in your personal startup file maxima-init.mac (see Ch. 1, Introduction to Maxima
for more information about this).

Otherwise you need to provide a complete path in double quotes,
as in load("c:/work3/qbromberg.mac"),
 
We always use the brief load version in our examples, which are generated 
using the Xmaxima graphics interface on a Windows XP computer, and copied
into a fancy verbatim environment in a latex file which uses the fancyvrb
and color packages.

We use qdraw.mac for plots (see Ch.5), which uses draw2d defined
in share/draw/draw.lisp.
\end{myVerbatim2s}  
\smallskip
\begin{myVerbatim2s}
  Maxima, a Computer Algebra System.
  Some numerical results depend on the Lisp version used.
  This chapter uses Version 5.18.1 (2009) using Lisp GNU 
  Common Lisp (GCL) GCL 2.6.8 (aka GCL).
  http://maxima.sourceforge.net/
\end{myVerbatim2s}
\newpage
\setcounter{section}{9}
\pagestyle{headings}
\subsection{Introduction}
This chapter is divided into two sections.\\

\noindent In the first section we discuss the use of \mv|bfloat|, including examples which also
  involve \mv|fpprec|,  \mv|bfloatp|, \mv|bfallroots|, \mv|fpprintprec|, \mv|print|,
   and  \mv|printf|.
The second section of Chapter 9 presents examples of the use of Maxima for arbitrary
  precision quadrature (numerical integration).
(In Chapter 8, we gave numerous examples of numerical integration using
  the Quadpack functions such as \mv|quad_qags| as well as \mv|romberg|.
Those examples all accepted the default floating point precision of Maxima).\\

\noindent Chapter 10 covers both Fourier transform and Laplace
  transform type integrals.  
Chapter 11 presents tools for the use of fast Fourier
  transforms with examples of use.\\
  
\noindent Software files developed for Ch. 9 and available on the author's web page
 include:\\
 \mv+ 1. fdf.mac, 2. qbromberg.mac. + ,\\
  \mv+ 3. quad_de.mac, 4. quad_ts.mac, 5. quad_gs.mac.  +  
\subsection{  The Use of Bigfloat Numbers in Maxima}
\subsubsection{Bigfloat Numbers Using \textbf{bfloat}, \textbf{fpprec}, and \textbf{fpprintprec} } \label{bf1}
\textbf{bfloat(expr)} converts all numbers and functions of numbers in \textbf{expr}
   to bigfloat numbers.
You can enter explicit bigfloat numbers using the notation \textbf{2.38b0}, or \textbf{2.38b7},
  or \textbf{2.38b-4}, for example.
\textbf{bfloatp(val)} returns \textbf{true} if \textbf{val} is a bigfloat number,
  otherwise \textbf{false} is returned.\\
  
\noindent The number of significant digits in the resulting bigfloat is specified by
  the parameter \textbf{fpprec}, whose default value is \mv|16|. 
The setting of \textbf{fpprec} does not affect computations on ordinary floating point numbers. \\

\noindent The underlying Lisp code has two variables: 1. \textbf{\$fpprec}, which
  defines the Maxima variable \textbf{fpprec} (which determines the number
  of DECIMAL digits requested for arithmetic), and 2. \textbf{fpprec} which is related to the number
  of bits used for the fractional part of the bigfloat, and which can be accessed from
  Maxima using \textbf{?fpprec}. 
We can also use the \textbf{:lisp foobar} construct to look at a lisp variable \mv|foobar|
  from ``inside'' the Lisp interpreter.
You do not end with a semi-colon; you just press ENTER to get the response,
  and the input line number of the Maxima prompt does not advance. 
\begin{myVerbatim}
(%i1) fpprec;
(%o1)                                 16
(%i2) ?fpprec;
(%o2)                                 56
(%i3) :lisp $fpprec
16
(%i3) :lisp fpprec
56
\end{myVerbatim} 
We discuss some effects of the fact that computer arithmetic is carried out
  with binary bit representation of decimal numbers in Sec. \ref{gaps}.
\newpage
\subsubsection*{\textbf{Controlling Printed Digits with fpprintprec}}
When using bigfloat numbers, the screen can quickly fill up with numbers
  with many digits, and \textbf{fpprintprec} allows you to control
  how many digits are displayed.
For bigfloat numbers, when \textbf{fpprintprec} has a value between \textbf{2} and \textbf{fpprec}
   (inclusive), the number of digits printed is equal to \textbf{fpprintprec}.  
Otherwise, \textbf{fpprintprec} can be \textbf{0}, or greater than \textbf{fpprec},
  in which case the number of digits printed is equal to \textbf{fpprec}. 
\textbf{fpprintprec} cannot be \textbf{1}. 
The setting of \textbf{fpprintprec} does not affect the precision of the
  bigfloat arithmetic carried out, only the setting of \textbf{fpprec} matters.\\
  
\noindent The parameter \textbf{fpprec} can be used as a local variable in a function
  defined using \textbf{block}, and set to a local value which does not affect
  the global setting.
Such a locally defined value of \textbf{fpprec} governs the bigfloat calculations
  in that function and in any functions called by that function, and in any third layer
  functions called by the secondary layer functions, etc.
\textbf{fpprintprec} can be set to a value inside a function defined with \textbf{block},without
  changing the global value, provided you include the name \textbf{fpprintprec} in the
  local variables bracket \textbf{[ \; ]}:
\begin{myVerbatim}
(%i1) [fpprec,fpprintprec];
(%o1)                               [16, 0]
(%i2) piby2 : block([fpprintprec,fpprec:30,val],
                 val:bfloat(%pi/2),
                 fpprintprec:8,
                 disp(val),
                  print("  ",val),
                  val);
                                  1.5707963b0

   1.5707963b0 
(%o2)                  1.57079632679489661923132169164b0
(%i3) [fpprec,fpprintprec];
(%o3)                               [16, 0]
\end{myVerbatim} 
For simple \textbf{bfloat} uses in interactive mode, one can use the
  syntax \textbf{bfloat-job, fpprec:fp ;} which implicitly uses the
   \textbf{ev(...)} construct with temporary settings of global flags as in 
\begin{myVerbatim}
(%i1) bfloat(%pi),fpprec:20;
(%o1)                       3.1415926535897932385b0
(%i2) slength(string(%));
(%o2)                                 23
(%i3) fpprec;
(%o3)                                 16
(%i4) bfloat(%pi),fpprec:40;
(%o4)             3.141592653589793238462643383279502884197b0
(%i5) slength(string(%));
(%o5)                                 43
(%i6) fpprec;
(%o6)                                 16
(%i7) tval : bfloat(integrate(exp(x),x,-1,1)),fpprec:30;
(%o7)                  2.35040238728760291376476370119b0
(%i8) slength(string(%));
(%o8)                                 33
\end{myVerbatim} 
\newpage
\noindent Next we illustrate passing both a bigfloat number as well as local values 
  of \textbf{fpprec} and \textbf{fpprintprec} to a second function. 
Function \mv+f1+ is designed to call function \mv+f2+:
\begin{myVerbatim}
(%i1) f2(w) := block([v2 ],
            disp(" in f2 "),
            display([w,fpprec,fpprintprec]),
             v2 : sin(w),
             display(v2),
             print("  "),
              v2 )$
(%i2) f1(x,fp,fprt) :=
           block([fpprintprec,fpprec:fp,v1],
             fpprintprec:fprt,
             disp(" in f1 "),
              display([x,fpprec,fpprintprec]),
              v1 : f2(bfloat(x))^2,
              print(" in f1, v1 = ",v1),
              v1 )$
\end{myVerbatim}
%\newpage
And here we call \mv+f1+ with values \mv|x = 0.5, fp = 30, fprt = 8|:
\begin{myVerbatim}
(%i3) f1(0.5,30,8);
                                     in f1 
                    [x, fpprec, fpprintprec] = [0.5, 30, 8]
                                     in f2 
                  [w, fpprec, fpprintprec] = [5.0b-1, 30, 8]
                               v2 = 4.7942553b-1   
 in f1, v1 =  2.2984884b-1 
(%o3)                2.29848847065930141299531696279b-1
(%i4) [fpprec,fpprintprec];
(%o4)                              [16, 0]
\end{myVerbatim} 
We see that the called function (\mv+f2+) \tcbr{maintains} the values of
  \mv+fpprec+ and \mv+fpprintprec+ which exist in the calling function (\mv+f1+).\\
  
\noindent Bigfloat numbers are ``contagious'' in the sense that, for example, multiplying
  (or adding) an integer or ordinary float number with a bigfloat results in a bigfloat
  number.
In the above example \textbf{sin(w)} is a bigfloat since \textbf{w} is one. 
\begin{myVerbatim}
(%i5) 1 + 2.0b0;
(%o5)                                3.0b0
(%i6) 1.0 + 2.0b0;
(%o6)                                3.0b0
(%i7) sin(0.5b0);
(%o7)                        4.79425538604203b-1
\end{myVerbatim} 
The Maxima symbol \textbf{\%pi} is not automatically converted by contagion (in
  the present version of Maxima), and an extra use of \textbf{bfloat} does the conversion. 
\begin{myVerbatim}
(%i8) a:2.38b-1$
(%i9) exp(%pi*a);
                                   2.38b-1 %pi
(%o9)                           %e
(%i10) bfloat(%);
(%o10)                        2.112134508503361b0
\end{myVerbatim} 
A recent comment by Maxima developer Volker van Nek on the Maxima mailing list
  illustrates how the setting of \textbf{fpprec} can affect  what is printed
  to the screen when doing bigfloat arithmetic. 
\begin{myVerbatim}
(%i11) 1b0 + 1b-25,fpprec:26;
(%o11)                   1.0000000000000000000000001b0
(%i12) slength(string(%));
(%o12)                                29
\end{myVerbatim} 
Note that \textbf{1b-25} is the bigfloat version of $\mathbf{10^{-25}}$.
In the case above the value of \textbf{fpprec} is large enough to see the effect of
  the bigfloat addition.
The number of digits identified by the conversion of the number to a string and
  then finding the length of the string is $\mathbf{29 - 3 = 26}$ since the three 
  string characters $\mathbf{[., b, 0]}$ do not contribute
  to the precision of the number.  
%\newpage
\noindent If we leave the \textbf{fpprec} at \textbf{26} and add to \textbf{1}
  the bigfloat equivalent of $\mathbf{10^{-26}}$, Maxima prints out a bigfloat version of \textbf{1}, but
  if we subtract from that answer $\mathbf{1b0}$ we do not get zero: 
\begin{myVerbatim}
(%i13) 1b0 + 1b-25,fpprec:26;
(%o13)                   1.0000000000000000000000001b0
(%i14) slength(string(%));
(%o14)                                29
(%i15) 1b0 +10b0^(-26),fpprec:26;
(%o15)                               1.0b0
(%i16) % - 1b0,fpprec:26;
(%o16)                  9.6935228033557930648993206b-27
(%i17) slength(string(%));
(%o17)                                31
\end{myVerbatim} 
Of the thirty one characters present in this string, the five characters $\mathbf{[., b, -, 2, 7]}$
   do not contribute to the precision, , leaving \textbf{26} digits of precision.
Note that the setting of \textbf{fpprintprec} was \textbf{0} in the above, which provides for
  printing of the same number of significant digits as the value of \textbf{fpprec}.\\
  
\noindent Here are three ways to enter the bigfloat equivalent of $\mathbf{1/10 = 10^{-1} = 0.1}$: 
\begin{myVerbatim}
(%i18) [1b-1,bfloat(10^(-1)),10b0^(-1)];
(%o18)                     [1.0b-1, 1.0b-1, 1.0b-1]
\end{myVerbatim} 
We next show that once bigfloat numbers are in play inside a \textbf{block} function
  in which a local value of \textbf{fpprec} is set, the subsequent arithmetic is the same
  whether or not you wrap each operation with \textbf{bfloat}. 
\begin{myVerbatim}
(%i19) test1(h,j,fp):=
  block([fpprec:fp,u,a,x,w],
   u:bfloat(h*j),
   a:bfloat(exp(-u)),
   x:bfloat(exp(u-a)),
   w:bfloat(exp(-a)+x),
   [x,w])$
(%i20) test2(h,j,fp):=
  block([fpprec:fp,u,a,x,w],
   u:bfloat(h*j),
   a:exp(-u),
   x:exp(u-a),
   w:exp(-a)+x,
   [x,w])$
(%i21) [xt,wt]:test1(1/16,9,60)$
(%i22) [x1,w1]:test1(1/16,9,40)$
(%i23) [x2,w2]:test2(1/16,9,40)$
\end{myVerbatim}
\newpage
\begin{myVerbatim}
(%i24) [x1,w1]-[xt,wt],fpprec:60;
(%o24) [- 2.29699398191727953386796229322249912456243487302509210328785b-42, 
           - 9.07907621847706474924315736438559811433348894389391230442376b-42]
(%i25) [x2,w2]-[xt,wt],fpprec:60;
(%o25) [- 2.29699398191727953386796229322249912456243487302509210328785b-42, 
           - 9.07907621847706474924315736438559811433348894389391230442376b-42]
(%i26) bfloat([x1,w1]-[xt,wt]),fpprec:60;
(%o26) [- 2.29699398191727953386796229322249912456243487302509210328785b-42, 
           - 9.07907621847706474924315736438559811433348894389391230442376b-42]
(%i27) map('bfloatp,%o5);
(%o27)                            [true, true]
(%i28) map('bfloatp,%o4);
(%o28)                           [true, true]
\end{myVerbatim} 
In the above, we used the ``completely \textbf{bfloat} wrapped'' version test1(..) to
  define answers with \textbf{60} digit precision, and then used test1(..) and test2(..) to
  compute comparison answers at \textbf{40} digit precision.
We see that there is no difference in precision between the answers returned by the two test versions
  (each using 40 digit precision).\\

\noindent We also see, from the output \textbf{\%08}, that \mv|arith_job, fpprec:60;| using
  interactive mode produces the same answer (with bigfloats already in play) whether or
  not the ``\mv|arithmetic_job|'' is wrapped in \textbf{bfloat}.
The numbers returned by both test versions are bigfloats, as indicated by the
  last two outputs. 
\subsubsection{Using \textbf{print} and \textbf{printf} with Bigfloats}
In Sec. \ref{bf1} we described the relations between the settings of
  \textbf{fpprec} and \textbf{fpprintprec}.
Once you have generated a bigfloat with some precision, it is convenient to
  be able to control how many digits are displayed.
We start with the use of \mv|print|.
If you start with the global default value of \textbf{16} for \textbf{fpprec} and the
  default value of \textbf{0} for \textbf{fpprintprec}, you can use a simple one line
  command for a low number of digits, as shown in the following. 
We first define a bigfloat \mv+bf1+ to have \mv|fpprec = 45| digits of
  precision:
\begin{myVerbatim}
(%i1) [fpprec,fpprintprec];
(%o1)                               [16, 0]
(%i2) bf1:bfloat(integrate(exp(x),x,-1,1)),fpprec:45;
(%o2)          2.35040238728760291376476370119120163031143596b0
(%i3) slength(string(%));
(%o3)                                 48
\end{myVerbatim}
We then use \mv|print| with \mv|fpprintprec| to get increasing numbers
  of digits on the screen:
\begin{myVerbatim}
(%i4) print(bf1),fpprintprec:12$
2.35040238728b0 
(%i5) [fpprec,fpprintprec];
(%o5)                               [16, 0]
(%i6) print(bf1),fpprintprec:15$
2.3504023872876b0 
(%i7) [fpprec,fpprintprec];
(%o7)                               [16, 0]
(%i8) print(bf1),fpprintprec:16$
2.35040238728760291376476370119120163031143596b0 
(%i9) [fpprec,fpprintprec];
(%o9)                               [16, 0]
(%i10) slength(string(%o8));
(%o10)                                48
\end{myVerbatim} 
As you see above, when \textbf{fpprintprec} reaches the global value of
   \textbf{fpprec = 16} \tcbr{all} \textbf{45} digits are printed.
To control the number of printed digits, you need to locally set
   the value of \textbf{fpprec} as shown here: 
\begin{myVerbatim}
(%i11) print(bf1),fpprec:20,fpprintprec:18$
2.35040238728760291b0 
\end{myVerbatim} 
To use this construct in a \textbf{do} loop, wrap it in \textbf{ev(...)}: 
\begin{myVerbatim}
(%i12) for j:14 thru 18 do ev(print(bf1),fpprec:j+2,fpprintprec:j)$
2.3504023872876b0 
2.3504023872876b0 
2.350402387287602b0 
2.3504023872876029b0 
2.35040238728760291b0 
\end{myVerbatim} 
A more formal approach is to define a small function which we call
  \textbf{bfprint}: 
\begin{myVerbatim}
(%i13) bfprint(bf,fpp):=
  block([fpprec, fpprintprec ],
        fpprec : fpp+2,
        fpprintprec:fpp,
        print("   number of digits = ",fpp),
        print("  ",bf) )$
\end{myVerbatim} 
with the behavior: 
\begin{myVerbatim}
(%i14) bfprint(bf1,24)$
   number of digits =  24 
   2.35040238728760291376476b0 
\end{myVerbatim} 
\subsubsection*{Using \textbf{printf} with bigfloats}
We first show some interactive use of \textbf{printf} with bigfloats. 
\begin{myVerbatim}
(%i1) bf:bfloat(exp(-20)),fpprec:30;
(%o1)                 2.06115362243855782796594038016b-9
(%i2) slength(string(%));
(%o2)                                 34
(%i3) printf(true,"~d~a",3,string(bf))$
32.06115362243855782796594038016b-9
\end{myVerbatim} 
The format string is enclosed in double quotes, with \textbf{\~ \,d} used for an integer,
  \textbf{\~ \,f} used for a floating point number, \textbf{\~ \,a} used for a Maxima string,
   \textbf{\~ \,e} used for exponential display of a floating point number, and
   \textbf{\~ \,h} used for a bigfloat number.
You can include the newline instruction with \textbf{\~ \,\%} anywhere and as many times as you 
  wish.
In the example above, we used the string formatting to display the bigfloat
  number bf, which required that bf be converting to a Maxima string using \textbf{string}.
Because we did not include any spaces between the integer format instruction \textbf{\~ \,d} and
  the string format character \textbf{\~ \,a}, we get \textbf{32.0...} instead of \textbf{3 \;  2.0...}. 
\begin{myVerbatim}
(%i4) printf(true,"  ~d~a",3,string(bf))$
  32.06115362243855782796594038016b-9
(%i5) printf(true,"  ~d    ~a",3,string(bf))$
  3    2.06115362243855782796594038016b-9
\end{myVerbatim}
\newpage
\begin{myVerbatim}
(%i6) (printf(true,"  ~d    ~a",3,string(bf)),
           printf(true,"  ~d    ~a",3,string(bf)))$
  3    2.06115362243855782796594038016b-9  3    2.06115362243855782796594038016b-9
(%i7) (printf(true,"  ~d    ~a~%",3,string(bf)),
           printf(true,"  ~d    ~a",3,string(bf)))$
  3    2.06115362243855782796594038016b-9
  3    2.06115362243855782796594038016b-9
\end{myVerbatim} 
To get the output on successive lines we had to include the newline 
  instruction \textbf{\~ \,\%}.
To control the number of significant figures displayed, we use \textbf{fpprintprec}: 
\begin{myVerbatim}
(%i8) fpprintprec:8$
(%i9) printf(true,"  ~d    ~a",3,string(bf))$
  3    2.0611536b-9
\end{myVerbatim} 
Next let's show what we get if we use the other options: 
\begin{myVerbatim}
(%i10) printf(true,"  ~d    ~f",3,bf)$
  3    0.0000000020611536224385579
(%i11) printf(true,"  ~d    ~e",3,bf)$
  3    2.0611536224385579E-9
(%i12) printf(true,"  ~d    ~h",3,bf)$
  3    0.0000000020611536
\end{myVerbatim} 
\subsubsection*{A Table of Bigfloats using \textbf{block} and \textbf{printf} }
Here is an example of using \textbf{printf} with
  bigfloats inside a \textbf{block} to make a table. 
\begin{myVerbatim}
(%i1) print_test(fp) :=
        block([fpprec,fpprintprec,val],
         fpprec : fp,
         fpprintprec : 8,
         display(fpprec),
         print("  k          value "),
         print("  "),
         for k thru 4 do
          ( val : bfloat(exp(k^2)), 
            printf(true,"  ~d       ~a  ~%",k,string(val) ) ))$
(%i2) print_test(30)$
                                  fpprec = 30

  k          value  
   
  1       2.7182818b0  
  2       5.459815b1  
  3       8.1030839b3  
  4       8.8861105b6  
\end{myVerbatim} 
Note the crucial use of the newline instruction \textbf{\~ \,\%} to get the table output.
Some general use examples of \textbf{printf} can be found in the Maxima manual and
  in the file 
\begin{myVerbatim2}
C:\Program Files\Maxima-5.17.1\share\maxima\5.17.1\share\
                            contrib\stringproc\rtestprintf.mac
\end{myVerbatim2} 
We can use \textbf{printf} for the titles and empty lines with the alternative
  version.
We first define an alternative function \mv|print_test2|:
\begin{myVerbatim}
(%i3) print_test2(fp) :=
        block([fpprec,fpprintprec,val],
         fpprec : fp,
         fpprintprec : 8,
         display(fpprec),
         printf(true,"~%  ~a          ~a ~%~%",k,value), 
         for k thru 4 do
          ( val : bfloat(exp(k^2)), 
            printf(true,"  ~d       ~a  ~%",k,string(val) ) ))$
\end{myVerbatim}
Here we try out the alternative function with \mv|fp = 30|:
\begin{myVerbatim}
(%i4) print_test2(30)$
                                  fpprec = 30

  k          value 

  1       2.7182818b0  
  2       5.459815b1  
  3       8.1030839b3  
  4       8.8861105b6  
\end{myVerbatim} 
\subsubsection{Adding Bigfloats Having Differing Precision}
If \mv|A| and \mv|B| are bigfloats with different precisions, the precision of
  the sum \mv|(A + B)| is  the precision of the least precise number.
As an example, we calculate an approximation to $\mathbf{\boldsymbol{\pi}}$ using both
  30 and 50 digit precision, and add the numbers using 40 digit precision,
  and then using 60 digit precision.
In both cases, the result has 30 digit precision.
 
\begin{myVerbatim}
(%i1) fpprintprec:8$
(%i2) pi50 : bfloat(%pi),fpprec:50;
(%o2)                             3.1415926b0
(%i3) pi30 : bfloat(%pi),fpprec:30;
(%o3)                             3.1415926b0
(%i4) abs(pi30 - pi50),fpprec:60;
(%o4)                            1.6956855b-31
(%i5) twopi : bfloat(2*%pi),fpprec:60;
(%o5)                             6.2831853b0
(%i6) pisum40 : pi30 + pi50,fpprec:40;
(%o6)                             6.2831853b0
(%i7) abs(pisum40 - twopi),fpprec:60;
(%o7)                            1.6956855b-31
(%i8) pisum60 : pi30 + pi50,fpprec:60;
(%o8)                             6.2831853b0
(%i9) abs(pisum60 - twopi),fpprec:60;
(%o9)                            1.6956855b-31
\end{myVerbatim} 
\newpage
\subsubsection{Polynomial Roots Using \textbf{bfallroots}}  
The Maxima function \mv|bfallroots| has the same syntax as \mv|allroots|,
  and computes numerical approximations of the real and complex roots of
  a polynomial or polynomial expression of one variable. 
In all respects, \mv|bfallroots| is identical to \mv|allroots| except 
   that \mv|bfallroots| computes the roots using bigfloats, and to take
   advantage of bigfloats you need to set both ffprec and ratepsilon to
   compatible values (as our example shows).
The source code of \mv|bfallroots| with some comments is in the file \mvs|cpoly.lisp| in 
   the \mv|src| directory.\\
   
\noindent Our example is a cubic equation whose three degenerate roots
  are simply $\boldsymbol{\pi}$.
We are using Maxima 5.23.2 for this revised section, with 
  \mv|display2d:false| set in our init file.
We first compute a 50 digit approximation to the true root.
\begin{myVerbatim}
(%i1) fpprec;
(%o1) 16
(%i2) pi50 : ev (bfloat (%pi), fpprec:50);
(%o2) 3.1415926535897932384626433832795028841971693993751b0
(%i3) slength( string (%));
(%o3) 53
\end{myVerbatim}
We next define the symbolic cubic expression whose roots we
  would like to approximately calculate.
\begin{myVerbatim}
(%i4) e : expand ( (x-%pi)^3);
(%o4) x^3-3*%pi*x^2+3*%pi^2*x-%pi^3
\end{myVerbatim}
As a warm-up, we use the default 16 digit floating point precision
  and find the root(s) using both \mv|allroots| and \mv|bfallroots|.
We first need to turn the symbolic expression into a polynomial whose
  coefficients have the default 16 digit accuracy.
\begin{myVerbatim}
(%i6) e_f16 : float (e);
(%o6) x^3-9.424777960769379*x^2+29.60881320326807*x-31.00627668029982
\end{myVerbatim}
Now find the approximate roots of this numerical polynomial in \mv|x|
  using \mv|allroots|.
\begin{myVerbatim}
(%i7) sar16 : map ('rhs, allroots (%i*e_f16));
(%o7) [3.14159265358979-1.8873791418627661E-15*%i,
       9.9920072216264089E-16*%i+3.141592653589795,
       8.8817841970012523E-16*%i+3.141592653589795]
\end{myVerbatim}
We first check to see how well the approximate solutions behave
  as far as causing the approximate numerical polynomial to be zero (as
  roots should do).
\begin{myVerbatim}
(%i8) for s in sar16 do (subst (s,x,e_f16), disp (expand (%%)))$
6.3108872417680944E-30*%i

-6.3108872417680944E-30*%i

-3.1554436208840472E-30*%i

\end{myVerbatim}
 which is very good root behavior.
\newpage
\noindent We next compare the approximate roots  (taking realpart) to pi50.
\begin{myVerbatim}
(%i9) for s in sar16 do disp (pi50 - realpart(s))$
3.663735981263017b-15

-1.665334536937735b-15

-1.665334536937735b-15
\end{myVerbatim}
The above accuracy in finding $\boldsymbol{\pi}$ corresponds to 
  the default floating point precision being used.\\
  
\noindent Retaining the default precision, we try out
  \mv|bfallroots|.  
\begin{myVerbatim}
(%i10) sbfar16 : map ('rhs, bfallroots (%i*e_f16));
(%o10) [3.141592653589788b0-1.207367539279858b-15*%i,
        5.967448757360216b-16*%i+3.141592653589797b0,
        6.106226635438361b-16*%i+3.141592653589795b0]
\end{myVerbatim}
We then again check the roots against the expression:
\begin{myVerbatim}
(%i11) for s in sbfar16 do (subst (s,x,e_f16), disp (expand (%%)))$
7.888609052210118b-31*%i+2.664535259100376b-15

1.332267629550188b-15

1.332267629550188b-15-3.944304526105059b-31*%i
\end{myVerbatim}
 and compare the accuracy against our ``true value''.
\begin{myVerbatim}
(%i12) for s in sbfar16 do disp (pi50 - realpart(s))$
5.662137425588298b-15

-3.774758283725532b-15

-1.554312234475219b-15
\end{myVerbatim}
Thus we see that \mv|bfallroots| provides no increased accuracy unless
  we set \mv|fpprec| and \mv|ratepsilon| to values which will cause Maxima to
  use higher precision.\\
  
\noindent In order to demonstrate the necessity of setting \mv|ratepsilon|, we
  first try out \mv|bfallroots| using only the \mv|fpprec| setting.
Let's try to solve for the roots with 40 digit accuracy, first converting
  the symbolic cubic to a numerical cubic with coefficients having
  40 digit accuracy.
\begin{myVerbatim}
(%i13) e_f40 : ev (bfloat (e),fpprec : 40);
(%o13) x^3-9.424777960769379715387930149838508652592b0*x^2
          +2.960881320326807585650347299962845340594b1*x
          -3.100627668029982017547631506710139520223b1
\end{myVerbatim}
The coefficients are now big floats, with the tell-tale \mv|b0| or \mv|b1| 
  power of ten factor attached to the end.
\newpage
\noindent Now set \mv|fpprec : 40| and use \mv|bfallroots|:
\begin{myVerbatim}
(%i16) fpprec:40$
(%i17) sbfar40 : map ('rhs, bfallroots (%i*e_f40));
`rat' replaced -3.100627668029982017547631506710139520223B1 
     by -14821/478 = -3.100627615062761506276150627615062761506B1
`rat' replaced 2.960881320326807585650347299962845340594B1 
      by 32925/1112 = 2.960881294964028776978417266187050359712B1
`rat' replaced -9.424777960769379715387930149838508652592B0 
   by -103993/11034 = -9.424777959035707812216784484321189052021B0
(%o17) [5.444584912690149273860860372375096164019b-3*%i
         +3.138436376741899641306089676703354562429b0,
        3.138436376741899641306089676703354687072b0
         -5.444584912690149273860860372375167833945b-3*%i,
        7.166992586513730038213070271942264501126b-35*%i
         +3.14790520555190852960460513091447980252b0]
\end{myVerbatim}
Check the 40 digit expression using these roots
\begin{myVerbatim}
(%i18) for s in sbfar40 do (subst (s,x,e_f40), disp (expand (%%)))$
1.321649848909340099656224265221838153703b-9*%i
 +2.492462307131280208085778687675933071752b-7

2.492462307131280208085778687675940418592b-7
 -1.321649848909340099656224265204773970573b-9*%i

8.567776759672472832399742389749133357356b-39*%i
 +2.515445418347819864275611335161734182841b-7
\end{myVerbatim}
 and check the closeness of the roots to the ``true value'',
\begin{myVerbatim}
(%i19) for s in sbfar40 do disp (pi50 - realpart(s))$
3.156276847893597156553706576148321768144b-3

3.156276847893597156553706576148197124968b-3

-6.312551962115291141961747634976918322456b-3
\end{myVerbatim} 
  which are really poor results, apparently caused by inaccurate 
   \mv|rat| replacement of decimal coefficients by ratios of whole numbers.
Look, for example, at the third rat replacemnt above and its difference 
  from the actual 40 digit accurate number:
\begin{myVerbatim}
(%i20) bfloat (9.424777960769379715387930149838508652592B0 -
                  103993/11034 );
(%o20) 1.733671903171145665517319600570931418138b-9
(%i21) ratepsilon;
(%o21) 2.0E-8
\end{myVerbatim}
\newpage
\noindent So we are driven to the conclusion that, with the present design of
  Maxima, we must set \mv|ratepsilon| to a small number which somehow
  ``matches'' the setting of \mv|fpprec|.
\begin{myVerbatim}
(%i22) ratepsilon : 1.0e-41$
(%i23) sbfar40 : map ('rhs, bfallroots (%i*e_f40));
`rat' replaced -3.100627668029982017547631506710139520223B1 
   by -689775162029634828708/22246307389364524529 
       = -3.100627668029982017547631506710139520223B1
`rat' replaced 2.960881320326807585650347299962845340594B1 
  by 1094430324967716480409/36962991979932468848 
         = 2.960881320326807585650347299962845340594B1
`rat' replaced -9.424777960769379715387930149838508652592B0 
   by -787357891006146598194/83541266890691994833
        = -9.424777960769379715387930149838508652592B0
(%o23) [3.141592653589793238462643383279502884197b0,
        3.141592653589793238462643383279502884192b0
         -2.066298663554802260101294694335978730541b-40*%i,
        2.066298663554802260101294694335978730541b-40*%i
         +3.141592653589793238462643383279502884203b0]
(%i24) for s in sbfar40 do (subst (s,x,e_f40), disp (expand (%%)))$
2.20405190779178907744138100729171064591b-39

0.0b0

7.346839692639296924804603357639035486367b-40

(%i25) for s in sbfar40 do disp (pi50 - realpart(s))$
9.183549615799121156005754197048794357958b-41

5.050952288689516635803164808376836896877b-39

-5.510129769479472693603452518229276614775b-39
\end{myVerbatim}
  which provides roughly 40 digit accuracy solutions for the roots.\\
  
\noindent Of course, you can use \mv|ratprint : false| to avoid those
  pesky \mv|rat| conversion messages.

\newpage
\subsubsection{Bigfloat Number Gaps and Binary Arithmetic}  \label{gaps}
\textbf{fpprec} as set by the user is the number of DECIMAL digits being requested.
In fact, the actual arithmetic is carried out with binary arithmetic.
Due to the inevitably finite number of binary bits used to represent a floating point
  number there will be a range of floating point numbers which are not recognised
  as different.\\

\noindent For a simple example, let's take the case \textbf{fpprec = 4}.
Consider the gap around the number \mvf|x:bfloat(2/3)|
  whose magnitude is less than \textbf{1}.
We will find that \textbf{?fpprec} has the value \textbf{16} and that Maxima 
  behaves as if (for this case)
  the fractional part of a bigfloat number is represented by the state of a system
  consisting of \textbf{18} binary bits.\\
  
\noindent Let $\mathbf{u = 2^{-18}}$.
If we let $\mathbf{x1 = x + u}$ we get a number which is treated as having a nonzero
 difference from $\mathbf{x}$.
However, if we let $\mathbf{w}$ be a number which is one significant digit less than $\mathbf{u}$,
  and define $\mathbf{x2 = x + w}$, $\mathbf{x2}$ is treated as having \textbf{zero} difference
 from $\mathbf{x}$.
Thus the gap in bigfloats around our chosen $\mathbf{x}$ is 
 $\mathbf{ulp = 2 \cdot 2^{-18} = 2^{-17}}$,
  and this gap should be the same size (as long as \textbf{fpprec = 4}) for any bigfloat
  with a magnitude less than \textbf{1}.\\
  
\noindent If we consider a bigfloat number whose decimal magnitude is less than \textbf{1},
  its value is represented by a ``fractional binary number''.
For the case that this fractional binary number is the state of \textbf{18} (binary) bits,
  the smallest base 2 number which can occur
   is the state in which all bits are off (0) except the least significant bit which
   is on (1), and the decimal equivalent of this fractional binary number is
   precisely $\mathbf{2^{-18}}$.
Adding two bigfloats (each of which has a decimal magnitude less than \textbf{1}) 
  when each is represented by the
  state of an \textbf{18} binary bit system (interpreted as a fractional binary number), 
  it is not possible to increase the value of
  any one bigfloat by less than this smallest base \textbf{2} number. 
\begin{myVerbatim}
(%i1) fpprec:4$
(%i2) ?fpprec;
(%o2)                                 16
(%i3) x :bfloat(2/3);
(%o3)                              6.667b-1
(%i4) u : bfloat(2^(-18));
(%o4)                              3.815b-6
(%i5) x1 : x + u;
(%o5)                              6.667b-1
(%i6) x1 - x;
(%o6)                              1.526b-5
(%i7) x2 : x + 3.814b-6;
(%o7)                              6.667b-1
(%i8) x2 - x;
(%o8)                                0.0b0
(%i9) ulp : bfloat(2^(-17));
(%o9)                              7.629b-6
\end{myVerbatim}
In computer science \textbf{Unit in the Last Place}, or 
 \textbf{Unit of Least Precision},  \textbf{ulp(x)},  associated with a floating
 point number \textbf{x} is the gap between the two floating-point numbers closest to the value \textbf{x}.
 We assume here that the magnitude of \textbf{x} is less than 1.
 These two closest numbers will be $\mathbf{x+u}$ and $\mathbf{x - u}$ where $\mathbf{u}$
 is the smallest positive floating point number which can be accurately represented by 
 the systems of binary bits whose states are used to represent the fractional parts of the
 floating point numbers.\\
 
\noindent The amount of error in the evaluation of a floating-point operation is often 
expressed in ULP. 
We see that for \textbf{fpprec = 4}, 1 ULP is about $\mathbf{8 \cdot 10^{-6}}$.
An average error of 1 ULP is often seen as a tolerable error.
\newpage
\noindent We can repeat this example for the case \textbf{fpprec = 16}.
\begin{myVerbatim}
(%i10) fpprec:16$
(%i11) ?fpprec;
(%o11)                                56
(%i12) x :bfloat(2/3);
(%o12)                       6.666666666666667b-1
(%i13) u : bfloat(2^(-58));
(%o13)                       3.469446951953614b-18
(%i14) x1 : x + u;
(%o14)                       6.666666666666667b-1
(%i15) x1 - x;
(%o15)                       1.387778780781446b-17
(%i16) x2 : x + 3.469446951953613b-18;
(%o16)                       6.666666666666667b-1
(%i17) x2 - x;
(%o17)                               0.0b0
(%i18) ulp : bfloat(2^(-57));
(%o18)                       6.938893903907228b-18
\end{myVerbatim}   
\subsubsection{Effect of Floating Point Precision on Function Evaluation}  \label{uncert}
Increasing the value of \textbf{fpprec} allows a more accurate numerical value
  to be found for the value of a function at some point.
A simple function which allows one to find the absolute value of the change
  produced by increasing the value of \textbf{fpprec} has been presented by
  Richard Fateman.\footnote{see his draft paper ``Numerical Quadrature in a Symbolic/Numerical
  Setting'', Oct. 16, 2008, available as the file \textbf{quad.pdf} in the folder:
 \textbf{http://www.cs.berkeley.edu/$\mathbf{\sim}$fateman/papers/}} 
This function is \textbf{uncert( f, arglist)}, 
in which \textbf{f} is a Maxima function, depending on one or more variables, and
  \textbf{arglist} is the n-dimensional point at which one wants the change in value
  of \textbf{f} produced by an increase of \textbf{fpprec} by \textbf{10}.
This function returns a two element list consisting of the numerical value
  of the function at the requested point and also the absolute value of the
  difference induced by increasing the value of the current \textbf{fpprec} setting
  by the amount \textbf{10}.\\
  
\noindent We present here a version of Fateman's function which has an additional
  argument to control the amount of the increase of \textbf{fpprec}, and also has
  been simplified to accept only a function of one variable.\\

\noindent The function \textbf{fdf} is available in the Ch. 8 files \textbf{fdf.mac},
  \textbf{qbromberg.mac}, \mv|quad_ts.mac|, and \mv|quad_de.mac|, 
  and is defined by the code 
\begin{myVerbatim2}
fdf (%ff, %xx, %dfp) :=
   block([fv1,fv2,df],
    fv1:bfloat(%ff(bfloat(%xx))),
    block([fpprec:fpprec + %dfp ],
     fv2:  bfloat(%ff(bfloat(%xx))),
     df: abs(fv2 - fv1) ),
   [bfloat(fv2),bfloat(df)] )$     
\end{myVerbatim2} 
Here is an example of how this function can be used.    
\begin{myVerbatim}
(%i1) (fpprintprec:8,load(fdf))$
(%i2) fpprec;
(%o2)                                 16
(%i3) g(x) := sin(x/2)$
\end{myVerbatim}
\newpage
\begin{myVerbatim}
(%i4) fdf(g,1,10);
(%o4)                    [4.7942553b-1, 1.834924b-18]
(%i5) fdf(g,1,10),fpprec:30;
(%o5)                    [4.7942553b-1, 2.6824592b-33]
(%i6) fpprec;
(%o6)                                 16
\end{myVerbatim} 
In the first example, \textbf{fpprec} is \textbf{16}, and increasing the value to \textbf{26}
  produces a change in the function value of about $\mathbf{2 \times 10^{-18}}$.
In the second example, \textbf{fpprec} is \textbf{30}, and increasing the value to \textbf{40}
  produces a change in the function value of about $\mathbf{3 \times 10^{-33}}$.\\
  
\noindent In the later section describing the ``tanh-sinh'' quadrature method, 
  we will use this function for a heuristic estimate of the contribution of floating 
  point errors to the approximate numerical value produced for an integral by that method.    

\subsection{  Arbitrary Precision Quadrature with Maxima}
\subsubsection{Using \textbf{bromberg} for Arbitrary Precision Quadrature }  \label{bromberg}
A bigfloat version of the \textbf{romberg} function is defined in
  the file \mvs|brmbrg.lisp| located in \mvs|share/numeric|.
You need to use \textbf{load(brmbrg)} or \textbf{load("brmbrg.lisp")}
  to use the function \textbf{bromberg}.\\
  
\noindent The use of \textbf{bromberg} is identical to the use of the
 \textbf{romberg} which we discussed in Chapter 8 (Numerical Integration),
  except that \textbf{rombergtol} (used for a relative error precision return) is
  replaced by the bigfloat \textbf{brombergtol} with a default value of \textbf{1.0b-4}, and
  \textbf{rombergabs} (used for an absolute error return) is replaced by the bigfloat \textbf{brombergabs}
  which has the default value \textbf{0.0b0}, and \textbf{rombergit} (which causes
  an return after halving the step size that many times) is replaced by the integer
  \textbf{brombergit} which has the default value \textbf{11}, and finally,
  \textbf{rombergmin} (the minimum number of halving iterations)  is replaced by the
   integer  \textbf{brombergmin} which has the default value \textbf{0}.\\
  
\noindent If the function being integrated has a magnitude of order one over the domain
   of integration, then an absolute error precision of a given size is approximately 
   equivalent to a relative error precision of the same size.
We will test \textbf{bromberg} using the function \textbf{exp(x)} over the domain
  \textbf{[-1,\, 1]}, and use only the absolute error precision parameter \textbf{brombergabs},
  setting \textbf{brombergtol} to \textbf{0.0b0} so that the relative error test cannot
  be satisfied.  
Then the approximate value of the integral is returned when the absolute value
  of the change in value from one halving iteration to the next is less than the bigfloat
  number \textbf{brombergabs}.\\  
  
\noindent We explore the use and behavior of \textbf{bromberg} for the simple
  integral $\mathbf{\int_{-1}^{1} e^{x}\,dx}$, binding a value accurate to
  42 digits to \textbf{tval}, defining parameter values, calling \textbf{bromberg}
  first with \textbf{fpprec} equal to 30 together with 
  \textbf{brombergabs} set to \textbf{1.0b-15} and find an actual error (compared
  with \textbf{tval}) of about $\mathbf{7 \times 10^{-24}}$.  
\begin{myVerbatim}
(%i1) (fpprintprec:8,load(brmbrg));
(%o1) C:/PROGRA~1/MAXIMA~3.1/share/maxima/5.18.1/share/numeric/brmbrg.lisp
(%i2) [brombergtol,brombergabs,brombergit,brombergmin,fpprec,fpprintprec];
(%o2)                    [1.0b-4, 0.0b0, 11, 0, 16, 8]
(%i3) tval: bfloat(integrate(exp(x),x,-1,1)),fpprec:42;
(%o3)                             2.3504023b0
(%i4) fpprec;
(%o4)                                 16
\end{myVerbatim}
\newpage
\begin{myVerbatim}
(%i5) (brombergtol:0.0b0,brombergit:100)$
(%i6) b15:(brombergabs:1.0b-15,bromberg(exp(x),x,-1,1) ),fpprec:30;
(%o6)                             2.3504023b0
(%i7) abs(b15 - tval),fpprec:42;
(%o7)                            6.9167325b-24
(%i8) b20:(brombergabs:1.0b-20,bromberg(exp(x),x,-1,1) ),fpprec:30;
(%o8)                             2.3504023b0
(%i9) abs(b20 - tval),fpprec:42;
(%o9)                            1.5154761b-29
\end{myVerbatim} 
We see that, for the case of this test integral involving a well behaved integrand,
  the actual error of the result returned by \textbf{bromberg} is much smaller
  than the requested ``difference error'' supplied by the parameter \textbf{brombergabs}.\\
  
\noindent For later use, 
 we define \textbf{qbromberg} (in a file \textbf{qbromberg.mac}) with the code: 
\begin{myVerbatim2}
qbromberg(%f,a,b,rprec,fp, itmax ) :=
        block([brombergtol,brombergabs,brombergit,
                   fpprec:fp ],
          if rprec > fp then
           ( print(" rprec should be less than fp "),
              return(done) ),
          brombergabs : bfloat(10^(-rprec)),
          brombergtol : 0.0b0,
          brombergit : itmax,
          bromberg(%f(x),x,a,b) )$   
\end{myVerbatim2} 
This function, with the syntax
\begin{myVerbatim2s}
    qbromberg ( f, a, b, rprec, fp, itmax )
\end{myVerbatim2s}
  uses the Maxima function \mv|bromberg| to integrate the Maxima function \mv|f|
  over the interval \mv|[a, b]|, setting the local value of \mv|fpprec| to \mv|fp|,
  setting \mv|brombergtol| to \mv|0|, setting \mv|brombergabs| to $\mathbf{10^{-rprec}}$,
  where \mv|rprec| is called the ``requested precision''.\\

\noindent Here is a test of \textbf{qbromberg} for this simple integral. 
\begin{myVerbatim}
(%i10) load(qbromberg)$
(%i11) qbr20 : qbromberg(exp,-1,1,20,40,100);
(%o11)                             2.3504023b0
(%i12) abs(qbr20 - tval);
(%o12)                                0.0b0
(%i13) abs(qbr20 - tval),fpprec:40;
(%o13)                           1.0693013b-29
\end{myVerbatim}   
We have to be careful in the above step-by-step method to set
  \textbf{fpprec} to a large enough value to see the actual
  size of the error in the returned answer.  
\newpage
\noindent Instead of the work involved in the above step by step method, it
  is more convenient to define a function \textbf{qbrlist} which is passed a desired
  \mv|fpprec| as well as a list of requested precision goals for \textbf{bromberg}.
The function \mv|qbrlist| then assumes a sufficiently accurate
  \textbf{tval} is globally defined, and proceeds 
  through the list to calculate the \textbf{bromberg} value for each
  requested precision, computes the error in the result, and prints a line
  containing (rprec,  fpprec,  value,  value-error).
Here is the code for such a function, available in \textbf{qbromberg.mac}: 
\begin{myVerbatim2s}
qbrlist(%f,a,b,rplist,fp,itmax) :=
  block([fpprec:fp,fpprintprec,brombergtol,
             brombergabs,brombergit,val,verr,pr],
     if not listp(rplist) then (print("rplist # list"),return(done)),     
     brombergtol : 0.0b0,
     brombergit : itmax,
     fpprintprec:8,
     print(" rprec   fpprec     val             verr "),
     print(" "),
     for pr in rplist do
      ( brombergabs : bfloat(10^(-pr)),
        val: bromberg(%f(x),x,a,b),
        verr: abs(val - tval),
        print("  ",pr,"   ",fp,"   ",val,"   ",verr) ) )$
\end{myVerbatim2s} 
and here is an example of use of \textbf{qbrlist}
 in which the requested precision \mv|rprec| (called \mv|pr| in the code) is set to three
  different values supplied by the list \mv|rplist| for each setting of \mv|fpprec| used.
We first define an accurate comparison value \mv|tval|:  
\begin{myVerbatim}
(%i1) (fpprintprec:8, load(brmbrg), load(qbromberg))$
(%i2) tval: bfloat(integrate(exp(x),x,-1,1)),fpprec:42;
(%o2)                             2.3504023b0
\end{myVerbatim}
Here is our test for three different values of \mv|fpprec|:
\begin{myVerbatim}
(%i3) qbrlist(exp,-1,1,[10,15,17 ],20,100)$
 rprec   fpprec     val             verr    
   10     20     2.3504023b0     4.5259436b-14 
   15     20     2.3504023b0     1.3552527b-20 
   17     20     2.3504023b0     1.3552527b-20 
(%i4) qbrlist(exp,-1,1,[10,20,27 ],30,100)$
 rprec   fpprec     val             verr    
   10     30     2.3504023b0     4.5259437b-14 
   20     30     2.3504023b0     1.4988357b-29 
   27     30     2.3504023b0     5.5220263b-30 
(%i5) qbrlist(exp,-1,1,[10,20,30,35],40,100)$
 rprec   fpprec     val             verr    
   10     40     2.3504023b0     4.5259437b-14 
   20     40     2.3504023b0     1.0693013b-29 
   30     40     2.3504023b0     1.1938614b-39 
   35     40     2.3504023b0     1.1938614b-39 
\end{myVerbatim} 
We see that with \textbf{fpprec} equal to \textbf{40}, increasing \textbf{rprec} 
  from \textbf{30} to \textbf{35} 
  results in no improvement in the actual error of the result.
\newpage
\subsubsection*{When \textbf{bromberg} Fails}
If the integrand has end point algebraic and/or logarithmic singularities, \textbf{bromberg} may fail.
Here is an example in which the integrand has a
  logarithmic singularity at the lower end point:
   $\mathbf{\int_{0}^{1} \sqrt{t}\,\boldsymbol{\ln}(t)\,dt}$.
The \textbf{integrate} function has no problem with this integral. 
\begin{myVerbatim}
(%i6) g(x):= sqrt(x)*log(x)$
(%i7) integrate(g(t),t,0,1);
                                        4
(%o7)                                 - -
                                        9
(%i8) (load(brmbrg),load(qbromberg))$
(%i9) qbromberg(g,0,1,30,40,100);
log(0) has been generated.
#0: qbromberg(%f=g,a=0,b=1,rprec=30,fp=40,itmax=100)
 -- an error.  To debug this try debugmode(true);
\end{myVerbatim} 
You can instead use the tanh-sinh quadrature method for this integral (see Sec. \ref{tanh1}).
\subsubsection{A \textbf{Double Exponential} Quadrature Method for $\mathbf{a \leq x < \infty}$} \label{de1}
This method (H. Takahasi and M. Mori, 1974; see Sec \ref{tanh1}) is effective for integrands
  which contain a factor with some sort of exponential damping as the integration variable
  becomes large.\\

\noindent An integral of the form $\mathbf{\int_{a}^{\infty}\,g(y)\,dy}$ can be converted into the integral
  $\mathbf{\int_{0}^{\infty}\,f(x)\,dx}$ by making the change of variable of integration
  $\mathbf{y \rightarrow x }$ given by $\mathbf{y = x + a}$.  
Then $\mathbf{f(x) = g(x + a) }$.\\

\noindent The double exponential method used here then converts the 
   integral $\mathbf{\int_{0}^{\infty} f(x)\,dx}$ into the integral
   $\mathbf{\int_{-\infty}^{\infty} F(u)\,du}$ using a variable transformation
   $\mathbf{x \rightarrow u}$:
\begin{equation}
\mathbf{ x(u) = exp(u - exp(-u)) }
\end{equation}
and hence
\begin{equation}
\mathbf{F(u) = f(x(u))\,w(u), \qquad \text{where} \qquad w(u) = \frac{dx}{du} = 
   exp(- exp(-u)) + x(u) }.
\end{equation}
You can confirm that $\mathbf{x(0) = exp(-1), \quad w(0) = 2\,x(0)}$ and that
  $\mathbf{x(-\infty) = 0, \quad x(\infty) = \infty}$.\\
  
\noindent Because of the rapid decay of the integrand when the magnitude of $\mathbf{u}$ is large,
  one can approximate the value of the infinite domain $\mathbf{u}$ integral by using a trapezoidal
  numerical approximation with step size $\mathbf{h}$ using a modest number 
   ($\mathbf{2\,N + 1}$) of function evaluations.
\begin{equation}  
\mathbf{I(h,N) \simeq h \sum_{j\, =\, -N}^{N}\,F(u_{j}) \quad \text{where} \quad u_{j} = j\,h }
\end{equation}
This method is implemented in the Ch. 8 file \mv|quad_de.mac|.
We first demonstrate the available functions on the simple integral
  $\mathbf{\int_{0}^{\infty} e^{-x}\,dx = 1}$.
 
\begin{myVerbatim}
(%i1) fpprintprec:8$
(%i2) g(x):= exp(-x)$
(%i3) tval : bfloat(integrate(g(x),x,0,inf)),fpprec:45;
(%o3)                                1.0b0
\end{myVerbatim}
\newpage
\begin{myVerbatim}
(%i4) load(quad_de);
(%o4)                        c:/work3/quad_de.mac
(%i5) quad_de(g,0,30,40);
(%o5)                      [1.0b0, 4, 4.8194669b-33]
(%i6) abs(first(%) - tval),fpprec:45;
(%o6)                            9.1835496b-41
\end{myVerbatim} 
The package function \mv|quad_de(f, a, rp, fp)| integrates the Maxima function \mv|f| over the domain
  $\mathbf{[x \geq a]}$, using \textbf{fpprec : fp}, and returns a three element list
  when \textbf{vdiff} (the absolute value of the difference obtained for the integral in
  successive k levels) becomes less than or equal to $\mathbf{10^{-rp}}$.
The parameter \textbf{rp} is called the ``requested precision'', and the value of \textbf{h}
  is repeatedly halved until the \textbf{vdiff} magnitude either satisfies this criterion
  or starts increasing.
The first element is the appoximate value of the integral.
The second element (4 above) is the ``final k-level'' used, where $\mathbf{h = 2^{-k}}$.
The third and last element is the final value of \textbf{vdiff}.
We see in the above example that requesting precision $\mathbf{rp = 30}$ and using
  floating point precision \textbf{fpprec : 40} results in an answer good to
  about \textbf{40} digits.
This sort of accuracy is typical.\\

\noindent The package function \mv|idek(f, a, k, fp)| integrates the Maxima function \mv|f|
  over the domain $\mathbf{[a,\infty]}$ using a ``k-level approximation'' with
  $\mathbf{h = 1/2^{k}}$ and \textbf{fpprec : fp}. 
\begin{myVerbatim}
(%i7) idek(g,0,4,40);
(%o7)                                1.0b0
(%i8) abs(% - tval),fpprec:45;
(%o8)                            9.1835496b-41
\end{myVerbatim} 
The package function \mv|idek_e(f, a, k, fp)| does the same calculation
  as \textbf{idek(f, a, k, fp)}, but returns both the approximate value of
  the integral and also a rough estimate of the amount of the error which is
  due to the floating point arithmetic precision being used.
(The error of the approximation has three contributions: 1. the quadrature algorithm
  being used, 2. the step size \textbf{h} being used, and 3. the precision of the floating
  point arithmetic being used.) 
\begin{myVerbatim}
(%i9) idek_e(g,0,4,40);
(%o9)                      [1.0b0, 8.3668155b-42]
(%i10) abs(first(%) - tval),fpprec:45;
(%o10)                           9.1835496b-41
\end{myVerbatim} 
The package function \mv|ide(f, a, rp, fp)| follows the same path
  as \mv|quad_de(f, a, rp, fp)|, but shows the progression toward
  success as the k level increases ( and \textbf{h} decreases ): 
\begin{myVerbatim}
(%i11) ide(g,0,30,40);   
      rprec =  30  fpprec =  40 
 k     value           vdiff      
 1    1.0b0           
 2    1.0b0          4.9349774b-8    
 3    9.9999999b-1   4.8428706b-16   
 4    1.0b0          4.8194669b-33   
\end{myVerbatim} 
\newpage
\noindent The package function \mv|ide_test(f, a, rp, fp)| follows
  the path of \mv|ide(f, a, rp, fp)|, but adds to the table
  the value of the error of the approximate result for each
  \textbf{k} level attempted.
The use of this function depends on an accurate value of the
  integral being bound to the global variable \textbf{tval}. 
\begin{myVerbatim}
(%i12) ide_test(g,0,30,40);   
      rprec =  30  fpprec =  40 
 k     value           vdiff            verr      
 1    1.0b0                           4.9349775b-8    
 2    1.0b0          4.9349774b-8     4.8428706b-16   
 3    9.9999999b-1   4.8428706b-16    4.8194668b-33   
 4    1.0b0          4.8194669b-33    9.1835496b-41   
\end{myVerbatim} 
\subsubsection*{Test Integral 1}
Here we test this double exponential method code with the known
  integral
\begin{equation}
\mathbf{\int_{0}^{\infty} \frac{e^{-t}}{\sqrt{t}} \, dt = \sqrt{\boldsymbol{\pi}}}
\end{equation} 
\begin{myVerbatim}
(%i13) g(x):= exp(-x)/sqrt(x)$
(%i14) integrate(g(t),t,0,inf);
(%o14)                             sqrt(%pi)
(%i15) tval : bfloat(%),fpprec:45;
(%o15)                            1.7724538b0
(%i16) quad_de(g,0,30,40);
(%o16)                  [1.7724538b0, 4, 1.0443243b-34]
(%i17) abs(first(%) - tval),fpprec:45;
(%o17)                           1.8860005b-40
(%i18) idek_e(g,0,4,40);
(%o18)                   [1.7724538b0, 2.7054206b-41]
\end{myVerbatim} 
Again we see that the combination \textbf{rp = 30, fp = 40} leads to an
  answer good to about \textbf{40} digits of precision.
\subsubsection*{Test Integral 2}
Our second known integral is
\begin{equation}
\mathbf{\int_{0}^{\infty} e^{-t^{2}/2} \, dt = \sqrt{\boldsymbol{\pi}/2} }
\end{equation} 
\begin{myVerbatim}
(%i19) g(x) := exp(-x^2/2)$
(%i20) tval : bfloat(sqrt(%pi/2)),fpprec:45$
(%i21) quad_de(g,0,30,40);
(%o21)                  [1.2533141b0, 5, 1.099771b-31]
(%i22) abs(first(%) - tval),fpprec:45;
(%o22)                           2.1838045b-40
(%i23) idek_e(g,0,5,40);
(%o23)                   [1.2533141b0, 1.3009564b-41]
\end{myVerbatim} 
\newpage
\subsubsection*{Test Integral 3}
Our third test integral is
\begin{equation}
\mathbf{\int_{0}^{\infty} e^{-t} \, \boldsymbol{\cos}\,t\,dt = 1/2 }
\end{equation} 
\begin{myVerbatim}
(%i24) g(x) := exp(-x)*cos(x)$
(%i25) integrate(g(x),x,0,inf);
                                       1
(%o25)                                 -
                                       2
(%i26) tval : bfloat(%),fpprec:45$
(%i27) quad_de(g,0,30,40);
(%o27)                    [5.0b-1, 5, 1.7998243b-33]
(%i28) abs(first(%) - tval),fpprec:45;
(%o28)                           9.1835496b-41
(%i29) idek_e(g,0,5,40);
(%o29)                      [5.0b-1, 9.8517724b-42]
\end{myVerbatim} 
\subsubsection{The \textbf{tanh-sinh} Quadrature Method for $\mathbf{a \leq x \leq b}$} \label{tanh1}
H. Takahasi and M. Mori (1974: see references at the end of this section) presented an 
  efficient method for the
  numerical integration of the integral of a function over a finite domain.
This method is known under the names ``tanh-sinh method'' and ``double exponential method''.
This method can handle integrands which have algebraic and logarithmic end point
  singularities, and is well suited for use with arbitrary precision work.\\
  
\noindent Quoting (loosely) David Bailey's (see references below) slide presentations 
on this subject:
\begin{quote}
The tanh-sinh quadrature method can accurately handle all ``reasonable functions'',
  even those with ``blow-up singularities'' or vertical slopes at the end points
  of the integration interval.
In many cases, reducing the step size h by half doubles the number of correct
  digits in the result returned ("quadratic convergence").
\end{quote}  
An integral of the form $\mathbf{\int_{a}^{b}\,g(y)\,dy}$ can be converted into the integral
  $\mathbf{\int_{-1}^{1}\,f(x)\,dx}$ by making the change of variable of integration
  $\mathbf{y \rightarrow x }$ given by $\mathbf{y = \boldsymbol{\alpha}\,x + \boldsymbol{\beta}}$
  with $\mathbf{\boldsymbol{\alpha} = (b - a)/2}$ and $\mathbf{\boldsymbol{\beta} = (a + b)/2}$.
Then $\mathbf{f(x) = \boldsymbol{\alpha}\,g(\boldsymbol{\alpha}\,x + \boldsymbol{\beta}) }$.\\

\noindent The tanh-sinh method introduces a change of variables
  $\mathbf{x \rightarrow u}$ which implies
\begin{equation}
\mathbf{\int_{-1}^{1} f(x)\,dx = \int_{-\infty}^{\infty} F(u)\,du. }
\end{equation}
The change of variables is expressed by
\begin{equation}
\mathbf{ x(u) = \boldsymbol{\tanh} \left( \frac{ \boldsymbol{\pi} }{2}\, \boldsymbol{\sinh} \,u \right) }
\end{equation}
and you can confirm that
\begin{equation}
 \mathbf{u=0 \Rightarrow x = 0 , \qquad u \rightarrow -\infty  \Rightarrow  x \rightarrow -1  ,
  u \rightarrow \infty  \Rightarrow  x \rightarrow 1 }
\end{equation}
We also have $\mathbf{x(-u) = - x(u)}$.\\

\noindent The ``weight'' $\mathbf{w(u) = dx(u)/du}$ is
\begin{equation}
\mathbf{w(u)} = \mathbf{\frac{\frac{\boldsymbol{\pi}}{2} \, \boldsymbol{\cosh} \,u } 
               {\boldsymbol{\cosh}^{2}\left(\frac{\boldsymbol{\pi}}{2} \, \boldsymbol{\sinh} \,u\right)} }
\end{equation}
with the property $\mathbf{w(-u) = w(u)}$, in terms of which
 $\mathbf{F(u) = f(x(u)\, w(u) }$.
Moreover, $\mathbf{F(u)}$ has ``double exponential behavior'' of the form
\begin{equation}
\mathbf{F(u) \approx exp\left(-\frac{\boldsymbol{\pi}}{2}\,exp(\,\vert\,u\,\vert\,)\right) \qquad
            \text{for} \quad u \rightarrow \pm \infty}.
\end{equation}
Because of the rapid decay of the integrand when the magnitude of $\mathbf{u}$ is large,
  one can approximate the value of the infinite domain integral by using a trapezoidal
  numerical approximation with step size $\mathbf{h}$ using a modest number 
   ($\mathbf{2\,N + 1}$) of function evaluations.
\begin{equation}  \label{Eq:usum1}
\mathbf{I(h,N) \simeq h \sum_{j\, =\, -N}^{N}\,F(u_{j}) \quad \text{where} \quad u_{j} = j\,h }
\end{equation}
This method is implemented in the Ch.8 file \mv|quad_ts.mac| and we will illustrate
  the available functions using the simple integral $\mathbf{\int_{-1}^{1} e^{x}\,dx}$.\\ 

\noindent The package function \mv|quad_ts (f, a, b, rp, fp )| is the most useful
  workhorse for routine use, and uses the tanh-sinh method to integrate the Maxima
  function \mv|f| over the finite interval \mv|[a, b]|, stopping when the absolute value of the
  difference $\mathbf{(I_{k} - I_{k-1})}$ is less than $\mathbf{10^{-rp}}$ 
  (rp is the ``requested precision'' for the result), using \mv|fp| digit precision 
  arithmetic (\textbf{fpprec} set to \mv|fp|, and  \textbf{bfloat} being used to 
  enforce this arithmetic precision).
This function returns the list\\
 \mv|[ approx-value, k-level-used, abs(vdiff) ]|,
  where the last element should be smaller than $\mathbf{10^{-rp}}$ .  
 
\begin{myVerbatim}
(%i1) fpprintprec:8$
(%i2) tval : bfloat( integrate( exp(x),x,-1,1 )),fpprec:45;
(%o2)                             2.3504023b0
(%i3) load(quad_ts);
   _kmax% =  8   _epsfac% =  2 
(%o3)                        c:/work3/quad_ts.mac
(%i4) bfprint(tval,45)$
   number of digits =  45 
   2.35040238728760291376476370119120163031143596b0 
(%i5) quad_ts(exp,-1,1,30,40);
 construct _yw%[kk,fpprec] array for kk =  8  and fpprec =  40   ...working... 
(%o5)                       [2.3504023b0, 5, 0.0b0]
(%i6) abs(first(%) - tval),fpprec:45;
(%o6)                            2.719612b-40
\end{myVerbatim}
 
A value of the integral accurate to about 45 digits is bound to the
  symbol \textbf{tval}.
The package function \mvs|bfprint(bf, fpp)| allows controlled printing
  of \textbf{fpp} digits of the ``true value'' \textbf{tval} to the screen.
We then compare the approximate quadrature result with this ``true value''.
The package \mv|quad_ts.mac| defines two global parameters.
\mv|_kmax%| is the 
  maximum ``k-level'' possible (the defined default is 8, which means the minimum step
  size for the transformed ``u-integral'' is $\mathbf{du = h = 1/2^{8} = 1/256}$.
The actual ``k-level'' needed to return a result with the requested precision \mv|rp| is the
  integer in the second element of the returned list.
The global parameter \mv|_epsfac%| (default value \mv|2|) is used to decide how many
  \mv|(y, w)| numbers to pre-compute (see below).\\
  
\noindent We see that a ``k-level'' approximation with $\mathbf{k = 5}$ and $\mathbf{h = 1/2^{5} = 1/32}$
  returned an answer with an actual precision of about 40 digits (when $\mathbf{rp = 30}$
  and $\mathbf{fp = 40}$).\\
  
\noindent The first time 40 digit precision arithmetic is called for, 
  a set of (y, w) numbers are calculated and stored in an array which we call
  \mv|_yw%[8, 40]|.
The \mv|y(u)| values will later be converted to \mv|x(u)| numbers using high precision, and the
  original integrand function \mv|f(x(u))| is also calculated at high precision.
The \mv|w(u)| numbers are what we call ``weights'', and are needed for the numbers
  $\mathbf{F(u) = f( x(u) )\,w(u) }$ used in the trapezoidal rule evaluation.
The package precomputes pairs \mv|(y, w)| for larger and larger values of \mv|u| until
  the magnitude of the weight \mv|w| becomes less than \mv|eps|, where $\mathbf{eps = 10^{- n\,p}}$, where
  \mv|n| is the global parameter \mv|_epsfac%| (default \mv|2|) and
  \mv|p| is the requested floating point precision \mv|fp|.\\

\noindent Once the set of \mv|40-digit| precision \mv|(y, w)| numbers have been ``pre-computed'', they
  can be used for the evaluation of any similar precision integrals later, 
  since these numbers are independent
  of the actual function being integrated, but depend only on the nature of the tanh-sinh
  transformation being used.\\
  
\noindent The package function \mv|qtsk(f, a, b, k, fp)| (note: arg \textbf{k} replaces \textbf{rp} )
  integrates the Maxima function \mv|f|
  over the domain $\mathbf{[a, b]}$ using a ``k-level approximation'' with
  $\mathbf{h = 1/2^{k}}$ and \mvs|fpprec : fp|.
 
\begin{myVerbatim}
(%i7) qtsk(exp,-1,1,5,40);
(%o7)                             2.3504023b0
(%i8) abs(% - tval),fpprec:45;
(%o8)                            2.719612b-40
\end{myVerbatim}
    
A heuristic value of the error contribution due to the arithmetic precision being used (which is
  separate from the error contribution due to the nature of the algorithm and the step size being
  used) can be found by using the package function \mv|qtsk_e(f, a, b, k, fp);|.
  The first element of the returned list is the value of the integral, the second 
  element of the returned list is a rough estimate of the contribution of the floating
  point arithmetic precision being used to the error of the returned answer.
 
\begin{myVerbatim}
(%i9) qtsk_e(exp,-1,1,5,40);
(%o9)                    [2.3504023b0, 2.0614559b-94]
(%i10) abs(first(%) - tval),fpprec:45;
(%o10)                           2.719612b-40
\end{myVerbatim}
 
The very small estimate of the arithmetic precision contribution (two parts in $\mathbf{10^{94}}$)
 to the error of the
  answer is due to the high precision being used to convert from the pre-computed \textbf{y}
  to the needed abcissa \textbf{x} via $\mathbf{ x : bfloat( 1 - y) }$ and the subsequent
  evaluation $\mathbf{f(x)}$.
The precision being used depends on the size of the smallest \textbf{y} number,
  which will always be that appearing in the last element of the hashed array
   \mv|_yw%[8, 40]|.
 
\begin{myVerbatim}
(%i11) last(_yw%[8,40]);
(%o11)                  [4.7024891b-83, 8.9481574b-81]
\end{myVerbatim} 
(In Eq. (\ref{Eq:usum1}) we have separated out the \textbf{(u = 0, x = 0)} term, and
  used the symmetry properties $\mathbf{x(-u) = -x(u)}$, and $\mathbf{w(-u) = w(u)}$ to write the
  remainder as a sum over positive values of \textbf{u} (and hence positive values
  of \textbf{x}) so only the large \textbf{u} values of \textbf{y(u)} need to
  be pre-computed).\\
  
\noindent We see that the smallest \textbf{y} number is about $\mathbf{5 \times 10^{-83}}$
   and if we subtract this from \mv|1| we will get \mv|1| unless we use a very high precision.
It turns out that as \textbf{u} approaches plus infinity, \textbf{x} (as used here) approaches \textbf{b}
 (which is \mv|1| in our example) from values less than \textbf{b}.
Since a principal virtue of the tanh-sinh method is its ability to handle integrands
  which ``blow up'' at the limits of integration, we need to make sure we stay away (even
  if only a little) from those end limits.\\
  
\noindent We can see the precision with which the arithmetic is being carried out in this
  crucial step by using the \textbf{fpxy(fp)} function    
\begin{myVerbatim}
(%i12) fpxy(40)$
 the last y value =  4.7024891b-83 
 the fpprec being used for x and f(x) is  93 
\end{myVerbatim} 
and this explains the small number returned (as the second element) by
  \mv|qtsk_e(exp, -1, 1, 5, 40);|.\\  

\noindent The package function \mv|qts(f, a, b, rp, fp)| follows the same path
  as \mv|quad_ts(f, a, b, rp, fp)|, but shows the progression toward
  success as the \mv|k| level increases ( and \verb|h| decreases ): 
\begin{myVerbatim}
(%i13) qts(exp,-1,1,30,40)$
  rprec =  30   fpprec =  40 
  k       newval         vdiff     
  1    2.350282b0 
  2    2.3504023b0    1.2031242b-4 
  3    2.3504023b0    8.136103b-11 
  4    2.3504023b0    1.9907055b-23 
  5    2.3504023b0    0.0b0
\end{myVerbatim} 
The package function \mv|qts_test(f, a, b, rp, fp)| follows
  the path of \mv|qts(f, a, b, rp, fp)|, but adds to the table
  the value of the error of the approximate result for each
  \mv|k| level attempted.
The use of this function depends on an accurate value of the
  integral being bound to the global variable \textbf{tval}. 
\begin{myVerbatim}
(%i14) qts_test(exp,-1,1,30,40)$
      rprec =  30  fpprec =  40 
 k     value           vdiff            verr      
 1    2.350282b0                      1.2031234b-4    
 2    2.3504023b0    1.2031242b-4     8.136103b-11    
 3    2.3504023b0    8.136103b-11     1.9907055b-23   
 4    2.3504023b0    1.9907055b-23    2.7550648b-40   
 5    2.3504023b0    0.0b0            2.7550648b-40 
\end{myVerbatim} 
\subsubsection*{Test Integral 1}
Here we test this tanh-sinh method code with the known
  integral which confounded \textbf{bromberg} in Sec. \ref{bromberg} :
\begin{equation}
\mathbf{\int_{0}^{1} \sqrt{t}\,\boldsymbol{\ln}(t)\,dt = -4/9}
\end{equation} 
\begin{myVerbatim}
(%i15) g(x):= sqrt(x)*log(x)$
(%i16) tval : bfloat(integrate(g(t),t,0,1)),fpprec:45;
(%o16)                          - 4.4444444b-1
(%i17) quad_ts(g,0,1,30,40);
(%o17)                [- 4.4444444b-1, 5, 3.4438311b-41]
(%i18) abs(first(%) - tval),fpprec:45;
(%o18)                           4.4642216b-41
(%i19) qtsk_e(g,0,1,5,40);
(%o19)                  [- 4.4444444b-1, 7.6556481b-43]
\end{myVerbatim} 
Requesting thirty digit accuracy with forty digit arithmetic returns a value
  for this integral which has about forty digit precision.
Note that ``vdiff'' is approximately the same as the actual absolute error.
\subsubsection*{Test Integral 2}
Consider the integral
\begin{equation}
\mathbf{\int_{0}^{1} \frac{\boldsymbol{\arctan}(\sqrt{2 + t^{2}})}{(1 + t^{2})\,\sqrt{2+t^{2}}}\,dt = 5\,\boldsymbol{\pi}^{2}/96}.
\end{equation} 
\begin{myVerbatim}
(%i20) g(x):= atan(sqrt(2+x^2))/(sqrt(2+x^2)*(1+x^2))$
(%i21) integrate(g(t),t,0,1);
                           1
                          /              2
                          [   atan(sqrt(t  + 2))
(%o21)                    I  --------------------- dt
                          ]    2            2
                          /  (t  + 1) sqrt(t  + 2)
                           0
(%i22) quad_qags(g(t),t,0,1);
(%o22)                 [0.514042, 5.70701148E-15, 21, 0]
(%i23) float(5*%pi^2/96);
(%o23)                             0.514042
(%i24) tval: bfloat(5*%pi^2/96),fpprec:45;
(%o24)                           5.1404189b-1
(%i25) quad_ts(g,0,1,30,40);
(%o25)                 [5.1404189b-1, 5, 1.5634993b-36]
(%i26) abs(first(%) - tval),fpprec:45;
(%o26)                           7.3300521b-41
(%i27) qtsk_e(g,0,1,5,40);
(%o27)                   [5.1404189b-1, 1.3887835b-43]
\end{myVerbatim} 
\subsubsection*{Test Integral 3}
We consider the integral
\begin{equation}
\mathbf{\int_{0}^{1} \frac{\sqrt{t}}{\sqrt{1-t^{2}}}\,dt = 2 \, \sqrt{\boldsymbol{\pi}} \,\Gamma(3/4)/\Gamma(1/4)}
\end{equation} 
\begin{myVerbatim}
(%i28) g(x):= sqrt(x)/sqrt(1 - x^2)$
(%i29) quad_qags(g(t),t,0,1);
(%o29)                [1.1981402, 8.67914629E-11, 567, 0]
(%i30) integrate(g(t),t,0,1);
                                       1  3
                                  beta(-, -)
                                       2  4
(%o30)                            ----------
                                      2
(%i31) tval : bfloat(%),fpprec:45;
(%o31)                             1.1981402b0
(%i32) quad_ts(g,0,1,30,40);
(%o32)                  [1.1981402b0, 5, 1.3775324b-40]
(%i33) abs(first(%) - tval),fpprec:45;
(%o33)                           1.8628161b-40
(%i34) qtsk_e(g,0,1,5,40);
(%o34)                   [1.1981402b0, 1.5833892b-45]
\end{myVerbatim} 
\newpage
\noindent An alternative route to the ``true value'' is to convert \mv|beta| to
  \mv|gamma|'s using \mv|makegamma|:
\begin{myVerbatim}
(%i35) makegamma(%o30);
                                               3
                             2 sqrt(%pi) gamma(-)
                                               4
(%o35)                       --------------------
                                         1
                                   gamma(-)
                                         4
(%i36) float(%);
(%o36)                             1.1981402
(%i37) bfloat(%o11),fpprec:45;
(%o37)                            1.1981402b0
\end{myVerbatim} 
\subsubsection*{Test Integral 4}
We next consider the integral
\begin{equation}
\mathbf{\int_{0}^{1} \boldsymbol{\ln}^{2}\,t\,dt = 2}
\end{equation} 
\begin{myVerbatim}
(%i38) g(x) := log(x)^2$
(%i39) integrate(g(t),t,0,1);
(%o39)                                 2
(%i40) quad_ts(g,0,1,30,40);
(%o40)                         [2.0b0, 5, 0.0b0]
(%i41) abs( first(%) - bfloat(2) ),fpprec:45;
(%o41)                           1.8367099b-40
(%i42) qtsk_e(g,0,1,5,40);
(%o42)                      [2.0b0, 1.2570464b-42]
\end{myVerbatim} 
\subsubsection*{Test Integral 5}
We finally consider the integral
\begin{equation}
\mathbf{\int_{0}^{\boldsymbol{\pi}/2} \boldsymbol{\ln(\cos}\,t)\,dt =
        -\boldsymbol{\pi \,\ln}(2)/2}
\end{equation}		 
\begin{myVerbatim}
(%i43) g(x) := log( cos(x) )$
(%i44) quad_qags(g(t),t,0,%pi/2);
(%o44)               [- 1.088793, 1.08801856E-14, 231, 0]
(%i45) integrate(g(t),t,0,%pi/2);
                               %pi
                               ---
                                2
                              /
                              [
(%o45)                        I    log(cos(t)) dt
                              ]
                              /
                               0
\end{myVerbatim}
\newpage
\begin{myVerbatim}
(%i46) float(-%pi*log(2)/2);
(%o46)                            - 1.088793
(%i47) tval : bfloat(-%pi*log(2)/2),fpprec:45;
(%o47)                           - 1.088793b0
(%i48) quad_ts(g,0,%pi/2,30,40);
(%o48)         [1.2979374b-80 %i - 1.088793b0, 5, 9.1835496b-41]
(%i49) ans: realpart( first(%) );
(%o49)                           - 1.088793b0
(%i50) abs(ans - tval),fpprec:45;
(%o50)                           1.9661653b-40
(%i51) qtsk_e(g,0,%pi/2,5,40);
(%o51)          [1.2979374b-80 %i - 1.088793b0, 2.4128523b-42]
\end{myVerbatim} 
We see that the tanh-sinh result includes a tiny imaginary part due to
  bigfloat errors, and taking the real part produces an answer good
  to about 40 digits (using \mv|rp = 30, fp = 40|).
\subsubsection*{References for the \textbf{tanh-sinh} Quadrature Method }
\small
This method was initially described in the article \textbf{Double Exponential Formulas
  for Numerical Integration}, by Hidetosi Takahasi and Masatake Mori, in the
  journal Publications of the Research Institute for Mathematical Sciences ( Publ. RIMS), vol.9,
  Number 3, (1974), 721-741, Kyoto University, Japan.
A recent summary by the second author is \textbf{Discovery of the Double Exponential Transformation
  and Its Developments}, by Masatake Mori, Publ. RIMS, vol.41, Number 4, (2005), 897-935.
Both of the above articles can be downloaded from the Project Euclid RIMS webpage 
\begin{myVerbatim2f}
http://projecteuclid.org/
              DPubS?service=UI&version=1.0&verb=Display&page=past&handle=euclid.prims
\end{myVerbatim2f} 
A good summary of implementation ideas can be found in the report
  \textbf{Tanh-Sinh High-Precision Quadrature}, by David H. Bailey, Jan. 2006, LBNL-60519,
  which can be downloaded from the webpage 
\begin{myVerbatim2f}
  http://crd.lbl.gov/~dhbailey/dhbpapers/
\end{myVerbatim2f} 
\subsubsection*{Further Improvements for the \textbf{tanh-sinh} Quadrature Method }
The code provided in the file \mv|quad_ts.mac| has been lightly tested, and should
 be used with caution.\\
 
\noindent No proper investigation has been made of the efficiency of choosing
  to use a floating point precision (for all terms of the sum) based on the small
   size of the smallest $\mathbf{y}$ value.\\

\noindent No attempt has been made to translate into Lisp and compile the code
  to make timing trials for comparison purposes.\\

\noindent These (and other) refinements are left to the initiative of the hypothetical
  alert reader of limitless dedication (HAROLD).   
\normalsize
\subsubsection{The \textbf{Gauss-Legendre} Quadrature Method for $\mathbf{a \leq x \leq b}$ }
Loosely quoting from David Bailey's slide presentations (see references at the
  end of the previous section)
\begin{quote}
The Gauss-Legendre quadrature method is an efficient method for continuous, well-behaved
  functions.
In many cases, doubling the number of points at which the integrand is 
  evaluated doubles the number of correct digits in the result.
This method performs poorly for functions with algebraic and/or logarithmic end point
  singularities.
The cost of computing the zeros of the Legendre polynomials and the corresponding
  ``weights'' increases as $\mathbf{n^{2}}$ and thus becomes impractical for use
  beyond a few hundred digits.
\end{quote}
Since one can always make a change of integration variable from the domain \textbf{[a,\, b]}
  to the integration domain \textbf{[-1,\, 1]}, this method approximates an integral 
  over \textbf{[-1,\,1]}  as the sum
\begin{equation}
\mathbf{\int_{-1}^{1} f(x)\,dx \approx \sum_{j = 1}^{N} w_{j}\,f(x_{j}) }
\end{equation}
where the $\mathbf{x_{j}}$ are the roots of the N-th degree Legendre polynomial
  $\mathbf{P_{N}(x)}$ on \textbf{[-1,\, 1]}, and the weights $\mathbf{w_{j}}$ are 
\begin{equation}
\mathbf{w_{j} = \frac{-2}{(N+1)\,P'_{N}(x_{j})\,P_{N+1}(x_{j}) } }
\end{equation}
This method is implemented with our Ch.9 package file \mv|quad_gs.mac|, 
  and Richard Fateman's lisp file:
\begin{myVerbatim2s}
           http://www.cs.berkeley.edu/~fateman/generic/quad-maxima.lisp
\end{myVerbatim2s}
which should be downloaded to use this package.\\

\noindent Except for the ``driver'' Maxima function \mv|quad_gs(f,a,b,rp)|, 
  all the Maxima functions in \mv|quad_gs.mac| are functions defined by Fateman
  in either the comment section of the above lisp file, or in the file  \\
  \mv|http://www.cs.berkeley.edu/~fateman/papers/quadmax.mac|.\\
  
\noindent An introduction to arbitrary precision code and some background to the problem of
  arbitrary precision quadrature has been provided by Richard Fateman in his 
  draft paper \textbf{Numerical Quadrature in a Symbolic/Numerical
  Setting}, \textbf{quad.pdf} (see Sec. \ref{uncert}). \\
   
\noindent We will illustrate the available functions using the simple 
   integral $\mathbf{\int_{-1}^{1} e^{x}\,dx}$.\\ 
   
\noindent The package function \textbf{gaussunit(f, N)} integrates the Maxima function
  \textbf{f} over the domain $\mathbf{[-1, 1]}$ using N point Gauss-Legendre quadrature. 
\begin{myVerbatim}
(%i1) load("quad-maxima.lisp");
(%o1)                          quad-maxima.lisp
(%i2) fpprintprec:8$
(%i3) tval : bfloat(integrate(exp(x),x,-1,1)),fpprec:45;
(%o3)                             2.3504023b0
(%i4) load(quad_gs);
(%o4)                        c:/work3/quad_gs.mac
(%i5) arrays;
(%o5)                            [ab_and_wts]
(%i6) arrayinfo(ab_and_wts);
(%o6)                             [hashed, 2]
(%i7) gaussunit(exp,4);
(%o7)                             2.350402b0
(%i8) abs(% - tval),fpprec:45;
(%o8)                            2.9513122b-7
(%i9) fpprec;
(%o9)                                 16
(%i10) arrayinfo(ab_and_wts);
(%o10)                       [hashed, 2, [4, 16]]
(%i11) first( ab_and_wts[4, 16] );
(%o11)                   [8.6113631b-1, 3.3998104b-1]
(%i12) second( ab_and_wts[4, 16] );
(%o12)                   [3.4785484b-1, 6.5214515b-1]
\end{myVerbatim}
\newpage
\begin{myVerbatim}
(%i13) lp4 : legenp(4,x);
                                   4       2
                               35 x    15 x    3
(%o13)                         ----- - ----- + -
                                 8       4     8
(%i14) float(solve(lp4));
(%o14)   [x = - 0.861136, x = 0.861136, x = - 0.339981, x = 0.339981]
\end{myVerbatim} 
With the default value of \mv|fpprec = 16| and using only four integrand evaluation
  points, the error is about 2 parts in $\mathbf{10^{7}}$.
The first element of the two index hashed array \mv|ab_and_wts[4,16]| is a list of the 
  positive zeros of the fourth order Legendre polynomial $\mathbf{P_{4}(x)}$ ( calculated with
  the arithmetic precision \textbf{fpprec = 16}).\\
  
\noindent That fourth order Legendre polynomial $\mathbf{P_{4}(x)}$  can be displayed
  with this package using \mv|legenp(4, x)|.
Using \textbf{solve} we see that the roots for negative $\mathbf{x}$ are simply the
  the positive roots with a minus sign, so the algorithm used makes use of this
  symmetry and keeps track of only the positive roots.\\
  
\noindent We can verify that the list of roots returned is correct to within
  the global floating point precision (we do this two different ways): 
\begin{myVerbatim}
(%i15) lfp4(x) := legenp(4,x)$
(%i16) map('lfp4,%o11);
(%o16)                     [0.0b0, - 3.4694469b-18]
(%i17) map( lambda([z],legenp(4,z)),%o11 );
(%o17)                     [0.0b0, - 3.4694469b-18]
\end{myVerbatim}    
The second element of  \mv|ab_and_wts[4,16]| is a list of the weights which are
  associated with the positive roots (the negative roots have the same weights), with
  order corresponding to the order of the returned positive roots.\\
  
\noindent The package function \mv|gaussunit_e(f, N)| does the same job
  as \mv|gaussunit(f, N)|, but returns a rough estimate of the amount
  contributed to the error by the floating point precision used (as the second
  element of a list: 
\begin{myVerbatim}
(%i18) gaussunit_e(exp,4);
(%o18)                    [2.350402b0, 1.2761299b-17]
\end{myVerbatim} 
We see that the error attributable to the floating point precision used is
  insignificant compared to the error due to the low number of integrand evaluation
  points for this example.\\

\noindent An arbitrary finite integration interval is allowed with the functions
  \mv|gaussab(f, a, b, N)| and\\
  \mv|gaussab_e(f, a, b, N)| which use \mv|N|  point 
  Gauss-Legendre quadrature over the interval $\mathbf{[a, b]}$, with the latter function
  being the analog of \mv|gaussunit_e(f, N)|.  
\begin{myVerbatim}
(%i19) gaussab(exp,-1,1,4);
(%o19)                            2.350402b0
(%i20) abs(% - tval),fpprec:45;
(%o20)                           2.9513122b-7
(%i21) gaussab_e(exp,-1,1,4);
(%o21)                    [2.350402b0, 1.2761299b-17]
\end{myVerbatim} 
\newpage
\noindent The package function \mv|quad_gs (f, a, b, rp )| 
  integrates the Maxima function \mv|f| over the finite interval
  $\mathbf{[a, b]}$, successively doubling the
  number of integrand evaluation points, stopping when the absolute value of the
  difference $\mathbf{(I_{n} - I_{n/2})}$ is less than $\mathbf{10^{-rp}}$ 
  (\mv|rp| is the ``requested precision'' for the result), using the global setting of
   \textbf{fpprec} to use the corresponding  precision 
  arithmetic.
We emphasize that Fateman's code uses a global setting of \textbf{fpprec} to achieve higher precision
  quadrature, rather than the method used in the previous two sections in which \textbf{fpprec}
  was set ``locally'' inside a \textbf{block}.  
This function returns the list\\
 \mv|[ approx-value, number-function-evaluations, abs(vdiff) ]|,
  where the last element should be smaller than $\mathbf{10^{-rp}}$.\\
  
\noindent Here we test this function for \mv|fpprec = 16, 30, and 40|.
\begin{myVerbatim}
(%i22) quad_gs(exp,-1,1,10);
                                  fpprec = 16

(%o22)                 [2.3504023b0, 20, 6.6613381b-16]
(%i23) abs(first(%) -tval),fpprec:45;
(%o23)                           6.2016267b-16
(%i24) fpprec:30$
(%i25) quad_gs(exp,-1,1,20);
                                  fpprec = 30

(%o25)                 [2.3504023b0, 20, 1.2162089b-24]
(%i26) abs(first(%) -tval),fpprec:45;
(%o26)                           2.2001783b-30
(%i27) fpprec:40$
(%i28) quad_gs(exp,-1,1,30);
                                  fpprec = 40

(%o28)                 [2.3504023b0, 40, 1.8367099b-40]
(%i29) abs(first(%) -tval),fpprec:45;
(%o29)                           2.7905177b-40
(%i30) gaussab_e(exp,-1,1,40);
(%o30)                   [2.3504023b0, 1.8492214b-41]
\end{myVerbatim} 
We have checked the contribution to the error due to the forty digit arithmetic
  precision used, with $\mathbf{N = 40}$ point Gauss-Legendre quadrature  (remember
  that \textbf{N} is the middle element of the list returned by\\
  \mv|quad_gs (f, a, b, rp )|
  and is also the last argument of the function \mv|gaussab_e(f, a, b, N)|.\\
  
\noindent We see that requesting \textbf{30} digit precision for the answer while using
  the global \textbf{fpprec} set to \textbf{40} results in an answer good to
  about \textbf{39} digits.
Finally, let's check on what abscissae and weight arrays have been calculated
  so far: 
\begin{myVerbatim}
(%i31) arrayinfo(ab_and_wts);
(%o31) [hashed, 2, [4, 16], [10, 16], [10, 30], [10, 40], [20, 16], [20, 30], 
                                                            [20, 40], [40, 40]]
\end{myVerbatim} 
Using \mv|quad_gs (f, a, b, rp )| with \mv|fpprec = 16|
  led to the calculation of the abscissae and weight array for the index pairs
  \mv|[10,16]| and \mv|[20,16]| before the requested precision was achieved (the function
  always starts with \mv|N = 10| point quadrature and then successively doubles
   that number  until success is achieved).\\
  
\noindent Using \mv|quad_gs (f, a, b, rp )| with \mv|fpprec = 30|
  led to the calculation of the abscissae and weight array for the index pairs
  \mv|[10,30]| and \mv|[20,30]| before the requested precision was achieved.\\
  
\noindent Using \mv|quad_gs (f, a, b, rp )| with \mv|fpprec = 40|
  led to the calculation of the abscissae and weight array for the index pairs
  \mv|[10,40]|, \mv|[20,40]|, and \mv|[40,40]| before 
  the requested precision was achieved.\\
  
\noindent Finally, we have the function \mv|quad_gs_table(f, a, b, rp)| which
  prints out a table showing the progression toward success:    
\begin{myVerbatim}
(%i32) quad_gs_table(exp,-1,1,30)$
                                  fpprec = 40
       new val        N           vdiff    
   2.3504023b0       10 
   2.3504023b0       20       1.2162183b-24 
   2.3504023b0       40       1.8367099b-40 
\end{myVerbatim}   
\end{document}